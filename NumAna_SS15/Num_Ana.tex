\documentclass[a4paper, pagesize=pdftex, pdftex, twoside, headsepline, index=totoc,toc=listof, fontsize=10pt, cleardoublepage=empty, headinclude, DIV=13, BCOR=13mm]{scrartcl}

\usepackage[ngerman]{babel}
\usepackage{scrtime} % Bestandteil von KOMA-Skript, ermoeglicht Zugriff auf Uhrzeit des Kompilierens 
\usepackage{scrpage2} % ermöglicht Bearbeiten von Kopf- und Fusszeilen (wie fancyhdr, nur optimiert auf KOMA-Skript, leich andere Syntax)
\usepackage[utf8]{inputenc} % Gibt an in welcher Textcodierung der Code verstandne werden soll
\usepackage{etex} % sehr technisch, ermöglicht LaTeX mehr Speicher zu belegen
\usepackage[T1]{fontenc} % auch sehr technisch; ist wichtig, um die Schriftarten richtig zu behandeln
\usepackage{textcomp} %verhindert ein paar Fehler bei den Fonts
\usepackage{mathtools} % Packet der American Mathematical Society, das viele Mathematik-Umgebungen und -Befehle definiert
\usepackage{amssymb} %zusätzliche Symbole
\usepackage{latexsym} % nochmal zusätzliche Symbole
\usepackage{stmaryrd} % nochmal mehr zusätzliche Symbole, u.a. Blitz für Widerspruchsbeweise ;)
\usepackage{nicefrac} % schräge Brüche, benutzte ich für Quotienvektorräume
\usepackage{paralist} % redefiniert alle Listenbefehle, sodass diese einen optionalen Parameter haben, der die Nummerierung angibt
\usepackage{dsfont} % Schriftart für N,Z,Q,R die ich momentan benutze (mittels \mathds{R} z.B)
\usepackage[pdftex]{graphicx} % Packet, dass das Einbinden von Grafiken aus Dateien ermöglicht
\usepackage{makeidx}% ermöglicht das automatische Anlegen eines Index 
\usepackage{extarrows}
\usepackage{bbold}
\usepackage[hyphens]{url}
\usepackage{algorithmicx}
\usepackage{algpseudocode}


%\usepackage{MnSymbol}
\flushbottom
\usepackage[normalem]{ulem}
\setlength{\ULdepth}{1.8pt}

%--Indexverarbeitung
\newcommand{\bet}[1]{\textbf{#1}} %Betonung von Text
\newcommand{\Index}[1]{\textbf{#1}\index{#1}} % Befehl, der gleichzeitg das Argument hervorhebt und in den Index mitaufnimmt
\makeindex % startet das automatische Sammeln der Index-Einträge
% Ein kleiner Text am Anfang des Index
\setindexpreamble{{\noindent \itshape Die \emph{Seitenzahlen} sind mit Hyperlinks zu den entsprechenden Seiten versehen, also anklickbar!} \par \bigskip}
\renewcommand{\indexpagestyle}{scrheadings} % Seitenstil für den Index festlegen

%--Farbdefinitionen
\usepackage[usenames, table, x11names]{xcolor} %usenames und x11names, aktivieren viele Farben; siehe Dokumentation von xcolor
% Es lassen sich natürlich auch eigene Farben definieren (hier nur Graustufen)
\definecolor{dark_gray}{gray}{0.45}
\definecolor{light_gray}{gray}{0.7}

%--Zum Zeichnen (ich habe es jetzt mal mit aufgenommen, aber es ist eigentlich nochmal ein ganz anderes Thema, sodass ich da jetzt nicht viel zu sagen werde)
\usepackage{tikz} % TikZ steht übrigens für "TikZ ist kein Zeichenprogramm", ein rekursives Akronym ...
\tikzset{>=latex}
\usetikzlibrary{shapes,arrows}
\usetikzlibrary{calc}
\usetikzlibrary{decorations.pathreplacing}
% Hiermit kann man ganz leicht kommutative Diagramme zeichnen (deswegen auch "cd")
\usepackage{tikz-cd}

%--Marginnote, ermöglicht es kleine Notizen an neben den eigentlichen Textkörper zu setzten
\usepackage{marginnote}
\renewcommand*{\marginfont}{\color{Honeydew4} \footnotesize }

%--Schriftarten
\usepackage{lmodern} % neuere Version der Standard-LaTeX-Schriftarten
\renewcommand{\familydefault}{\sfdefault} %Standardschriftart auf die serifenlose Schriftart setzen

%--Hyperref; aktiviert Hyperlinks in der erzeugten PDF-Datei und definiert deren Aussehen
\usepackage[hidelinks, pdfpagelabels,  bookmarksopen=true, bookmarksnumbered=true, linkcolor=black, urlcolor=SkyBlue2, plainpages=false,pagebackref, citecolor=black, hypertexnames=true, pdfauthor={Tobias Wedemeier}, pdfborderstyle={/S/U}, linkbordercolor=SkyBlue2, colorlinks=false]{hyperref}
%--Römische Zahlen
\newcommand{\RM}[1]{\MakeUppercase{\romannumeral #1{}}}



%-- Definitionen von weiteren Mathe-Befehlen, die dann das "richtige" Aussehen haben. Hier sind der Phantasie keine Grenzen gesetzt
\DeclareMathOperator{\id}{id} %identische Abbildung
\DeclareMathOperator{\End}{End} %Endomorphismen
\DeclareMathOperator{\rg}{rg} %Rang
\DeclareMathOperator{\diam}{diam} %Durchmesser
\DeclareMathOperator{\dist}{dist} %Distanz
\DeclareMathOperator{\grad}{grad} %Gradient
\DeclareMathOperator{\rot}{rot} %Rotation
\DeclareMathOperator{\hess}{Hess} %Hesse-Matrix
\DeclareMathOperator{\supp}{supp}
\DeclareMathOperator{\aut}{Aut}
\DeclareMathOperator{\inn}{Inn}
\DeclareMathOperator{\sym}{Sym}
\DeclareMathOperator{\syl}{Syl}
\DeclareMathOperator{\alt}{Alt}
\DeclareMathOperator{\sign}{sign}
\DeclareMathOperator{\Sl}{Sl}
\DeclareMathOperator{\Gl}{Gl}
\DeclareMathOperator{\Quot}{Quot}
\DeclareMathOperator{\ggT}{ggT}
\DeclareMathOperator{\cone}{cone}
\DeclareMathOperator{\Char}{char}
\DeclareMathOperator{\im}{Im}
\DeclareMathOperator{\re}{Re}
\DeclareMathOperator{\Bin}{Bin}
\DeclareMathOperator{\acl}{acl}
\DeclareMathOperator{\cov}{cov}
\DeclareMathOperator{\argmin}{argmin}
\DeclareMathOperator{\argmax}{argmax}
\DeclareMathOperator{\Tol}{TOL}
\DeclareMathOperator{\RK}{RK}
\DeclareMathOperator{\divv}{div}
\DeclareMathOperator{\tr}{tr}
\DeclareMathOperator{\spann}{span}
\DeclareMathOperator{\esssup}{ess sup}
\DeclareMathOperator{\bild}{bild}
\DeclareMathOperator{\cond}{cond}
\DeclareMathOperator{\train}{train}
\DeclareMathOperator{\spur}{spur}
\DeclareMathOperator{\diag}{diag}
%--Skalarprodukt (cooler Befehl, den ich im Internet gefunden habe; benutzt TeX-Befehle)
\makeatletter
\newcommand{\sprod}[2]{\ensuremath{%
  \setbox0=\hbox{\ensuremath{#2}}
  \dimen@\ht0
  \advance\dimen@ by \dp0
  \left\langle \left.#1 \,\rule[-\dp0]{0pt}{\dimen@}\right|#2\right\rangle}}
\makeatother

%--Norm (auch aus dem Internet, wird auch auf der Beispielseite verwandt)
\newcommand{\norm}[1]{
	\ensuremath{\left\Vert#1\right\Vert}
}
\newcommand{\ener}[1]{
	\ensuremath{\left|\left|\left|#1\right|\right|\right|_\mu}
}

%--selbstgeschriebenen Befehle
%--Betrag
\newcommand{\abs}[1]{\ensuremath{\left\vert#1\right\vert}}

%--Umklammern mit passender Größe der Klammern
\newcommand{\enbrace}[1]{\ensuremath{\left( #1\right)}}

%--Mengen
\newcommand{\penbrace}[1]{\ensuremath{\left\{#1\right\}}}
\newcommand{\cenbrace}[1]{\ensuremath{\left[#1\right]}}

%--Differential
\newcommand{\diff}[2]{\ensuremath{\frac{\partial #1}{\partial #2} }}
\newcommand{\difff}[2]{\ensuremath{\frac{\dint #1}{\dint #2} }}

\newcommand{\zz}{\ensuremath{\mathrm{Z\kern-.3em\raise-0.5ex\hbox{$Z$}}}} % zu zeigen ZZ aus dem inet
\setlength{\parindent}{0pt}%absatz nicht einrücken
\newcommand{\lh}[1]{\langle #1 \rangle} %lineare Hülle
\newcommand{\nt}{\trianglelefteqslant} %normalteiler
\newcommand{\pfs}{\mathds{P}-\text{f.s.}} %P fast sicher Konvergenz
\newcommand{\dint}{\mathrm{d}} % d des integrals

\newcommand{\xfrac}[2]{%
	\mbox{\raisebox{-0.4ex}{\ensuremath{\displaystyle #1}\hspace{0.2ex}}%
		{\raisebox{-0.1ex}{\big \backslash}}%
		\raisebox{0.6ex}{\ensuremath{\displaystyle #2}}%
	}%
}
\newcommand{\Pw}{\mathds{P}}
\newcommand{\V}{\mathds{V}}
\newcommand{\E}{\mathds{E}}
\newcommand{\R}{\mathds{R}}
\newcommand{\N}{\mathds{N}}
\newcommand{\Z}{\mathds{Z}}
\newcommand{\Q}{\mathds{Q}}
\newcommand{\G}{\mathcal{G}}
\newcommand{\F}{\mathcal{F}}
\newcommand{\D}{\mathcal{D}}
\newcommand{\C}{\mathds{C}}
\newcommand{\A}{\mathcal{A}}
\newcommand{\mc}[1]{\mathcal{#1}}
\newcommand{\Pfs}[1][\relax]{\ensuremath{
	\ifx#1\relax \Pw-\text{f.s.}
	\else \Pw^{#1}-\text{f.s.}
	\fi}}
\newcommand{\bgl}[1]{\stackrel{#1}{=}}
\newcommand{\dt}[1]{#1 \dots #1}
\newcommand{\ablim}[1]{\limits_{\mathclap{#1}}}
\newcommand{\degree}{\ensuremath{^\circ}}
\newcommand{\adj}{\ensuremath{^\ast}}



\newcommand{\sect}[1]{\section*{#1}\addcontentsline{toc}{section}{#1}}
\newcommand{\ssect}[2]{ \subsubsection*{#1 #2}\addcontentsline{toc}{subsubsection}{#1}}


\usepackage{listings}
\definecolor{light_green}{rgb}{0,0.5,0}
\definecolor{grey}{rgb}{.5,.5,.5}
\lstset{language=Python, commentstyle=\color{light_green}\bfseries,
	keywordstyle=\color{blue}\bfseries,
	stringstyle=\ttfamily\color{orange},
	morekeywords={as},
	deletendkeywords={range,abs,set},
	escapeinside={\%*}{\&*}
}

\lstset{showspaces=false,
	showstringspaces=false, tabsize=4, breaklines=true, rulecolor=\color{black}}
\lstset{numbers=left,basicstyle=\ttfamily\footnotesize, numberstyle=\tiny\color{grey}, numbersep=10pt}

\renewcommand{\lstlistlistingname}{Programmcodeverzeichnis}
\renewcommand{\lstlistingname}{Programmcode}








\newcommand{\vorlesung}{Numerische Analysis}
\newcommand{\Prof}{Prof. Dr. Ohlberger}
\newcommand{\subt}{Mitschrift der Tafelnotizen}

\input{../!config/Tazdr/extra_files/headings.tex}



\begin{document}
\maketitle
\thispagestyle{empty}
\cleardoubleoddemptypage

\thispagestyle{empty}
\vspace*{\fill}
\begin{center}
	Hierbei handelt es sich um eine \subt von \textbf{\Prof}, WWU Münster, aus der Vorlesung \textbf{\vorlesung} im Wintersemester 2014/15. 
	Dies ist kein Skript der Vorlesung und keine eigene Arbeit des Autors.\\
	\vspace{2cm}
	Für Fehler in der Mitschrift wird keine Haftung übernommen. 
	Hinweise auf Fehler sind gerne gesehen, hierfür kann man mich in der Uni ansprechen oder alternativ eine e-Mail an: \textit{tobias.wedemeier@gmx.de}\\
	Auch ist eine Mitarbeit über Github möglich.\\
	\vspace{2cm}
	Wenn Teile aus der Vorlesung selber fehlen, können diese gerne an meine e-Mail versandt werden. 
	Ich werde diese dann einarbeiten.\\
\end{center}
\vspace*{\fill}
\cleardoubleoddemptypage

\pagenumbering{Roman}

\tableofcontents
\cleardoubleoddemptypage %sorgt dafür, dass alles folgende erst auf der nächsten freien "rechten" Seite steht

\thispagestyle{empty}

\setcounter{section}{-1}

\section{Einleitung}
\label{sec:einleitung}

\subsection{Variationsprinzip und Galerkinapproximation}
\label{sub:vartiation_galerkin}
\bet{Beispiel:} Elastizitätstheorie in der Physik:\\
Gesucht: $u:\R^d\to \R,~ d=1,2,3$, Gegeben: Energiefunktional $E:\R\to \R$\\
Aufgabe: Finde $\argmin\ablim{u\in X}E(u)$\\
$u$ entspricht der Auslenkung/Verschiebevektor, $\nabla u$ der Gradient (Jacobimatrix); der Symmetrische Gradient $\frac{1}{2}(\nabla u+ \nabla u^T)=:\epsilon(u)$, dann ist die elastische Gesamtenergie:
\[
E(u):= \frac{1}{2}\int\limits_{\Omega} \Theta:\epsilon(u)\dint x-\int\limits_{\Omega}f(x)u(x)\dint x
\]\marginnote{: ist das Skalarprodukt}
mit symmetrischem \Index{Spannungstensor} $\Theta$ und äußerer Kraft $f:\R^d\to \R^d$.\\
Materialgesetz: Der Spannungstensor ist proportional zum \Index{Verzerrungstensor}:
\begin{equation*}
\begin{aligned}
	\Theta(u) &= A \epsilon(u)\\
	\Theta(u)_{i,j} &= A_{ijkl}\epsilon(u)_{kl}~\forall i,j,k,l=1\dt{,}d
\end{aligned}
\end{equation*}
\[
\Rightarrow E(u)=\frac{1}{2}\int\limits_{\Omega} A\epsilon(u):\epsilon(u)\dint x-\int\limits_{\Omega}f(x)u(x)\dint x
\]
%sub end

\subsection{Definition 1 (Energieminimierung/Variationsprinzip)}
\label{sub:def_1}\index{Variationsprinzip}
\begin{enumerate}[(a)]
	\item Physikalisches Prinzip: Ein physikalisches System strebt immer in einen Zustand minimaler Energie.
	\item Mathematisches Prinzip: Sei $\bar{u}(x,t)$ eine Zustandsvariable und $E(u)$ die Energie eines Systems, das durch $\bar{u}$ repräsentiert wird.
	Dann strebt $\bar{u}$ gegen ein $u=u(x)$, der die Energie minimiert, d.h. falls $E$ genügend glatt ist gilt:
	\[
	\difff{ }{\epsilon}E(u+\epsilon\varphi)|_{\epsilon=0}=0\qquad \forall \text{zulässigen Variationen von }\varphi
	\]
	Elastizität:
	\begin{equation*}
	\begin{aligned}
		\difff{ }{\epsilon}E(u+\epsilon\varphi)|_{\epsilon=0} &= \difff{ }{\epsilon}\cenbrace{\frac{1}{2}\int\limits_{\Omega} A\epsilon(u+\epsilon\varphi):\epsilon(u+\epsilon\varphi)\dint x-\int\limits_{\Omega}f(x)u(x)\dint x}_{\epsilon=0}\\
		&=\int\limits_{\Omega}A\epsilon(u):\epsilon(\varphi)\dint x-\int\limits_{\Omega}f\varphi \bgl{!} 0\\
		&\Rightarrow -\nabla(A\epsilon(u))=f \text{ Dgl.}
	\end{aligned}
	\end{equation*}
	Diese Dgl. gliedert sich auf in 
	\[
	-\sum_{i=1}^{d}\sum_{k,l=1}^{d}\partial_{x_i}A_{ijkl}\epsilon(u)_{kl}=f_i\qquad \forall j=1\dt{,}d
	\]
	Im 1D ergibt sich $-\difff{ }{x}\enbrace{A \difff{ }{x}u}=f$, mit $A\in \R$.
	Für $A=1:~ -u''(x)=f$.\\
	Im 2D ergibt sich:
	\[
	-\nabla(A\nabla u)=f;~A=\id \Rightarrow -\Delta u=f
	\]
\end{enumerate}
%sub end

\subsection{Galerkinverfahren}
\label{sub:galerkin}
\uline{Idee:} Energieminimierung in endlich-dimensionalen Teilräumen.
Sei $X$ Funktionenraum und $E:V\to \R$ ein Energiefunktional.
Gesucht ist $u=\argmin\ablim{v\in X}E(v)$.\\
Sei $X_h\subseteq X$ endlich-dimensionaler Teilraum von $X$.
Wir erhalten die \Index{Galerkin-Approximation}
\[
u_h\in X_h:~ u_h=\argmin\ablim{v_h\in X_h} E(v_h)
\]
\[
\Rightarrow \difff{ }{\epsilon}E(u_h+\epsilon v_h)|_{\epsilon=0}=0~\forall v_h\in X_h
\]
$X_h$ endl.-dim. $\Rightarrow \exists$ Basis $\Phi:=\penbrace{\varphi_i|i=1\dt{,}N:=\dim(X_h)}$, mit der Basisdarstellung $u_h(x)=\sum_{i=1}^{N}u_i\varphi_i(x)$, $u_i\in \R,~i=1\dt{,}N$.
\[
\Rightarrow \difff{ }{\epsilon}E\enbrace{\sum_{i=1}^{N}u_i\varphi_i+\epsilon \varphi_j}_{\epsilon=0}=0~\forall j=1\dt{,}N
\]
Dies ist ein Gleichungssystem mit $N$ Unbekannten und $N$ Gleichungen. 
Allgemein ist das System nicht linear.
%sub end

\subsection{Beispiel Elastizität in 1D}
\label{sub:bsp_elastizitaet}
\[
E(u)= \frac{1}{2}\int\limits_{0}^{1}(u'(x))^2+ fu;~A=1
\]
Betrachte
\begin{equation*}
\begin{aligned}
	\difff{ }{\epsilon}E(u+\epsilon\varphi)|_{\epsilon=0} &= \difff{ }{\epsilon}\frac{1}{2}\int_{0}^{1}(u'+\epsilon\varphi')^2-fu|_{\epsilon=0}\\
	&= \int\limits_{0}^{1}u'\varphi'-f\varphi
\end{aligned}
\end{equation*}
Sei $(u,v):=\int_{0}^{1}uv$ das $L^2$-Skalarprodukt, so folgt
\[
(u',\varphi')=(f,\varphi)~\forall \varphi\in X
\]
Analog folgt für $u_h\in X_h$:
\[
(u_h',\varphi_h')=(f,\varphi_h)~\forall \varphi_h\in X_h
\]
Sei $\varphi_1\dt{,}\varphi_N$ Basis von $X_h$, $u_h=\sum_{i=1}^{N}u_i\varphi_i$.
Dann folgt
\begin{equation*}
\begin{aligned}
	\enbrace{\sum_{i=1}^{N}u_i\varphi_i',\varphi_j'}&=(f,\varphi_j),~j=1\dt{,}N\\
	\Rightarrow \sum_{i=1}^{N}u_i(\varphi_i',\varphi_j')&=(f,\varphi_j),~j=1\dt{,}N\\
	U_i=u_i,~i=1\dt{,}N;~U&\in \R^N,~S_{ij}=(\varphi_i',\varphi_j'),~S\in \R^{N\times N}\\
	F_j=(f,\varphi_i&),~F\in \R^N\\
	\Rightarrow SU=F &\text{ lin. Gleichungssystem}
\end{aligned}
\end{equation*}
%sub end
%sec end
\section{Interpolation}
\label{sec:interpolation}
Sei $\penbrace{\Phi(x,a_0\dt{,}a_n|a_0\dt{,}a_n\in \R)}$ eine Familie von Funktionen mit $x\in \R$.
Ein Element aus dieser Familie ist durch $(n+1)$ Parameter $a_0\dt{,}a_n\in \R$ charakterisiert.\\
\uline{Aufgabe:} Zu $(x_k,f_k)\in \R^2,~k=0\dt{,}n$ mit $x_i\neq x_k$ für $i\neq k$, finde Parameter $a_0\dt{,}a_n\in \R$, so dass
\[
\Phi(x_k,a_0\dt{,}a_n)=f_k,~k=0\dt{,}n.
\]
Dies ist ein Gleichungssystem mit $(n+1)$ Gleichungen und Unbekannten.\\
Familie von linearen prarmeterabh. Funktionen:
Sei $f\in C^0(\R)$ und $V\subseteq C^0(\R)$ sei ein Teilraum mit $\dim(V)=n+1$.
Sei $\varphi_0\dt{,}\varphi_n$ eine Basis von $V$, so setze
\[
\Phi(x,a_0\dt{,}a_n)=\sum_{i=0}^{n}a_i\varphi_i(x)
\]

\subsection{Beispiel: Polynominterpolation}
\label{sub:polynom}
Hier wählt man $V=\Pw_n$ und z.B. $\varphi_i(x)=x^i;~i=0\dt{,}n$.
\[
\Rightarrow \Phi(x,a_0\dt{,}a_n)=\sum_{i=0}^{n}a_ix^i=:p(x)
\]
Aufgabe: Finde $p\in \Pw$ mit $p(x_i)=f_i,~i=0\dt{,}n$.

\subsection{Beispiel: Trigonometrische Interpolation}
\label{sub:trigonometrische}
$\Phi(x,a_0\dt{,}a_n)=a_0+a_1e^{ix}\dt{+}a_ne^{nix}=\sum_{j=0}^{n}a_je^{jix}=a_0=\sum_{k=1}^{n}a_k(\cos(kx)+i\sin(kx))$

\subsection{Beispiel: Nicht lineare Interpolation}
\label{sub:nicht_linear}
Exponentielle Interpolation:
\[
\Phi(x,a_0\dt{,}a_n)=a_0e^{\lambda_0x}\dt{+}a_ne^{\lambda_nx}
\]
mit $\lambda_0\dt{,}\lambda_n\in \R$ fest gewählt oder
\[
\Phi(x,a_0\dt{,}a_n)= \sum_{i=0}^{n}a_ie^{\lambda_ix}\text{ und }(m+1)\cdot 2=n+1
\]

\subsection{Beispiel: Rationale Interpolation}
\label{sub:rational}
\begin{equation*}
\begin{aligned}
\Phi(x,a_0\dt{,}a_n,b_0\dt{,}b_m) &= \frac{a_0\dt{+}a_nx^n}{b_0\dt{+}b_mx^m}
\end{aligned}
\end{equation*}
mit $(m+1)\cdot 2=n+1$.

\subsection{Erweitertes Problem: Hermite-Interpolation}
\label{sub:hermite}\index{Interpolation!Hermite-}
Aufgabe: zu Stützstellen $x_0\dt{,}x_n$ seien die Funktionswerte $f_0\dt{,}f_n$ und Ableitungen $f_0^{(p)}\dt{,}f_n^{(p)},~p=1\dt{,}p_{\max}$ gegeben.
Ist $p_{\max}=1$, so suchen wir ein Interpolationsproblem
\[
p(x)=\sum_{i=0}^N a_ix^i
\]
mit $(n+1)\cdot 2=N+1$ mit $p(x_k)=f_k,~p'(x_k)=f'(x_k),~k=0\dt{,}n$.

\subsection{Beispiel: Spline-Interpolation}
\label{sub:spline}\index{Interpolation!Spline-}
Gesucht: $\Phi\in C^q(\R)$ mit $q$ fest gewählt mit 
\[
\Phi(x_k)=f_k \text{ und } \Phi|_{[x_k,x_{k+1}]}\in \Pw_r.
\] 
Das heißt $(q,r)$ bestimmen die Klasse von \Index{Splines}.

\subsection{Polynominterpolation}
\label{sub:polynominterpolation}
Gegeben: $(x_0,f_0)\dt{,}(x_n,f_n)\in \R^2$ mit $x_i\neq x_k,~i\neq k$.\\
Gesucht: $p\in Pw_N$ mit $p(x_i)=f_i,~i=0\dt{,}n$ und $N$ minimal gewählt.\\
Beispiel: $(x_0,f_0)=(0,0),~(x_1,f_1)=(1,1)$ dann folgt $p\in\Pw_1,~p=x$ ist eindeutiges Interpolationspolynom, aber jedes Monom $x^k$ erfüllt die Interpolationsaufgabe.

\subsection{Satz 1}
\label{sub:satz_1}
Es existiert genau ein $p\in \Pw_n$ mit 
\[
p(x_i)=f_i,~i=0\dt{,}n.
\]

\bet{Beweis:}\\
Sei $\varphi_0\dt{,}\varphi_n$ eine Basis von $\Pw_n$.
Dann ist das Interpolationsproblem äquivalent zu einem linearem Gleichungssystem:
\[
A\cdot a=f \text{ mit }A\in \R^{(n+1)\times(n+1)},~a\in \R^{(n+1)},~f\in \R^{(n+1)}
\]
so dass $p(x)=\sum_{i=0}^{n}a_i\varphi_i(x)$ und $A_{ik}=\varphi_k(x_i)~\forall k,i=0\dt{,}n$, dann folgt
\[
(A\cdot a)_j=\enbrace{\sum_{k=0}^{n}A_{ik}a_k}_j=p(x_j)=f_j
\]
Zeige: $A$ ist regulär.
Sei $a=(a_0\dt{,}a_n)^T$ Lösung der Gleichung $Aa=0$, das heißt
\[
\sum_{k=0}^{n}a_k\varphi_k(x_i)=0~\forall i=0\dt{,}n
\]
Es ist $p(x)=\sum_{k=0}^{n}a_k\varphi_k(x)\in \Pw_n$.
Dann hat $p\in \Pw_n$ mindestens $n+1$ Nullstellen.
Mit dem Fundamentalsatz der Algebra folgt $p\equiv 0$ und somit $a_0\dt{=}a_n=0$.
Also ist $A$ regulär und somit $p\in \Pw_n$ eindeutig bestimmt.
\hfill $\square$

\minisec{Bemerkung}
Interpolation $\Leftrightarrow Aa=f$ mit $A_{ik}=\varphi_k(x_i),~i,k=0\dt{,}n$.\\
1. Ansatz: Monombasis $\varphi_k(x)=x^k \rightsquigarrow p(x)=\sum_{i=0}^{n}a_ix^i$ \Index{Normalform} von $p\in \Pw_N$.
\begin{equation*}
\begin{aligned}
	\Rightarrow A= \begin{pmatrix}
	\varphi_0(x_0) & \dots & \varphi_n(x_0)\\
	\ddots & & \ddots\\
	\varphi_0(x_n) & \dots & \varphi_n(x_n)
	\end{pmatrix} = \begin{pmatrix}
	1 & x_0 & \dots & x_0^n\\
	1 & x_1 & \dots & x_1^n\\
	\ddots & & & \ddots\\
	1 & x_n & \dots & x_n^n
	\end{pmatrix}
\end{aligned}
\end{equation*}
Dies ist die \Index{Vandermondsche Matrix}, insbesondere ist $A$ vol besetzt und sie ist schlecht konditioniert.\\
Idee: Konstruiere eine Basis $\varphi_0\dt{,}\varphi_n$ so, dass gilt
\[
A=\id
\]
Dann wäre $a=f$, d.h. $a_i=f_i~\forall i=0\dt{,}n$ die Lösung des Interpolationsproblems.
\begin{enumerate}[(a)]
	\item Lagrange-Form des Interpolationsproblems:
	\[
	A=\id \Leftrightarrow \varphi_k(x_i)=\delta_{ik}~(0\le k,i\le n)
	\]
	Ansatz: $\varphi_k(x)=c\cdot \prod_{i=0,i\neq k}^{n}(x-x_i) \Rightarrow \varphi_k(x_i)=0~\forall i\neq k$.
	Aus $\varphi_k(x_k)=1$ folgt
	\[
	c=\enbrace{\prod_{i=0, i\neq k}^{n}(x_k-x_i)}^{-1}
	\]
	\[
	\Rightarrow \varphi_k(x)= \prod_{i=0,i\neq k}^{n}\frac{(x-x_i)}{(x_k-x_i)},~k=0\dt{,}n
	\]
\end{enumerate}
\subsection{Defintion 2 (Lagrange-Polynome)}
\label{sub:def_2}
Die Polynome 
\[
l_k^n(x)= \prod_{i=0,i\neq k}^{n}\frac{(x-x_i)}{(x_k-x_i)}
\]
heißen \Index{Lagrange-Polynome} $(l_0^n\dt{,}l_n^n)$ bilden eine Basis von $\Pw_n$ und 
\[
p(x)= \sum_{k=0}^n a_k l_k^n(x)
\]
heißt \Index{Lagrange-Form} von $p\in \Pw_n$.
Es ist 
\[
p(x)= \sum_{k=0}^{n}f_kl_k^n(x)
\]
die Lösung des Interpolationsproblems zu $(x_0,f_0)\dt{,}(x_n,f_n)$.
Für die Lagrange-Polynome gilt 
\[
l_i^n(x_j)=\delta_{ij}.
\]

\minisec{Bemerkung:}
Diese Darstellung ist insbesondere für die Theorie sehr nützlich, edr Nachteil ist, dass die Polynome sich bei Hinzunahme von Stützstellen ändern.

\begin{enumerate}[(b)]
	\item Newton-Form des Interpolationsproblems:\\
	Wähle eine Basis von $\Pw_n$, so dass $A$ eine untere Dreiecksmatrix wird:
	\[
	\varphi_k(x)= \prod_{j=0}^{k-1}(x-x_j) ~k=0\dt{,}n
	\]
	Dann gilt $\varphi_k\in \Pw_n$.
	Dann ist 
	\begin{equation*}
	\begin{aligned}
		\varphi_0(x)&=1 \text{ (verwende die Konvention, dass $\prod_{j=j_0}^{j_n}a_j=1$, falls $j_n<j_0)$}\\
		\varphi_1(x)&=(x-x_0)\\
		\varphi_2(x)&=(x-x_0)(x-x_1)
		&\ddots
	\end{aligned}
	\end{equation*}
	Es gilt $\varphi_k(x_i)=0$ für $i<k \Rightarrow A$ ist eine untere Dreiecksmatrix.
\end{enumerate}

\subsection{Definiton 3 (Newton-Polynome)}
\label{sub:def_3}
Die Polynome
\[
N_k^n:= \prod_{j=0}^{k-1}(x-x_j)
\]
heißen \Index{Newton-Polynome} und 
\[
p(x)=\sum_{k=0}^{n}a_kN_k^n(x)
\]
heißt \Index{Newton-Form} von $p\in \Pw_n$.
Für das Interpolationsproblem gilt:
\begin{equation*}
\begin{aligned}
	a_0&= \frac{f_0}{\varphi_0(x_0)}=f_0\\
	a_1&= \frac{f_1-\varphi_0(x_1a_0)}{\varphi_1(x_1)}=\frac{f_1-f_0}{x_1-x_0}=: f[x_0,x_1]\\
	a_2 &= \dots = \frac{f[x_1,x_2]-f[x_0,x_1]}{x_2-x_0} =: f[x_0,x_1,x_2]
\end{aligned}
\end{equation*}
Die Koeffizienten $a_0\dt{,}a_n$ werden iterativ über die sogenannten \Index{dividierten Differenzen} $f[x_0\dt{,}x_n]$ berechnet ($\rightsquigarrow$ § 3).

\section{Funktionsinterpolation durch Polynome}
\label{sec:funktionsinterpolation}
Gegeben: $x_0\dt{,}x_m$ und $f\in C^0(\R)$\\
Gesucht: Interpolationspolynom zu $(x_0,f(x_0))\dt{,}(x_n,f(x_n))$\\
Frage: Approximationsfehlerabschätzung:
\[
||f-p||_{\infty}\le ??
\]

\subsection{Satz 4 (Fehlerdarstellung)}
\label{sub:satz_4}
Sei $f\in C^{n+1}(a,b)$ und $p\in \Pw_n$ das Interpolationsproblem zu den Stützstellen $x_0\dt{,}x_n$ (paarweise verschieden). 
Dann existiert zu jedem $x\in (a,b)$ ein $\xi_x\in (a,b)$ mit
\[
f(x)-p(x)= \frac{1}{(n+1)!}f^{(n+1)}(\xi_x)\cdot \underbracket{\prod_{k=0}^{n}(x-x_k) }_{\text{Knotenpolynom}}\tag{$\ast$}
\]






























\cleardoubleoddemptypage
\pagenumbering{Alph}
\setcounter{page}{1}


\printindex
\listoffigures
\end{document}