\setcounter{section}{-1}
\section{Einführung}
\label{sec:para0}
\subsection{Einführung und erste Beispiele}
\label{sub:abschnitt_0.1}
\begin{enumerate}
	\item Wir betrachten die algebraische Gruppe $\GL_n = \GL_n(k)$;\marginnote{13.10.} meist ist $k = \CC$ oder allgemeiner $k = \overline{k}$ algebraisch abgeschlossen und $\Char(k) = 0$. $\GL_n = \GL_n(\CC) \subseteq M_n(\CC)$ ist (bzgl. der üblichen Topologie) eine offene Teilmenge. Später betrachten wir die viel gröbere \Index{Zariski-Topologie}. \\
	Sei $m = n^2$ und $\aff^m \subseteq \CC^m$ ein $m$-dimensionaler affiner Vektorraum. Wir können $\GL_n(\CC)$ auch folgendermaßen charakterisieren:
	\[ \GL_n(\CC) = M_n(\CC) \setminus V(\det = 0), \]
	wobei $V(f = 0) = \{x \in \aff^m : f(x) = 0 \}$ die Nullstellenmenge oder \Index{Verschwindungsmenge} der polynomialen Funktion $f\colon \aff^m \rightarrow \CC$, $f \in k[x_1,\dots,x_m]$ bezeichnet.
	\item \textbf{Beispiel:} Sei $\aff^m = M_n(\CC)$. $G := \GL_n(\CC)$ ist eine Gruppe. Können wir die Gruppenverknüpfung, die Inversenbildung und das neutrale Element als polynomiale Funktion auffassen? \setlength{\arraycolsep}{1pt}
	\[ \begin{array}{rrclcrrclcrrcl}
		\mu\colon & G \times G & \longrightarrow & G & \phantom{abstand} & i\colon & G & \longrightarrow & G & \phantom{abstand} & e\colon & \{x\} & \longrightarrow & G \\
		 & (g,h) & \longmapsto & gh & & & g & \longmapsto & g^{-1} & & & x & \longmapsto & e = \ind_n
	\end{array} \]
	$e$ ist offensichtlich polynomial und $\mu$ auch (vgl. Matrixmultiplikation). Jedoch ist $i$ nicht polynomial, da nach Cramerscher Regel gilt: \setlength{\arraycolsep}{3pt}
	\[ A^{-1} = \textcolor{red}{\frac{1}{\det(A)}} \cdot \operatorname{adj}(A) \]
	Daher müssen wir auch Quotienten polynomialer Abbildungen mit Nenner ungleich 0 zulassen.
\end{enumerate}

\minisec{Idee}
	Eine lineare algebraische Gruppe ist abgeschlossen in $\GL_n$ und die Gruppenabbildungen sind Quotienten polynomialer Abbildungen, sodass folgende Eigenschaften erfüllt sind (d.h. die folgenden Diagramme sind kommutativ):
	\begin{description}
		\item[Assoziativität:] $\mu \circ (\mu,\id) = \mu \circ (\id,\mu)$
		\[ \begin{tikzcd}
		G \times G \times G \rar{(\mu,\id)} \dar{(\id,\mu)} & G \times G \dar{\mu} \\
		G \times G \rar{\mu} & G
		\end{tikzcd} \qquad	\begin{tikzcd}
		(g,h,l) \rar[mapsto] \dar[mapsto]& (gh, l) \arrow[d,end anchor={[xshift=-1.5em,yshift=-0.5em]north east},mapsto]\\
		(g, hl) \rar[mapsto] & g(hl) \stackrel{!}{=} (gh)l
		\end{tikzcd} \]
		\item[Inverse:] $e \circ p = \mu \circ (\id,i) \circ \Delta$ bzw. $e \circ p = \mu \circ (i,\id) \circ \Delta$
		\[ \begin{tikzcd}
		\mbox{} & G \times G \dar{(\id, i)} \\
		G \rar \dar{p}  \urar{\Delta} & G \times G \dar{\mu} \\
		\{x\}  \rar{e} & G
		\end{tikzcd} \qquad	\begin{tikzcd}
		\mbox{} & (g,g) \dar[mapsto] \\
		g \rar[mapsto] \urar[mapsto] \dar[mapsto]& (g, g ^{-1}) \arrow[d,end anchor={[xshift=0.5em,yshift=-0.5em]},mapsto] \\
		e \rar[mapsto]& e \stackrel{!}{=} g \cdot g ^{-1}
		\end{tikzcd} \]
		\item[Neutrales Element:] $\mu \circ (e,\id) = \id = \mu \circ (\id,e)$
		\[ \begin{tikzcd}
		G \rar{(e,\id)} \arrow[bend right, swap,end anchor={[xshift=0.5em]}]{rr}{\id} & G \times G \rar{\mu} &  G
		\end{tikzcd} \qquad \begin{tikzcd}
		g \rar[mapsto] \arrow[bend right,mapsto,end anchor={[xshift=2.5em]}]{rr} & (e,g) \rar[mapsto] & eg \stackrel{!}{=} g
		\end{tikzcd} \]
	\end{description}
	
\begin{defn}[polynomiale und rationale Funktion]
	\begin{itemize}
		\item $f \colon \aff^n \rightarrow \aff^m$ heißt \bet{polynomial} bzw. \bet{Morphismus affiner Varietäten}, falls
		\[ \enbrace[\big]{f_1(x_1,\dots,x_n),\dots,f_m(x_1,\dots,x_n)} = f(x_1,\dots,x_n) \]
		mit $f_j \in k[x_1,\dots,x_n]$ für alle $j \in \{1,\dots,n\}$. \index{polynomiale Funktion} \index{Morphismus!affiner Varietäten} 
		\item Sei $U \subseteq \aff^n$ offen (im klassischen Sinne oder bzgl. Zariski-Topologie). $f\colon U \rightarrow \aff^m$ heißt \bet{polynomial}, falls 
		$f = \frac{h}{g}$ mit $h,g \in k[x_1,\dots,x_n]$ und $g(x) \neq 0$ für alle $x \in U$. \hfill $\lceil f_j = \frac{h_j}{g_j}$ komponentenweise$\rfloor$
		\item $f$ wie oben heißt \bet{rational}, fall $f = \frac{h}{g}$ wie oben mit $g \not\equiv 0$ auf $U$. $f$ ist im Allgemeinen keine Abbildung, sondern nur 
		auf $U \setminus V(g = 0)$ definiert. \index{rationale Funktion}
	\end{itemize}
\end{defn}

\begin{bsp}[Beispiele für polynomiale und rationale Funktionen]
	\begin{itemize}
		\item $\GL_n \rightarrow \GL_n$ mit $g \mapsto g^{-1}$ ist polynomial, $M_n \rightarrow M_n$ mit \enquote{$g \mapsto g^{-1}$} ist rational.
		\item $\det\colon \GL_n \rightarrow \CC$ und $\det\colon M_n \rightarrow \CC$ ist polynomial.
		\item $\det^{-1}\colon M_n \rightarrow \CC$ ist rational und  $\det^{-1}\colon \GL_n \rightarrow \CC$ ?
		\item $\aff^1 = \CC \rightarrow \CC^\times$ mit $z \mapsto e^{2\pi i z}$ ist nicht polynomial.
	\end{itemize}
\end{bsp}

\begin{defn}[Zariski-Topologie] \label{1.3}
	Seien $g_1,\dots,g_m \in k[x_1,\dots,x_n]$\marginnote{16.10.} und $I = \sprod{g_1,\dots,g_m}$ das von $g_1,\dots,g_m$ erzeugte Ideal in $k[x_1,\dots,x_n]$.\linebreak $M \subseteq \aff^n$ heißt abgeschlossen bzgl. der \Index{Zariski-Topologie}, falls gilt:
	\[ M = V(I) = \{x \in \aff^n : g_i(x) = 0 \text{ für } 1 \leq i \leq m \}\]
\end{defn}

\minisec{Beispiel}
	\begin{itemize}
		\item Es ist $V(0) = \aff^n$ und $V(1) = \emptyset$.
		\item $GL_n(\CC)$ ist offen in $M_n(\CC)$.
		\item $GL_n(\CC)$ lässt sich jedoch auch "abschließen" vermöge:
		\begin{equation}
		\begin{aligned}
			\GL_n(\CC) &\hookrightarrow M_n(\CC) \times \CC \\
			A &\mapsto \enbrace*{A, \frac{1}{\det(A)}}
		\end{aligned}
		\end{equation}
	\end{itemize}

\setlength{\fboxsep}{10pt}
\setlength{\fboxrule}{3pt}
\begin{center}
	\fbox{\textbf{Ab jetzt immer, wenn nichts anderes gesagt, offen und abgeschlossen bezüglich Zariski-Topologie!}}
\end{center}

\begin{defn}[Lie-Algebra und Lie-Klammer] \label{1.4}
	Sei $V$ ein (endlichdimensionaler) $k$-Vektorraum. Eine Abbildung
	\begin{equation}
	\begin{aligned}
		[\cdot,\cdot]\colon V \times V &\longrightarrow V \\
		(v,w) &\longmapsto [v,w]
	\end{aligned}
	\end{equation}
	heißt \Index{Lie-Klammer}, falls sie folgende Eigenschaften erfüllt:
	\begin{enumerate}[1)]
		\item bilinear
		\item schiefsymmetrisch, d.h. $[v,w] = -[w,v]$
		\item \Index{Jacobi-Identität}: $\benbrace*{[u,v],w}+\benbrace*{[v,w],u}+\benbrace*{[w,u],v} = 0$
	\end{enumerate}
	Das Paar $(V,[\cdot,\cdot])$ heißt \Index{Lie-Algebra}.
\end{defn}

\minisec{Beispiele}
	\begin{itemize}
		\item Der $k$-Vektorraum $V$ mit der trivialen Lie-Klammer $[v,w] := 0$ heißt abelsche Lie-Algebra.
		\item Die allgemeine lineare Lie-Algebra $\gl_n(k) := M_n(k)$ mit der Lie-Klammer $[A,B] := AB-BA$.
		\item Die spezielle lineare Lie-Algebra $\slie_n(k) := \SL_n(k)$ als Lie-Unter-Algebra von $\gl_n(k)$, zum Beispiel für $n=2$ und $k = \CC$:
		\[\slie_2(\CC) := \{A \in M_2(\CC) : \tr(A) = 0 \} = \sprod*{
			h = \begin{pmatrix} 1 & 0 \\ 0 & -1	\end{pmatrix},
			e = \begin{pmatrix}	0 & 1 \\ 0 & 0 \end{pmatrix},
			f = \begin{pmatrix}	0 & 0 \\ 1 & 0 \end{pmatrix}} \]
		Es ist $[h,e] = 2e, [h,f] = -2f$ und $[e,f] = h$, also besitzt die Abbildung $[h,\cdot]\colon V \rightarrow V$ die Eigenwerte $2$ und $-2$.
	\end{itemize}
	
\begin{defn}[lineare algebraische Gruppe]\label{1.5}
	Sei $G \leq \GL_n(\CC)$ eine abgeschlossene Untergruppe, dann heißt $G$ \Index{lineare algebraische Gruppe}.
\end{defn}

\begin{center}
	\fbox{\textbf{Ab jetzt: "algebraische Gruppe" = "lineare algebraische Gruppe" = "affine algebraische Gruppe"}}
\end{center}
	
	Zu einer algebraischen Gruppe $G$ definieren wir die zugehörige Lie-Algebra durch
	\[ \Lie(G) := \g := T_eG, \]
	wobei $T_eG$ den Tangentialraum in $e \in G$ bezeichnet.
	
	\[ T_eV(f_1,\dots,f_n) = \penbrace*{x \in \aff^n : \frac{d}{dt} f_i(e+tx) \bigg|_{t=0} = 0} \text{ über } k = \CC \]
	
\minisec{Beispiele}
	\begin{itemize}
		\item Betrachte $f = x_1^2-x_2$ in $\aff^2$, dann ist
		\begin{equation}
		\begin{aligned}
			T_{(0,0)}V(f) &= \penbrace*{x \in \aff^2 : \frac{d}{dt} f( (0,0) + t(x_1,x_2)) \big|_{t=0} = 0 } = \penbrace*{x \in \aff^2 : \frac{d}{dt} (tx_1)^2 - tx_2 \big|_{t=0} = 0} \\
			&= \penbrace*{x \in \aff^2 : 2tx_1 - x_2 \big|_{t=0} = 0} = \{ x \in \aff^2 : x_2 = 0 \} \\
			T_{(1,1)}V(f) &= \{ x \in \aff^2 : 2x_1 - x_2 = 0 \}
		\end{aligned}
		\end{equation}
		\item $T_{E}\SL_n(\CC) = \slie_n(\CC) = \penbrace*{A \in M_n(\CC) : \frac{d}{dt} \det(E + tA) - 1 \big|_{t=0} = 0} = \{A \in M_n(\CC) : \tr(A) = 0 \}$, denn:
		\[ \det(E+tA) = \det \underbrace{\begin{pmatrix}
		1+ta_{11} &  & ta_{ij} \\ 
		& \ddots &  \\ 
		ta_{ij} &  & 1+ta_{nn}
		\end{pmatrix}}_{=:(c_ij)_{ij}} = \sum\limits_{\sigma \in S_n} \sgn(\sigma) \prod\limits_{i=1}^n c_{i,\sigma(i)} = 1 + t \cdot \tr(A) + t^2\dots + t^n \]
		\[ \Rightarrow \frac{d}{dt} \det(E+tA) \big|_{t=0} = \tr(A) \]
		\item Die Untergruppe
		\[ B := \penbrace*{ b \in \GL_n : b_{ij} = 0 \text{ für } i>j} = \penbrace*{(b_{ij}) \in M_n(\CC) : b_{ii} \in \CC^\times, b_{ij} = 0 \text{ für } i>j} \subseteq \GL_n = G\]
		heißt \Index{Borel-Untergruppe}. Die zugehörige Lie-Algebra ist gegeben durch:
		\[ \Lie(B) = \mathfrak{b} = \penbrace{ (b_{ij}) : b_{ij} = 0 \text{ für } i > j} \supseteq B \text{ offen} \]
		$G/B$ ist eine algebraische Varietät und nicht affin.
		\[U := \penbrace*{(b_{ij}) \in M_n(\CC) : b_{ii} = 1, b_{ij} = 0 \text{ für } i>j} \subseteq B\]
		heißt \Index{unipotente Gruppe}.
	\end{itemize}

\subsection{Aktionen algebraischer Gruppen}
\label{sub:abschnitt_0.2}

\begin{defn}[Darstellung einer linearen algebraischen Gruppe] \label{def_2.1}
	Sei $G$ eine lineare algebraische Gruppe und $X = V(f_1,\dots,f_r)$ abgeschlossene Teilmenge in $\AA^n$. \marginnote{20.10.} Eine \Index{Darstellung} von $G$ auf einem $\CC$-Vektorraum $W$ ist ein polynomialer Gruppenhomomorphismus $G \rightarrow \GL(W)$. $G$ operiert dann auf $W$ vermöge
	\begin{equation}
	\begin{aligned}
		G \times W &\longrightarrow W \\
		(g,w) &\longmapsto f(g)w
	\end{aligned}
	\end{equation}
	Eine Aktion von $G$ auf $x$ ist eine polynomiale Abbildung $G \times X \rightarrow X$, sodass $(gh)x = g(hx)$ und $ex = x$ für alle $g,h \in G, x \in X$.
\end{defn}

\begin{bsp}
	\begin{enumerate}[1)]
		\item $G = \GL_n(\CC) = \GL(\CC^n) = \GL(W)$ mit Standardaktion auf $W = \CC^n$. \\
		Orbiten: $\setnull, W \setminus \setnull$ \\
		Stabilisatoren: $\Stab_{\GL_n}(\setnull) = \GL_n$ und
		\[ \Stab_{\GL_n}(e_1) = \penbrace*{\left( \begin{BMAT}[5pt]{cc}{cc}
		1 & * \\
		0 & \GL_{n-1} 
		\addpath{(0,1,|)rr} \addpath{(1,2,|)dd}
		\end{BMAT}\right)} = P(1,n-1) \qquad \text{(parabolische Untergruppe)}\]
		\item \setlength{\arraycolsep}{1pt} \[\begin{array}{rcll}
		\GL_n & \longrightarrow & \CC^\times  & \qquad \text{Polynomialer Gruppenhomomorphismus gegeben durch } \det^m \text{ mit } m \in \ZZ \\ 
		 &  &  &  \\ 
		B & \longrightarrow & \CC^\times & \qquad B \twoheadrightarrow H \rightarrow \CC^\times, A \mapsto \lambda_1^{a_1} \cdots \lambda_n^{a_n}, a_i \in \ZZ (B/H =U) \\ 
		 &  &  &  \\ 
		H & \longrightarrow & \CC^\times & \qquad \text{z.B. } (t_1,\dots,t_n) \mapsto t_1^{a_1} \cdots t_n^{a_n}, a_i \in \ZZ \\ 
		 &  &  &  \\ 
		\CC^\times & \longrightarrow & \CC^\times & \qquad t \mapsto t^m, m \in \ZZ
		\end{array} \]
	\end{enumerate}
\end{bsp}
\newpage