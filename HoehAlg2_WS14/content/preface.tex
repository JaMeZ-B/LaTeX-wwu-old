\section*{Vorwort}
\label{sec:preface}
	Der vorliegende Text ist eine Mitschrift zur Vorlesung Höhere Algebra II, gelesen von Prof. Dr. Lutz Hille an der WWU Münster im Wintersemester 2014/2015. Der Inhalt entspricht weitestgehend dem Tafelanschrieb während der Vorlesung. Für die Korrektheit des Inhalts wird keinerlei Garantie übernommen. Bemerkungen, Korrekturen und Ergänzungen kann man folgenderweise loswerden:
	\begin{itemize}
		\item persönlich durch Überreichen von Notizen oder per E-Mail
		\item durch Abändern der entsprechenden TeX-Dateien und Versand per E-Mail an mich
		\item direktes Mitarbeiten via GitHub. Dieses Skript befindet sich im \texttt{latex-wwu}-Repository von JaMeZ-B:
		\begin{center}
			\url{https://github.com/JaMeZ-B/latex-wwu}
		\end{center}
	\end{itemize}

\subsection*{Themenübersicht}
\label{sub:content}
	Hier kommt eine Themenübersicht hin -- oder vielleicht auch nicht.

\subsection*{Literatur}
\label{sub:lit}
	Hier kommen Literaturempfehlungen des Dozenten hin, sofern dieser welche angibt.
%\begin{itemize}
%	\item F. Ischebeck: \href{http://wwwmath.uni-muenster.de/u/ischebeck/}{Einladung zur Zahlentheorie}
%	\item R. Remmert, P. Ullrich: \href{http://link.springer.com/book/10.1007/978-3-7643-7731-1}{Elementare Zahlentheorie}
%	\item A. Scholz, B. Schöneberg: Einführung in die Zahlentheorie
%	\item K. Halupczok: \href{http://wwwmath.uni-muenster.de/u/karin.halupczok/ElZthSS2009Skript.pdf}{Skript zur Elementaren Zahlentheorie}
%\end{itemize}

\subsection*{Vorlesungswebsite}
\label{sub:link}
	Folgt noch.
%\begin{center}
%	\url{\homepage}
%\end{center}



\vfill
\begin{flushright}
	Phil Steinhorst \\
	p.st@wwu.de
\end{flushright}
\newpage