% % % % % % 17. Juni
\section{Parabolische PDGL}
\label{sec:para11}
	
\begin{defn}[Wärmeleitungsgleichung]
	Die \Index{Wärmeleitungsgleichung} \marginnote{17. Jun}
	\begin{equation}
		u_t - \Delta u = 0 \label{eq_32}
	\end{equation}
	beschreibt die zeitliche Entwicklung einer diffundierenden Größe (z.B. Wärme).
	\begin{tabbing}
	\hspace{3cm}\=\kill
	Temperatur: \> $u \colon (0,T) \times \Omega \rightarrow \RR$ (in einem Stück Material $\Omega$ zur Zeit $(0,T)$)\\ 
	Leitfähigkeit: \> $a > 0$ (Materialparameter)\\ 
	Wärmekapazität: \> $\kappa > 0$ (Materialparameter)\\ 
	Wärmefluss: \> $F = -a \nabla u$ (in Richtung des negativen Temperaturgradienten)\\ 
	Nettofluss: \> $- \int_{\der V} F \cdot \nu dx$ nach $V \subset \Omega$\\ 
	Wärmeänderung: \> $\frac{d}{dt} \int_V \kappa u dx$ in $V$ 
	\end{tabbing} 
	Wärmeänderung = Nettofluss $\Rightarrow$
	\[ 0 = \frac{d}{dt} \int_V \kappa u dx + \int_{\der V} F \cdot \nu dx = \int_V \kappa u_t + \diver F dx = \int_V \kappa u_t - a \Delta u dx \]
	und \eqref{eq_32} folgt für $\kappa = a$, da $V$ beliebig ist.
\end{defn}
	
	\mbox{} \\
	Parabolische Gleichungen sind eng mit elliptischen verwandt, daher folgen wir etwa den elliptischen Methoden. Diesmal beginnen wir jedoch mit der Fundamentallösung, da Mittelwertformeln und Maximumsprinzipien dann einfacher folgen.
	
	\mbox{} \\
	Suche nach einer einfachen, radialsymmetrischen Lösung, z.B. der Form
	\[ u(t,x) = t^{-\alpha} v(t^{-\beta} r) \]
	mit $r = |x|$. Mit $\Delta = \der_r^2 + \frac{n-1}{r} \der_r$ und $y = t^{-\beta} r$ ergibt sich
	\[ -\alpha t^{-\alpha - 1} v(y) - \beta t^{-\alpha-1} y v'(y) - t^{-\alpha - 2\beta} v''(y) + \frac{n-1}{y} t^{-\alpha - 2\beta} v'(y) = 0\]
	Wähle $\beta = \frac{1}{2}$. Mit $\alpha = \frac{n}{2}$ wird dies zu
	\[ 0 = [y^{n-1} v' + \frac{1}{2} y^n v]' = [vy^{n-1} (\log v + \frac{y^2}{4})']' \]
	d.h. eine Lösung ist $v = c \exp \enbrace*{-\frac{y^2}{4}}$.
	
\begin{defn}[Fundamentallösung] \label{def_83}
	Die Funktion \marginnote{[83]}
	\[ \Phi(t,x) = \begin{cases}
		\frac{1}{(4\pi t)^{n/2}} \exp \enbrace*{ - \frac{|x|^2}{4t}} & (t > 0) \\
		0 & (t < 0) \end{cases} \]
	löst \eqref{eq_32} auf $(\RR \setminus \setnull) \times \RR^n$ und heißt \Index{Fundamentallösung} der Wärmeleitungsgleichung.
\end{defn}
	
\begin{bem} \label{bem_84}
	Zur Zeit $t > 0$ \marginnote{[84]} ist die Fundamentallösung offfensichtlich eine $n$-dimensionale Normalverteilung/Gaußsche Glockenkurve mit Mittelwert 0 und Standardabweichung $\sqrt{2t}$.
\end{bem}
	
\minisec{Beweis}
	\begin{description}
		\item[$n=1$:]
		\[ \enbrace*{ \int_R \frac{1}{\sqrt{4 \pi t}} e^{-\frac{x^2}{4t}} dx}^2 = \int_{\RR} \frac{1}{4\pi t} e^{-\frac{x^2+y^2}{4t}} dxdy \overset{PK}{=} \int_{r=0}^{\infty} 2\pi r \frac{1}{4\pi t} e^{-\frac{r^2}{4t}} dr = \left \lbrack -e^{- \frac{r^2}{4t}} \right\rbrack_{r=0}^\infty = 1 \]
		\item[$n > 1$:]
		\[ \int_{\RR^n} \frac{1}{\sqrt{4\pi t}^n} e^{- \frac{|x|^2}{4t}} dx = \prod\limits_{i=1}^{n} \int_{\RR} \frac{1}{\sqrt{4\pi t}} e^{-\frac{|x_i|^2}{4t}} dx_i = 1 \]
	\end{description}
	
\mbox{} \\
Definiere: \begin{itemize}
	\item $\Omega_T = \Omega \times (0,T]$ \\
	\item $\Gamma_T = \overline{\Omega}_T \setminus \Omega_T = \setnull \times \Omega \cup [0,T] \times \der \Omega$
	\item \Index{Wärmeball} für $t \in \RR, x \in \RR^n$ (ein Levelset der Fundamentallösung)
	\[ E(t,x,r) = \penbrace*{(s,y) \in \RR^{n+1} : s \leq t, \Phi(t-s,x-y) \geq \frac{1}{r^n}} \]
\end{itemize}

\begin{thm}[Mittelwertformel] \label{thm_85}
	Sei $u \in C^1((0,T]),C^2(\Omega,\RR)$ eine Lösung von \eqref{eq_32} in $\Omega_T$, dann gilt \marginnote{[85]}
	\begin{equation}
		u(t,x) = \frac{1}{4r^n} \int_{E(t,x,r)} u(s,y) \frac{|x-y|^2}{(t-s)^2} dyds =: U(r) \label{eq_33}
	\end{equation}
	für jedes $E(x,r,t) \subset \Omega_T$. (Beachte: Formel integriert nur über Vergangenheit!)
\end{thm}
	
\minisec{Beweis}
	O.B.d.A sei $t=0, x=0$. $U(r) = \frac{1}{4} \int_{E(0,0,r)} u(r^2 \tilde{s},r \tilde{y}) \frac{|0- \tilde{y|^2}}{(0-\tilde{s})^2} d\tilde{y} d\tilde{s}$. \\
	\begin{equation}
	\begin{aligned}
	U'(r) &= \frac{1}{4} \int_{E(0,0,1)} 2rs u_t(r^2 s,ry) \frac{y^2}{s^2} + yu_x(r^2s,ry) \frac{y^2}{s^2} dyds \\ \notag 
	&= \frac{1}{4r^{n+1}} \int_{E(0,0,r)} \nabla u(s,y) \cdot y \frac{|y|^2}{s^2} + 2u_s \frac{|y|^2}{s} dyds =: A+B.
	\end{aligned}
	\end{equation}
	Sei $\psi := \log(r^n \Phi(-s,y)) = -\frac{n}{2} \log (-urs) + \frac{|y|^2}{4^s} + n \log r$. Wegen $\Phi(-s,y) = r^{-n}$ auf $\der E(0,0,r)$ ist dort $\psi = 0$.
	\begin{equation}
	\begin{aligned}
		B &= \frac{1}{4r^{n+1}} \int_{E(0,0,r)} 4u_s y \cdot \nabla \psi dy ds \overset{P.I.}{=} -\frac{1}{4r^{n+1}} \int_{E(0,0,r)} \psi n u_s + \psi y \cdot \nabla u_s dy ds \\ \notag
		&\overset{P.I.}{=} \frac{1}{r^{n+1}} \int_{E(0,0,r)} -\psi n u_s + \underbrace{\psi_s}_{-\frac{n}{2s} - \frac{|y|^2}{4s^2}} y \cdot \nabla u dy dy \\
		&= \frac{1}{r^{n+1}} \int_{E(0,0,r)} -\psi n \Delta u - \frac{n}{2s} y \cdot \nabla u dy ds - A = \frac{1}{r^{n+1}} \int_{E(0,0,1)} n \nabla \psi \cdot \nabla u - \frac{n}{2s} y \cdot \nabla u dy ds -A = -A
	\end{aligned}
	\end{equation}
	$\Rightarrow U'(r) = 0 \Rightarrow U(r) = \lim\limits_{\rho \rightarrow 0} U(\rho) = u(0,0) \lim\limits_{\rho \rightarrow 0} \underbrace{\frac{1}{4\rho^n} \int_{E(0,0,\rho)} \frac{|y|^2}{s^2} dyds}_{=\frac{1}{4} \int_{E(0,0,1)} \frac{|y|^2}{s^2} dyds = 1}$ \qed

\begin{thm}[Starkes Maximumprinzip] \label{thm_86}
	$u \in C^1((0,T];C^2(\Omega,\RR)) \cap C(\Omega)$ sei eine Lösung von \eqref{eq_32} in $\Omega_T$, dann gilt: \marginnote{[86]}
	\[ \max_{\overline{\Omega_T}} = \max_{\Gamma_T} u \]
	Ist $u(t,x) = \max_{\overline{\Omega_T}} u$ für $(t,x) \in \Omega_T$, so ist $u$ konstant auf $\Omega_t$, falls $\Omega$ zusammenhängend ist. \\
	(Analog für Minima durch Ersetzen von $u$ mit $-u$.)
\end{thm}

\minisec{Beweis} 
	Sei $M = u(x,t)$ \marginnote{20. Jun} ein Maximum. Nach Theorem \ref{thm_85} ist $M = \frac{1}{4r^n} \int_{E(t,x,r)} u(s,y) \frac{|x-y|^2}{|t-s|^2} dyds \leq M$, da \linebreak $\int_{E(t,x,r)} \frac{|x-y|^2}{(t-s)^2} dyds = 1$. Gleichheit gilt genau dann, wenn $u \equiv M$ auf $E(t,x,r)$. Um jeden Punkt $(s,y)$ von $E(t,x,r)$ kann wieder ein Wärmeball gezeichnet werden, auf dem $u \equiv M$ ist. So kann $\Omega_t$ mit Wärmebällen überdeckt werden, auf denen $u$ jeweils gleich $M$ ist. \qed

\begin{bem}[Unendliche Ausbreitungsgeschwindigkeit] \label{bem_87}
	Sei $u \in C^1((0,T];C^2(\Omega,\RR)) \cap C(\Omega_T)$ eine Lösung des Anfangs-Randwertproblems \marginnote{[87]}
	\[ \begin{cases}
		u_t - \Delta u = 0 & \text{ auf } \Omega_T \\
		u = 0 & \text{ auf } [0,T] \times \der \Omega \\
		u = g \text{ auf } \setnull \times \Omega \end{cases} \]
	für $g \geq 0$. Aus dem Maximumsprinzip folgt, dass $u > 0$ auf $\Omega_T$, falls ein $x \in \Omega$ existiert mit $g(x) > 0$, d.h. die Anfagswert-Information ist unendlich schnell überall hingeflossen.
\end{bem}
	
\begin{defn}[Inhomogene Wärmeleitung]
	Die inhomogene Wärmeleitungsgleichung
	\begin{equation}
		u_t(t,x) - \Delta u(t,x) = f(t,x) \label{eq_34}
	\end{equation}
	hat einen Quellterm $f$. Adäquate Randbedingungen für parabolische Gleichungen sind eine Anfangsbedingung
	\begin{equation}
		u=g \text{ für } t = 0 \label{eq_35}
	\end{equation}
	und eine Dirichlet- oder Neumann-Randbedingung
	\begin{equation}
		u = h \text{ auf } \der \Omega \label{eq_36}
	\end{equation}
	oder
	\begin{equation}
		\der u / \der v = h \text{ auf } \der \Omega \label{eq_37}
	\end{equation}
\end{defn}

\begin{thm}[Eindeutigkeit] \label{thm_88}
	Sei $g \in C(\Omega), h \in C([0,T] \times \der \Omega)$, \marginnote{[88]} dann ist eine Lösung $u \in C^1((0,T];C^2(\Omega,\RR)) \cap C(\Omega_T)$ von \eqref{eq_34}, \eqref{eq_35} und \eqref{eq_36} eindeutig.
\end{thm}

\minisec{Beweis}
	Das folgt aus dem starken Maximumsprinzip für $u - \tilde{u}$, wenn $u,\tilde{u}$ zwei Lösungen sind. \qed
	
\begin{thm}[Fundamentallösung] \label{thm_89}
	Es gilt \marginnote{[89]}
	\[ \begin{cases}
		\mathcal{L}[\Phi] := \Phi_t - \Delta \Phi = 0 & \text{ auf } (0,\infty) \times \RR^n \\
		\Phi(0,x) = \delta(x) \end{cases} \]
	in dem Sinn, dass für jedes $f \in C_c^1(\RR^n)$ gilt:
	\[ \lim_{t\rightarrow 0} \int_{\RR^n} \Phi(t,x) f(x) dx = f(0) \]
\end{thm}
		
\minisec{Beweis}
	Nebenrechnung:
	\begin{equation}
	\begin{aligned}
		& \ \enbrace*{ \overbrace{\int_{\RR \setminus [-\sqrt[4]{t},\sqrt[4]{t}]} \frac{1}{\sqrt{4 \pi t}} e^{- \frac{x^2}{4t}} dx}^{=:A_t}}^2 = \int_{\RR^2 \setminus [-\sqrt[4]{t},\sqrt[4]{t}] \times [-\sqrt[4]{t},\sqrt[4]{t}]} \frac{1}{\sqrt{4 \pi t}} e^{- \frac{x^2+y^2}{4t}} dxdy \\ \notag
		\leq \ & \ \int_{r=2 \sqrt[4]{t}}^{\infty} 2\pi r \frac{1}{4\pi t} e^{- \frac{r^2}{4t}} dr = \left\lbrack -e^{-\frac{r^2}{4t}} \right\rbrack_{r = 2\sqrt[4]{t}}^\infty = e^{-\frac{1}{\sqrt{t}}}
	\end{aligned}
	\end{equation}
	\begin{equation}
		\Rightarrow \int_{\RR^n} \Phi(t,x) f(x) dx = \int_{\RR^n \setminus [-\sqrt[4]{t},\sqrt[4]{t}]^n} \Phi(t,x) f(x) dx + \int_{[-\sqrt[4]{t},\sqrt[4]{t}]^n} \Phi(t,x) f(x) dx \notag
	\end{equation}
	Der erste Summand ist betragsäßig kleiner als $A_t^n \max_\RR^n |f| \xlongrightarrow[t \rightarrow 0]{} 0$. Der zweite Summen liegt zwischen zwischen dem Minimum und dem Maximum von $f(1-A_t)$ auf $[-\sqrt[4]{t},\sqrt[4]{t}]$; beides strebt gegen $f(0)$ für $t \rightarrow 0$. \qed

% % % % 24. Juni
\begin{bem} \label{bem_90}
	Alternativ (und vielleicht näher zu unserer Vorgehensweise für elliptische Gleichungen) kann $\Phi$ \marginnote{24. Jun \\ \ [90]} aufgefasst werden als Lösung zu
	\[ \begin{cases}
		\Phi_t - \Delta \Phi = \delta & \text{ auf } \RR \times \RR^n \\
		\Phi(0,x) = 0 \end{cases} \]
\end{bem}

\mbox{} \\
Im selben Sinn wie zuvor, betrachte nun für festes $(s,y)$ die Lösung des zu \eqref{eq_34}-\eqref{eq_36} \textit{adjungierten Problems} (eine Diffusion rückwärts in der Zeit)
\begin{equation}
	\begin{cases}
		\mathcal{L}^*[G^{(s,y)}] := -G_t^{(s,y)} - \Delta G^{(s,y)} = 0 & \text{auf } (0,s) \times \Omega \\
		G^{(s,y)}(t,x) = 0 & \text{auf } \der \Omega \\
		G^{(s,y)}(t,x) = \delta(x-y) & \text{zu } t=s \end{cases} \label{eq_38}
\end{equation}
Motivation: wenn wir $G^{(s,y)}$ für alle $(s,y)$ finden, gilt (informell; formaler Beweis wie bei elliptischen Differentialgleichungen)
\begin{equation}
\begin{aligned}
	\int_{\Omega_s} (G^{(s,y)} \mathcal{L}[u] - u \mathcal{L}^*[G^{(s,y)}]) dxdt
	&= \int_{\Omega_s} \der_t (G^{(s,y)} u) + \nabla(uG_x^{(s,y)} - G^{(s,y)} u_x) dx dt \\ \notag
	&= \int_{\Omega_s} \begin{pmatrix} \der_t \\ \nabla \end{pmatrix} \cdot \begin{pmatrix} G^{(s,y)} u \\ uG_x^{(s,y)} - G^{(s,y)} u_x \end{pmatrix} dxdt \\
	&= \int_{[0,s]\times \der\Omega} u \der_v G^{(s,y)} - G^{(s,y)} \der_v u dx dt + \int_\Omega G^{(s,y)} u \bigg|_{t=s} - G^{(s,y)} u \bigg|_{t=0} dx
\end{aligned}
\end{equation}
und somit
\[ u(s,y) = \int_{\Omega_s} G^{(s,y)} f dx dt + \int_{\setnull \times \Omega} G^{(s,y)} gdx - \int_{[0,s]\times \der \Omega} \der_v G^{(s,y)} dx dt,\]
eine \textit{Greensche Darstellung} der Lösung.

\begin{bsp}[Greensche Funktion für Halbraum] \label{bsp_91}
	$G(s,y)$ für $\Omega = \{x \in \RR^n : x_n > 0\}$ kann wieder mit der Spiegelungsmethode (vgl. \ref{bsp_49}) gefunden werden:
	\[ G^{(s,y)}(t,x) = \Phi(s-t,x-y) - \Phi(s-t,x+y) \]
	erfüllt \eqref{eq_38}.
\end{bsp}
\newpage