\section{Definitionsgebiet}
\label{sec:para4}
Den Bereich, auf dem die Lösung einer PDGL wohldefiniert ist, nennen wir Definitionsbereich. Dieser ist im Allgemeinen beschränkt -- für Probleme erster Ordnung durch verschiedene Umstände:
	\begin{itemize}
		\item Die charakteristischen Kurven $x(s)$, die $\Gamma$ schneiden, durchdringen nicht ganz $\RR^n$. Nur der von solchen charakteristischen Kurven erreichte Teil des $\RR^n$ kann zum Definitionsgebiet gehören.
		\item Entlang einer charakteristischen Kurve kann die Lösung $z(s)$ der GDGL aufhören zu existieren.
		\item Charakteristische Kurven $x(s)$ können sich schneiden. Da zu jedem $x(s)$ auch ein Funktionswert $z(s)$ der Lösung gehört,	würden an einem Schnittpunkt mehrere Funktionswerte vorliegen. An den Schnittpunkten ist die Parametrisierung $\widehat{x}$ des $\RR^n$ nicht mehr wohldefiniert, d.h. wir haben typischerweise $\det D\widehat{x} = 0$.
	\end{itemize}

\subsection{Beispiel}
\label{bsp_11}
	Betrachte das Cauchy-Problem \marginnote{[11]}
	\[ \begin{cases}
		u_{x_1} + u_{x_2} = u^3 \\
		u = x_2	\text{ auf } \Gamma = \{x_1 = 0, 0 < x_2 < 3\}
		\end{cases} \]
	Parametrisiere $\Gamma$ durch $x_0(y) = (0,y), 0 < y < 3$. Die (vereinfachten) charakteristischen Gleichungen
	\[ \frac{\der}{\der s} \begin{pmatrix} x \\ s \end{pmatrix} = \begin{pmatrix} (1,1) \\ z^3 \end{pmatrix} \]
	mit Anfangswerten $x(0) = (0,y), z(0) = y$ haben die Lösung
	\[ \begin{pmatrix} x \\ z \end{pmatrix} = \begin{pmatrix} (s,s+y) \\ y/\sqrt{1-2y^2s} \end{pmatrix} \]
	Das Definitionsgebiet liegt zwischen den charakteristischen Kurven durch den Rand von $\Gamma$, $x(s) = (s,s)$ und $x(s) = (s,3+s)$, und endet auf jeder charakteristischen Kurve, wenn $z \rightarrow \infty$, d.h. für $s = \frac{1}{2y^2}$. Insgesamt ergibt sich das Definitionsgebiet
	\[ \left\{ x \in \RR^2 : x_1 + 3 > x_2 > x_1, x_2 < x_1 + \frac{1}{\sqrt{2x_1}} \right\} \]

\subsection{Beispiel}
\label{bsp_12}
	Betrachte das Cauchy-Problem \marginnote{[12]}
	\[ \begin{cases}
		u_t + uu_x = 0 \\
		u(t,x) = \sin(x) \text{ auf } \Gamma = \setnull \times (0,2\pi)
		\end{cases} \]
	Die Lösung der charakteristischen Gleichungen liefert
	\[ \begin{pmatrix} (t,x) \\ x \end{pmatrix} = \begin{pmatrix} (s,y+s \sin(y)) \\ \sin(y) \end{pmatrix} \]
	Entlang der Kurve mit $0 = \det D\widehat{x} = \det(\der(t,x)/\der(y,s)) = -1 -s \cos(y)$ beginnen die charakteristischen Kurven, sich zu schneiden. Das Definitionsgebiet liegt zwischen den randständigen charakteristischen Kurven $x=0$ und $x=2\pi$ und der Kurve $\{(t(y,s),x(y,s)) : s = -\frac{1}{cos(y)}\}$,
	\[ \{(t,x) \in \RR^2 : 0 < x < 2\pi, t < - \frac{1}{\cos(y)} \text{ für } x=y-\tan(y) \} \]
\newpage