\section{Die prime Restklassengruppe $\modu m$}
\label{sec:para4}

\begin{defn}[prime Restklassengruppe]
	Sei $m \in \NN$, $m > 1$. Dann heißt $(\ZZ/m\ZZ)^\times$ die \Index{prime Restklassengruppe} $\modu m$. Wir wissen: \begin{enumerate}[(1)]
		\item $\overline{a} \in (\ZZ/m\ZZ)^\times \Leftrightarrow (a,m)=1$
		\item $M := \{k \in \ZZ : 0 \leq k < m, (k,m) = 1 \}$ ist ein Vertretersystem von $(\ZZ/m\ZZ)^\times$. \\
		$(\ZZ/m\ZZ)^\times$ hat $\varphi(m) = \# M$ Elemente und ist eine abelsche Gruppe der Ordnung $\varphi(m)$.
		\item $\overline{a} = \alpha \in (\ZZ/m\ZZ)^\times \Rightarrow \alpha^{\varphi(m)} = 1 \Leftrightarrow a^{\varphi(m)} \kon 1 \modu m$ (Satz von Euler-Fermat)
	\end{enumerate}
\end{defn}

\begin{defn}[Primitivwurzel]
	Ein $\omega \in (\ZZ/m\ZZ)^\times$ heißt eine \Index{Primitivwurzel} von $(\ZZ/m\ZZ)^\times$, wenn sich jedes Element $\alpha \in (\ZZ/m\ZZ)^\times$ in der Form
	\[ \alpha = \omega^i \text{ für ein } i \in \NN_0 \]
	schreiben lässt; jedes $g \in \ZZ$ mit $\omega = \overline{g} = g \modu m$ heißt dann eine Primitivwurzel $\modu m$.
\end{defn}

\begin{satz}[Satz von Gauß] \label{satz_4.1}
	Ist $p$ eine Primzahl, so besitzt $(\ZZ/p\ZZ)^\times$ eine Primitivwurzel. Es gibt also ein $\omega \in (\ZZ/p\ZZ)^\times$, sodass sich jedes $\alpha \in (\ZZ/p\ZZ)^\times$ darstellen lässt in der Form
	\begin{equation}
		\alpha = \omega^i \text{ mit } 0 \leq i < p-1 \label{eq_4.1}
	\end{equation}
	Die Darstellung \eqref{eq_4.1} ist unter der Bedingung $0 \leq i < p-1$ eindeutig; $i = i(\alpha) = i_\omega(\alpha)$ heißt der \Index{Index} von $\alpha$ bezüglich $\omega$.
\end{satz}

Wählt man ein $g \in \ZZ$ mit $\omega = g \modu m$, so gilt also: Zu jedem $a \in \ZZ$ mit $p \not | a$ gibt es genau ein $i \in \ZZ$ mit
\[ a \kon gî \modu p, 0 \leq i < p-1 \]
$i = i(a) = i_g(a)$ heißt der Index von $a$ bzgl. $g$.

\minisec{Zusatz}
	Es gibt genau $\varphi(p-1)$ verschiedene Primitivwurzeln von $(\ZZ/p\ZZ)^\times$.

\subsection{Gruppentheoretische Vorbereitungen}
\begin{defn}[Ordnung eines Gruppenelements] \label{def_ordnung}
	Sei $G$ eine (abelsche) Gruppe der Ordnung $n$, d.h. $\#G = n$. Sei $\alpha \in G$. Wir wissen: $\alpha^n = 1$. Unter allen $m \in \NN$ mit $\alpha^m = 1$ sei nun $k$ das kleinste. Setze dann
	\[ \ord(\alpha) := k, \]
	die \Index{Ordnung} von $\alpha$. $\sprod{\alpha} := \penbrace{\alpha^j : j \in \ZZ}$ ist offenbar eine Untergruppe von $G$.
\end{defn}

\begin{lemma} \label{lemma_4.1}
	In der Situation von Definition \ref{def_ordnung} gelten: \begin{enumerate}[(1)]
		\item $\sprod{\alpha} = \{1, \alpha, \alpha^2, \dots, \alpha^{k-1}\}$, insbesondere $\ord(\alpha) = \ord(\sprod{\alpha})$.
		\item $\alpha^m = 1$ für $m \in \ZZ \Rightarrow \ord(\alpha) | m$
		\item Sei $\ord(\alpha) = k$ wie oben, dann vermittelt der Gruppenhomomorphismus
		\begin{equation}
		\begin{aligned}
			\ZZ &\longrightarrow \sprod{\alpha} \\
			j &\longmapsto \alpha^j
		\end{aligned}
		\end{equation}
		einen Gruppenisomorphismus $\ZZ/k\ZZ \rightarrow \sprod{\alpha}$, also $\sprod{\alpha} \simeq \ZZ/k\ZZ$.
		\item $G$ zyklisch $\Leftrightarrow \exists \alpha \in G$ mit $\ord(\alpha) = \ord(G)$. \\
		Eine Gruppe $G$ heißt \Index{zyklisch}, wenn es ein $\alpha \in G$ gibt mit $G = \sprod{\alpha}$. $\alpha$ heißt dann ein \Index{Erzeuger} von $G$.
	\end{enumerate}
\end{lemma}

\minisec{Bemerkung}
	Definitionsgemäß gilt:
	\[ (\ZZ/m\ZZ)^\times \text{ besitzt Primitivwurzel} \quad \Leftrightarrow \quad (Z/m\ZZ)^\times \text{ ist zyklisch}, \]
	und nach dem zuvor Gesagten:
	\[ \omega \text{ ist Primitivwurzel von } (\ZZ/m\ZZ)^\times \quad \Leftrightarrow \quad \ord(\omega) = \varphi(m)\]
	
\begin{defn}[Gruppenexponent]
	Sei $G$ eine endliche Gruppe. Das kgV aller $\ord(\alpha), \alpha \in G$ heißt der \Index{Exponent} $e = e(G)$ der Gruppe $G$.
\end{defn}

\minisec{Bemerkung}
	Ist $n = \ord(G), e=e(G)$, so gilt stets $e|n$, denn für jedes $\alpha \in G$ gilt $\alpha^n=1 \Rightarrow \ord(\alpha) | n \Rightarrow e|n$.
	
\begin{falko} \label{F4.1}
	Sei $G$ eine endliche abelsche Gruppe und sei $e$ ihr Exponent. Dann gibt es ein Element $\omega \in G$ mit $\ord(\omega) = e$.
\end{falko}

\setcounter{countsatz}{0}
\begin{satz}
	Sei $K$ ein Körper und $G$ eine endliche Untergruppe von $K^\times$. Dann ist $G$ zyklisch.
\end{satz}

\subsection{Restklassengruppen}
\begin{defn}[Restklassen, Restklassenabbildung]
	Sei $G$ eine Gruppe und $H \subseteq G$ eine Untergruppe. \marginnote{21.11. \\ \ [12]} Für $x,y \in G$ definiere eine Relation:
	\[ x \overset{H}{\sim} y :\Leftrightarrow yx^{-1} \in H (\Leftrightarrow y \in Hx), \]
	oder für eine abelsche Gruppe mit $+$ statt $\cdot$ als Verknüpfungssymbol:
	\[ x \overset{H}{\sim} y :\Leftrightarrow y-x \in H (\Leftrightarrow y \in H+x), \]
	$\overset{H}{\sim}$ ist eine Äquivalenzrelation. Mit $G/H$ bezeichnen wir die Menge der zugehörigen Äquivalenzklassen (\bet{Restklassen}). Die Abbildung \index{Restklasse}
	\begin{equation}
	\begin{aligned}
		G &\longrightarrow G/H \\
		x &\longmapsto \overline{x} := Hx
	\end{aligned}
	\end{equation}
	heißt \Index{Restklassenabbildung}.
\end{defn}

\minisec{Bemerkung}
	Die Relation $\overset{H}{\sim}$ ist verträglich mit der Multiplikation, falls $G$ abelsch ist. In diesem Fall ist $G/H$ eine Gruppe und die Restklassenabbildung ein Homomorphismus. \\
	Ist $G$ eine beliebige Gruppe, so gilt gleiches für $G/H$ genau dann, wenn für jedes $x \in G$ gilt: $Hx = xH$.
	
\begin{falko} \label{F4.2}
	Sei $G$ eine abelsche Gruppe der Ordnung $n$ und $n = p_1^{\nu_1} p_2^{\nu_2} \dots p_r^{\nu_r}$ die Primfaktorzerlegung von $n$. Für $1 \leq i \leq r$ sei
	\[ G_{p_i} := \{\alpha \in G : \alpha^{p_i^{\nu_i}} = 1 \} \leq G \]
	Dann ist die Abbildung
	\begin{equation}
	\begin{aligned}
		f\colon \prod_{i=1}^{r} G_{p_i} &\longrightarrow G \\
		(\alpha_1,\dots,\alpha_r) &\longmapsto \alpha_1 \alpha_2 \cdots \alpha_r \label{eq_F4.2}
	\end{aligned}
	\end{equation}
	ein Isomorphismus von Gruppen. Ferner gilt $\# G_{p_i} = p_i^{\nu_i}$ für $1 \leq i \leq r$.
\end{falko}

\minisec{Bemerkung 1}
	$G$ endliche Gruppe, $\alpha \in G$, $j \in \ZZ$. Dann gilt:
	\[ \ord(\alpha^j) = \frac{\ord(\alpha)}{(\ord(\alpha),j)} \]
	
\minisec{Bemerkung 2}
	Eine zyklische Gruppe der Ordnung $n$ hat genau $\varphi(n)$ Elemente der Ordnung $n$, also $\varphi(n)$ Erzeuger. \\
	
	
Wir werden jetzt die Struktur der primen Restklassengruppe modulo $p^\nu$ \marginnote{25.11. \\ \ [13]}
\[ G = G_\nu = (\ZZ/p^\nu\ZZ)^\times, p \text{ Primzahl}, \nu \in \NN, \nu > 1 \]
untersuchen. Es ist
\[ \ord(G) = \varphi(p^\nu) = \# \{0 \leq a < p^\nu : p \not | \ a \} = p^\nu - \# \{0 \leq a < p^\nu : p | a \} = p^\nu - p^{\nu-1} = (p-1) p^{\nu-1} \]
Damit gilt für jedes $n \in \NN$:
\[ \varphi(n) = \varphi \enbrace*{ \prod_{p|n} p^{w_p(n)}} = \prod_{p | n} \varphi \enbrace*{p^{w_p(n)}} = \prod_{p | n} \enbrace*{p^{w_p(n)}-p^{w_p(n)-1}} = \prod_{p | n} p^{w_p(n)} \enbrace*{1- \frac{1}{p}} = n \prod_{p | n} \enbrace*{1- \frac{1}{p}}\]

\begin{falko} \label{F4.3}
	Für jedes $n \in \NN$ gilt:
	\[ \varphi(n) = \prod_{p|n} \enbrace*{p^{w_p(n)} - p^{w_p(n)-1}} = n \prod_{p | n} \enbrace*{1 - \frac{1}{p}} \]
\end{falko}

\begin{defn}[1-Einheit und 1-Einheitengruppe]
	Sei $G_\nu = (\ZZ/p^\nu\ZZ)^\times$. Der Kern $G_\nu^{(1)}$ des Homomorphismus \index{1-Einheit}
	\begin{equation}
	\begin{aligned}
		(\ZZ/p^\nu\ZZ)^\times &\longrightarrow (\ZZ/p\ZZ)^\times \\
		a \modu p^\nu &\longmapsto a \modu p
	\end{aligned}
	\end{equation}
	heißt die \bet{Gruppe der 1-Einheiten} von $(\ZZ/p^\nu \ZZ)^\times$. Sie besteht aus den Elementen $a \modu p^\nu$ von $(\ZZ/p^\nu\ZZ)^\times$ mit $a \kon 1 \modu p$. Es ist $\ord(G_\nu^{(1)}) = p^{\nu-1}$.
\end{defn}

\newpage
\begin{lemma} \label{lemma_4.2}
	Sei $p$ Primzahl, $j \in \NN$, $a \in \ZZ$. Es gelte
	\begin{equation}
		a \kon 1 \modu p^j, \text{ aber } a \not\kon 1 \modu p^{j+1} \label{eq_lemma_4.2_1}
	\end{equation}
	Dann folgt -- außer für $p=2$ und $j = 1$:
	\begin{equation}
		a^p \kon 1 \modu p^{j+1}, \text{ aber } a^p \not\kon 1 \modu p^{j+2} \label{eq_lemma_4.2_2}
	\end{equation}
\end{lemma}

\begin{falko} \label{F4.4}
	Sei $\nu > 1$. \begin{enumerate}[(i)]
		\item Im Fall $p \neq 2$ ist für jedes $a$ der Gestalt $a = 1+cp$ mit $p \nmid c$ die Restklasse $a \modu p^\nu$ ein Element der Ordnung $p^{\nu - 1}$ in der 1-Einheitengruppe von $(\ZZ/p^\nu\ZZ)^\times$. Insbesondere gilt dies für $a = 1+p$. \\
		Die 1-Einheitengruppe von $(\ZZ/p^\nu\ZZ)^\times$ ist also für $p \neq 2$ zyklisch mit kanonischem Erzeuger $1+p \modu p^\nu$.
		\item Im Falle $p=2$ gilt: Für $\nu \geq 3$ ist $5 \modu 2^\nu$ ein Element der Ordnung $2^{\nu-2}$ in $(\ZZ/2^\nu\ZZ)^\times$. \\
		Für $\nu=2$: $(\ZZ/4\ZZ)^\times$ ist zyklisch mit $-1 \modu 4$ als Erzeuger.
	\end{enumerate}
\end{falko}

\begin{satz} \label{satz_4.2}
	Sei $p \neq 2$. Auch für $\nu \geq 2$ ist dann $(\ZZ/p^\nu\ZZ)^\times$ zyklisch. Mit anderen Worten: $(\ZZ/p^\nu\ZZ)^\times$ besitzt eine Primitivwurzel. Es existiert also ein $g \in \ZZ$, sodass es zu jedem $a \in \ZZ$ mit $(a,p) = 1$ genau ein $i \in \ZZ$ gibt mit
	\[ a \kon g^i \modu p^\nu \quad \text{und} \quad 0 \leq i < \varphi(p^\nu) \]
	Es gibt genau $\varphi(\varphi(p^\nu)) = \varphi((p-1)p^{\nu-1}) = \varphi(p-1) \varphi(p^{\nu-1})$ Primitivwurzeln von $(\ZZ/p^\nu)^\times$.
\end{satz}

\minisec{Zusatz}
	Ist schon eine Primitivwurzel $g_0 \modu p$ bekannt, so findet man eine Primitivwurzel $\modu p^\nu$ wie folgt: Ist $g_0^{p-1} \not\kon 1 \modu p^2$, so ist $g = g_0$ eine Primitivwurzel $\modu p^\nu$. Ist $g_0^{p-1} \kon 1 \modu p^2$, so ist $g = g_0 + p$ eine Primitivwurzel $\modu p^\nu$.
	
\minisec{Bemerkung}
	Folgende Aussagen sind für $p \neq 2$ und $g \in \ZZ$ äquivalent: \begin{enumerate}[(i)]
		\item $g$ ist Primitivwurzel $\modu p$ und $g^{p-1} \not \kon 1 \modu p^2$.
		\item $g$ ist Primitivwurzel $\modu p^n$ für alle $n \in \NN$.
		\item $g$ ist Primitivwurzel $\modu p^2$.
	\end{enumerate}
	
\begin{satz} \label{satz_4.3}
	Sei $\nu \in \NN, \nu \geq 3$. Zu jeder ungeraden Zahl $a \in \ZZ$ gibt es eindeutig bestimmte $k \in \{0,1\}$ und \linebreak $j \in \{0,1,\dots,2^{\nu-2}-1\}$ mit
	\[ a \kon (-1)^k 5^j \modu 2^\nu \]
	Mit anderen Worten: Die Abbildung
	\begin{equation}
	\begin{aligned}
		\ZZ/2\ZZ \times \ZZ/2^{\nu-2}\ZZ &\longrightarrow (\ZZ/2^\nu\ZZ)^\times \\
		(k \modu 2, j \modu 2^{\nu-2}) &\longmapsto (-1 \modu 2^\nu)^k \cdot (5 \modu 2^\nu)^j \label{eq_satz_4.3}
	\end{aligned}
	\end{equation}
	ist ein Isomorphismus von Gruppen. Es ist also
	\[ (\ZZ/2^\nu\ZZ)^\times = \sprod{-1 \modu 2^\nu} \times \sprod{5 \modu 2^\nu} \simeq \ZZ/2\ZZ \times \ZZ/2^{\nu-2}\ZZ \]
	Insbesondere ist $(\ZZ/2^\nu\ZZ)^\times$ nicht zyklisch.
\end{satz}

\setcounter{countsatz}{2}
\begin{satz} \label{satz_4.3a}
	Sei $p$ Primzahl mit $p \neq 2$, $\nu \in \NN$, $\nu \geq 2$. Dann existiert eine Primitivwurzel $g \modu p$, sodass die Abbildung
	\begin{equation}
	\begin{aligned}
		\ZZ/(p-1)\ZZ \times \ZZ/p^{\nu-1}\ZZ &\longrightarrow (\ZZ/p^\nu\ZZ)^\times \\
		(i \modu (p-1), j \modu p^{\nu-1}) &\longmapsto g^i(1+p)^j \modu p^\nu \label{eq_satz_4.3a}
	\end{aligned}
	\end{equation}
	wohldefiniert und ein Isomorphismus von Gruppen ist. Insbesondere ist $(\ZZ/p^\nu\ZZ)^\times$ zyklisch.
\end{satz}

\minisec{Bemerkung}
	Das direkte Produkt $G_1 \times G_2 \times \dots \times G_r$ \marginnote{28.11. \\ \ [14]} zyklischer Gruppen $G_i$ mit paarweise teilerfremden Ordnungen $m_i$ ist zyklisch von der Ordnung $m_1m_2\cdots m_r$. 
	
\begin{falko} \label{F4.5}
	Seien $G_1,G_2,\dots,G_r$ endliche abelsche Gruppen der Ordnungen $m_1,m_2,\dots,m_r$. Wenn $G:= G_1 \times \dots \times G_r$ zyklisch ist, so sind die $m_1,\dots,m_r$ paarweise teilerfremd und die $G_i$ sind zyklisch.
\end{falko}

\begin{falko} \label{F4.6}
	Sei $G$ eine zyklische Gruppe der Ordnung $n$. Dann ist jede Untergruppe $H$ von $G$ zyklisch mit $\ord(H) | n$. Die Abbildung $H \mapsto \ord(H)$ ist eine Bijektion zwischen der Menge aller Untergruppen $H$ von $G$ und der Menge aller natürlichen Teiler $d$ von $n$, und zwar ist $H_d := \{x \in G : x^d = 1\}$ \underline{die} Untergruppe der Ordnung $d$ von $G$. Es ist
	\[ H_{\frac{n}{d}} = \{x \in G : x^{\frac{n}{d}}=1 \} \overset{!}{=} \{y^d ; y \in G\} \]
	die Untergruppe der $d$-ten Potenzen in $G$.
\end{falko}

\minisec{Korollar}
	Für beliebige $n \in \NN$ gilt:
	\[ \sum_{d | n} \varphi(d) = n \]
	
\minisec{Bemerkung}
	Sei $G$ eine beliebige endliche Gruppe der Ordnung $n$. Für jedes $d | n, d \in \NN$ habe $G$ höchstens $d$ Elemente $x$ mit $x^d = 1$. Dann ist $G$ zyklisch.
	
\begin{falko} \label{F4.7}
	Sei $G$ eine beliebige endliche Gruppe der Ordnung $n$. Folgende Aussagen sind äquivalent:
	\begin{enumerate}[(i)]
		\item Für jedes $d \mid n$ gilt $\#\penbrace*{x \in G : x^d = 1} \leq d$.
		\item Für jedes $d \mid n$ hat $G$ höchstens eine Untergruppe der Ordnung $d$.
		\item $G$ ist zyklisch.
	\end{enumerate}
\end{falko}
	
\begin{satz} \label{satz_4.4}
	Sei $m \in \NN, m > 1$. Genau dann besitzt $(\ZZ/m\ZZ)^\times$ eine Primitivwurzel, wenn $m$ eine der Zahlen folgender Gestalt ist (mit einer Primzahl $p \neq 2$ und $\nu \geq 1$):
	\[ 2, \qquad 4, \qquad p^\nu, \qquad 2p^\nu \]
\end{satz}
\newpage