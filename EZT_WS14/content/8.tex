\section{Multiplikative zahlentheoretische Funktionen}
\label{sec:para8}
	Jede Funktion $f \colon \NN \rightarrow \CC$ heiße eine \Index{zahlentheoretische Funktion}. Manchmal ist der Definitionsbereich auch $\NN_0$ oder $\ZZ$ anstelle von $\NN$.

\begin{defn}[multiplikative zahlentheoretische Funktion]	\label{def_8.1}
	$f$ wie oben heißt (zahlentheoretisch) \Index{multiplikativ}, wenn $f \neq 0$ und $f(n_1n_2) = f(n_1)f(n_2)$, falls $(n_1,n_2) = 1$. (daraus folgt stets $f(1) = 1$.)
\end{defn}

\minisec{Beispiel}
	\[ \begin{array}{c}
	\varphi(n) \text{ Eulersche } \varphi\text{-Funktion} \\ 
	\sigma(n) = \sum\limits_{d \mid n} d \\ 
	\tau(n) = \sum\limits_{d \mid n} 1 = \#\{d \in \NN : d \mid n\}
	\end{array} \]

\minisec{Bemerkungen}
	\begin{enumerate}[1)]
		\item $f$ multiplikativ $\Leftrightarrow f(n) = \prod\limits_{n} f(p^{w_p(n)})$ für alle $n \in \NN$
		\item Eine multiplikative Funktion ist festgelegt durch ihre Werte auf den Primzahlpotenzen. Ordnet man umgekehrt jeder Primzahlpotenz $> 1$ eine Zahl $\neq 0$ zu, so existiert genau eine multiplikative Funktion mit den vorgegebenen Werten.
		\item $f, g$ multiplikativ $\Rightarrow fg$ multiplikativ
		\item $f \neq 0$ heißt vollständig multiplikativ, wenn $f(n_1n_2) = f(n_1)f(n_2)$ für alle $n_1, n_2 \in \NN$. Dann gilt $f(p^k) = f(p)^k$.
	\end{enumerate}

\newpage