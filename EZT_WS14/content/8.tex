\section{Multiplikative zahlentheoretische Funktionen}
\label{sec:para8}
	Jede Funktion $f \colon \NN \rightarrow \CC$ heiße eine \Index{zahlentheoretische Funktion}. Manchmal ist der Definitionsbereich auch $\NN_0$ oder $\ZZ$ anstelle von $\NN$.

\begin{defn}[multiplikative zahlentheoretische Funktion]	\label{def_8.1}
	$f$ wie oben heißt (zahlentheoretisch) \Index{multiplikativ}, wenn $f \neq 0$ und $f(n_1n_2) = f(n_1)f(n_2)$, falls $(n_1,n_2) = 1$. (daraus folgt stets $f(1) = 1$.)
\end{defn}

\minisec{Beispiel}
	\[ \begin{array}{c}
	\varphi(n) \text{ Eulersche } \varphi\text{-Funktion} \\ 
	\sigma(n) = \sum\limits_{d \mid n} d \\ 
	\tau(n) = \sum\limits_{d \mid n} 1 = \#\{d \in \NN : d \mid n\}
	\end{array} \]

\minisec{Bemerkungen}
	\begin{enumerate}[1)]
		\item $f$ multiplikativ $\Leftrightarrow f(n) = \prod\limits_{n} f(p^{w_p(n)})$ für alle $n \in \NN$
		\item Eine multiplikative Funktion ist festgelegt durch ihre Werte auf den Primzahlpotenzen. Ordnet man umgekehrt jeder Primzahlpotenz $> 1$ eine Zahl $\neq 0$ zu, so existiert genau eine multiplikative Funktion mit den vorgegebenen Werten.
		\item $f, g$ multiplikativ $\Rightarrow fg$ multiplikativ
		\item $f \neq 0$ heißt vollständig multiplikativ, wenn $f(n_1n_2) = f(n_1)f(n_2)$ für alle $n_1, n_2 \in \NN$. Dann gilt $f(p^k) = f(p)^k$.
	\end{enumerate}
	
\begin{falko}[Summatorische Funktion] \label{F8.1}
	Ist $f$ multiplikativ, so auch die zugehörige \Index{summatorische Funktion} $g$, definiert durch \marginnote{09.01. \\ \ [22]}
	\[ g(n) = \sum\limits_{d \mid n} f(d) \]
\end{falko}

\minisec{Beispiel (Teilersummenfunktionen)}
	\[ \sigma_\alpha(n) = \sum\limits_{d \mid n} d^\alpha = \prod\limits_{p} \frac{p^{\alpha(w_p(n)+1)}-1}{p^\alpha - 1}, \quad \alpha \in \CC \index{Teilersummenfunktion}\]

\begin{falko} \label{F8.2}
	Ist $f$ multiplikativ, so gilt für jedes $n = \prod_p p^{w_p(n)}$ die Formel
	\[ \sum\limits_{d \mid n} f(d) = \prod\limits_{p \mid n} (1+f(p) + f(p^2) + \dots + f(p^{w_p(n)})) \]
\end{falko}

\begin{defn}[Möbiusfunktion] \label{def_8.2}
	Wir definieren die multiplikative \Index{Möbiusfunktion} $\mu$ durch (vgl. Bemerkung 2 zur Definition~\ref{def_8.1}):
	\[ \mu(p^\nu) = \begin{cases}
		-1 & \text{für } \nu = 1 \\
		0 & \text{für } \nu > 1 \\
		1 & \text{für } \nu = 0 \\
	\end{cases} \]
	Also für $n \in \NN$ mit Primfaktorzerlegung $n = p_1^{k_1} \cdots p_r^{k_r}$:
	\[ \mu(n) = \begin{cases}
		(-1)^r, & \text{falls } k_1 = \dots = k_r = 1, \text{ d.h. } n \text{ quadratfrei} \\
		0, & \text{sonst}
	\end{cases} \]
\end{defn}

\begin{falko} \label{F8.3}
	Setze $\varepsilon(n) := \sum\limits_{d \mid n} \mu(d)$, d.h. $\varepsilon$ sei die zur Möbiusfunktion gehörige summatorische Funktion. Dann gilt:
	\[ \varepsilon(n) = \begin{cases}
		1 & \text{für } n=1 \\
		0 & \text{für } n \neq 1
	\end{cases} \]
	Also ist $\varepsilon$ die charakteristische Funktion zur Teilmenge $\{1\}$ von $\NN$. $\varepsilon$ ist vollständig multiplikativ (im Gegensatz zu $\mu$).
\end{falko}

\begin{falko} \label{F8.4}
	Ist $f$ multiplikativ, so gilt
	\[ \sum\limits_{d \mid n} \mu(d) f(d) = \prod\limits_{p \mid n} (1-f(p)) \]
\end{falko}

\begin{defn}[Dirichlet-Faltung] \label{def_8.3}
	Für zahlentheoretische Funktionen $f,g$ definiere die \Index{Dirichlet-Faltung} $f * g$ durch
	\[ (f * g)(n) = \sum\limits_{d \mid n} f(d) g\enbrace*{\frac{n}{d}} = \sum\limits_{\substack{a,b \in \NN \\ ab = n}} f(a) g(b) \]
	Es gelten $f * g = g*f, (f*g)*h = f*(g*h)$ sowie $\varepsilon*f = f*\varepsilon = f$.
\end{defn}

\minisec{Bemerkung}
	Die Menge $R$ der zahlentheoretischen Funktionen $f \colon \NN \rightarrow \CC$ bildet mit $+$ und $*$ einen kommutativen Ring mit Einselement $\varepsilon$.

\begin{falko} \label{F8.5}
	$f,g$ multiplikativ $\Rightarrow f * g$ multiplikativ.
\end{falko}

\begin{satz}[Möbius-Umkehrformeln] \label{satz_8.1}
	Für zahlentheoretische Funktionen $f,g$ sind folgende Aussagen äquivalent: \index{Möbius-Umkehrformeln} \begin{enumerate}[(i)]
		\item $g(n) = \sum\limits_{d \mid n} f(d)$, d.h. $g$ ist summatorische Funktion zu $f$, d.h. $g = f * i_0$
		\item $f(n) = \sum\limits_{d \mid n} \mu\enbrace*{\frac{n}{d}} g(d)$, d.h. $f = g * \mu$.
	\end{enumerate}
\end{satz}

\minisec{Korollar}
	Jede zahlentheoretische Funktion $g$ ist summatorische Funktion zu genau einer Funktion $f$. Dabei gilt:
	\[ g \text{ multiplikativ } \Leftrightarrow f \text{ multiplikativ} \]
	Ist $g$ multiplikativ so gilt für $f$ außerdem noch die Formel
	\[ f \enbrace*{\prod_p p^{\nu_p}} = \prod\limits_{p \mid n} (g(p^{\nu_p}) - g(p^{\nu_p - 1})) \]
	
\minisec{Bemerkung}
	Zu jeder zahlentheoretischen Funktion $f$ mit $f(1) \neq 0$ gibt es eine (eindeutig bestimmte) Funktion $h$ mit $f * h = h * f = \varepsilon$.
	
\minisec{Bemerkung}
	Nach dem Korollar zu Satz~\ref{satz_8.1} gibt es zu $i(n) = n$ eine eindeutig bestimmte Funktion $\varphi\colon \NN \rightarrow \CC$ mit \begin{enumerate}[(1)]
		\item $\sum\limits_{d \mid n} \varphi(d) = n$, und es gilt weiter
		\item $\varphi$ ist multiplikativ
		\item $\varphi(n) = \sum\limits_{d \mid n} \mu\enbrace*{\frac{n}{d}}d = \sum\limits_{d \mid n} \mu(d) \frac{n}{d} = n \sum\limits_{d \mid n} \frac{\mu(d)}{d}$
		\item $\varphi\enbrace*{\prod_p p^{\nu_p}} = \prod\limits_{p \mid n} \enbrace*{p^{\nu_p} - p^{\nu_p - 1}}$ bzw.1
		$\varphi(n) = \prod\limits_{p \mid n} \enbrace*{p^{w_p(n)} - p^{w_p(n)-1}} = n \prod\limits_{p \mid n} \enbrace*{1 - \frac{1}{p}}$, speziell \\
		$\varphi(p^\nu) = p^\nu - p^{\nu-1}, \nu \geq 1$
		\item $\varphi(n) = \# \{k \in \NN : 1 \leq k \leq n \text{ mit } (k,n)=1\}$
	\end{enumerate}
	
\begin{falko} \label{F8.6}
	Sei $f$ multiplikativ. Dann gilt:
	\[ f \text{ vollständig multiplikativ } \Leftrightarrow \mu f * f = \varepsilon \]
\end{falko}

\subsection{Summe von zwei Quadraten (Fortsetzung)}
Wie in §\ref{sec:para5} definiere für $n \in \NN$: \marginnote{13.01. \\ \ [23]}
\[ R(n) = \#\{(a,b) \in \ZZ \times \ZZ : n=a^2+b^2, (a,b) = 1\} \]
Eigentlich interessiert uns aber
\[ r(n) = \#\{(a,b) \in \ZZ \times \ZZ : n=a^2+b^2\} \]
Sei $n = a^2+b^2, d = (a,b)$ und $a' = \frac{a}{d}, b' = \frac{b}{d}$. Dann ist $\frac{n}{d^2} = a'^2+b'^2$ mit $(a',b')=1$. Also:
\begin{equation}
	r(n) = \sum\limits_{d^2 \mid n} R\enbrace*{\frac{n}{d^2}} \label{eq_8_1}
\end{equation}
\begin{equation}
	r(n) = \sum\limits_{r \mid n} q(x) R\enbrace*{\frac{n}{x}} \text{ mit } q(n) = \begin{cases}
		1, & \text{falls } n \text{ Quadrat} \\
		0, & \text{sonst}
	\end{cases} \label{eq_8_2}
\end{equation}
\begin{equation}
	r = q * R \label{eq_8_2'}
\end{equation}

In §\ref{sec:para5} gezeigt: hat $n$ keine Primteiler $p \kon 3 \modu 4$ und ist $4 \nmid n$, so gilt:
\[ R(n) = 2^{s+2} \text{ mit } s = \text{ Anzahl der ungeraden Primteiler von } n \]
\[ R(n) = 0 \text{ für alle anderen } n \] 
Definiere $\rho(n) := \# \{x \modu n : x^2+1 \kon 0 \modu n \}$. Dann gilt:
\begin{equation}
	R(n) = 4 \rho(n) \label{eq_8_3}
\end{equation}
\begin{equation}
	r = 4(q * \rho) \label{eq_8_3'}
\end{equation}
$q$ und $\rho$ sind multiplikativ, also ist auch $q * \rho $ multiplikativ (aber $r$ nicht).

\begin{falko} \label{F8.7}
	Sei $n = 2^\mu m$ mit $2 \nmid m$ und $m$ habe nur Primteiler $p \kon 1 \modu 4$. Dann ist
	\[ r(n) = 4 \tau(m) \text{ mit } \tau(m) = \text{ Teileranzahl von } m \]
\end{falko}

\begin{satz} \label{satz_8.2}
	Es sei $\chi = \chi_4$ definiert durch
	\[ \chi(n) = \begin{cases}
		(-1)^{\frac{n-1}{2}}, & \text{falls } n \text{ ungerade} \\
		0, & \text{falls } n \text{ gerade}
	\end{cases} \]
	$\chi$ ist vollständig multiplikativ, und zwar auch auf $\ZZ$. Für die Anzahl $r(n)$ der Darstellungen von $n \in \NN$ als Summe von zwei Quadraten gilt die Formel
	\[ r(n) = 4 \sum\limits_{d \mid n} \chi(d) \]
\end{satz}

\subsection{Nachtrag zu §1}
Wie oft kommt $p \in \PP$ in $n!, n\in \NN$ vor?
\setcounter{countfalko}{8}
\begin{falko} 
	\[ w_p(n!) = \sum\limits_{j \geq 1} \benbrace*{\frac{n}{p^j}} \]
\end{falko}

\minisec{Korollar}
	Hat $n$ die $p$-adische Entwicklung $n = a_r p^r + \dots + a_1p + a_0$ mit $a_i \in \ZZ, 0 \leq a_i \leq p-1, a_r \neq 0$, so gilt
	\[ w_p(n!) = \frac{n - s_n}{p-1} \text{ mit } s_n = a_0 + a_1 + \dots + a_r \]
	
\begin{lemma}
	Für $x,y \in \RR$ gilt $[x+y] - [x] - [y] \in \{0,1\}$.
\end{lemma}

\begin{falko} \label{F8.10}
	\[ w_p\enbrace*{ \binom{n}{k}} \leq \frac{\log(n)}{\log(p)}, \quad n \in \NN, p \in \PP, k \in \NN_0, k \leq n \]
\end{falko}

\begin{falko} \label{F8.11}
	\[ \binom{n}{k} \leq n^{\pi(n)}, \quad n,p,k \text{ wie oben}, \pi(n) = \#\{p \in \PP : p \leq n\} \]
\end{falko}

\begin{falko} \label{F8.12}
	\[ \pi(x) \geq \frac{\log(2)}{2} \frac{x}{\log(x)} \text{ für alle reellen } x \geq 2\]
\end{falko}

\begin{falko} \label{F8.13}
	\[ \pi(2^k) < \frac{2^{k+2}}{k} \text{ für alle } k \in \NN \marginnote{16.01. \\ \ [24]} \]
\end{falko}

\begin{falko} \label{F8.14}
	Für alle reellen $x > 1$ gilt:
	\begin{equation}
		\pi(x) < (8 \log 2) \frac{x}{\log 2} \label{eq_F8.14}
	\end{equation}
\end{falko}

\begin{satz} \label{satz_8.3}
	Für alle reellen $x \geq 2$ gilt
	\begin{equation}
		\frac{\log 2}{2} \cdot \frac{x}{\log x} \leq \pi(x) < (8 \log 2) \frac{x}{\log x} \label{eq_satz_8.3}
	\end{equation}
	Insbesondere sind also die Funktionen $\pi(x)$ und $\frac{x}{\log x}$ von der gleichen Größenordnung.
\end{satz}

\subsection{Nachtrag zu §2: Periodische Kettenbrüche und quadratische Irrationalzahlen}
	$\alpha \in \RR \setminus \QQ$ heißt (reelle) \Index{quadratische Irrationalzahl}, wenn $\alpha$ einer Gleichung der Gestalt
	\begin{equation}
		a \alpha^2 + b \alpha + c = 0 \label{eq_quad_gl}
	\end{equation}
	genügt mit $a,b,c \in \ZZ$ und $a > 0$. Sei ohne Einschränkung $\ggT(a,b,c) = 1$, dann ist $a,b,c$ eindeutig durch $\alpha$ bestimmt.
	
\begin{falko} \label{F8.1_2}
	Ein periodischer Kettenbruch stellt eine quadratische Irrationalzahl dar. \marginnote{F1 im Skript}
\end{falko}
\newpage