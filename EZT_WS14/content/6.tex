\section{Quadratische Reste}
\label{sec:para6}

\minisec{Vorbemerkungen}
	Sei $m \in \NN$, $m > 1$. Wir untersuchen Kongruenzen über $\ZZ$ der Gestalt
	\begin{align}
		& aX^2 + bX + c \kon 0 \modu m, \quad a \neq 0 \label{eq_6.1} \\ 
		\Leftrightarrow \quad & 4a^2X^2 + 4abX + 4ac \kon 0 \modu 4am \\ 
		\Leftrightarrow \quad & (2aX + b)^2 \kon b^2 - 4ac \modu 4am \label{eq_6.2} \\ 
		\Leftrightarrow \quad & \begin{cases}
			Y^2 \kon D := b^2 - 4ac \modu 4am \\
			Y \kon b \modu 2a
		\end{cases} \label{eq_6.3}
	\end{align}

\minisec{Bemerkung}
	\begin{enumerate}[1)]
		\item Für $(a,m) = 1$: \eqref{eq_6.1} ist äquivalent zu $X^2 + \frac{b}{a}X + \frac{c}{a} \kon 0 \modu m$.
		\item Für $m, a$ ungerade: \eqref{eq_6.1} ist äquivalent zu $(aX + \frac{b}{2})^2 - \enbrace*{ \enbrace*{\frac{b}{2}}^2 - ac} \kon 0 \modu am$.
	\end{enumerate}

\begin{falko} \label{F6.1}
	Die Kongruenz
	\begin{equation}
		X^2 \kon D \modu m \text{ mit } (D,m) = d = d_1^2 d_0 \text{ und } d_0 \text{ quadratfrei} \label{eq_6.4}
	\end{equation}
	ist genau dann lösbar, wenn $\enbrace*{\frac{m}{d},d_0} = 1$ und
	\begin{equation}
		X^2 \kon d_0 \frac{D}{d} \modu \frac{m}{d} \label{eq_6.5}
	\end{equation}
	lösbar ist. Hier sind $d_0 \frac{D}{d}$ und $\frac{m}{d}$ teilerfremd! (Denn $\frac{m}{d}$ prim zu $\frac{D}{d}$ und wegen $\enbrace*{\frac{m}{d},d_0} = 1$ auch zu $d_0$.)
\end{falko}

Damit ist alles reduziert auf eine Kongruenz der Gestalt
\begin{equation}
	X^2 \kon a \modu m \text{ mit } (a,m) = 1 \label{eq_6_stern}
\end{equation}

\begin{defn}[Quadratischer Rest]
	Ist \eqref{eq_6_stern} lösbar, d.h. existiert ein $b \in \ZZ$ mit $b^2 \kon a  \modu m$, so heißt $a$ ein \Index{Quadratischer Rest} (QR) modulo $m$, andernfalls heißt $a$ ein \bet{quadratischer Nichtrest} modulo $m$.
\end{defn}

\minisec{Probleme}
	\begin{enumerate}[1)]
		\item Sei $m$ gegeben. Man verschaffe sich eine Übersicht über die sämtlichen quadratischen Reste modulo $m$.
		\item Sei $a$ gegeben. Für welche (zu $a$ teilerfremden) natürlichen Zahlen $m > 1$ ist $a$ quadratischer Rest modulo $m$?
	\end{enumerate}

Problem 2) ist schwieriger und tiefer. Eine Antwort liefert das \bet{quadratische Reziprozitätsgesetz}.\\ Zuerst Problem 1): \index{quadratisches Reziprozitätsgesetz}

\begin{falko} \label{F6.2}
	$a$ ist quadratischer Rest modulo $m$ genau dann, wenn gilt:
	\begin{enumerate}[1)]
		\item $a$ ist quadratischer Rest modulo $p$ für jeden ungeraden Primteiler $p$ von $m$.
		\item $\begin{cases}
			a \kon 1 \modu 4, & \text{ falls } 4 | m, 8 \not| \ m \\
			a \kon 1 \modu 8, & \text{ falls } 8 | m
		\end{cases}$
	\end{enumerate}
	Ist $a$ quadratischer Rest modulo $m$, so hat \eqref{eq_6_stern} genau $2^{s+t}$ Lösungen modulo $m$; dabei ist $s$ die Anzahl der ungeraden Primteiler von $m$ und
	\begin{equation}
	\begin{aligned}
		t &= 2 \text{ für } w_2(m) \geq 3 \\
		t &= 1 \text{ für } w_2(m) = 2 \\
		t &= 0 \text{ für } w_2(m) \leq 1.
	\end{aligned}
	\end{equation}
\end{falko}