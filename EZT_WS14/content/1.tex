\section{Fundamentalsatz der elementaren Arithmetik}
\label{sec:para1}

\minisec{Terminologie}
	Sei $R$ ein kommutativer Ring mit $1 \neq 0$. $R$ heißt \Index{Integritätsring} bzw. \bet{nullteilerfrei}, wenn gilt: \index{Nullteiler} \marginnote{14.10.}
	\[ a \cdot b = 0 \quad \Rightarrow \quad a = 0 \text{ oder } b = 0.\]

\begin{bsp} \label{bsp_integritaetsringe}
	\begin{itemize}
		\item $\ZZ$
		\item $\ZZ[\sqrt{2}] := \{a + b\sqrt{2} : a,b \in \ZZ \} \subseteq \RR$ \\
			$\ZZ[i] := \{a + bi : a,b \in \ZZ\} \subseteq \CC$ \\
			$\ZZ[\sqrt{-5}] := \dots$
		\item $K[X]$ für $K$ Körper \\
			$\ZZ[X]$
		\item $K$ Körper
		\item $\CC \sprod{z} := \penbrace{\text{konvergente Potenzreihen } \sum\limits_{n=0}^{\infty} a_n z^n}$
		\item Nicht nullteilerfrei ist z.B. $\mathcal{C}[0,1] := \{f \colon [0,1] \rightarrow \RR \text{ stetig} \}$
	\end{itemize}
\end{bsp}

\begin{defn}[Teilbarkeit] \label{def_1.1}
	Seien $a,b \in R$. $a$ heißt ein \Index{Teiler} von $b$, wenn ein $q \in R$ existiert mit $b = qa$, und schreiben:
	\[ a | b \]
	Ist $R$ nullteilerfrei und $a \neq 0$, so ist $q$ eindeutig bestimmt.
\end{defn}

\begin{falko}[Triviale Teilbarkeitsregeln] \label{F1.1}
	\begin{enumerate}[(i)]
		\item $a | 0, 1 | a, a | a$
		\item $a | b, b | c \quad \Rightarrow \quad a | c$
		\item $a | b, a | c \quad \Rightarrow \quad a | b+c, a | b-c$
		\item $a_1 | b_1, a_2 | b_2 \quad \Rightarrow \quad a_1 a_2 | b_1 b_2$
		\item $ac | bc \quad \Rightarrow \quad a | b$, falls $c \neq 0$ und $R$ nullteilerfrei.
	\end{enumerate}	
\end{falko}

\begin{defn}[Einheit, assoziiert] \label{def_1.2}
	\begin{enumerate}[(i)]
		\item $e \in R$ heißt eine \Index{Einheit} in $R$, falls $e | 1$ gilt, d.h. falls ein $f \in R$ existiert mit $ef = 1$. $f$ ist eindeutig bestimmt. Wir setzen $e^{-1} := f$ und schreiben auch $\frac{1}{e}$ für $e^{-1}$. \\
		Wir bezeichnen die \Index{Einheitengruppe} von $R$ mit $R^\times := \{x \in R : x \text{ ist Einheit in } R\}$.
		\item $a \in R$ heißt \Index{assoziiert} zu $b \in R$, falls $a | b$ und $b | a$ gilt. Schreibe: $a \assoz b$.
	\end{enumerate}
\end{defn}

\begin{bsp}
	\begin{enumerate}[1)]
		\item Sei $K$ ein Körper, dann ist $K^\times = K \setminus \setnull$. \qquad $\ZZ^\times = \{1,-1\}$, \qquad $K[X]^\times = K^\times$, \\
		$\mathcal{C}[0,1]^\times = \{f \in \mathcal{C}[0,1] : f(x) \neq 0 \text{ für alle } x \in [0,1]\}$, \qquad  $\ZZ[\sqrt{2}]^\times = \{\pm (1+\sqrt{2})^k : k \in \ZZ \}$ \\
		$\ZZ[X]^\times = \{1, -1\}$ \qquad $\CC \sprod{z}^\times = \penbrace*{ \sum a_n z^n \in \CC \sprod{z} : a_0 \neq 0}$
		\item $e \in R^\times \quad \Leftrightarrow \quad e | a$ für jedes $a \in R$.
	\end{enumerate}
\end{bsp}

\begin{falko} \label{F1.2}
	Sei $R$ ein Integritätsring, $a, b \in R$ und $b \neq 0$. Dann gilt:
	\[ a \assoz b \quad \Leftrightarrow \quad \exists e \in R^\times \text{ mit } b = ea \]
\end{falko}

\minisec{Beweis}
	\begin{description}
		\item[\bewrueck] $a | b, e^{-1}b = a, b | a$
		\item[\bewhin] Da $a | b$ und $b | a$, existieren $e, f \in R$, sodass $b = ea$ und $a = fb$. $\Rightarrow b = efb \Rightarrow ef = 1$, da $b \neq 0$ und $R$ nullteilerfrei. \qed
	\end{description}

\setlength{\fboxsep}{10pt}
\setlength{\fboxrule}{3pt}
\begin{center}
	\fbox{\textbf{Ab jetzt ist, wenn nichts anderes gesagt, $R$ ein Integritätsring!}}
\end{center}

\begin{defn}[unzerlegbar, irreduzibel, zusammengesetzt] \label{def_1.3}
	Sei $a \in R \setminus R^\times$. $a$ heißt \Index{unzerlegbar} oder \Index{irreduzibel} in $R$, wenn gilt:
	\[ a = bc \text{ in } R \quad \Rightarrow \quad b \in R^\times \text{ oder } c \in R^\times. \]
	Andernfalls heißt $a$ \bet{zerlegbar}, \bet{zusammengesetzt} oder \bet{reduzibel}.
\end{defn}

\minisec{Bemerkung}
	$a$ unzerlegbar $\quad \Leftrightarrow \quad $ jeder Teiler von $a$ ist Einheit oder assoziiert zu $a$ \\
	$a$ zerlegbar $\quad \Leftrightarrow \quad a$ hat echten Teiler, d.h. einen Teiler, der weder eine Einheit ist noch assoziiert zu $a$

\setcounter{countdef}{2}
\begin{defn}[Primzahl] \label{def_1.3'}
	Ein $p \in \ZZ$ heißt \Index{Primzahl}, wenn $p \in \NN$ und $p$ unzerlegbar in $\ZZ$. Wir beziechnen mit $\PP$ die Menge der Primzahlen von $\ZZ$. $a$ unzerlegbar in $\ZZ \Leftrightarrow a = p$ oder $a = -p$ mit $p \in \PP$.
\end{defn}

\minisec{Bemerkung}
	$a \in \ZZ$ sei zerlegbar, $a \neq 0$. Dann gibt es eine Primzahl $p$ mit $p | a$ und $p \leq \sqrt{|a|}$.

\begin{defn}[Zerlegung in unzerlegbare Faktoren] \label{def_1.4}
	Wir sagen, $a \in R$ besitzt in $R$ eine \Index{Zerlegung in unzerlegbare Faktoren}, wenn
	\begin{equation}
	\begin{aligned}
		a = ep_1p_2\dots p_r \text{ mit } e \in R^\times \text{ und } p_1,\dots,p_r \text{ unzerlegbar} \label{eq_def_1.4}
	\end{aligned}
	\end{equation}
	\eqref{eq_def_1.4} heißt eine Zerlegung von $a$ in unzerlegbare Faktoren. Auch $r = 0$ ist erlaubt.
\end{defn}

\begin{falko} \label{F1.3}
	In $\ZZ$ besitzt jedes $a \neq 0$ eine Zerlegung in unzerlegbare Faktoren.
\end{falko}	

\setcounter{countfalko}{2}
\begin{falko} \label{F1.3'}
	Jede natürliche Zahl $a > 1$ besitzt eine Zerlegung $a = p_1p_2 \dots p_r$ mit Primzahlen $p_1,\dots,p_r$ und $r \geq 1$.
\end{falko}

\minisec{Bemerkung}
	\begin{enumerate}[1)]
		\item Die Aussage F1.3 gilt auch für die Beispiele zu Beginn, mit Ausnahme von $\mathcal{C}[0,1]$.
		\item Sei $R$ ein Integritätsring, der die \Index{Teilbarkeitsbedingung für Hauptideale} erfüllt, so besitzt jedes $a \neq 0$ aus $R$ eine Zerlegung in unzerlegbare Faktoren.
		\item Primzahlen sind die multiplikativen Bausteine (Atome) von $\NN$.
		\item Im Beispiel $\CC\sprod{z}$ von oben gibt es (bis auf Assoziiertheit) nur das einzige unzerlegbare Element $z$. Dieses ist ein \Index{Primelement} (der Begriff folgt weiter unten).
	\end{enumerate}
	
\begin{satz}[Existenz unendlich vieler Primzahlen] \label{satz_1.1}
	Es gibt unendlich viele Primzahlen.
\end{satz}

\textbf{Bemerkungen} \\
	Es sei $p_1,p_2,\dots$ die aufsteigend sortierte Folge der Primzahlen.
	\begin{enumerate}[1)]
		\item $a_n := p_1p_2\dots p_n +1$ ist Primzahl für $n \leq 5$, aber z.B. nicht für $n = 6$. Unklar ist, ob unendlich viele $a_n$ Primzahlen oder keine Primzahlen sind.
		\item Für $x \in \RR_{>0}$ definieren wir:
		\[ \pi(x) := \# \penbrace{p \in \PP : p \leq x}\]
	\end{enumerate}

\minisec{Primzahlsatz (Gauß, Legendre)}
	\begin{equation}
		\pi(x) \sim \frac{x}{\log x}, \text{ d.h. } \lim\limits_{x \rightarrow \infty} \frac{\pi(x)}{x / \log x} = 1 \label{eq_primzahlsatz1}
	\end{equation}
	\begin{equation}
		\pi(x) \sim \int_{2}^{x} \frac{1}{\log t} dt =: \li(x) \label{eq_primzahlsatz_2}
	\end{equation}
	\begin{equation}
	\pi(x) > \frac{x}{\log x} \text{ für alle } x \geq 17 \label{eq_primzahlsatz_3}
	\end{equation}
	\begin{equation}
	\pi(n) > \frac{n}{\log n} \text{ für alle } n \in \NN, n \geq 11 \label{eq_primzahlsatz_4}
	\end{equation}
	
\begin{defn}[eindeutige Zerlegung] \label{def_1.5}
	Sei $R$ ein kommutativer Ring mit $1 \neq 0$. Wir sagen, $a \in R \setminus \setnull$ hat eine \bet{eindeutige} \Index{Zerlegung in unzerlegbare Faktoren}, wenn $a$ eine Zerlegung
	\[ a = ep_1p_2 \dots p_r \]
	in unzerlegbare Faktoren besitzt und eine solche im folgendem Sinne eindeutig ist: Ist auch
	\[ a = e'p_1'p_2'\dots p'_{r'} \]
	eine solche Zerlegung, so gilt $r = r'$ und nach Umnummerierung $p_i' \assoz p_i$ für alle $1 \leq i \leq r$.
\end{defn}

\begin{falko}\label{F1.4}
	In dem Integritätsring $R$ besitze jedes Element $a \neq 0$ eine Zerlegung in unzerlegbare Faktoren. Dann sind äquivalent: \begin{enumerate}[(i)]
		\item Jedes $a \neq 0$ aus $R$ hat eindeutige Zerlegung in unzerlegbare Faktoren.
		\item Ist $p$ unzerlegbar, so gilt: $p | ab \Rightarrow p | a$ oder $p | b$.
	\end{enumerate}
\end{falko}

\begin{defn}[Primelement] \label{def_1.6}
	Sei $R$ ein kommutativer Ring mit $1 \neq 0$. Ein $p \in R\setminus R^\times$ heißt \Index{Primelement} von $R$, wenn für alle $a, b \in R$ gilt:
	\begin{equation}
		p | ab \quad \Leftrightarrow \quad p | a \text{ oder } p | b \label{eq_def_1.6}
	\end{equation}
\end{defn}

\minisec{Bemerkung}
	\begin{enumerate}[1)]
		\item $0$ ist Primelement in $R \Leftrightarrow R$ ist Integritätsring
		\item In einem Integritätsring $R$ gilt: Jedes Primelement $p \neq 0$ ist unzerlegbar.
	\end{enumerate}

\begin{lemma}
	Seien $a, b \in \NN$. Sei $m = \kgV(a,b) \in \NN$. Dann gilt:
	\[ a|c \text{ und } b|c \quad \Rightarrow \quad m | c \]
	$m$ ist also auch minimal bzgl. der Teilbarkeitsrelation $|$.
\end{lemma}

\begin{falko}[Satz von Euklid] \label{F1.5}
	Jede Primzahl $p$ ist ein Primelement von $\ZZ$, d.h. es gilt stets \eqref{eq_def_1.6}. (Das gleiche gilt für $-p$, also für jedes unzerlegbare Element von $\ZZ$.) \index{Satz von Euklid}
\end{falko}

\minisec{Fundamentalsatz der elementaren Arithmetik}
	In $\ZZ$ hat jedes $a \neq 0$ eine eindeutige Zerlegung in unzerlegbare Faktoren. \index{Fundamentalsatz der elementaren Arithmetik}

\minisec{Bemerkung}
	Eindeutige Zerlegung in unzerlegbare Faktoren hat man zum Beispiel auch für die Ringe $\ZZ[\sqrt{2}], \ZZ[i], K[X]$ und $K$ für $K$ Körper, $\ZZ[X]$ und $\CC\sprod{z}$, nicht aber für $\ZZ[\sqrt{-5}]$:
	\[ 3 \cdot 3 = 9 = (2+ \sqrt{-5})(2 - \sqrt{-5}) \]
	Dies sind zwei wesentlich verschiedene Zerlegungen in unzerlegbare Faktoren.

\begin{defn}[Exponent] \label{def_1.7}
	Sei $p$ eine Primzahl und $a \in \ZZ \setminus \setnull$. Dann heißt
	\[ w_p(a) :=\max\{k \in \NN_0 : p^k |a \} \]
	der \Index{Exponent} von $p$ in $a$. Wir setzen $w_p(0) := \infty$.
\end{defn}

\begin{falko}[Eigenschaften der Exponentfunktion] \label{F1.6}
	Die Funktion $w_p \colon \ZZ \rightarrow \NN_0 \cup \{\infty\}$ hat folgende Eigenschaften:
	\begin{enumerate}[(i)]
		\item $w_p(a+b) \geq \min(w_p(a),w_p(b))$ und Gleichheit, falls $w_p(a) \neq w_p(b)$.
		\item $w_p(ab) = w_p(a) + w_p(b)$
	\end{enumerate}
\end{falko}

\begin{satz}[Fundamentalsatz der elementaren Arithmetik] \label{satz_1.2}
	Für jedes $a \in \ZZ \setminus \setnull$ gilt $w_p(a) > 0$ nur für endlich viele $p$. Es ist
	\begin{equation}
		a = \sgn(a) \cdot \prod\limits_p^{} p^{w_p(a)} \label{eq_satz_1.2}
	\end{equation}
\end{satz}

\minisec{Bemerkung}
	\begin{enumerate}[1)]
		\item $w_p$ lässt sich eindeutig zu einer Abbildung $w_p \colon \QQ \rightarrow \ZZ \cup \{\infty\}$ fortsetzen, sodass (ii) für alle $a,b \in \QQ$ gilt. Es gilt dann auch (i). Für $a \in \QQ \setminus \setnull$ ist $w_p(a) \neq 0$ nur für endlich viele $p$, und die Formel \eqref{eq_satz_1.2} gilt entsprechend.  Ferner gilt: $a \in \ZZ \Leftrightarrow w_p(a) \geq 0$ für alle $p$.
		\item Sei
		\[ \NN_0^{(\PP)} := \{ (e_p)_{p \in \PP} : e_p \in \NN_0, e_p = 0 \text{ für fast alle } p \}. \]
		Nach Satz \ref{satz_1.2} sind $(\NN,\cdot)$ und $(\NN_0^{(\PP)},+)$ zwei zueinander isomorphe Halbgruppen. Nach Bemerkung 1) sind $\QQ^\times$ und $\{1,-1\} \times \ZZ^{(\PP)}$ sogar zwei zueinander isomorphe Gruppen.
	\end{enumerate}
	
\begin{defn}[faktorieller Ring, Vertretersystem für Primelemente] \label{def_1.8}
	Ein Integritätsring $R$ heißt \Index{faktoriell}, wenn jedes $a \in R \setminus \setnull$ eine eindeutige Zerlegung in unzerlegbare Faktoren hat. Man spricht dann auch von eindeutiger Primfaktorzerlegung in $R$. \\
	$P$ heißt \bet{Vertretersystem für die Primelemente $\neq 0$} von $R$, wenn:
	\begin{enumerate}[(1)]
		\item Zu jedem Primelement $q \neq 0$ von $R$ gibt es ein $p \in P$ mit $q \assoz p$. 
		\item Für $p, p' \in P$ mit $p \assoz p'$ gilt $p = p'$, d.h. $p$ in (1) ist eindeutig bestimmt durch $q$.
	\end{enumerate}
	Für $R = \ZZ$ nehme man stets $P = \PP$. Für $K$ Körper und $R = K[X]$ nimmt man $P = \{p \in K[X] : p \text{ irreduzibel und normiert}\}$. \index{Primelement} \index{Vertretersystem}
\end{defn}

\begin{falko} \label{F1.7}
	Sei $R$ faktoriell und $P$ ein Vertretersystem für Primelemente. Es gibt zu jedem $p \in P$ eine Funktion $w_p \colon R \rightarrow \NN_0 \cup \{\infty\}$ mit den Eigenschaften (i) und (ii) aus F\ref{F1.6}, sodass gilt: \begin{enumerate}[a)]
		\item Für jedes $a \in R \setminus \setnull$ ist $w_p(a) > 0$ nur für endlich viele $p \in P$.
		\item Für jedes $a \in R \setminus \setnull$ gilt
		\begin{equation}
			a = e \prod\limits_{p \in P}^{} p^{w_p(a)} \label{eq_F1.7}
		\end{equation}
		mit eindeutigem $e \in \RR^\times$.
	\end{enumerate}
\end{falko}

\begin{defn}[ggT und kgV] \label{def_1.9}
	Sei $R$ ein kommutativer Ring mit $1 \neq 0$. Gegeben $a_1, \dots, a_n \in R$.
	\begin{enumerate}[a)]
		\item Ein $d \in R$ heißt ein \Index{größter gemeinsamer Teiler} (ggT) von $a_1,\dots,a_n$, falls:
		\begin{center}
			1. \quad $d | a_i$ für alle $i$ \hspace{3cm} 2. \quad $t | a_i$ für alle $i \Rightarrow t | d$
		\end{center}
		\item Ein $m \in R$ heißt ein \Index{kleinstes gemeinsames Vielfaches} (kgV) von $a_1,\dots,a_n$, falls:
		\begin{center}
			1. \quad $a_i | m$ für alle $i$ \hspace{3cm} 2. \quad $a_i | c$ für alle $i \Rightarrow m | c$
		\end{center}
	\end{enumerate}
\end{defn}

\minisec{Bemerkung}
	\begin{enumerate}[1)]
		\item $d, d'$ ggT von $a_1, \dots, a_n \Rightarrow d \assoz d'$ und $m, m'$ kgV von $a_1, \dots, a_n \Rightarrow m \assoz m'$
		\item Im Allgemeinen ist die Existenz eines ggT und kgV nicht gesichert. In faktoriellen Ringen existieren sie aber immer, siehe dazu folgende Feststellung.
	\end{enumerate}
\newpage