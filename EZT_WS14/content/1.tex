\section{Fundamentalsatz der elementaren Arithmetik}
\label{sec:para1}

\minisec{Terminologie}
	Sei $R$ ein kommutativer Ring mit $1 \neq 0$. $R$ heißt \Index{Integritätsring} bzw. \bet{nullteilerfrei}, wenn gilt: \index{Nullteiler}
	\[ a \cdot b = 0 \quad \Rightarrow \quad a = 0 \text{ oder } b = 0.\]

\begin{bsp} \label{bsp_integritaetsringe}
	\begin{itemize}
		\item $\ZZ$
		\item $\ZZ[\sqrt{2}] := \{a + b\sqrt{2} : a,b \in \ZZ \} \subseteq \RR$ \\
			$\ZZ[i] := \{a + bi : a,b \in \ZZ\} \subseteq \CC$ \\
			$\ZZ[\sqrt{-5}] := \dots$
		\item $K[X]$ für $K$ Körper \\
			$\ZZ[X]$
		\item $K$ Körper
		\item $\CC \sprod{z} := \penbrace{\text{konvergente Potenzreihen } \sum\limits_{n=0}^{\infty} a_n z^n}$
		\item Nicht nullteilerfrei ist z.B. $\mathcal{C}[0,1] := \{f \colon [0,1] \rightarrow \RR \text{ stetig} \}$
	\end{itemize}
\end{bsp}

\begin{defn}[Teilbarkeit] \label{def_1.1}
	Seien $a,b \in R$. $a$ heißt ein \Index{Teiler} von $b$, wenn ein $q \in R$ existiert mit $b = qa$, und schreiben:
	\[ a | b \]
	Ist $R$ nullteilerfrei und $a \neq 0$, so ist $q$ eindeutig bestimmt.
\end{defn}

\begin{falko}[Triviale Teilbarkeitsregeln] \label{F1.1}
	\begin{enumerate}[(i)]
		\item $a | 0, 1 | a, a | a$
		\item $a | b, b | c \quad \Rightarrow \quad a | c$
		\item $a | b, a | c \quad \Rightarrow \quad a | b+c, a | b-c$
		\item $a_1 | b_1, a_2 | b_2 \quad \Rightarrow \quad a_1 a_2 | b_1 b_2$
		\item $ac | bc \quad \Rightarrow \quad a | b$, falls $c \neq 0$ und $R$ nullteilerfrei.
	\end{enumerate}	
\end{falko}