\section{Der euklidische Algorithmus}
\label{sec:para2}
	Sei $R$ kommutativer Ring mit $1 \neq 0$. Für beliebiges $a \in R$ betrachte man die Menge der Vielfachen von $a \in R$, also
	\[ Ra := {xa : x \in R} = {b \in R : a|b} \]
	Die Teilmenge $I = Ra$ hat folgende Eigenschaften:
	\begin{enumerate}[(i)]
		\item $0 \in I$
		\item $b_1,b_2 \in I \Rightarrow b_1+b_2 \in I$
		\item $c \in R, b \in I \Rightarrow cb \in I$
	\end{enumerate}
	
\begin{defn}[Ideal, Hauptideal] \label{def_2.1}
	Eine Teilmenge $I$ von $R$ heißt ein \Index{Ideal} in $R$, falls die Eigenschaften (i), (ii), (iii) erfüllt sind. $I$ heißt \Index{Hauptideal}, wenn es ein $a \in R$ gibt mit $I = Ra$. Wir verwenden die Bezeichnung
	\[ (a) := Ra \]
	und nennen $(a)$ das von $a \in R$ erzeugte Hauptideal.
\end{defn}

\minisec{Bemerkung}
	\begin{enumerate}[(1)]
		\item $(b) \subseteq (a) \Leftrightarrow a | b$
		\item $a \assoz b \Leftrightarrow (a) = (b)$
		\item $c$ ist gemeinsames Vielfaches von $a_1,\dots,a_n \Leftrightarrow (c) \subseteq (a_1) \cap \dots \cap (a_n)$
		\item $m$ ist ein kgV von $a_1,\dots,a_n \Leftrightarrow (a_1) \cap \dots \cap (a_n) = (m)$
		\item $d$ ist ein gemeinsamer Teiler von $a_1,\dots,a_n \Leftrightarrow (a_i) \subseteq (d)$ für $1 \leq i \leq n$
		\item $d$ ist ein gemeinsamer Teiler von $a_1,\dots,a_n \Leftrightarrow Ra_1 + Ra_2 + \dots + Ra_n \subseteq (d)$
		\item $d$ ist ein ggT von $a_1,\dots,a_n \Leftrightarrow (d)$ ist das kleinste Hauptideal mit $Ra_1 + \dots Ra_n \subseteq (d)$.
	\end{enumerate}
	Ein ggT lässt sich also idealtheoretisch nicht so einfach charakterisieren wie oben ein kgV durch (4). Am schönsten wäre es, wenn $Ra_1 + \dots + Ra_n$ ein Hauptideal wäre, dann würde (7) übergehen in:
	\[ d \text{ ist ein ggT von } a_1,\dots,a_n \Leftrightarrow Ra_1+Ra_2+ \dots + Ra_n = (d) \]
	
\begin{defn}[Hauptidealring] \label{def_2.2}
	Ein Integritätsring $R$ heißt ein \Index{Hauptidealring}, wenn jedes Ideal $I$ von $R$ ein Hauptideal ist.
\end{defn}

\minisec{Bezeichnung}
	Für Elemente $a_1,\dots,a_n$ in einem beliebigen kommutativen Ring $R$ mit $1 \neq 0$ setze
	\[ (a_1,\dots,a_n) := Ra_1 + \dots + Ra_n \]
	Man nennt $(a_1,\dots,a_n)$ das von $a_1,\dots,a_n$ erzeugte Ideal in $R$.
\newpage