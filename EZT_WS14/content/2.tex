\section{Der euklidische Algorithmus}
\label{sec:para2}
	Sei $R$ kommutativer Ring mit $1 \neq 0$. Für beliebiges $a \in R$ betrachte man die Menge der Vielfachen von $a \in R$, also
	\[ Ra := {xa : x \in R} = {b \in R : a|b} \]
	Die Teilmenge $I = Ra$ hat folgende Eigenschaften:
	\begin{enumerate}[(i)]
		\item $0 \in I$
		\item $b_1,b_2 \in I \Rightarrow b_1+b_2 \in I$
		\item $c \in R, b \in I \Rightarrow cb \in I$
	\end{enumerate}
	
\begin{defn}[Ideal, Hauptideal] \label{def_2.1}
	Eine Teilmenge $I$ von $R$ heißt ein \Index{Ideal} in $R$, falls die Eigenschaften (i), (ii), (iii) erfüllt sind. $I$ heißt \Index{Hauptideal}, wenn es ein $a \in R$ gibt mit $I = Ra$. Wir verwenden die Bezeichnung
	\[ (a) := Ra \]
	und nennen $(a)$ das von $a \in R$ erzeugte Hauptideal.
\end{defn}

\minisec{Bemerkung}
	\begin{enumerate}[(1)]
		\item $(b) \subseteq (a) \Leftrightarrow a | b$
		\item $a \assoz b \Leftrightarrow (a) = (b)$
		\item $c$ ist gemeinsames Vielfaches von $a_1,\dots,a_n \Leftrightarrow (c) \subseteq (a_1) \cap \dots \cap (a_n)$
		\item $m$ ist ein kgV von $a_1,\dots,a_n \Leftrightarrow (a_1) \cap \dots \cap (a_n) = (m)$
		\item $d$ ist ein gemeinsamer Teiler von $a_1,\dots,a_n \Leftrightarrow (a_i) \subseteq (d)$ für $1 \leq i \leq n$
		\item $d$ ist ein gemeinsamer Teiler von $a_1,\dots,a_n \Leftrightarrow Ra_1 + Ra_2 + \dots + Ra_n \subseteq (d)$
		\item $d$ ist ein ggT von $a_1,\dots,a_n \Leftrightarrow (d)$ ist das kleinste Hauptideal mit $Ra_1 + \dots Ra_n \subseteq (d)$.
	\end{enumerate}
	Ein ggT lässt sich also idealtheoretisch nicht so einfach charakterisieren wie oben ein kgV durch (4). Am schönsten wäre es, wenn $Ra_1 + \dots + Ra_n$ ein Hauptideal wäre, dann würde (7) übergehen in:
	\[ d \text{ ist ein ggT von } a_1,\dots,a_n \Leftrightarrow Ra_1+Ra_2+ \dots + Ra_n = (d) \]
	
\begin{defn}[Hauptidealring] \label{def_2.2}
	Ein Integritätsring $R$ heißt ein \Index{Hauptidealring}, wenn jedes Ideal $I$ von $R$ ein Hauptideal ist.
\end{defn}

\minisec{Bezeichnung}
	Für Elemente $a_1,\dots,a_n$ in einem beliebigen kommutativen Ring $R$ mit $1 \neq 0$ setze
	\[ (a_1,\dots,a_n) := Ra_1 + \dots + Ra_n \]
	Man nennt $(a_1,\dots,a_n)$ das von $a_1,\dots,a_n$ erzeugte Ideal in $R$.
	
\begin{falko}[Satz vom größten gemeinsamen Teiler] \label{F2.1}
	Sei $R$ ein Hauptidealring.\marginnote{24.10.} Dann gilt: Zu jedem System $a_1,\dots,a_n$ von Elementen aus $R$ existiert ein ggT $d$ von $a_1,\dots,a_n$ und jedes solche $d$ besitzt eine Darstellung der Gestalt
	\begin{equation}
		d = x_1a_1+\dots+x_na_n \quad \text{mit } x_i \in R \label{eq_F2.1}
	\end{equation}
	Wir sagen, in $R$ gelte der \Index{Satz vom größten gemeinsamen Teiler}.
\end{falko}

\minisec{Bemerkung}
	Sei $R$ ein beliebiger Integritätsring. Ist $d$ ein gemeinsamer Teiler von $a_1,\dots,a_n$ aus $R$ und gibt es eine Darstellung der Form \eqref{eq_F2.1}, so ist $d$ ein ggT von $a_1,\dots,a_n$.

\begin{satz} \label{satz_2.1}
	$\ZZ$ ist ein Hauptidealring.
\end{satz}

\minisec{Definition (Gaußklammer)}
	Für $x \in \RR$ setze
	\[ [x] = \max \{g \in \ZZ : g \leq x\} \in \ZZ \]
	$[x]$ ist charakterisiert durch folgende zwei Eigenschaften: \begin{enumerate}[(1)]
		\item $[x] \in \ZZ$
		\item $[x] \leq x < [x] + 1$
	\end{enumerate}

\begin{falko}[Division mit Rest in $\ZZ$] \label{F2.2}
	Gegeben $a, b \in \ZZ$, $a \neq 0$. Dann gibt es eine Darstellung
	\begin{equation}
		b = qa +r \quad \text{mit } 0 \leq r < |a| \text{ und } q,r \in \ZZ \label{eq_F2.2}
	\end{equation}
\end{falko}

\minisec{Bemerkung}
	\begin{enumerate}[1)]
		\item Die Darstellung \eqref{eq_F2.2} ist eindeutig.
		\item Es gibt eine Darstellung
		\begin{equation}
			b = qa + r \quad \text{mit } |r| < |a|; q,r \in \ZZ, \label{eq_D'}
		\end{equation}
		doch diese ist nicht mehr eindeutig, z.B. $27 = 4 \cdot 6 + 3 = 5 \cdot 6 - 3$.
		\item Es gibt eine Darstellung
		\begin{equation}
			b = qa + r \quad \text{mit } -\frac{|a|}{2} < r \leq \frac{|a|}{2}; q,r \in \ZZ,
		\end{equation}
		und diese ist eindeutig.
		\item Es gibt eine Darstellung
		\[ b = qa + r \quad \text{mit } |r| \leq \frac{|a|}{2}; q,r \in \ZZ, \]
		doch diese ist nicht eindeutig, falls $a$ gerade.
	\end{enumerate}

\begin{defn}[euklidischer Ring] \label{def_2.3}
	Ein Integritätsring $R$ heißt ein \Index{euklidischer Ring}, falls eine Funktion $\nu\colon R \rightarrow \NN_0$ mit $\nu(0) = 0$ existiert, sodass gilt: Zu $a, b \in R$ mit $a \neq 0$ existieren $q,r \in R$ mit
	\[b = qa + r \text{ und } \nu(r) = \nu(a) \]
\end{defn}

\minisec{Beispiele}
	\begin{enumerate}[(1)]
		\item $R = \ZZ$ mit $\nu(a) = |a|$.
		\item $R = K[X], K$ Körper, mit $\nu(g) = \deg(g) + 1$ für $g \neq 0$, $\nu(0) = 0$.
		\item $R = \ZZ[i]$ mit $\nu(z) = N(z) = z \overline{z} = |z|^2$.
	\end{enumerate}
	
\begin{falko} \label{F2.3}
	Jeder euklidische Ring ist ein Hauptidealring.
\end{falko}

\begin{falko} \label{F2.4}
	Jeder Hauptidealring ist faktoriell.
\end{falko}

Im Folgenden sei $R$ ein euklidischer Ring mit euklidischer Normfunktion $\nu$. Allgemein gilt folgende elementare Umformung:
\begin{equation}
	(a_1,a_2,\dots,a_n) = (a_1, a_2-y_2a_1,\dots,a_n-y_na_1) \text{ für bel. } y_i \in R \label{eq_U}
\end{equation}

\minisec{Euklidischer Algorithmus}
	Gegeben $a_1,\dots,a_n \in R$. Wir wollen $d \in R$ bestimmen mit \index{Euklidischer Algorithmus}
	\[ (a_1,\dots,a_n) = (d) \]
	Sind alle $a_i = 0$, so ist $d = 0$ und wir sind fertig. Sei daher ohne Einschränkung
	\[ a_1 \neq 0 \text{ und } \nu(a_1) \leq \nu(a_i), \text{ falls } a_i \neq 0 \]
	Sei $a_i = q_ia_1 + r_i$ mit $\nu(r_i) < \nu(a_1)$ für $i \geq 2$. Dann ist
	\[ (a_1,\dots,a_n) \stackrel{\eqref{eq_U}}{=} (a_1,r_2,\dots,r_n) \]
	Fortsetzung des Verfahrens liefert
	\[ (d,0,0,\dots,0) = (d) \]
	
\minisec{Beispiel}
	\[\begin{array}{rl}
		(\textcolor{red}{27},63,114) & 63 = 2 \cdot 27 + 9, 114 = 4 \cdot 27 + 6 \\ 
		= (27,9,\textcolor{red}{6}) & 27 = 4 \cdot 6 + 3, 9 = 1 \cdot 6 + 3 \\ 
		= (\textcolor{red}{3},3,6) & 3 = 1 \cdot 3 + 0, 6 = 2 \cdot 3 + 0 \\ 
		= (3, 0, 0) = (3) & 
	\end{array}\]

\minisec{Beispiel im Fall $\boldsymbol{n=2}$}
	Sei $a,b \in R \setminus \setnull$. \marginnote{28.10.}
	\[\begin{array}{lll}
		b = q_0 a + r_1 & \nu(r_1) < \nu(a) & \text{Falls } r_1 = 0, \text{ dann Schluss. Sonst weiter:} \\ 
		a = q_1 r_1 + r_2 & \nu(r_2) < \nu(r_1) &  \\ 
		r_1 = q_2 r_2 + r_3 & \nu(r_3) < \nu(r_2) &  \\ 
		\qquad \vdots &  &  \\ 
		r_{n-2} = q_{n-1} r_{n-1} + r_n & \nu(r_n) < \nu(r_{n-1}) &  \\ 
		r_{n-1} = q_n r_n + 0 &  & 
	\end{array} \]
	Also:
	\[ (a,b) = (a,r_1) = (r_1,r_2) = \dots = (r_{n-1},r_n) = (r_n) \]
	
\begin{falko} \label{F2.5}
	$r_n$ ist ein größter gemeinsamer Teiler von $a$ und $b$. Es ist
	\[ r_n = xa + yb \text{ mit } x,y \in R, \]
	wobei $x$ und $y$ aus obiger Rechnung rekursiv bestimmbar sind.
\end{falko}

\minisec{Bemerkung}
\begin{enumerate}[1)]
	\item Von neuem erhalten wir für jeden euklidischen Ring also den Satz vom größten gemeinsamen Teiler. (Satz \ref{F2.1})
	\item Sei $R = \ZZ$. Verlangen wir $0 \leq r_i$ in obiger Rechnung, so sind $q_0,q_1,\dots,q_n$ sowie die $r_1,\dots,r_n$ eindeutig bestimmt.
\end{enumerate}

\minisec{Beispiel}
	Sei $a = 84, b=133$.
	\begin{equation}
	\begin{aligned}
		133 &= 1 \cdot 84 + 49 \\
		84 &= 1 \cdot 49 + 35 \\
		49 &= 1 \cdot 35 + 14 \\
		35 &= 2 \cdot 14 + 7 \\
		14 &= 2 \cdot 7	\qquad \Rightarrow n=4, r_4 = 7
	\end{aligned}
	\end{equation}
	Also ist $(133,84) = (7)$.

Wir können den euklidischen Algorithmus für $a,b$ auch wie folgt aufschreiben:
	\[\begin{array}{lll}
	\frac{b}{a} = q_0 + \frac{r_1}{a} & q_0 = \benbrace*{\frac{b}{a}} & 0 < \frac{r_1}{a} < 1, \text{ falls } r_1 \neq 0 \\ 
	\frac{a}{r_1} = q_1 + \frac{r_2}{r_1} & q_1 = \benbrace*{\frac{a}{r_1}} &  \\ 
	\frac{r_1}{r_2} = q_2 + \frac{r_3}{r_2} & &  \\ 
	\qquad \vdots &  &  \\ 
	\frac{r_{n-2}}{r_{n-1}} = q_{n-1} + \frac{r_n}{r_{n-1}} & &  \\ 
	\frac{r_{n-1}}{r_n} = q_n &  & 
	\end{array} \]
Zusammengefasst erhalten wir die \Index{Kettenbruchentwicklung} von $\frac{b}{a}$:
	\[ \frac{b}{a} = q_0 + \cfrac{1}{q_1 + \cfrac{1}{\textcolor{white}{q_2 + \cfrac{\textcolor{black}{\ddots}}{\textcolor{black}{+\cfrac{1}{q_{n-1}+\cfrac{1}{q_n}}}}}}} \]
	
Statt einer rationalen Zahl sei jetzt $\alpha$ allgemeiner eine beliebige reelle Zahl. \\
Es ist $\alpha = [\alpha] + \varepsilon$ mit $0 \leq \varepsilon < 1$. Falls $\alpha \notin \ZZ$, d.h. $\varepsilon > 0$, setze $q_0 := [\alpha]$ und $\rho_1 := \frac{1}{\varepsilon}$. Dann:
\[ \begin{array}{lll}
\alpha = q_0 + \frac{1}{\rho_1} & \text{ mit } \rho_1 > 1. & \text{Falls } \rho_1 \notin \ZZ, \text{ so setze } [\rho_1] =: q_1 \\ 
\rho_1 = q_1 + \frac{1}{\rho_2} & \text{ mit } \rho_2 > 1. & \text{usw.} \\ 
\qquad \vdots &  &  \\ 
\rho_k = q_k + \frac{1}{\rho_{k+1}} & \text{ mit } \rho_{k+1} > 1.  & 
\end{array} \]
Abbrechen, wenn $\rho_{n+1} \in \ZZ$, sonst weiter. Jedenfalls:
\[ \alpha = \frac{b}{a} = q_0 + \cfrac{1}{q_1 + \cfrac{1}{\textcolor{white}{q_2 + \cfrac{\textcolor{black}{\ddots}}{\textcolor{black}{+\cfrac{1}{q_k+\cfrac{1}{\rho_{k+1}}}}}}}} \]

\begin{defn}[Kettenbruch, $k$-ter Rest] \label{def_2.4}
	\begin{enumerate}[1)]
		\item $q_0,q_1,\dots,q_n$ seien reelle Zahlen mit $q_1,\dots,q_n > 0$. Unter dem \bet{endlichen Kettenbruch} \index{Kettenbruch}
		\begin{equation}
			[q_0,q_1,\dots,q_n] \label{eq_2.1}
		\end{equation}
		mit den Teilquotienten $q_i$ verstehen wir sowohl das $(n+1)$-Tupel $(q_0,q_1,\dots,q_n)$, als auch seinen wie folgt definierten Wert:
		\begin{equation}
			[q_0,q_1,\dots,q_n] = q_0 + \cfrac{1}{q_1 + \cfrac{1}{\textcolor{white}{q_2 + \cfrac{\textcolor{black}{\ddots}}{\textcolor{black}{+\cfrac{1}{q_n}}}}}} \label{eq_2.2}
		\end{equation}
		Für $0 \leq k \leq n$ nennen wir den Kettenbruch
		\begin{equation}
			\rho_k := [q_k,q_{k+1},\dots,q_n] \label{eq_2.3}
		\end{equation}
		den \bet{$k$-ten Rest} des Kettenbruchs \eqref{eq_2.1}. Für den Wert \eqref{eq_2.1} des Kettenbruchs \eqref{eq_2.1} gilt: \index{$k$-ter Rest}
		\begin{equation}
			[q_0,q_1,\dots,q_n] = [q_0,q_1,\dots,q_{k-1},\rho_k] \text{ für } 0 \leq k \leq n \label{eq_2.4}
		\end{equation}
		Man kann den Wert \eqref{eq_2.2} des Kettenbruchs \eqref{eq_2.1} durch \eqref{eq_2.4} mit \eqref{eq_2.3} rekursiv definieren: Es ist $[q_0] = q_0, [q_0,q_1] = q_0 + \frac{1}{q_1}$, also:
		\begin{equation}
			[q_0,q_1,\dots,q_n] = [q_0, \rho_1] = q_0 + \frac{1}{\rho_1} \text{ für } n \geq 1 \label{eq_2.5}
		\end{equation}
		\item Gegeben sei eine Folge $(q_k)_{k \geq 0}$ in $\RR$ mit $q_k > 0$ für $k \geq 1$. Unter dem \bet{unendlichen Kettenbruch} \index{Kettenbruch}
		\begin{equation}
			[q_0,q_1,q_2,\dots] \label{eq_2.6}
		\end{equation}
		verstehen wir die Folge der
		\begin{equation}
			[q_0,q_1,\dots,q_n] \qquad n=0,1,2,\dots \label{eq_2.7}
		\end{equation}
		Falls diese Folge in $\RR$ konvergiert, so bezeichnen wir auch deren Limes mit $[q_0,q_1,q_2,\dots]$. Der unendliche Kettenbruch
		\begin{equation}
			\rho_k := [q_k,q_{k+1},\dots] \qquad k = 0,1,2,\dots \label{eq_2.8}
		\end{equation}
		heißt der \bet{$k$-te Rest} von \eqref{eq_2.6}. Formal gilt: \index{$k$-ter Rest}
		\begin{equation}
			[q_0,q_1,q_2,\dots] = [q_0,q_1,\dots,q_{k-1},\rho_k] \label{eq_2.9}
		\end{equation}
		Später werden wir sehen, dass \eqref{eq_2.9} auch für die Werte der entsprechenden Kettenbrüche gilt, wenn \eqref{eq_2.8} konvergiert.
	\end{enumerate}
\end{defn}

\begin{defn}[Näherungsbruch] \label{def_2.5}
	Jedem endlichen Kettenbruch $[q_0,q_1,\dots,q_k]$ ordnen wir rekursiv ein Paar $\binom{c}{d} \in \RR \times \RR_{>0}$ reeller Zahlen zu mit
	\begin{equation}
		[q_0,q_1,\dots,q_k] = \frac{c}{d} \label{eq_2.10}
	\end{equation}
	$k = 0$: Für $[q_0]$ sei $\binom{c}{d} = \binom{q_0}{1}$. Es gilt dann in der Tat $[q_0] = q_0 = \frac{q_0}{1}$. \\
	$k \geq 1$: Zuerst Motivation (Heuristik):
	\begin{equation}
		[q_0,q_1,\dots,q_k] = [q_0,\rho_1] = q_0 + \frac{1}{\rho_1} \label{eq_2.11}
	\end{equation}
	mit $\rho_1 = [q_1,q_2,\dots,q_k]$. Gehöre $\binom{c'}{d'}$ zu $\rho_1$. Dann gilt
	\[ [q_0,q_1,\dots,q_k] = q_0 + \frac{d'}{c'} = \frac{q_0 c' + d'}{c'} \]
	Wir ordnen nun also $[q_0,q_1,\dots,q_k]$ das Tupel
	\begin{equation}
		\binom{c}{d} = \binom{q_0 c' + d'}{c'} = \underbrace{\begin{pmatrix}
		q_0 & 1 \\ 
		1 & 0
		\end{pmatrix}}_{=: M_1} \binom{c'}{d'} \label{eq_2.12}
	\end{equation}
	zu. Dann gilt \eqref{eq_2.10}. Sei jetzt
	\begin{equation}
		[q_0,q_1,\dots] \label{eq_2.13}
	\end{equation}
	ein endlicher oder unendlicher Kettenbruch. Das dem $k$-ten Abschnitt
	\begin{equation}
		[q_0,q_1,\dots,q_k] \label{eq_2.14}
	\end{equation}
	von \eqref{eq_2.13} zugeordnete 2-Tupel
	\begin{equation}
		\binom{c_k}{d_k} \label{eq_2.15}
	\end{equation}
	heißt der \bet{$k$-te Näherungsbruch} von \eqref{eq_2.13}. Auch $\frac{c_k}{d_k}$ heißt $k$-ter Näherungsbruch von \eqref{eq_2.13}. Ist \eqref{eq_2.13} der endliche Kettenbruch $[q_0,q_1,\dots,q_n]$, so ist der $n$-te Näherungsbruch $\frac{c_n}{d_n}$ gleich dem Wert dieses Kettenbruchs. Allgemein ist $\frac{c_k}{d_k}$ der Wert des Kettenbruchs \eqref{eq_2.14}. Aus formalen Gründen definieren wir noch
	\begin{equation}
		\binom{c_{-1}}{d_{-1}} = \binom{1}{0} \qquad \binom{c_{-2}}{d_{-2}} = \binom{0}{1} \label{eq_2.16}
	\end{equation}
\end{defn}

\begin{falko}[Rekursionsformeln für Näherungsbrüche] \label{F2.6}
	Mit den Bezeichnungen wie oben gilt:
	\begin{equation}
	\begin{aligned}
		c_k &= q_k c_{k-1} + c_{k-1}	\label{eq_2.17}
		d_k &= q_k d_{k-1} + d_{k-2}
	\end{aligned}
	\end{equation}
	Dies schreiben wir auch in Matrizenform:
	\begin{equation}
		\binom{c_k}{d_k} = \underbrace{\begin{pmatrix}
		c_{k-1} & c_{k-2} \\ 
		d_{k-1} & d_{k-2}
		\end{pmatrix}}_{=:M_k} \binom{q_k}{1}
	\end{equation}
\end{falko}

\minisec{Bemerkung}
	$d_k > 0$ für $k \geq 0$ (vgl. Definition \ref{def_2.5}, oder auch \eqref{eq_2.17}).
	
\begin{falko} \label{F2.7}
	Mit den obigen Bezeichnungen gilt: \marginnote{31.10.}
	\begin{enumerate}[(i)]
		\item $M_{k+1} = M_k \cdot \begin{matrix}
			q_k & 1 \\ 1 & 0 
		\end{matrix}$, also:
		\item $M_{k+1} = \begin{pmatrix} q_0 & 1 \\ 1 & 0 \end{pmatrix} \begin{pmatrix} q_1 & 1 \\ 1 & 0 \end{pmatrix} \dots \begin{pmatrix} q_k & 1 \\ 1 & 0 \end{pmatrix}$
		\item $d_kc_{k-1} - c_kd_{k-1} = (-1)^k$ für $k \geq -1$
		\item $\frac{c_{k-1}}{d_{k-1}} - \frac{c_k}{d_k} = \frac{(-1)^k}{d_k d_{k-1}}$ für $k \geq 1$ \hfill $d_{-1} = 0$
		\item $d_kc_{k-2} - c_kd_{k-2} = (-1)^{k-1} q_k$ für $k \geq 0$
		\item $\frac{c_{k-2}}{d_{k-2}} - \frac{c_k}{d_k} = \frac{(-1)^{k-1} q_k}{d_k d_{k-1}}$ für $k \geq 0$, aber $k \neq 1$
	\end{enumerate}
\end{falko}

\begin{falko} \label{F2.8}
	\begin{enumerate}[(i)]
		\item $\enbrace*{ \frac{c_{2m}}{d_{2m}}}_{m \geq 0}$ ist streng monoton steigend
		\item $\enbrace*{ \frac{c_{2m+1}}{d_{2m+1}}}_{m \geq 0}$ ist streng monoton fallend
		\item $\frac{c_{2m}}{d_{2m}} < \frac{c_{2n+1}}{d_{2n+1}}$ für alle $m \geq 0, n \geq 0$
	\end{enumerate}
\end{falko}

\begin{falko} \label{F2.9}
	\begin{enumerate}[(i)]
		\item $[q_0,q_1,\dots,q_n] = \frac{\rho_k c_{k-1} + c_{k-2}}{\rho_k d_{k-1} + d_{k-2}}$ für $1 \leq k \leq n$.
		\item $[q_k,q_{k-1}, \dots, q_1] = \frac{d_k}{d_{k-1}}$ für $k \geq 1$
	\end{enumerate}
\end{falko}

\begin{falko} \label{F2.10}
	Gegeben sei ein unendlicher Kettenbruch
	\begin{equation}
		\alpha = [q_0,q_1,\dots] \label{eq_2.24}
	\end{equation}
	Dann gelten: \begin{enumerate}[(i)]
		\item $\alpha$ konvergent $\Rightarrow$ jeder Rest $\rho_n = [q_n, q_{n+1}, \dots]$ ist konvergent.
		\item $\rho_n$ konvergent für ein $n \Rightarrow \alpha$ ist konvergent
		\item Ist $\alpha$ konvergent, so gilt für die Werte
		\begin{equation}
			\alpha = \frac{c_{n-1} \rho_n + c_{n-2}}{d_{n-1} \rho_n + d_{n-2}} \quad n \geq 1 \label{eq_F2.10}
		\end{equation}
		d.h. $[q_0,q_1,\dots] = [q_0, q_1, \dots, q_{n-1}, \rho_0]$.
		\item Ist $\alpha$ konvergent, so gilt $\frac{c_{2n}}{d_{2n}} < \alpha < \frac{c_{2m+1}}{d_{2m+1}}$ für alle $n,m \geq 0$.
	\end{enumerate}
\end{falko}

\begin{falko} \label{F2.11}
	Der Wert $\alpha$ eines konvergenten unendlichen Kettenbruchs genügt den Ungleichungen
	\begin{equation}
		\abs{\alpha - \frac{c_k}{d_k}} < \frac{1}{d_k d_{k+1}} \text{ für jedes } k \geq 0 \label{eq_2.26}
	\end{equation}
\end{falko}

\begin{defn}[natürlicher Kettenbruch]
	Ein Kettenbruch $[q_0,q_1,\dots]$ -- endlich oder unendlich -- heißt \Index{natürlicher Kettenbruch}, wenn $q_k \in \ZZ$ für alle $k \geq 0$. Nach wie vor setzen wir $q_k > 0$ für $k \geq 1$ voraus!
\end{defn}

Im weiteren betrachten wir nur natürliche Kettenbrüche und sprechen dann schlechthin von Kettenbrüchen. Nach F\ref{F2.6} ist dann
\[ c_k, d_k \in \ZZ \text{ für alle } k \geq -2, \quad d_k \in \NN \text{ für } k \geq 0, \quad d_k = q_kd_{k-1} + d_{k-2} \geq d_{k+1} + 1 > d_{k-1} \text{ für } k \geq 1 \]
\begin{equation}
	d_k > d_{k-1} \text{ für } k \geq 2, \quad d_k \geq k \text{ für } k \geq 1 \label{eq_2.28}
\end{equation} 
\eqref{eq_2.28} gilt im Allgemeinen nicht für $k = 1$. Denn $d_0 = 1$, und es ist $d_1 = 1$ möglich.

\minisec{Bemerkung}
	Induktiv folgt leicht $d_k > 2^{\frac{k-1}{2}}$ für $k \geq 2$.
	
\begin{falko} \label{F2.12}
	Jeder unendliche natürliche Kettenbruch ist konvergent.
\end{falko}

\begin{falko} \label{F2.13}
	Die Näherungsbrüche eines natürlichen Kettenbruchs lassen sich nicht kürzen, d.h. $c_k$ und $d_k$ sind teilerfremd für jedes $k \geq -2$. \\
	Wir können also wirklich $\binom{c_k}{d_k}$ mit $\frac{c_k}{d_k}$ identifizieren.
\end{falko}

\begin{falko} \label{F2.14}
	Jede rationale Zahl ist durch einen endlichen natürlichen Kettenbruch darstellbar.
\end{falko}

\newpage