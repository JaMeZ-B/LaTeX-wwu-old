\section{Der euklidische Algorithmus}
\label{sec:para2}
	Sei $R$ kommutativer Ring mit $1 \neq 0$. Für beliebiges $a \in R$ betrachte man die Menge der Vielfachen von $a \in R$, also
	\[ Ra := {xa : x \in R} = {b \in R : a|b} \]
	Die Teilmenge $I = Ra$ hat folgende Eigenschaften:
	\begin{enumerate}[(i)]
		\item $0 \in I$
		\item $b_1,b_2 \in I \Rightarrow b_1+b_2 \in I$
		\item $c \in R, b \in I \Rightarrow cb \in I$
	\end{enumerate}
	
\begin{defn}[Ideal, Hauptideal] \label{def_2.1}
	Eine Teilmenge $I$ von $R$ heißt ein \Index{Ideal} in $R$, falls die Eigenschaften (i), (ii), (iii) erfüllt sind. $I$ heißt \Index{Hauptideal}, wenn es ein $a \in R$ gibt mit $I = Ra$. Wir verwenden die Bezeichnung
	\[ (a) := Ra \]
	und nennen $(a)$ das von $a \in R$ erzeugte Hauptideal.
\end{defn}

\minisec{Bemerkung}
	\begin{enumerate}[(1)]
		\item $(b) \subseteq (a) \Leftrightarrow a | b$
		\item $a \assoz b \Leftrightarrow (a) = (b)$
		\item $c$ ist gemeinsames Vielfaches von $a_1,\dots,a_n \Leftrightarrow (c) \subseteq (a_1) \cap \dots \cap (a_n)$
		\item $m$ ist ein kgV von $a_1,\dots,a_n \Leftrightarrow (a_1) \cap \dots \cap (a_n) = (m)$
		\item $d$ ist ein gemeinsamer Teiler von $a_1,\dots,a_n \Leftrightarrow (a_i) \subseteq (d)$ für $1 \leq i \leq n$
		\item $d$ ist ein gemeinsamer Teiler von $a_1,\dots,a_n \Leftrightarrow Ra_1 + Ra_2 + \dots + Ra_n \subseteq (d)$
		\item $d$ ist ein ggT von $a_1,\dots,a_n \Leftrightarrow (d)$ ist das kleinste Hauptideal mit $Ra_1 + \dots Ra_n \subseteq (d)$.
	\end{enumerate}
	Ein ggT lässt sich also idealtheoretisch nicht so einfach charakterisieren wie oben ein kgV durch (4). Am schönsten wäre es, wenn $Ra_1 + \dots + Ra_n$ ein Hauptideal wäre, dann würde (7) übergehen in:
	\[ d \text{ ist ein ggT von } a_1,\dots,a_n \Leftrightarrow Ra_1+Ra_2+ \dots + Ra_n = (d) \]
	
\begin{defn}[Hauptidealring] \label{def_2.2}
	Ein Integritätsring $R$ heißt ein \Index{Hauptidealring}, wenn jedes Ideal $I$ von $R$ ein Hauptideal ist.
\end{defn}

\minisec{Bezeichnung}
	Für Elemente $a_1,\dots,a_n$ in einem beliebigen kommutativen Ring $R$ mit $1 \neq 0$ setze
	\[ (a_1,\dots,a_n) := Ra_1 + \dots + Ra_n \]
	Man nennt $(a_1,\dots,a_n)$ das von $a_1,\dots,a_n$ erzeugte Ideal in $R$.
	
\begin{falko}[Satz vom größten gemeinsamen Teiler] \label{F2.1}
	Sei $R$ ein Hauptidealring. Dann gilt: Zu jedem System $a_1,\dots,a_n$ von Elementen aus $R$ existiert ein ggT $d$ von $a_1,\dots,a_n$ und jedes solche $d$ besitzt eine Darstellung der Gestalt
	\begin{equation}
		d = x_1a_1+\dots+x_na_n \quad \text{mit } x_i \in R \label{eq_F2.1}
	\end{equation}
	Wir sagen, in $R$ gelte der \Index{Satz vom größten gemeinsamen Teiler}.
\end{falko}

\minisec{Bemerkung}
	Sei $R$ ein beliebiger Integritätsring. Ist $d$ ein gemeinsamer Teiler von $a_1,\dots,a_n$ aus $R$ und gibt es eine Darstellung der Form \eqref{eq_F2.1}, so ist $d$ ein ggT von $a_1,\dots,a_n$.

\begin{satz} \label{satz_2.1}
	$\ZZ$ ist ein Hauptidealring.
\end{satz}

\minisec{Definition (Gaußklammer)}
	Für $x \in \RR$ setze
	\[ [x] = \max \{g \in \ZZ : g \leq x\} \in \ZZ \]
	$[x]$ ist charakterisiert durch folgende zwei Eigenschaften: \begin{enumerate}[(1)]
		\item $[x] \in \ZZ$
		\item $[x] \leq x < [x] + 1$
	\end{enumerate}

\begin{falko}[Division mit Rest in $\ZZ$] \label{F2.2}
	Gegeben $a, b \in \ZZ$, $a \neq 0$. Dann gibt es eine Darstellung
	\begin{equation}
		b = qa +r \quad \text{mit } 0 \leq r < |a| \text{ und } q,r \in \ZZ \label{eq_F2.2}
	\end{equation}
\end{falko}

\minisec{Bemerkung}
	\begin{enumerate}[1)]
		\item Die Darstellung \eqref{eq_F2.2} ist eindeutig.
		\item Es gibt eine Darstellung
		\begin{equation}
			b = qa + r \quad \text{mit } |r| < |a|; q,r \in \ZZ, \label{eq_D'}
		\end{equation}
		doch diese ist nicht mehr eindeutig, z.B. $27 = 4 \cdot 6 + 3 = 5 \cdot 6 - 3$.
		\item Es gibt eine Darstellung
		\begin{equation}
			b = qa + r \quad \text{mit } -\frac{|a|}{2} < r \leq \frac{|a|}{2}; q,r \in \ZZ,
		\end{equation}
		und diese ist eindeutig.
		\item Es gibt eine Darstellung
		\[ b = qa + r \quad \text{mit } |r| \leq \frac{|a|}{2}; q,r \in \ZZ, \]
		doch diese ist nicht eindeutig, falls $a$ gerade.
	\end{enumerate}

\begin{defn}[euklidischer Ring] \label{def_2.3}
	Ein Integritätsring $R$ heißt ein \Index{euklidischer Ring}, falls eine Funktion $\nu\colon R \rightarrow \NN_0$ mit $\nu(0) = 0$ existiert, sodass gilt: Zu $a, b \in R$ mit $a \neq 0$ existieren $q,r \in R$ mit
	\[b = qa + r \text{ und } \nu(r) = \nu(a) \]
\end{defn}

\minisec{Beispiele}
	\begin{enumerate}[(1)]
		\item $R = \ZZ$ mit $\nu(a) = |a|$.
		\item $R = K[X], K$ Körper, mit $\nu(g) = \deg(g) + 1$ für $g \neq 0$, $\nu(0) = 0$.
		\item $R = \ZZ[i]$ mit $\nu(z) = N(z) = z \overline{z} = |z|^2$.
	\end{enumerate}
	
\begin{falko} \label{F2.3}
	Jeder euklidische Ring ist ein Hauptidealring.
\end{falko}

\begin{falko} \label{F2.4}
	Jeder Hauptidealring ist faktoriell.
\end{falko}

Im Folgenden sei $R$ ein euklidischer Ring mit euklidischer Normfunktion $\nu$. Allgemein gilt folgende elementare Umformung:
\begin{equation}
	(a_1,a_2,\dots,a_n) = (a_1, a_2-y_2a_1,\dots,a_n-y_na_1) \text{ für bel. } y_i \in R \label{eq_U}
\end{equation}

\minisec{Euklidischer Algorithmus}
	Gegeben $a_1,\dots,a_n \in R$. Wir wollen $d \in R$ bestimmen mit \index{Euklidischer Algorithmus}
	\[ (a_1,\dots,a_n) = (d) \]
	Sind alle $a_i = 0$, so ist $d = 0$ und wir sind fertig. Sei daher ohne Einschränkung
	\[ a_1 \neq 0 \text{ und } \nu(a_1) \leq \nu(a_i), \text{ falls } a_i \neq 0 \]
	Sei $a_i = q_ia_1 + r_i$ mit $\nu(r_i) < \nu(a_1)$ für $i \geq 2$. Dann ist
	\[ (a_1,\dots,a_n) \stackrel{\eqref{eq_U}}{=} (a_1,r_2,\dots,r_n) \]
	Fortsetzung des Verfahrens liefert
	\[ (d,0,0,\dots,0) = (d) \]
	
\minisec{Beispiel}
\[\begin{array}{rl}
	(\textcolor{red}{27},63,114) & 63 = 2 \cdot 27 + 9, 114 = 4 \cdot 27 + 6 \\ 
	= (27,9,\textcolor{red}{6}) & 27 = 4 \cdot 6 + 3, 9 = 1 \cdot 6 + 3 \\ 
	= (\textcolor{red}{3},3,6) & 3 = 1 \cdot 3 + 0, 6 = 2 \cdot 3 + 0 \\ 
	= (3, 0, 0) = (3) & 
\end{array}\]
\newpage