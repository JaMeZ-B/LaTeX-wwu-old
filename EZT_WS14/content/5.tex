\section{Summen von zwei Quadraten in $\ZZ$ und der Gaußsche Zahlring $\ZZ[i]$}
\label{sec:para5}
	Ausgangspunkt ist F\ref{F3.8}:	\marginnote{02.12. \\ \ [15]}
	\[ p \kon 1 \modu 4 \quad \Rightarrow \quad \exists c \in \ZZ \text{ mit } c^2 \kon -1 \modu p, \]
	d.h. $c^2 + 1 = kp$ mit einem $k \in \ZZ$.
	
\begin{satz}[Fermat, Euler] \label{satz_5.1}
	Sei $p$ eine Primzahl. Ist $p \kon 1 \modu 4$, so gibt es $x,y \in \ZZ$ mit
	\begin{equation}
		p = x^2 + y^2 \label{eq_5.1}
	\end{equation}
	Ist umgekehrt $p$ in der Gestalt \eqref{eq_5.1} darstellbar, so ist $p \kon 1 \modu 4$ oder $p = 2$.
\end{satz}

\begin{defn}[Gaußscher Zahlring]
	\[ \ZZ[i] := \{a + bi : a,b \in \ZZ\} \]	
	heißt \Index{Gaußscher Zahlring}. Es ist $\ZZ[i]^\times = \{1, -1, i, -i\}$ und $(a+bi)(a-bi) = a^2+b^2$.
\end{defn}

\begin{satz}[Gaußscher Zahlring ist euklidisch] \label{satz_5.2}
	$\ZZ[i]$ ist ein euklidischer Ring mit euklidischer Normfunktion $\nu$ definiert durch \index{euklidischer Ring}
	\[ \nu(z) = z \cdot \overline{z} =: N(z), z \in \ZZ[i] \]
\end{satz}

\begin{falko} \label{F5.1}
	Sei $\pi$ ein Primelement $\neq 0$ von $\ZZ[i]$. Dann gibt es genau eine Primzahl $p$ mit $\pi | p$ in $\ZZ[i]$. Es gilt entweder $N(\pi)=p$ oder $N(\pi)=p^2$. Im ersten Fall nennen wir $\pi$ vom Grad 1, im zweiten Fall vom Grad 2.
\end{falko}

Um alle Primelemente $\pi$ von $\ZZ[i]$ zu finden, haben wir also die Primfaktorzerlegung aller $p \in \ZZ[i]$ zu untersuchen. Die $p$ heißen \bet{rationale Primzahlen}, die $\pi$ \bet{Gaußsche Primzahlen}. \index{rationale Primzahl} \index{Gaußsche Primzahl}

\begin{satz} \label{satz_5.3}
	Sei $p$ Primzahl sowie $\pi$ ein Primfaktor von $p$ in $\ZZ[i]$. Dann gibt es drei Fälle: \begin{enumerate}[(i)]
		\item $p \assoz \pi^2$ \quad ($p$ ist \bet{verzweigt} in $\ZZ[i]$)
		\item $p \assoz \pi$ \quad ($p$ ist \bet{träge} in $\ZZ[i]$, d.h. $p$ bleibt Primelement in $\ZZ[i]$)
		\item $p = \pi \overline{\pi}$ mit $pi \not \assoz \overline{\pi}$ \quad ($p$ \bet{zerfällt} in $\ZZ[i]$)
	\end{enumerate}
	Und zwar gilt: \index{Verzweigtheit} \index{Trägheit}
	\[\begin{array}{rclcl}
		\text{(i)} & \Leftrightarrow &  p = 2 &  & \\ 
		\text{(ii)} & \Leftrightarrow  & N(\pi) = p^2 & \Leftrightarrow &  p \kon 3 \modu 4 \\ 
		\text{(iii)} & \Leftrightarrow & N(\pi) = p & \Leftrightarrow &  p \kon 1 \modu 4 \\ 
	\end{array}\]
	Also ist z.B. $7$ auch in $\ZZ[i]$ ein Primelement, aber $5 = (2+i)(2-i)$ nicht.	
\end{satz}

\minisec{Korollar}
	Ist $p$ eine Primzahl mit $p \kon 1 \modu 4$, so ist $p$ in der Gestalt $p = a^2 + b^2$ mit $a,b \in \NN$ darstellbar. Bis auf Vertauschung von $a$ und $b$ ist diese Darstellung eindeutig. Ferner ist notwendigerweise $(a,b) = 1$.
	
\begin{satz} \label{satz_5.4}
	Sei $n \in \NN$. \begin{enumerate}[(i)]
		\item Genau dann ist $n$ eine Summe von zwei Quadraten in $\ZZ$, wenn für jede Primzahl $p \kon 3 \modu 4$ der Exponent $w_p(n)$ gerade ist.
		\item Besitzt $n$ eine primitive Darstellung als Summe von zwei Quadraten, d.h.
		\begin{equation}
			n = a^2 + b^2 \text{ mit teilerfremden } a,b \in \ZZ, \label{eq_satz_5.4_1}
		\end{equation}
		so folgt:
		\begin{equation}
			n \text{ hat keine Primteiler } p \kon 3 \modu 4, \text{ und es ist } 4 \not | \ n. \label{eq_satz_5.4_2}
		\end{equation}
		\item Umgekehrt: Gelte \eqref{eq_satz_5.4_2}, und bezeichne $s$ die Anzahl der ungeraden Primteiler von $n$. Für $n > 2$ hat dann $n$ genau $2^{s-1}$ primitive Darstellungen als Summe von zwei Quadraten, wenn nur wesentlich verschiedene Darstellungen gezählt werden. \\
		(Beachte: $n$ kann außerdem noch nicht-primitive Darstellungen haben, z.B. $50 = 7^2 + 1^2 = 5^2 + 5^2$.)
	\end{enumerate}
\end{satz}

\minisec{Korollar}
	Es sei $n$ \marginnote{05.12. \\ \ [16]} eine ungerade natürliche Zahl, $n > 1$. Besitzt $n$ im Wesentlichen nur eine einzige Darstellung als Summe von zwei Quadraten und ist diese Darstellung primitiv, so ist $n$ eine Primzahl (Umkehrung des Korollars von Satz \ref{satz_5.3}).
	
\minisec{Bemerkung}
	$45 = 6^2 + 3^2$ ist die einzige Darstellung von $45$ als Summe von zwei Quadraten, doch diese ist nicht primitiv. \\
	Im Übrigen ist die Voraussetzung, dass $n$ ungerade ist, wesentlich: Für $n = 10$ ist $10 = 3^2 + 1^2$ die im Wesentlichen einzige Darstellung von $10$ als Summe von zwei Quadraten und diese ist auch primitiv.
\newpage