\section*{Vorwort}
\label{sec:preface}
	Der vorliegende Text ist eine Zusammenfassung zur Vorlesung Elementare Zahlentheorie, gelesen von Prof. Dr. Falko Lorenz an der WWU Münster im Wintersemester 2014/2015. Der Inhalt entspricht weitestgehend dem Skript, welches auf der Vorlesungswebsite bereitsgestellt wird, jedoch wird auf Beweise weitestgehend verzichtet. Dieses Werk ist keine Eigenleistung des Autors und wird nicht vom Dozenten der Veranstaltung korrekturgelesen. Für die Korrektheit des Inhalts wird daher keinerlei Garantie übernommen. Bemerkungen, Korrekturen und Ergänzungen kann man folgenderweise loswerden:
	\begin{itemize}
		\item persönlich durch Überreichen von Notizen oder per E-Mail
		\item durch Abändern der entsprechenden TeX-Dateien und Versand per E-Mail an mich
		\item direktes Mitarbeiten via GitHub. Dieses Skript befindet sich im \texttt{latex-wwu}-Repository von Jannes Bantje:
		\begin{center}
			\url{https://github.com/JaMeZ-B/latex-wwu}
		\end{center}
	\end{itemize}

\subsection*{Literatur}
\label{sub:lit}
\begin{itemize}
	\item F. Ischebeck: \href{http://wwwmath.uni-muenster.de/u/ischebeck/}{Einladung zur Zahlentheorie}
	\item R. Remmert, P. Ullrich: \href{http://link.springer.com/book/10.1007/978-3-7643-7731-1}{Elementare Zahlentheorie}
	\item A. Scholz, B. Schöneberg: Einführung in die Zahlentheorie
	\item K. Halupczok: \href{http://wwwmath.uni-muenster.de/u/karin.halupczok/ElZthSS2009Skript.pdf}{Skript zur Elementaren Zahlentheorie}
\end{itemize}

\subsection*{Vorlesungswebsite}
\label{sub:link}
Das vollständige Skript des Dozenten sowie weiteres Material findet man unter folgendem Link:
\begin{center}
	\url{\homepage}
\end{center}

\subsection*{Titelbild}
\label{sub:titlepic}
Plot der Eulerschen $\varphi$-Funktion für $1 \leq n \leq 1000$, erstellt mit Maple. Zugehöriges Maple-Worksheet befindet sich im Git-Repository (siehe oben).

\vfill
\begin{flushright}
	Phil Steinhorst \\
	p.st@wwu.de
\end{flushright}
\newpage