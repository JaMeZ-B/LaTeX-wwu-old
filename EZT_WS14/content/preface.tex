\section*{Vorwort}
\label{sec:preface}
	Der vorliegende Text ist eine Zusammenfassung zur Vorlesung Elementare Zahlentheorie, gelesen von Prof. Dr. Falko Lorenz an der WWU Münster im Wintersemester 2014/2015. Der Inhalt entspricht weitestgehend dem Skript, welches auf der Vorlesungswebsite bereitsgestellt wird, jedoch wird auf Beweise weitestgehend verzichtet. Für die Korrektheit des Inhalts wird keinerlei Garantie übernommen. Bemerkungen, Korrekturen und Ergänzungen kann man folgenderweise loswerden:
	\begin{itemize}
		\item persönlich durch Überreichen von Notizen oder per E-Mail
		\item durch Abändern der entsprechenden TeX-Dateien und Versand per E-Mail an mich
		\item direktes Mitarbeiten via GitHub. Dieses Skript befindet sich im \texttt{latex-wwu}-Repository von Jannes Bantje:
		\begin{center}
			\url{https://github.com/JaMeZ-B/latex-wwu}
		\end{center}
	\end{itemize}

\subsection*{Themenübersicht}
\label{sub:content}
	Im Sommersemester 2013 wurden folgende Themen behandelt:
	\begin{itemize}
		\item Ein paar algebraische Grundlagen (Gruppen- und Ringtheorie, Ideale) 
		\item Fundamentalsatz der Arithmetik (Satz von der eindeutigen Primfaktorzerlegung) 
		\item Euklidischer Algorithmus, Kettenbruchdarstellung 
		\item Simultane Kongruenzen, Satz von Euler-Fermat, chinesischer Restsatz 
		\item Restklassengruppen, Hauptsatz über endliche abelsche Gruppen 
		\item Gaußscher Zahlenring $\ZZ[\mathfrak{i}]$ 
		\item Quadratische Reste, Quadratisches Reziprozitätsgesetz 
		\item Fermat- und Mersenne-Primzahlen 
		\item Zahlentheoretische Funktionen $\varphi\colon \NN \longrightarrow \CC$
		\item Satz von Lagrange ("Vier-Quadrate-Satz")	
	\end{itemize}

\subsection*{Literatur}
\label{sub:lit}
\begin{itemize}
	\item F. Ischebeck: \href{http://wwwmath.uni-muenster.de/u/ischebeck/}{Einladung zur Zahlentheorie}
	\item R. Remmert, P. Ullrich: \href{http://link.springer.com/book/10.1007/978-3-7643-7731-1}{Elementare Zahlentheorie}
	\item A. Scholz, B. Schöneberg: Einführung in die Zahlentheorie
	\item K. Halupczok: \href{http://wwwmath.uni-muenster.de/u/karin.halupczok/ElZthSS2009Skript.pdf}{Skript zur Elementaren Zahlentheorie}
\end{itemize}

\subsection*{Vorlesungswebsite}
\label{sub:link}
\begin{center}
	\url{\homepage}
\end{center}


\vfill
\begin{flushright}
	Phil Steinhorst \\
	p.st@wwu.de
\end{flushright}
\newpage