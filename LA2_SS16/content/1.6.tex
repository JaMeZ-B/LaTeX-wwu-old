\section{Lineare Abbildungen von $\mathbf{K^n}$ nach $\mathbf{K^m}$ und Matrizen}

\begin{definition}[Linearkombination]
	\label{def:I.6.1}
	Sei $V$ ein $K$-Vektorraum, $v_1,\dots,v_m \in V$ und $\lambda_1,\dots,\lambda_m \in K$, so heißt
	\[
		\sum_{i=1}^{m} \lambda_i v_i
	\]
	eine \Index{Linearkombination} der Vektoren $v_1,\dots,v_m$ in $V$.
\end{definition}

\setcounter{definition}{2}
\begin{lemma}
	\label{lemma:I.6.3}
	Sei $A \in M(m \times n,K)$, dann ist die Abbildung
	\begin{align*}
		F_A \colon K^n &\longrightarrow K^m \\
		x &\longmapsto Ax
	\end{align*}
	mit $Ax$ wie in \autoref{def:I.3.5}(3) linear.
\end{lemma}

\begin{satz}[Umkehrung von \autoref{lemma:I.6.3}]
	\label{satz:I.6.4}
	Ist $F\colon K^n \rightarrow K^m$ eine lineare Abbildung, dann ist $A_F = (F(e_1),\dots,F(e_n)) \in M(m\times n,K)$ die eindeutige Matrix mit $F(x) = A_F x$ für alle $x \in K^n$.
\end{satz}

\setcounter{definition}{5}
\begin{satz}
	\label{satz:I.6.6}
	Die Abbildung
	\begin{align*}
		\Phi \colon \Hom(K^n,K^m) &\longrightarrow M(m \times n,K) \\
		F &\longmapsto A_F
	\end{align*}
	ist ein Isomorphismus von $K$-Vektorräumen mit Umkehrabbildung $\Phi^{-1}(A) = F_A$.
	Für $A \in M(m\times n,A)$ setzen wir $\Kern(A) := \Kern(F_A)$ und $\Bild(A) := \Bild(F_A)$.
\end{satz}

\setcounter{definition}{7}
\begin{satz}
	\label{satz:I.6.8}
	Seien $A \in M(m\times n,K)$ und $b \in K^m$.
	Dann gilt:
	\begin{enumerate}[(1)]
		\item Das LGS $Ax = b$ besitzt genau dann eine Lösung $x \in K^n$, wenn $b \in \Bild(A)$.
		\item Ist $x_s \in K^n$ eine spezielle Lösung von $Ax=b$, so ist die Menge $\LL$ aller Lösungen von $Ax=b$ gegeben durch
		\[
			\LL = x_s + \Kern(A) = \{x_s + x : x \in \Kern(A)\}.
		\]
	\end{enumerate}
\end{satz}

\begin{korollar}
	\label{folg:I.6.9}
	Sei $A \in M(m\times n,K)$.
	Dann sind äquivalent:
	\begin{enumerate}[(1)]
		\item Die Gleichung $Ax=b$ besitzt für jedes $b \in K^m$ genau eine Lösung $x \in K^n$.
		\item Es gilt $\Kern(A) = \setzero$ und $\Bild(A) = K^m$.
		\item Die Abbildung $F_A$ ist bijektiv.
	\end{enumerate}
\end{korollar}

\setcounter{definition}{12}
\begin{definition}[Matrixmultiplikation]
	\label{def:I.6.13}
	Für $B \in M(l \times m,K)$ und $A \in (m \times n,K)$ definieren wir das Produkt $C := B \cdot A = (c_{ij})_{i,j} \in M(l \times n,K)$ durch \index{Matrixmultiplikation}
	\[
		c_{ij} := \sum_{k=1}^{m} b_{ik}a_{kj} \quad \text{für } 1 \leq i \leq l \text{ und } 1 \leq j \leq n.
	\]
\end{definition}

\begin{satz}
	\label{satz:I.6.14}
	Seien $F \colon K^n \rightarrow K^m, G \colon K^m \rightarrow K^l$ linear.
	Dann gilt $A_{G \circ F} = A_G \cdot A_F$. \\
	Ist umgekehrt $A \in M(m \times n,K), B \in M(l \times m,K)$, so ist $F_{BA} = F_B \circ F_A$.
\end{satz}

\setcounter{definition}{16}
\begin{definition}[Inverse Matrix]
	\label{def:I.6.17}
	$A \in M(n \times n,K)$ heißt \Index{invertierbar}, falls eine Matrix $B \in M(n\times n,K)$ existiert mit $BA = AB = E_n$.
	$B$ heißt die zu $A$ \Index{inverse Matrix} und ist eindeutig bestimmt.
	Wir schreiben $A^{-1} := B$.
\end{definition}

\setcounter{definition}{18}
\begin{satz}
	\label{satz:I.6.19}
	Sei $A \in M(n \times n,K)$.
	Folgende Aussagen sind äquivalent:
	\begin{enumerate}[(1)]
		\item $A$ ist invertierbar.
		\item $F_A\colon K^n \rightarrow K^n$ ist invertierbar und es gilt $(F_A)^{-1} = F_{A^{-1}}$.
		\item Für jedes $b \in K^n$ existiert genau eine Lösung $x \in K^n$ für das LGS $Ax = b$, und es gilt $x = A^{-1}b$.
		\item $A$ lässt sich durch elementare Zeilenumformungen in die Einheitsmatrix $E_n$
	\end{enumerate}
\end{satz}
\newpage
\setcounter{definition}{21}
\begin{definition}[allgemeine lineare Gruppe]
	\label{def:I.6.22}
	Sei $K$ ein Körper und sei $n \in \NN$.
	Die Menge
	\[
		\GL(n,K) := \{A \in M(n\times n,K) : A \text{ ist invertierbar}\} \subseteq M(n\times n,K)
	\]
	ist versehen mit der Matrixmultiplikation eine Gruppe -- die \Index{allgemeine lineare Gruppe}.
\end{definition}