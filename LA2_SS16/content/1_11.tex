\section{Der Entzerrungsalgorithmus und der Rang einer Matrix}

\setcounter{definition}{2}
\begin{definition}[Rang einer Matrix]
	\label{def:I.11.3}
	Sei $A \in M(m \times n,K)$.
	Dann heißt 
	\[
		\Rang(A) := \dim_K(\Bild(A))
	\]
	der \Index{Rang} der Matrix $A$.
\end{definition}

\begin{lemma}
	\label{lemma:I.11.4}
	Sei $A \in M(m \times n,K)$ und seien $D \in \GL(m,K)$ und $F \in \GL(n,K)$ invertierbare Matrizen.
	Dann gilt
	\[
		\Rang(A) = \Rang(DA) = \Rang(AF) = \Rang(DAF).
	\]
	Insbesondere ist $\Rang(A)$ invariant unter elementaren Zeilen- und Spaltenumformungen.
	
	Da wir $A$ durch elementare Zeilen- und Spaltenumformungen auf die Gestalt
	\[
		\enb{\begin{BMAT}[.25cm]{c|c}{c|c}
			E_k & 0 \\
			0 & 0
			\end{BMAT}} = \enb{ \begin{BMAT}(e)[1pt]{ccc|c}{ccc|c}
			1 & & & \\
			& \ddots & & 0 \\
			& & 1 & \\
			& 0 & & 0
		\end{BMAT}}
	\]
	bringen können, gilt $\Rang(A) = k$ sowie $\Rang(A) = \Rang(A^T)$.
\end{lemma}

\setcounter{definition}{8}
\begin{satz}
	\label{satz:I.11.9}
	Sei $A \in M(m \times n,K)$ und $b \in K^m$.
	Das LGS $Ax = b$ ist genau dann lösbar, wenn $\Rang(A \mid b) = \Rang(A)$ gilt.
	Ist zusätzlich $\Rang(A) = n$, so ist die Lösung eindeutig.
\end{satz}