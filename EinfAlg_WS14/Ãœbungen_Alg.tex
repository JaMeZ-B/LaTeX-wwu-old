\section*{Aussagen aus den Übungen}
\addcontentsline{toc}{section}{Aussagen aus den Übungen}
\label{sec:übungenalg}

\subsection*{Zettel 1}
\label{sub:zettel_1alg}
\minisec{Aufage 1.2}
Sei $G$ eine Gruppe. $A,B$ Untergruppen von $G$.\\
\zz: Wenn $A\cup B$ eine Untergruppe ist, dann gilt: $A\subseteq B$ oder $b\subseteq A$.\\

\bet{Beweis:}\\
Annahme: $A\not\subseteq B$. Also ex. ein $a\in A\backslash B$ und $b\in B$ beliebig. Betrachte $ab\in A\cup B$, da $AB$ Untergruppe. Also ist $ab\in A$ oder $ab\in B$.\\
Sei $ab\in B$ und $b^{-1}\in B$, da $B$ Untergruppe, folgt, dass $abb^{-1}=a\in B$. $\lightning$ zur Annahme.\\
Also $ab\in A$ und $a^{-1}\in A$, da $A$ Untergruppe, folgt, dass $a^{-1}ab=b\in A$. Da $b$ beliebig war, folgt $B\subseteq A$.
\hfill $\square$

\minisec{Aufgabe 1.4}
Gruppe $G$. $A,B$ Untergruppen.Wir definieren $AB:=\{ab~|~a\in A,b \in B \}$.\\
\begin{enumerate}[(i)]
	\item Die Menge $AB$ ist im allgemeinen keine Untergruppe.
	\item Wenn weiter gilt $AB=BA$, dann ist $AB$ eine Untergruppe.
\end{enumerate}
Beweise klar! (\checkmark)
% sub end

\subsection*{Zettel 2}
\label{sub:zettel_2lga}
\minisec{Aufgabe 2.1}
Eine Gruppe $G$ hat \Index{Exponent} $k$, wenn für jedes Gruppenelement $g\in G$ gilt: $g^k=e$.\\
\zz Gruppen mit Exponent 2 sind abelsch.\\

\bet{Beweis:}\\
Aus $g^2=e$ folgt $g=g^{-1}~\forall g\in G$. $a,b\in G$ beliebig\\
\[ ab=(ab)^{-1}\stackrel{\text{G Gruppe}}{=} b^{-1}a^{-1}=ba \]
\hfill $\square$\\
Anmerkung: Gruppen mit Exponenten 3 sind im allgemeinen nicht abelsch.

\minisec{Aufgabe 2.3}
Menge $X$ und $Sym(X)$. Der \Index{Träger einer Permutation} $\sigma \in Sym(X)$ ist definiert wie folgt: $\supp(\sigma):=\{x\in X~|~\sigma(x)\not=x \}$.
\begin{enumerate}[(i)]
	\item Wenn $\supp(\rho)\cap \supp(\sigma)=\emptyset$ für $\rho, \sigma \in Sym(X)$ gilt, dann folgt $\rho \circ \sigma=\sigma \circ \rho$.
	\item Wenn $\supp(\rho)\cap \supp(\sigma)=\emptyset$ und $\rho \circ \sigma=\id$ für $\sigma,\rho\in Sym(X)$ gilt, dann folgt $\rho=\sigma=\id$.
\end{enumerate}

\bet{Beweis:}\\
\begin{enumerate}[(i)]
	\item Es gilt: $\rho\circ\sigma=\left\{\begin{array}{cl} x, & \text{wenn } x\notin \supp(\rho),~\supp(\sigma)\\ \rho(x), & \text{wenn } x\in \supp(\rho)\\ \sigma(x), & \text{wenn } x\in \supp(\sigma) \end{array}\right.$
	oder 
	$\sigma\circ\rho=\left\{\begin{array}{cl} x, & \text{wenn } x\notin \supp(\rho),~\supp(\sigma)\\ \rho(x), & \text{wenn } x\in \supp(\rho)\\ \sigma(x), & \text{wenn } x\in \supp(\sigma) \end{array}\right.$\\
	
	Da $\supp(\rho)\cap \supp(\sigma)=\emptyset$ gilt und somit $x$ nicht von beiden Permutationen verändert wird. Da Permutationen bijektiv nach Definition sind, ist dies wohldefiniert.
	\item Nach (i) gilt, dass $\rho\circ\sigma=\left\{\begin{array}{cl} x, & \text{wenn } x\notin \supp(\rho),~\supp(\sigma)\\ \rho(x), & \text{wenn } x\in \supp(\rho)\\ \sigma(x), & \text{wenn } x\in \supp(\sigma) \end{array}\right.$ gilt.\\
	
	Also muss $\rho(x)=x$ gelten, da $\rho\circ\sigma$ gilt, analog $\sigma(x)=x$. Also folgt $\rho=\sigma=\id$.
\end{enumerate}
%sub end

\subsection*{Zettel 3}
\label{sub:zettel_3alg}




