\documentclass[a4paper, pagesize=pdftex, pdftex, twoside, headsepline, index=totoc,toc=listof, fontsize=10pt, cleardoublepage=empty, headinclude, DIV=13, BCOR=13mm]{scrartcl}

\usepackage[ngerman]{babel}
\usepackage{scrtime} % Bestandteil von KOMA-Skript, ermoeglicht Zugriff auf Uhrzeit des Kompilierens 
\usepackage{scrpage2} % ermöglicht Bearbeiten von Kopf- und Fusszeilen (wie fancyhdr, nur optimiert auf KOMA-Skript, leich andere Syntax)
\usepackage[utf8]{inputenc} % Gibt an in welcher Textcodierung der Code verstandne werden soll
\usepackage{etex} % sehr technisch, ermöglicht LaTeX mehr Speicher zu belegen
\usepackage[T1]{fontenc} % auch sehr technisch; ist wichtig, um die Schriftarten richtig zu behandeln
\usepackage{textcomp} %verhindert ein paar Fehler bei den Fonts
\usepackage{amsmath} % Packet der American Mathematical Society, das viele Mathematik-Umgebungen und -Befehle definiert
\usepackage{amssymb} %zusätzliche Symbole
\usepackage{latexsym} % nochmal zusätzliche Symbole
\usepackage{stmaryrd} % nochmal mehr zusätzliche Symbole, u.a. Blitz für Widerspruchsbeweise ;)
\usepackage{nicefrac} % schräge Brüche, benutzte ich für Quotienvektorräume
\usepackage{paralist} % redefiniert alle Listenbefehle, sodass diese einen optionalen Parameter haben, der die Nummerierung angibt
\usepackage{dsfont} % Schriftart für N,Z,Q,R die ich momentan benutze (mittels \mathds{R} z.B)
\usepackage[pdftex]{graphicx} % Packet, dass das Einbinden von Grafiken aus Dateien ermöglicht
\usepackage{makeidx}% ermöglicht das automatische Anlegen eines Index 
\usepackage{extarrows}
\usepackage{bbold}
\usepackage{mathtools}
%\usepackage{MnSymbol}

\flushbottom
\usepackage[normalem]{ulem}
\setlength{\ULdepth}{1.8pt}

%--Indexverarbeitung
\newcommand{\bet}[1]{\uline{\textbf{#1}}} %Betonung von Text
\newcommand{\Index}[1]{\uline{\textbf{#1}}\index{#1}} % Befehl, der gleichzeitg das Argument hervorhebt und in den Index mitaufnimmt
\makeindex % startet das automatische Sammeln der Index-Einträge
% Ein kleiner Text am Anfang des Index
\setindexpreamble{{\noindent \itshape Die \emph{Seitenzahlen} sind mit Hyperlinks zu den entsprechenden Seiten versehen, also anklickbar!} \par \bigskip}
\renewcommand{\indexpagestyle}{scrheadings} % Seitenstil für den Index festlegen

%--Farbdefinitionen
\usepackage[usenames, table, x11names]{xcolor} %usenames und x11names, aktivieren viele Farben; siehe Dokumentation von xcolor
% Es lassen sich natürlich auch eigene Farben definieren (hier nur Graustufen)
\definecolor{dark_gray}{gray}{0.45}
\definecolor{light_gray}{gray}{0.7}

%--Zum Zeichnen (ich habe es jetzt mal mit aufgenommen, aber es ist eigentlich nochmal ein ganz anderes Thema, sodass ich da jetzt nicht viel zu sagen werde)
\usepackage{tikz} % TikZ steht übrigens für "TikZ ist kein Zeichenprogramm", ein rekursives Akronym ...
\tikzset{>=latex}
\usetikzlibrary{shapes,arrows}
\usetikzlibrary{calc}
\usetikzlibrary{decorations.pathreplacing}
% Hiermit kann man ganz leicht kommutative Diagramme zeichnen (deswegen auch "cd")
\usepackage{tikz-cd}

%--Marginnote, ermöglicht es kleine Notizen an neben den eigentlichen Textkörper zu setzten
\usepackage{marginnote}
\renewcommand*{\marginfont}{\color{Honeydew4} \footnotesize }

%--Schriftarten
\usepackage{lmodern} % neuere Version der Standard-LaTeX-Schriftarten
\renewcommand{\familydefault}{\sfdefault} %Standardschriftart auf die serifenlose Schriftart setzen

%--Hyperref; aktiviert Hyperlinks in der erzeugten PDF-Datei und definiert deren Aussehen
\usepackage[colorlinks, pdfpagelabels, pdfstartview=FitH, bookmarksopen=true, bookmarksnumbered=true,linkcolor=black,urlcolor=SkyBlue2, plainpages=false, hypertexnames=false, citecolor=black, hypertexnames=true]{hyperref}

%--Römische Zahlen
\newcommand{\RM}[1]{\MakeUppercase{\romannumeral #1{}}}



%-- Definitionen von weiteren Mathe-Befehlen, die dann das "richtige" Aussehen haben. Hier sind der Phantasie keine Grenzen gesetzt
\DeclareMathOperator{\id}{id} %identische Abbildung
\DeclareMathOperator{\End}{End} %Endomorphismen
\DeclareMathOperator{\rg}{rg} %Rang
\DeclareMathOperator{\diam}{diam} %Durchmesser
\DeclareMathOperator{\dist}{dist} %Distanz
\DeclareMathOperator{\grad}{grad} %Gradient
\DeclareMathOperator{\rot}{rot} %Rotation
\DeclareMathOperator{\hess}{Hess} %Hesse-Matrix
\DeclareMathOperator{\supp}{supp}
\DeclareMathOperator{\aut}{Aut}
\DeclareMathOperator{\inn}{Inn}
\DeclareMathOperator{\sym}{Sym}
\DeclareMathOperator{\syl}{Syl}

%--Skalarprodukt (cooler Befehl, den ich im Internet gefunden habe; benutzt TeX-Befehle)
\makeatletter
\newcommand{\sprod}[2]{\ensuremath{%
  \setbox0=\hbox{\ensuremath{#2}}
  \dimen@\ht0
  \advance\dimen@ by \dp0
  \left\langle \left.#1 \,\rule[-\dp0]{0pt}{\dimen@}\right|#2\right\rangle}}
\makeatother

%--Norm (auch aus dem Internet, wird auch auf der Beispielseite verwandt)
\newcommand{\norm}[2][\relax]{
\ifx#1\relax \ensuremath{\left\Vert#2\right\Vert}
\else \ensuremath{\left\Vert#2\right\Vert_{#1}}
\fi}


%--selbstgeschriebenen Befehle
%--Betrag
\newcommand{\abs}[1]{\ensuremath{\left\vert#1\right\vert}}

%--Umklammern mit passender Größe der Klammern
\newcommand{\enbrace}[1]{\ensuremath{\left( #1\right)}}

%--Mengen
\newcommand{\penbrace}[1]{\ensuremath{\left\{#1\right\}}}

%--Differential
\newcommand{\diff}[2]{\ensuremath{\frac{\partial #1}{\partial #2} }}

\newcommand{\zz}{$\mathrm{Z\kern-.3em\raise-0.5ex\hbox{Z}}$} % zu zeigen ZZ aus dem inet
\setlength{\parindent}{0pt}%absatz nicht einrücken
\newcommand{\lh}[1]{\langle #1 \rangle} %lineare Hülle
\newcommand{\nt}{\trianglelefteqslant} %normalteiler
\newcommand{\pfs}{\mathds{P}-\text{f.s.}} %P-f.s. konvergenz
\newcommand{\dint}{\mathrm{d}} % d des integrals

\newcommand{\xfrac}[2]{%
	\mbox{\raisebox{-0.4ex}{\ensuremath{\displaystyle #1}\hspace{0.2ex}}%
		{\raisebox{-0.1ex}{\big \backslash}}%
		\raisebox{0.6ex}{\ensuremath{\displaystyle #2}}%
	}%
}
\newcommand{\Pw}{\mathds{P}}
\newcommand{\E}{\mathds{E}}
\newcommand{\R}{\mathds{R}}
\newcommand{\N}{\mathds{N}}
\newcommand{\Z}{\mathds{Z}}


\newcommand{\sect}[1]{\section*{#1}\addcontentsline{toc}{section}{#1}}
\newcommand{\ssect}[1]{\subsection*{#1}\addcontentsline{toc}{subsection}{#1}}

\newcommand{\vorlesung}{Übungen Einführung in die Algebra}
\newcommand{\subt}{Aufarbeitung der Übungsaufgaben}
\newcommand{\Prof}{Prof. Dr. Kramer}

\input{extra_files/headings.tex}



\begin{document}
\maketitle
\thispagestyle{empty}
\newpage
	
\thispagestyle{empty}
\vspace*{\fill}
\begin{center}
	Hierbei handelt es sich um eine \subt von \textbf{\Prof}, WWU Münster, aus der Vorlesung \textbf{\vorlesung} im Wintersemester 2014/15.\\
	\vspace{2cm}
	Für Fehler in der Aufarbeitung wird keine Haftung übernommen. Hinweise auf Fehler sind gerne gesehen, hierfür kann man mich in der Uni ansprechen oder alternativ eine e-Mail an: \textit{tobias.wedemeier@gmx.de}\\
	Auch ist eine Mitarbeit über Github möglich.\\
	\vspace{2cm}
	Die Beweise sind größtenteils aus den Musterlösungen oder vom Autor selber und sind verkürzt oder vereinfacht, für die Korrektheit wird keine Haftung übernommen. Sie sind nur zum Verständnis gedacht.
\end{center}
\vspace*{\fill}
\newpage
	
\pagenumbering{Roman}
	
\tableofcontents
\cleardoubleoddemptypage %sorgt dafür, dass alles folgende erst auf der nächsten freien "rechten" Seite steht
	
\pagenumbering{arabic}
\setcounter{page}{1}


\sect{Zettel 1}
\label{sub:zettel_1alg}
\ssect{Aufgabe 1.2}
Sei $G$ eine Gruppe. $A,B$ Untergruppen von $G$.\\
\zz: Wenn $A\cup B$ eine Untergruppe ist, dann gilt: $A\subseteq B$ oder $b\subseteq A$.\\

\bet{Beweis:}\\
Annahme: $A\not\subseteq B$. Also ex. ein $a\in A\backslash B$ und $b\in B$ beliebig. Betrachte $ab\in A\cup B$, da $AB$ Untergruppe. Also ist $ab\in A$ oder $ab\in B$.\\
Sei $ab\in B$ und $b^{-1}\in B$, da $B$ Untergruppe, folgt, dass $abb^{-1}=a\in B$. $\lightning$ zur Annahme.\\
Also $ab\in A$ und $a^{-1}\in A$, da $A$ Untergruppe, folgt, dass $a^{-1}ab=b\in A$. Da $b$ beliebig war, folgt $B\subseteq A$.
\hfill $\square$

\ssect{Aufgabe 1.4}
Gruppe $G$. $A,B$ Untergruppen.Wir definieren $AB:=\{ab~|~a\in A,b \in B \}$.\\
\begin{enumerate}[(i)]
	\item Die Menge $AB$ ist im allgemeinen keine Untergruppe.
	\item Wenn weiter gilt $AB=BA$, dann ist $AB$ eine Untergruppe.
\end{enumerate}
Beweise klar! (\checkmark)
% sub end

\newpage

\sect{Zettel 2}
\label{sub:zettel_2lga}
\ssect{Aufgabe 2.1}
Eine Gruppe $G$ hat \Index{Exponent} $k$, wenn für jedes Gruppenelement $g\in G$ gilt: $g^k=e$.\\
\zz Gruppen mit Exponent 2 sind abelsch.\\

\bet{Beweis:}\\
Aus $g^2=e$ folgt $g=g^{-1}~\forall g\in G$. $a,b\in G$ beliebig\\
\[ ab=(ab)^{-1}\stackrel{\text{G Gruppe}}{=} b^{-1}a^{-1}=ba \]
\hfill $\square$\\
Anmerkung: Gruppen mit Exponenten 3 sind im allgemeinen nicht abelsch.

\ssect{Aufgabe 2.3}
Menge $X$ und $Sym(X)$. Der \Index{Träger einer Permutation} $\sigma \in Sym(X)$ ist definiert wie folgt: $\supp(\sigma):=\{x\in X~|~\sigma(x)\not=x \}$.
\begin{enumerate}[(i)]
	\item Wenn $\supp(\rho)\cap \supp(\sigma)=\emptyset$ für $\rho, \sigma \in Sym(X)$ gilt, dann folgt $\rho \circ \sigma=\sigma \circ \rho$.
	\item Wenn $\supp(\rho)\cap \supp(\sigma)=\emptyset$ und $\rho \circ \sigma=\id$ für $\sigma,\rho\in Sym(X)$ gilt, dann folgt $\rho=\sigma=\id$.
\end{enumerate}

\bet{Beweis:}\\
\begin{enumerate}[(i)]
	\item Es gilt: $\rho\circ\sigma=\left\{\begin{array}{cl} x, & \text{wenn } x\notin \supp(\rho),~\supp(\sigma)\\ \rho(x), & \text{wenn } x\in \supp(\rho)\\ \sigma(x), & \text{wenn } x\in \supp(\sigma) \end{array}\right.$
	oder 
	$\sigma\circ\rho=\left\{\begin{array}{cl} x, & \text{wenn } x\notin \supp(\rho),~\supp(\sigma)\\ \rho(x), & \text{wenn } x\in \supp(\rho)\\ \sigma(x), & \text{wenn } x\in \supp(\sigma) \end{array}\right.$\\
	
	Da $\supp(\rho)\cap \supp(\sigma)=\emptyset$ gilt und somit $x$ nicht von beiden Permutationen verändert wird. Da Permutationen bijektiv nach Definition sind, ist dies wohldefiniert.
	\item Nach (i) gilt, dass $\rho\circ\sigma=\left\{\begin{array}{cl} x, & \text{wenn } x\notin \supp(\rho),~\supp(\sigma)\\ \rho(x), & \text{wenn } x\in \supp(\rho)\\ \sigma(x), & \text{wenn } x\in \supp(\sigma) \end{array}\right.$ gilt.\\
	
	Also muss $\rho(x)=x$ gelten, da $\rho\circ\sigma$ gilt, analog $\sigma(x)=x$. Also folgt $\rho=\sigma=\id$.
\end{enumerate}
%sub end

\newpage

\sect{Zettel 3}
\label{sub:zettel_3alg}

\ssect{Aufgabe 3.1}
Gegeben seinen Gruppen $G$ und $H$.\\
\begin{enumerate}[(i)]
	\item $G$ abelsch $\Leftrightarrow$ $f:G\to G,~ f(g)=g^{-1}$ ist Gruppenhomomorphismus.
	\item Gruppenhomomorphismus $\phi:G\to H$. Weiter sei $g\in G$ mit $o(g)<\infty$.\\
	Die Ordnung von $\phi(g)$ ist eine Teiler der Ordnung von $g$.
\end{enumerate}

\bet{Beweis:}\\
\uline{zu (i):} $\Rightarrow$: \[f(gh)=(gh)^{-1}=h^{-1}g^{-1}=g^{-1}h^{-1}=f(g)f(h) \]
Also $f$ ein Gruppenhomomorphismus.\\
$\Leftarrow$:Seien $g,h\in G$ bel. Da $f$ ein Gruppenhomomorphismus ist gilt \[gh=f((gh)^{-1})=f(h^{-1}g^{-1})=f(h^{-1})f(g^{-1})=hg \]
Also ist $G$ abelsch.
\hfill $\square$

\uline{zu (ii):} Da $o(g)<\infty$, ex. ein $n\in\N$ mit $o(g)=n$. Wir haben also \[e_H=\phi(e_G)=\phi(g^n)=\phi(g)^n \]
Also ist $o(\phi(g))<\infty$, folglich ex. ein $m\in \N$ mit $o(\phi(g))=m$. Angenommen $m\nmid n$. Dann ex. $k,l\in \N,~1\le k<m$, so dass gilt: $n=l\cdot m+k$. Dann folgt \[e_H=\phi(g^n)=\phi(g^{l\cdot m})\phi(g^k)=(\phi(g)^m)^l\phi(g)^k=\phi(g)^k\not=e_H~\lightning \]
\hfill $\square$

\ssect{Aufgabe 3.2}
Sei $G$ eine Gruppe und $H\subseteq G$ eine Untergruppe.
\begin{enumerate}[(i)]
	\item Die Abbildung \[\psi:\{gH~|~g\in G \}\to \{Hg~|~g\in G \},~gH\mapsto Hg^{-1} \]
	ist eine wohldefinierte Abbildung, die die Linksnebenklassen bijektiv auf die Rechtsnebenklassen abbildet.
	\item Es gelte $[G:H]=2$.\\
	Die Untergruppe $H$ ist ein Normalteiler in $G$.
\end{enumerate}

\bet{Beweis:}\\
\uline{zu (i):} Wohldefiniertheit:\\
Gegeben sei $gH=g'H$. \zz: $Hg^{-1}=Hg'^{-1}$.\\
\[ (gH)^{-1}=\{(gh)^{-1}~|~h\in H \}= \{h^{-1}g^{-1}~|~h\in H \}= \{hg^{-1}~|~h\in H \}=Hg^{-1} \]
Da $gH=g'H$ gilt, folgt $(gH)^{-1}=(g'H)^{-1}$ und mit der obigen Gleichung folgt $Hg^{-1}=Hg'^{-1}$. Also ist $\psi$ wohldefiniert (oder mit $gH=g'H$ folgt $g\in g'H$, also $\exists h\in H$ mit $g=g'h$).\\
$\psi$ injektiv: Sei $\psi(gH)=\psi(g'H)$. Also
\begin{equation*}
\begin{aligned}
	Hg^{-1}&=Hg'^{-1} \Rightarrow g^{-1}\in Hg'^{-1} \Rightarrow \exists h\in H:~g^{-1}=hg'^{-1}\\ &\Rightarrow g=g'h^{-1} \Rightarrow g\in g'H \Rightarrow gH=g'H
\end{aligned}
\end{equation*}
Also ist $\psi$ injektiv.\\
$\psi$ surjektiv: Sei $Hg\in \{Hg~|~g\in G\}$ beliebig. Dann ist $g^{-1}H$ ein Urbild von $Hg$ unter $\psi$.\\
Also ist $\psi$ bijektiv.
\hfill $\square$
\newpage

\uline{zu (ii):}












\cleardoubleoddemptypage
\pagenumbering{Alph}
\setcounter{page}{1}




\printindex
\listoffigures
\end{document}