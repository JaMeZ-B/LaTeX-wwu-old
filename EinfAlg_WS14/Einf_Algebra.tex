\documentclass[a4paper, pagesize=pdftex, pdftex, twoside, headsepline, index=totoc,toc=listof, fontsize=10pt, cleardoublepage=empty, headinclude, DIV=13, BCOR=13mm]{scrartcl}

\usepackage[ngerman]{babel}
\usepackage{scrtime} % Bestandteil von KOMA-Skript, ermoeglicht Zugriff auf Uhrzeit des Kompilierens 
\usepackage{scrpage2} % ermöglicht Bearbeiten von Kopf- und Fusszeilen (wie fancyhdr, nur optimiert auf KOMA-Skript, leich andere Syntax)
\usepackage[utf8]{inputenc} % Gibt an in welcher Textcodierung der Code verstandne werden soll
\usepackage{etex} % sehr technisch, ermöglicht LaTeX mehr Speicher zu belegen
\usepackage[T1]{fontenc} % auch sehr technisch; ist wichtig, um die Schriftarten richtig zu behandeln
\usepackage{textcomp} %verhindert ein paar Fehler bei den Fonts
\usepackage{amsmath} % Packet der American Mathematical Society, das viele Mathematik-Umgebungen und -Befehle definiert
\usepackage{amssymb} %zusätzliche Symbole
\usepackage{latexsym} % nochmal zusätzliche Symbole
\usepackage{stmaryrd} % nochmal mehr zusätzliche Symbole, u.a. Blitz für Widerspruchsbeweise ;)
\usepackage{nicefrac} % schräge Brüche, benutzte ich für Quotienvektorräume
\usepackage{paralist} % redefiniert alle Listenbefehle, sodass diese einen optionalen Parameter haben, der die Nummerierung angibt
\usepackage{dsfont} % Schriftart für N,Z,Q,R die ich momentan benutze (mittels \mathds{R} z.B)
\usepackage[pdftex]{graphicx} % Packet, dass das Einbinden von Grafiken aus Dateien ermöglicht
\usepackage{makeidx}% ermöglicht das automatische Anlegen eines Index 
\usepackage{extarrows}
\usepackage{bbold}
\usepackage{mathtools}
%\usepackage{MnSymbol}

\flushbottom
\usepackage[normalem]{ulem}
\setlength{\ULdepth}{1.8pt}

%--Indexverarbeitung
\newcommand{\bet}[1]{\uline{\textbf{#1}}} %Betonung von Text
\newcommand{\Index}[1]{\uline{\textbf{#1}}\index{#1}} % Befehl, der gleichzeitg das Argument hervorhebt und in den Index mitaufnimmt
\makeindex % startet das automatische Sammeln der Index-Einträge
% Ein kleiner Text am Anfang des Index
\setindexpreamble{{\noindent \itshape Die \emph{Seitenzahlen} sind mit Hyperlinks zu den entsprechenden Seiten versehen, also anklickbar!} \par \bigskip}
\renewcommand{\indexpagestyle}{scrheadings} % Seitenstil für den Index festlegen

%--Farbdefinitionen
\usepackage[usenames, table, x11names]{xcolor} %usenames und x11names, aktivieren viele Farben; siehe Dokumentation von xcolor
% Es lassen sich natürlich auch eigene Farben definieren (hier nur Graustufen)
\definecolor{dark_gray}{gray}{0.45}
\definecolor{light_gray}{gray}{0.7}

%--Zum Zeichnen (ich habe es jetzt mal mit aufgenommen, aber es ist eigentlich nochmal ein ganz anderes Thema, sodass ich da jetzt nicht viel zu sagen werde)
\usepackage{tikz} % TikZ steht übrigens für "TikZ ist kein Zeichenprogramm", ein rekursives Akronym ...
\tikzset{>=latex}
\usetikzlibrary{shapes,arrows}
\usetikzlibrary{calc}
\usetikzlibrary{decorations.pathreplacing}
% Hiermit kann man ganz leicht kommutative Diagramme zeichnen (deswegen auch "cd")
\usepackage{tikz-cd}

%--Marginnote, ermöglicht es kleine Notizen an neben den eigentlichen Textkörper zu setzten
\usepackage{marginnote}
\renewcommand*{\marginfont}{\color{Honeydew4} \footnotesize }

%--Schriftarten
\usepackage{lmodern} % neuere Version der Standard-LaTeX-Schriftarten
\renewcommand{\familydefault}{\sfdefault} %Standardschriftart auf die serifenlose Schriftart setzen

%--Hyperref; aktiviert Hyperlinks in der erzeugten PDF-Datei und definiert deren Aussehen
\usepackage[colorlinks, pdfpagelabels, pdfstartview=FitH, bookmarksopen=true, bookmarksnumbered=true,linkcolor=black,urlcolor=SkyBlue2, plainpages=false, hypertexnames=false, citecolor=black, hypertexnames=true]{hyperref}

%--Römische Zahlen
\newcommand{\RM}[1]{\MakeUppercase{\romannumeral #1{}}}



%-- Definitionen von weiteren Mathe-Befehlen, die dann das "richtige" Aussehen haben. Hier sind der Phantasie keine Grenzen gesetzt
\DeclareMathOperator{\id}{id} %identische Abbildung
\DeclareMathOperator{\End}{End} %Endomorphismen
\DeclareMathOperator{\rg}{rg} %Rang
\DeclareMathOperator{\diam}{diam} %Durchmesser
\DeclareMathOperator{\dist}{dist} %Distanz
\DeclareMathOperator{\grad}{grad} %Gradient
\DeclareMathOperator{\rot}{rot} %Rotation
\DeclareMathOperator{\hess}{Hess} %Hesse-Matrix
\DeclareMathOperator{\supp}{supp}
\DeclareMathOperator{\aut}{Aut}
\DeclareMathOperator{\inn}{Inn}
\DeclareMathOperator{\sym}{Sym}
\DeclareMathOperator{\syl}{Syl}

%--Skalarprodukt (cooler Befehl, den ich im Internet gefunden habe; benutzt TeX-Befehle)
\makeatletter
\newcommand{\sprod}[2]{\ensuremath{%
  \setbox0=\hbox{\ensuremath{#2}}
  \dimen@\ht0
  \advance\dimen@ by \dp0
  \left\langle \left.#1 \,\rule[-\dp0]{0pt}{\dimen@}\right|#2\right\rangle}}
\makeatother

%--Norm (auch aus dem Internet, wird auch auf der Beispielseite verwandt)
\newcommand{\norm}[2][\relax]{
\ifx#1\relax \ensuremath{\left\Vert#2\right\Vert}
\else \ensuremath{\left\Vert#2\right\Vert_{#1}}
\fi}


%--selbstgeschriebenen Befehle
%--Betrag
\newcommand{\abs}[1]{\ensuremath{\left\vert#1\right\vert}}

%--Umklammern mit passender Größe der Klammern
\newcommand{\enbrace}[1]{\ensuremath{\left( #1\right)}}

%--Mengen
\newcommand{\penbrace}[1]{\ensuremath{\left\{#1\right\}}}

%--Differential
\newcommand{\diff}[2]{\ensuremath{\frac{\partial #1}{\partial #2} }}

\newcommand{\zz}{$\mathrm{Z\kern-.3em\raise-0.5ex\hbox{Z}}$} % zu zeigen ZZ aus dem inet
\setlength{\parindent}{0pt}%absatz nicht einrücken
\newcommand{\lh}[1]{\langle #1 \rangle} %lineare Hülle
\newcommand{\nt}{\trianglelefteqslant} %normalteiler
\newcommand{\pfs}{\mathds{P}-\text{f.s.}} %P-f.s. konvergenz
\newcommand{\dint}{\mathrm{d}} % d des integrals

\newcommand{\xfrac}[2]{%
	\mbox{\raisebox{-0.4ex}{\ensuremath{\displaystyle #1}\hspace{0.2ex}}%
		{\raisebox{-0.1ex}{\big \backslash}}%
		\raisebox{0.6ex}{\ensuremath{\displaystyle #2}}%
	}%
}
\newcommand{\Pw}{\mathds{P}}
\newcommand{\E}{\mathds{E}}
\newcommand{\R}{\mathds{R}}
\newcommand{\N}{\mathds{N}}
\newcommand{\Z}{\mathds{Z}}


\newcommand{\sect}[1]{\section*{#1}\addcontentsline{toc}{section}{#1}}
\newcommand{\ssect}[1]{\subsection*{#1}\addcontentsline{toc}{subsection}{#1}}

\newcommand{\vorlesung}{Einführung in die Algebra}
\newcommand{\Prof}{Prof. Dr. Kramer}
\newcommand{\subt}{Aufarbeitung der Vorlesungsnotizen}

\input{extra_files/headings.tex}


\begin{document}
\maketitle
\thispagestyle{empty}
\newpage

\thispagestyle{empty}
\vspace*{\fill}
\begin{center}
	Hierbei handelt es sich um eine \subt von \textbf{\Prof}, WWU Münster, aus der Vorlesung \textbf{\vorlesung} im Wintersemester 2014/15. Dies ist kein Skript der Vorlesung und keine eigene Arbeit des Autors.\\
	\vspace{2cm}
	Für Fehler in der Aufarbeitung wird keine Haftung übernommen. Hinweise auf Fehler sind gerne gesehen, hierfür kann man mich in der Uni ansprechen oder alternativ eine e-Mail an: \textit{tobias.wedemeier@gmx.de}\\
	Auch ist eine Mitarbeit über Github möglich.\\
	\vspace{2cm}
	Wenn Teile aus der Vorlesung selber fehlen, können diese gerne an meine e-Mail versandt werden. Ich werde diese dann einarbeiten.\\
	\vspace{2cm}
	Eine Aufarbeitung der Übungen ist ebenfalls existent, diese sind auch in der Dropbox oder bei Github zu finden.
\end{center}
\vspace*{\fill}
\newpage

\pagenumbering{Roman}

\tableofcontents
\cleardoubleoddemptypage %sorgt dafür, dass alles folgende erst auf der nächsten freien "rechten" Seite steht

\pagenumbering{arabic}
\setcounter{page}{1}

\section{Elementare Gruppentheorie}
\label{sec:elementare_gruppentheorie}

\bet{Erinnerung:} eine \Index{Verknüpfung} auf einer nicht leeren Menge $X$ ist eine Abbildung
\begin{equation*}
\begin{aligned}
X\times X \to X , (x,y) \mapsto m(x,y).
\end{aligned}
\end{equation*}
Häufig schreibt man $m(x,y)= x\cdot y$ oder $ m(x,y) = x + y$, je nach Kontext. Die Schreibweise $m(x,y)=x+y$ wird eigentlich nur für kommutative Verknüpfungen benutzt, d.h. wenn $\forall x,y\in X$ gilt $m(x,y)=m(y,x)$.

\subsection{Definition Gruppe}
\label{sub: gef_gruppe}
Eine \Index{Gruppe} $(G,\cdot )$ besteht aus einer Verknüpfung $\cdot $ auf einer nicht leeren Menge $G$, mit folgenden Eigenschaften:
\begin{enumerate}[(G1)]
	\item Die Verknüpfung ist \uline{assoziativ}, d.h. $(x\cdot y)\cdot z = x \cdot (y \cdot z)$ gilt $\forall x,y,z \in G$.\\ 
	(Folglich darf man Klammern weglassen.)
	\item Es gibt ein \uline{neutrales Element} $e \in G$, d.h. es gilt $e\cdot x= x\cdot e= x \forall x\in G$
	\item Zu jedem $x\in G$ gibt es ein \uline{Inverses} $y \in G$, d.h. $xy=e=yx.$\\
	man schreibt dann auch $y=x^{-1}$ für das Inverse zu x.
\end{enumerate}
Fordert man von der Verknüpfung nur (G1) und (G2), so spricht man von einer Halbgruppe mit Eins oder einem \Index{Monoid}. Fordert man nur (G1), so spricht man von einer \Index{Halbgruppe}.\\
% sub end

\subsection{Beispiel 1}
\label{sub:beispiel_1}
\begin{itemize}
	\item $(\mathds{Z}, +), (\mathds{Q}, +)$ sind kommutative Gruppen.
	\item $(\mathds{Z},\cdot), (\mathds{N},\cdot), (\mathds{N}, +)$ sind Monoide.
\end{itemize}
%sub end

\subsection{Beobachtungen}
\label{sub: beobachtungen}
\begin{enumerate}[a)]
	\item Das Neutraleelement (einer Verknüpfung) ist eindeutig bestimmt: sind $e,e'$ beides Neutralelemente, so folgt: $e=ee'=e'$
	\item Das Inverse zu $x$ ist eindeutig bestimmt:\\
	$xy=e=xy'=y'x \Rightarrow y'=y'e=y'xy=ey=y$
\end{enumerate}
%sub end

\subsection{Lemma 1 (Sparsame Definition von Gruppen)}
\label{sub:lemma_1}
Sei $G \times G \to G$ eine assoziative Verknüpfung. Dann ist $G$ schon eine Gruppe, wenn gilt:
\begin{enumerate}[(i)]
	\item es gibt $e \in G$ so, dass $ex=x$ $\forall x \in G$ gilt.
	\item zu jedem $x\in G$ gibt es ein $y \in G$ mit $ yx=e$
\end{enumerate}
\bet{Beweis}\\
Sei $yx=e$, es folgt $yxy=y$. Wähle $z$ mit $zy=e$, es folgt $z\underbracket{yx}_{=e}y=zy=e \Rightarrow xy=e$\\
Weiter gilt $xe=xyx=ex=x$.
\hfill $\square$
%sub end

\subsection{Beispiel 2}
\label{sub: beispiel_2}
Sei $X$ eine nicht leere Menge, sei $X^X=\{f : X \to X\}$ die Menge aller Abbildungen von $X$ nach $X$. Als Verknüpfung auf $X$ nehmen wir die Komposition von Abbildungen. Dann gilt wegen $f=\id_X \circ f= f \circ \id_X$, dass $\id_X$ ein Neutralelement ist.\\
Damit haben wir ein Monoid $(X_X, \circ )$.\\
Sei $\sym(X)=\{f:X\to X~|~f$ bijektiv$\}$. Zu jedem $f\in \sym(X)$ gibt es also eine Umkehrabbildung $g:X\to X$ mit $f \circ g=g\circ f=\id_X$. Folglich ist $(\sym(X), \circ)$ eine Gruppe, die \bet{Symmetrische Gruppe}\index{Gruppe!symmetrische !}. Wenn $X$ endlich ist mit $n$ Elementen, so gibt es genau $n!=n(n-1)(n-2)\dots \cdot 2\cdot 1$ Permutationen, also hat $\sym(X)$ dann genau $n!$ Elemente.\\
Für $X=\{1,2,3,\dots,n\}$ schreibt man auch $\sym(X)=\sym(n)\big( =S_n \big)$.
%sub end

\subsection{Definition zentralisieren}
\label{sub:def_zentralisieren}
Sei $G \times G \to G$ eine Verknüpfung. Wir sagen, $x,y \in G$ vertauschen oder kommutieren oder $x$ \Index{zentralisiert} $y$, wenn gilt $xy=yx$.\\
Eine Gruppe, in der alle Elemente vertauschen heißt kommutativ oder \Index{abelsch}. 
%sub end

\subsection{Beispiel 3}
\label{sub:beispiel_3}
\begin{enumerate}[(a)]
	\item $(\mathds{Z}, +), (\mathds{Q}, +), (\mathds{Q}^*,\cdot)$ sind abelsche Gruppen.
	\item $K$ Körper, $G=Gl_2(K)=\{X \in K^{2\times 2}~ |~ \det(X)\not= 0 \}$ Gruppe der invertierbaren $2 \times 2$ Matrizen.\\
\[
	\begin{pmatrix}	1 & 1\\ 0 & 1 \end{pmatrix}
	\begin{pmatrix} 0 & 1\\	1 & 0 \end{pmatrix}
	=
	\begin{pmatrix}	1 & 1\\	1 & 0 \end{pmatrix}
\]
\[
	\begin{pmatrix} 0 & 1\\	1 & 0 \end{pmatrix}
	\begin{pmatrix} 1 & 1\\ 0 & 1 \end{pmatrix}
	=
	\begin{pmatrix} 0 & 1\\ 1 & 1 \end{pmatrix}
\]
	$\Rightarrow$ nicht abelsch, genauso $Gl_n(K)$ für $n\ge 2$.
	\item $\sym(2)$ ist abelsch, aber $\sym(3)$ nicht. Allgemein ist $\sym(X)$ nicht abelsch, falls $\#X \ge 3$ gilt.
\end{enumerate}
%sub end

\subsection{Definition Untergruppe}
\label{sub: def_untergruppe}
Sei $G$ eine Gruppe, sei $H\subseteq G$. Wir nennen $H$ \bet{Untergruppe} \index{Gruppe! Unter-!} von $G$, wenn gilt:
\begin{enumerate}[(UG1)]
	\item $e\in H$
	\item $x,y\in H \Rightarrow xy\in H$
	\item $x\in H \Rightarrow x^{-1} \in H$
\end{enumerate}
Offensichtlich ist eine Untergruppe dann wieder eine Gruppe, mit der von $G$ vererbten Verknüpfung.

\subsubsection*{Bsp}
\begin{enumerate}[(a)]
	\item $(\mathds{Q}, +)$. $\mathds{Z}$  ist Untergruppe, denn $0 \in \mathds{Z}, m,n \in \mathds{Z} \Rightarrow m+n\in \mathds{Z}$ und $n\in \mathds{Z} \Rightarrow -n\in \mathds{Z}$
	\item $(\mathds{Q}^*,\cdot)$. $\mathds{Z}^*$ ist keine Untergruppe, kein Inverses.
\end{enumerate}
%sub end

\subsection{Lemma 2}
\label{sub:lemma_2}
Sei $G$ eine Gruppe und sei $U$ eine nicht leere Menge von Untergruppen von $G$. Dann ist auch\\
$\bigcap U = \{g\in G~|~\forall H\in U$ gilt $g\in H\}$\\
eine Untergruppe von $G$.\\

\bet{Beweis}\\
Für alle $H\in U$ gilt $e\in H$, also $e\in \bigcap U$. Angenommen $x,y\in \bigcap U$. Dann gilt für alle $H\in U$, dass $xy\in H$ sowie $x^{-1}\in H$. Es folgt $xy\in \bigcap U$ sowie $x^{-1}\in \bigcap U$.
\hfill $\square$
%sub end

\subsection{Definition $\lh{X}$}
\label{sub:def_lhX}
Sei $G$ eine Gruppe und $X \subseteq G$ eine Teilmenge. Wir setzen:
\[\lh{X}=\bigcap\{H\subseteq G | H \text{ Untergruppe und } X \subseteq H\}\]
Ist nicht leer, da mindestens $G$ enthalten ist.
\begin{itemize}
	\item Es gilt z.B. $\lh{\emptyset}=\{e\}$, denn $\{e\}$ ist Untergruppe.
	\item Ist $H \subseteq G$ Untergruppe mit $X \subseteq H$, so folgt $X\subseteq \lh{X} \subseteq H$, insb. also $\lh{H}=H$.
\end{itemize}

\subsubsection*{Satz}
Sei $X \subseteq G$ und sei $W=\{x_1\cdot x_2,\cdots x_s | s\ge 1, x_i\in X \text{ oder } x_i^{-1}\in X ~\forall i=1,\dots,s\}$.\\
Dann gilt: $ \lh{X}=\{e\}\cup W$.\\

\bet{Beweis}\\
Wegen $X\subseteq \lh{X}$ und $e\in \lh{X}$ folgt $\{e\}\cup W\subseteq \lh{X}$. Ist $f,g\in W$, so folgt $fg\in W$ sowie $f^{-1}\in W$, also ist $H=\{e\}\cup W$ eine Untergruppe von $G$, mit $X\subseteq H$. Es folgt $\lh{X}\subseteq H=\{e\}\cup W$.
\hfill $\square$
%sub end

\subsection{Definition zyklische Gruppe}
\label{sub:def_zyklische_gruppen}
Sei $G$ eine Gruppe und sei $g\in G$. Für $n\ge 1$ setze $g^n=\underbrace{g\cdots g}_{n-mal}$ sowie $g^{-n}=\underbrace{g^{-1}\cdots g^{-1}}_{n-mal}$ und $g^0=e$.\\
Dann gilt $\forall k,l\in \mathds{Z}$, dass $g^k\cdot g^l = g^{k+l}$.\\
Sei $\lh{g}=\lh{\{g\}}\stackrel{1.10}{=}\{g^n | n\in \mathds{Z}\}$. Man nennt $\lh{g}$ die von $g$ erzeugte \bet{zyklische Gruppe}\index{Gruppe!zyklische !}. Wenn für ein $n\ge 1$ gilt $g^n=e$, so heißt $n$ ein \Index{Exponent} von $g$. Die \Index{Ordnung} von g ist der kleinste Exponent von g,
\[o(g)=\min\enbrace{\{n\ge 1 | g^n=1\}\cup \{\infty\}}\]
$o(g)=\infty$ bedeutet: $g^n\not= e~\forall n\ge 1$\\
$o(g)=1$ bedeutet: $g^n=g=e$
%sub end

\subsection{Zyklische Gruppen}
\label{sub:zyklische_gruppen}
Eine Gruppe $G$ heißt \Index{zyklisch}, wenn es ein $g\in G$ gibt mit $G=\lh{g}$. Wegen $g^kg^l=g^{k+l}=g^{l+k}=g^lg^k$ gilt: zyklische Gruppen sind abelsch.

\subsubsection*{Satz}
Sei $G=\lh{g}$ zyklisch mit $o(g)=n<\infty$. Dann gilt $\#G=n$ und $G=\{g,g^1,g^2,g^3,\dots,g^n\}$.\\

\bet{Beweis}\\
Jedes $m\in \mathds{Z}$ lässt sich schreiben als $m=kn+l$ mit $0\le l<n$ (Teilen mit Rest), also $g^m= \underbrace{g^{kn}}_{=e}.g^l=g^l$. Es folgt $G\subseteq \{g,g^2,\dots,g^n\}, g^n=g^0$.
Ist $g^k=g^l$ für $0\le k\le l<n$, so gilt $e=g^0=g^{l-k}$, also $l-k=0$ (wegen $l<n$), also $\#\{g,g^2,\dots,g^n=g^0\}=n$.
\hfill $\square$

\subsubsection*{Folgerung}
Ist $G$ endlich mit $\#G=n$ und ist $h\in G$ mit $o(h)=n$, so folgt $\lh{h}=G$. Insbesondere ist dann $G$ eine zyklische Gruppe.
%sub end

\subsection{Nebenklassen}
\label{sub:nebenklassen}
\index{Nebenklassen!Links-} \index{Nebenklassen!Rechts-}

Sei $G$ eine Gruppe und sei $H$ eine Untergruppe. Sei $a\in G$. Wir definieren:
\[aH=\{ah | h\in H\}\subseteq G\]
\[Ha=\{ha | h\in H\}\subseteq G\]
Man nennt $aH$ die \bet{Linksnebenklassen} von $a$ bzgl. $H$ (und $Ha$ die \bet{Rechtsnebenklassen}). In nicht abelschen Gruppen gilt im allgemeinen $aH\not=Ha$.

\subsubsection*{Lemma}
Sei $H\subseteq G$ Untergruppe der Gruppe $G$ und $a,b\in G$.\\
Dann sind äquivalent:
\begin{enumerate}[(i)]
	\item $b\in aH$
	\item $bH=aH$
	\item $bH \cap aH \not= \emptyset$
\end{enumerate}
\bet{Beweis}\\
\begin{itemize}
	\item$(i)\Rightarrow (ii):~b\in aH \Rightarrow b=ah$ für ein $h\in H \Rightarrow bH=\{ahh' | h' \in H\}\\
	\stackrel{H\text{ Untergruppe}}{=}\{ah'' | h''\in H\}=aH$
	\item$(ii) \Rightarrow (iii):$ klar
	\item$(iii) \Rightarrow (i):$ Sei $g \in bH \cap aH,~g=bh=ah' \Rightarrow b=ah'h^{-1} \in aH$, da $H$ Untergruppe
\end{itemize}
\hfill $\square$

\subsubsection*{Folgerung}
Jedes $g\in G$ liegt in genau einer Linksnebenklasse bzgl. $H$, nämlich $g \in gH$.
Entsprechendes gilt natürlich für Rechtsnebenklassen. Man setzt:\\
$\nicefrac{G}{H}=\{gH~|~g \in G \}$ Menge der Linksnebenklasse, Rechtsnebenklassen analog.

\subsubsection*{Lemma}
Sei $H\subseteq G$ Untergruppe der Gruppe $G$, sei $a \in G$.\\
Dann ist die Abbildung $H \to gH, h \mapsto gH$ bijektiv.\\

\bet{Beweis}\\
'Surjektiv' ist klar nach Definition von $gH$. Angenommen, $gh=gh' \Rightarrow h=g^{-1}gh'=h'$
\hfill $\square$
%sub end

\subsection{Satz 1, Satz von Lagrange}
\label{sub:satz_von_lagrange}
\index{Satz von Lagrange}
Sei $G$ eine Gruppe und $H\subseteq G$ eine Untergruppe. Wenn zwei der drei Mengen $G,H,\nicefrac{G}{H}$ endlich sind, dann ist die dritte ebenfalls endlich und es gilt:\\
\[\#G=\#H \cdot \#\nicefrac{G}{H} \]
Insbesondere ist dann $\#H$ eine \Index{Teiler} von $\#G$.\\
\vfill
\bet{Beweis}\\
Wenn $G$ endlich ist, dann sind auch $H$ und $\nicefrac{G}{H}$ endlich.\\
Angenommen, $\nicefrac{G}{H}$ und $H$ sind endlich. Dann ist auch $G= \bigcup \nicefrac{G}{H}=\bigcup\{gH~|~gH\in \nicefrac{G}{H} \}$ endlich, da $\#gH=\#H$ nach \hyperref[sub:nebenklassen]{1.13}.\\
Jetzt zählen wir genauer: sei $\#\nicefrac{G}{H}=m; \#H=n$ etwa $\nicefrac{G}{H}=\{g_1H,g_2H,\dots g_mH \}$.\\
$g_iH\stackrel{\hyperref[sub:nebenklassen]{1.13}}{=}n~~~~~g_iH\cap g_jH=\emptyset$ für $i\not=j$ nach \hyperref[sub:nebenklassen]{1.13}.\\
$G=g_1\cap g_2H\cap \dots \cap g_mH \Rightarrow \#G=m\cdot n$
\hfill $\square$

\subsubsection*{Bemerkung}
\begin{enumerate}[(1)]
	\item Eine entsprechende Aussage gilt für Rechtsnebenklassen.
	\item Die Abbildung $G \to G,~~g\mapsto g^{-1}$ bildet die Linksnebenklassen bijektiv auf die Rechtsnebenklassen ab:
	\[
	(gH)^{-1}=\{(gh)^{-1}~|~h \in H \} \stackrel{\text{Achtung!}}{=}\{h^{-1}g^{-1}~|~h \in H \}=\{hg^{-1}~|~h\in H \}=Hg^{-1}~~~~~~~~\text{(ÜA)}
	\]
\end{enumerate}

\subsubsection*{Korollar A (Lagrange)}
Sei $G$ eine endliche Gruppe und sei $g\in G$. Dann teilt $o(g)$ die Zahl $\#G$.\\

\bet{Beweis}\\
Da $G$ endlich ist, folgt $o(g)<\infty$. Nach dem \hyperref[sub:satz_von_lagrange]{Satz von Lagrange} ist $\#\lh{g}=o(g)$ ein Teiler von $\#G$.
\hfill $\square$

\subsubsection*{Korollar B}
Sei $G$ eine endliche Gruppe, sei $p$ eine \Index{Primzahl}  (d.h. die einzigen Teiler von $p$ sind 1 und $p$) und $p>1$. Wenn gilt $\#G=p$, dann ist $G$ zyklisch. Für jedes $g\in G \textbackslash\{e\}$ gilt $\lh{g}=G$.\\

\bet{Beweis}\\
Sei $g \in G\textbackslash\{e\}$. Dann ist $o(g)>1$ und $o(g)$ teilt $p$. Es folgt $o(g)=p$, also $G=\lh{g}$ vgl. \hyperref[sub:zyklische_gruppen]{1.12}.
\hfill $\square$

Für endliche Gruppen sind Teilbarkeitseigenschaften wichtig, wie wir sehen werden.\\
Die Zahl $\#\nicefrac{G}{H}:=[G:H]$ nennt man auch den \Index{Index von H in G}.

\subsubsection*{Wichtige Rechenregeln in Gruppen}
\begin{enumerate}[(a)]
	\item Man darf \uline{kürzen}
	\begin{equation*}
	\begin{aligned}
		ax &= ay \Rightarrow x=y\\
		xa &= ya \Rightarrow x=y
	\end{aligned}
	\end{equation*}
	(multipliziere beide Seiten von links/rechts mit $a^{-1}$)
	\item Es gilt $(x^{-1})^{-1}=x$   ($x^{-1}x=e=xx^{-1} \Rightarrow (x^{-1})^{-1}=x$)
	\item Beim Invertieren darf die Reihenfolge umgedreht werden:\\
	\[(ab)^{-1}=b^{-1}a^{-1}\\
	\enbrace{ab(b^{-1}a^{-1})=e=(b^{-1}a^{-1})ab \Rightarrow (ab)^{-1}=b^{-1}a^{-1}}
	\]
	(in abelschen Gruppen gilt natürlich damit $(ab)^{-1}=a^{-1}b^{-1}$ )
\end{enumerate}
% sub end
\subsection{Homomorphismen}
\label{sub:homomorphismen}
Seien $G,K$ Gruppen. Eine Abbildung $\varphi: G \to K$ heißt \bet{(Gruppen-)Homomorphismus}\index{Homomorphismus!Gruppen-}, wenn $\forall x,y \in G$ gilt
\[\varphi\underbrace{(x\cdot y)}_{\text{Verküpfung in G} } =\underbrace{\varphi(x)\varphi(y)}_{\text{Verknüpfung in K}} \]

\minisec{Bsp}
\begin{enumerate}[(a)]
	\item $id_G: G \to G$ ist Homomorphismus
	\item $H \subseteq G$ Untergruppe   $i:H \hookrightarrow G,~~~h \mapsto h$ Inklusion, ist Homomorphismus.
	\item $(G,\cdot)=(\mathds{Z},+)~~m\in \mathds{Z} ~~ \varphi:\mathds{Z} \to \mathds{Z}, x\mapsto mx$ ist Homomorphismus, denn $\phi(x+y)=m(x+y)=mx+my=\varphi(x)+\varphi(y)$
	\item $G$ Gruppe, $a \in G,~ a\not= e,~~ \lambda_a(x)=ax$.\\
	$\lambda: G \to G$ ist kein Homomorphismus, denn $\lambda_a(e)=a, \lambda(ee)=a$, aber $\lambda_a(e)\lambda_a(e)=aa\not=a$
\end{enumerate}

\subsubsection*{Lemma}
Sei $\varphi:G \to K$ ein Homomorphismus von Gruppen. Dann gilt $\varphi(e_G)=e_K$ und $\varphi(x^{-1})=\varphi(x)^{-1}~\forall x \in G$. ($e_G$ Neutralelement in $G$ und $e_K$ Neutralelement in $K$)\\
\bet{Beweis}\\
\[
	\varphi(e_G)=\varphi(e_G \cdot e_G)=\varphi(e_G) \cdot \varphi(e_G)
	\stackrel{\text{kürzen}}{\Rightarrow} e_K=\varphi(e_G)
\]
\[
	e_K=\varphi(e_G)=\varphi(x^{-1}x)=\varphi(x^{-1})\varphi(x) \Rightarrow \varphi(x)^{-1}=\varphi(x^{-1})
\]
\hfill $\square$

\uline{Achtung:} $\varphi(x)^{-1}$ ist das Inverse in $K$ von $\varphi(x)$ \uline{nicht} die Umkehrabbildung!\\

Das \Index{Bild} eines Homomorphismus $\varphi:G \to K$ ist $\varphi(G)\subseteq K$,\\
der \Index{Kern} ist $ker(\varphi)=\{x \in G~|~\varphi(x)=e_K \}\subseteq G$
%sub end

\subsection{Satz 2, Gruppenhomomorphismen}
\label{sub:satz_ghm}
Bild und Kern von Gruppenhomomorphismen sind Untergruppen.\\

\bet{Beweis}\\
Setze $H=\varphi(G)\subseteq K$. Es folgt $e_K \in H$. Für $\varphi(x),\varphi(y)\in H$ gilt $\varphi(x)\varphi(y)=\varphi(xy)\in H$ sowie $\varphi(x)^{-1}=\varphi(x^{-1}) \in H$, also ist $H$ Untergruppe. Betrachte jetzt $ker(\varphi)\subseteq G$. Es gilt $\varphi(e_G)=e_K$, also $e_G \in ker(\varphi)$. Ist $x,y \in ker(\varphi)$, so folgt 
\[\varphi(xy)=\varphi(x)\varphi(y)=e_K \cdot e_K=e_K \text{ , also } xy \in ker(\varphi)\]
\[\varphi(x^{-1})=\varphi(x)^{-1}=e_K^{-1}=e_K \text{ , also } x^{-1} \in ker(\varphi) \]
\hfill $\square$

\minisec{Bemerkung:}
\uline{Jede} Untergruppe von $H\subseteq G$ ist Bild eine geeigneten Homomorphismus (nämlich der Inklusion $H \hookrightarrow G$). Wir werden sehen, dass im allgemeinen \uline{nicht} jede Untergruppe $H\subseteq G$ Kern eines Homomorphismus ist.
% sub end

\subsection{Normalteiler}
\label{sub:normalteiler}
Sei $G$ eine Gruppe und $N\subseteq G$ eine Untergruppe. Wir nennen $N$ \Index{normal} in $G$ oder \Index{Normalteiler} in $G$, wenn eine der folgenden äquivalenten Bedingungen erfüllt ist:
\begin{enumerate}[(i)]
	\item für alle $a\in G$ gilt $aN=Na$ (Rechtsnebenklassen sind Linksnebenklassen)
	\item für alle $a\in G$ gilt $aNa^{-1}=N (aNa^{-1}=\{ana^{-1}~|~n\in N \})$
	\item für alle $a\in G$ gilt $aN\subseteq Na$
	\item für alle $a\in G$ gilt $aNa^{-1}\subseteq N$
\end{enumerate}

\bet{Beweis:}\\
(i) und (ii) sind äquivalent: multipliziere von rechts mit $a^{-1}$ bzw. $a$. Genauso sind (iii) und (iv) äquivalent.\\
Klar: (ii) $\Rightarrow$ (iv) $(\checkmark)$\\
Zeige (iv) $\Rightarrow$ (ii): Setze $b=a^{-1}$, es folgt aus (iv), dass $bNb^{-1}\subseteq N \rightsquigarrow N\subseteq b^{-1}Nb=aNa^{-1}$. Also gilt für alle $a\in G$, dass $N\subseteq aNa^{-1}$ und $aNa^{-1}\subseteq N$, damit gilt (ii)
\hfill $\square$

\subsubsection*{Lemma}
Ist $\varphi: G \to K$ ein Homomorphismus von Gruppen, dann ist $ker(\varphi)$ ein Normalteiler in $G$.\\

\bet{Beweis:}\\
Sei $N=ker(\varphi)=\{n\in G~|~\varphi(n)=e\}$, sei $a\in G$. Dann gilt 
\[
\varphi(ana^{-1})=\varphi(a)\underbrace{\varphi(n)}_{=e}\varphi(a^{-1})=\varphi(a)\varphi(a^{-1})=e
\]
also gilt $aNa^{-1}\subseteq N~~\forall a\in G$.
\hfill $\square$

\minisec{Achtung:}
\uline{Bilder} von Homomorphismen sind \uline{nicht} immer Normalteiler, nach Beispiel \hyperref[sub:homomorphismen]{1.15 (b)} ist \uline{jede} Untergruppe Bild eines Homomorphismus, aber nicht jede Untergruppe ist normal.

\minisec{Beispiel:}
$G=Sym(3)$, $g=(1,2)$ Transposition, die 1 und 2 vertauscht. $g^2=id$, $\lh{g}=\{g,id\}\subseteq \sym(3)$ ist Untergruppe, aber für $h=(2,3)$ gilt 
\[
h\lh{g}h^{-1}=\{hgh^{-1}, h~id~h^{-1} \} = \{\underbrace{(2,3)(1,2)(2,3)}_{=(3,1)}, id \} \not\subseteq \lh{g}
\]
also ist $\lh{g}$ kein Normalteiler in $\sym(3)$.\\

\bet{Schreibweise:} Ist $N\subseteq G$ ein Normalteiler, schreibt man kurz $N\trianglelefteqslant G$\\

\bet{Beachte:} Ist $G$ \uline{abelsch}, dann sind alle Untergruppen $H\subseteq G$ automatisch normal.
%sub end

\subsection{Definition Teilmengen assoziativ}
\label{sub:teilmengen}
Für Teilmengen $X,Y,Z \subseteq G$ in einer Gruppe schreibe kurz:\\
\[XY=\{xy~|~x\in X,~y\in Y\}\subseteq G \]
\[X^{-1}=\{x^{-1}~|~x\in X \}\subseteq G \]
Es gilt dann $(XY)Z=X(YZ)$, (weil die Verknüpfung assoziativ ist).
\subsubsection*{Satz}
Sei $N\trianglelefteqslant G$ Normalteiler in der Gruppe $G$. Dann ist $\nicefrac{G}{N}=\{gN~|~g\in G \}$ eine Gruppe mit der Verknüpfung $(gN)\cdot (hN)=ghN$\\
Das Neutralelement ist $eN=N$, das Inverse zu $gN$ ist $g^{-1}N$.\\
\bet{Beweis:}\\
Da $N$ Normalteiler ist, gilt für $g,h \in G$
\[gNhN=g(Nh)N\stackrel{\hyperref[sub:normalteiler]{1.17}}{=}g(hN)N=ghNN\stackrel{N\text{ Gruppe}}{=}ghN \]
Die Verknüpfung ist also einfach gegeben durch
\[gN\cdot hN=gNhN=ghN \]
und damit assoziativ nach obiger Bemerkung. Es gilt $NgN=gNN=gN=gNN$, also ist $N$ ein Neutralelement. Weiter gilt:
\[gNg^{-1}N=gg^{-1}N=N=g^{-1}gN=g^{-1}NgN \]
\hfill $\square$
%sub end

\subsection{Definition $\pi_H$}
\label{def_pi_H}
Ist $G$ eine Gruppe und $H$ eine Untergruppe, so definieren wir $\pi_H: G \to \nicefrac{G}{H}$ durch $\pi_H(g)=gH$.

\subsubsection*{Satz}
Ist $N \nt G$ ein Normalteiler, dann ist $\pi_N: G \to \nicefrac{G}{N}$ ein surjektiver Homomorphismus mit Kern $N=ker(\pi_N)$.\\
\bet{Beweis:}\\
$\pi_N$ ist nach Definition surjektiv und \[\pi_N(gh)=ghN=gNhN=\pi_N(g)\pi_N(h)\]
Weiter gilt \[\pi_N(g)=N \Longleftrightarrow gN=N \stackrel{\hyperref[sub:nebenklassen]{1.13}}{\Longleftrightarrow} g\in N\]
\hfill $\square$

\minisec{Folgerung:} 
Jeder Normalteiler ist auch ein Kern eines Homomorphismus.
%sub end

\subsection{Der Homomorphiesatz}
\label{sub:der_homomorphiesatz}
Sei $G \stackrel{\varphi}{\to} K$ ein Homomorphismus von Gruppen, sei $N \nt G$ ein Normalteiler. Wenn gilt $N\subseteq ker(\varphi)$, dann gibt es \uline{genau einen} Homomorphismus $\overline{\varphi}: \nicefrac{G}{H} \to K$ mit $\overline{\varphi} \circ \pi_H=\varphi$.

\begin{center}
	\begin{tikzcd}[column sep=small]
		G \ar{rr}{\varphi} \ar{rd}[below,left]{\pi_N} & & K\\
		& \nicefrac{G}{N} \ar{ru}[below,right]{\overline{\varphi}} &
	\end{tikzcd}
	\captionof{figure}{Homomorphiesatz}
\end{center}

\bet{Beweis:}\\
\uline{Existenz von $\overline{\varphi}$:}\\
Für $g \in G$ setze $\overline{\varphi}(gN)=\varphi(g)$. Das ist eine wohldefinierte Abbildung, denn angenommen, 
\[gN=g'N \Rightarrow g^{-1}g' \in N \subseteq ker(\varphi) \Rightarrow \varphi(g^{-1}g')=e \Rightarrow \varphi(g)=\varphi(g')\]
Es gilt damit
\[\overline{\varphi}(gNhN)=\overline{\varphi}(ghN)=\varphi(gh)=\varphi(g)\varphi(h)=\overline{\varphi}(gN)\overline{\varphi}(hN) \]
also ist $\overline{\varphi}$ ein Homomorphismus.\\

\uline{Eindeutigkeit von $\overline{\varphi}$:}\\ Sei $\psi: \nicefrac{G}{N} \to K$ ein Homomorphismus mit $\psi \circ \pi_N = \varphi$.\\
Es folgt 
\[
\psi(gN)=\psi(\pi_N(g))=\varphi(g)=\overline{\varphi}(gN) ~~~\forall g \in G
\]

\minisec{Bemerkung:}
In der Situation vom Homomorphiesatz gilt:
\begin{enumerate}[(i)]
	\item $ker(\varphi)=\pi_N^{-1}~ker(\overline{\varphi})$
	\item $ker(\overline{\varphi})=\pi_N~ker(\varphi)$
	\item $\varphi(G)=\overline{\varphi}(\nicefrac{G}{N})$
\end{enumerate}
\bet{Beweis:}\\
(iii) ist klar nach Konstruktion, $\overline{\varphi}(gN)=\varphi(g)$\\
(ii) $\overline{\varphi}(gN)=e=\varphi(g) \Leftrightarrow g \in ker(\varphi)$, also $ker(\overline{\varphi})=\pi_N(ker(\varphi))$\\
(i) $\varphi(g)=e \Rightarrow g \in ker(\varphi) \Rightarrow\pi_N(g) \in ker(\overline{\varphi}) \Rightarrow \varphi(g)=e$
\hfill $\square$

%sub end

\subsection{Definition Isomorphismus}
\label{sub:def_isomorph}
Ein Gruppenhomomorphismus $\varphi:G \to K$ heißt \bet{Mono/Epi/Isomorphismus}\index{Homomorphismen!Mono/Epi/Iso}, wenn $\varphi$ \uline{injektiv/surjektiv/bijektiv} ist.\\
(Klar: $\varphi$ Epimorphismus $\Leftrightarrow \varphi(G)=K$)\\
Für einen Mono / Epi / Isomorphismus schreibt man auch: \\
$\stackrel{\varphi}{\rightarrowtail}$  $\stackrel{\varphi}{\twoheadrightarrow}$  und  $\stackrel{\cong}{\to}$.

\subsubsection*{Lemma}
Ein Gruppenhomomorphismus $G \stackrel{\varphi}{\to}K$ ist genau dann injektiv, wenn gilt $ker(\varphi)=\{e_G\}$.\\

\bet{Beweis:}\\
Wenn $\varphi$ injektiv ist, dann ist $ker(\varphi)=\{e_G\}$ (klar). Angenommen, $ker(\varphi)=\{e_G\}$ und $a,b\in G$ mit $\varphi(a)=\varphi(b) \rightsquigarrow \varphi(a)\varphi(b)^{-1}=\varphi(ab^{-1})=e_K \Rightarrow ab^{-1}=e_G \Rightarrow a=b$
\hfill $\square$

%sub end

\subsection{Satz 3, Eigenschaften von Gruppenhomomorphismen}
\label{sub:satz_eigenschaften}
Sei $G\stackrel{\varphi}{\to}K $ ein Gruppenhomomorphismus. Dann gilt folgendes:
\begin{enumerate}[(i)]
	\item Ist $H\subseteq G$ Untergruppe, so ist $\varphi(H) \subseteq G$ Untergruppe. Wenn $H\nt G$, so gilt $\varphi(H)\nt\varphi(G)$
	\item Ist $L\subseteq K$ Untergruppe, so ist $\varphi^{-1}(L) \subseteq G$ Untergruppe. Ist $L\nt K$, so gilt $\varphi^{-1}(L) \nt G$.
\end{enumerate}

\bet{Beweis:}\\
\begin{enumerate}[(i)]
	\item Sei $a,b\in H$ und $g\in G$. Es gilt $\varphi(a)\varphi(b)=\varphi(ab)\in H,~~\varphi(a)^{-1}=\varphi(a^{-1})\in \varphi(H) $. $\varphi(e_G)=e_K \in \varphi(H) \Rightarrow \varphi(H)$ Untergruppe.\\
	Ist $H\nt G$, so folgt $\varphi(g)\varphi(H)\varphi(g)^{-1}=\varphi(gHg^{-1})\stackrel{H\nt G}{=}\varphi(H)$
	\hfill $\square$
	\item Sei $a,b \in \varphi^{-1}(L),~~g \in G$ (also $\varphi(a),\varphi(b) \in L$). Es folgt $\varphi(ab)\in L,~~\varphi(a^{-1})=\varphi(a)^{-1}\in L$ und $\varphi(e_G)=e_K \Rightarrow ab, a^{-1},e_G \in \varphi^{-1}(L) \rightsquigarrow$ Untergruppe.\\
	Angenommen, $L\nt K$.\\ Es folgt $\varphi(gag^{-1})=\varphi(g)\varphi(a)\varphi(g^{-1}) \in L$, also $g\varphi^{-1}(L)g^{-1} \subseteq \varphi^{-1}(L)$.
	\hfill $\square$
\end{enumerate}

\minisec{Beispiele}
Gruppe $(\mathds{Z},+)$, $\varphi: \mathds{Z} \to \mathds{Z} \text{ Homomorphismus, }\varphi(z)=m\cdot z$, $m \in \mathds{Z}$ fest.\\
$\varphi(\mathds{Z})=m\mathds{Z}=\{mz~|~z\in \mathds{Z}\}=(-m)\mathds{Z}$\\
z.B. $m=2~\rightsquigarrow 2\mathds{Z}=\{0,\pm 2,\pm 4,\pm 6,\dots\}$ gerade Zahlen\\
$ker(\varphi)=\left\{\begin{array}{cl} \{0\}, & \text{wenn }m\not=0\\ \mathds{Z}, & \text{wenn } m=0. \end{array}\right.$
$\varphi$ surjektiv $\Leftrightarrow~~m=\pm 1$\\
$\varphi$ injektiv $\Leftrightarrow~~m\not=0$\\

Angenommen, $m>0$, $a,b \in \mathds{Z}$\\
$a+m\mathds{Z}=b+m\mathds{Z} \text{ Nebenklassen }\stackrel{\hyperref[sub:nebenklassen]{1.13}}{\Leftrightarrow} a\in b+m\mathds{Z} \Leftrightarrow a-b \in m\mathds{Z}$\\
Folglich $\nicefrac{\mathds{Z}}{m\mathds{Z}}=\{m\mathds{Z},1+m\mathds{Z},2+m\mathds{Z},\dots,(m-1)+m\mathds{Z} \}$ insbesondere $\#\nicefrac{\mathds{Z}}{m\mathds{Z}}=m$.\\
Schreibe $\overline{k}=k+m\mathds{Z}$ \Index{Kongruenzklasse} von $k$ \Index{modulo} $m$.\\
$\nicefrac{\mathds{Z}}{m\mathds{Z}}=\{\overline{0},\overline{1},\dots,\overline{m-1} \}$ wird erzeugt von $\overline{1} \rightsquigarrow \nicefrac{\mathds{Z}}{m\mathds{Z}}=\lh{\overline{1}}$ \uline{zyklische Gruppe der Ordnung m}. $o(\overline{1})=m$. Später mehr dazu.
%sub end

\subsection{Die Isomorphiesätze}
\label{sub:isomorphiesaetze}
\subsubsection*{Lemma}
Sei $G$ eine Gruppe, seien $H,N \subseteq G$ Untergruppen. Wenn $N\nt G$ gilt, dann ist $ HN=NH \subseteq G$ eine Untergruppe.\\

\bet{Beweis:}\\
Es gilt $e=e\cdot e\in N\cdot H$. Weiter gilt für $h_1,h_2 \in H,~~n_1,n_2 \in N$, dass
\[h_1n_1h_2n_2=\underbracket[1pt][4pt]{h_1h_2}_{\in H} \underbracket[1pt]{h_2^{-1}n_1h_2}_{\in N}n_2 \in HN \]
\[(h_1n_1)^{-1}=n_1^{-1}h_1^{-1}=h_1^{-1}\underbracket[.7pt]{h_1n_1^{-1}h_1^{-1}}_{\in N} \in HN \]
\[(HN)^{-1}=N^{-1}H^{-1}=NH \subseteq HN \text{ genauso } HN \subseteq NH \]
\hfill $\square$

\subsubsection*{Satz}
Sei $G\stackrel{\varphi}{\to}K$ ein Epimorphismus von Gruppen. Sei $N=ker(\varphi)$. Dann ist die Abbildung $\overline{\varphi}:\nicefrac{G}{N}\to K$ aus dem \hyperref[sub:der_homomorphiesatz]{Homomorphisatz 1.20} ein Isomorphismus.\\

\bet{Beweis:}\\
$\overline{\varphi}(\nicefrac{G}{N})=\varphi(G)$ und $ker(\overline{\varphi})=\{N\}$ nach dem Beweis von \hyperref[sub:der_homomorphiesatz]{1.20}. Den Isomorphismus $\overline{\varphi}:\nicefrac{G}{ker(\varphi)} \stackrel{\cong}{\to}K$ nennt man \Index{kanonisch} oder \Index{natürlich}.

\subsubsection*{Theorem: 1. Isomorphiesatz}
Sei $G$ eine Gruppe, seien $H,N\subseteq G$ Untergruppen mit $N\nt G$. Dann gilt $H\cap N\nt H$, $N \nt NH$ und die Abbildung
\begin{equation*}
\begin{aligned}
	\nicefrac{H}{H\cap N} &\to \nicefrac{NH}{N}\\
	aH &\mapsto aNH
\end{aligned}
\end{equation*}
ist ein Isomorphismus. \qquad ("Kürzungsregel")\\

\bet{Beweis:}\\
Für alle $h\in H$ gilt $h(H\cap N)h^{-1}\subseteq N\cap H$ weile $N\nt G$ und $hHh^{-1}=H$. $\Rightarrow N\cap H \nt H$. Für alle $g\in NH$ gilt $gNg^{-1}\subseteq N \Rightarrow N\nt NH$
\hfill $\square$

\subsubsection*{Lemma}
Sei $G\stackrel{\varphi}{\to}K$ ein Gruppenhomomorphismus. Dann sind äquivalent:\\
\begin{enumerate}[(i)]
	\item $\varphi$ ist bijektiv
	\item es gibt ein Homomorphismus $\psi:K\to G$ mit $\varphi \circ \psi=\id_K$ und $\psi\circ\varphi=\id_G$.
\end{enumerate}
\bet{Beweis:}\\
\uline{(ii)$\Rightarrow$(i):} klar, aus $\varphi\circ\psi=\id_K$ folgt, dass $\varphi$ surjektiv ist und aus $\varphi\circ\psi=\id_G$ folgt, dass $\varphi$ injektiv ist.\\

\uline{(i)$\Rightarrow$(ii):} Sei $\psi:K\to G$ die eindeutig bestimmte Umkehrabbildung, also $\varphi \circ \psi=\id_K$ und $\psi\circ\varphi=\id_G$. Für $a,b\in K$ folgt $\psi(ab)=\psi(\varphi\psi(a)\varphi\psi(b)) \stackrel{\varphi \text{ Homo.}}{=}\underbracket{\psi(\varphi}_{\id}(\psi(a)\psi(b)))= \psi(a)\psi(b)$
\hfill $\square$
\vspace{2cm}
Betrachte die Abbildung $\varphi:H\to \nicefrac{HN}{N}\subseteq \nicefrac{G}{N},~h\mapsto hN$ das ist ein Homomorphismus,\\ weil $H\stackrel{i}{\to}G\stackrel{\pi_N}{\to}\nicefrac{G}{N}$ einer ist. Für $hn\in HN$ gilt $\varphi(h)=hN=hnN$, also ist $\varphi$ ein Epimorphismus. Der Kern ist $ker(\varphi)=\{h\in H~|~hN=N \}=H\cap N$. Also gilt nach dem vorigem Satz 
\[\nicefrac{H}{n\cap H}\xrightarrow[\cong]{\overline{\varphi}}\nicefrac{HN}{N}\]
\hfill $\square$
\newpage

\subsubsection*{Theorem: 2. Isomorphiesatz}
Sei $G$ Gruppe, seien $M,N\nt G$ Normalteiler mit $M\subseteq N\subseteq G$. Dann gilt $\nicefrac{N}{M}\nt \nicefrac{G}{M}$ und 
\[\nicefrac{\nicefrac{G}{M}}{\nicefrac{N}{M}}\cong \nicefrac{G}{N} \qquad\qquad \text{'Kürzungsregel'} \]

\bet{Beweis:}\\
Es gilt $\nicefrac{N}{M}=\{nM~|~n\in N\}=\pi_M(N)\subseteq \nicefrac{G}{M}$\\
Nach\hyperref[sub:satz_eigenschaften]{1.22(i)} gilt $\nicefrac{N}{M}\nt \nicefrac{G}{M}$.\\
Jetzt Homomorphiesatz \hyperref[sub:der_homomorphiesatz]{1.20}
\begin{center}
	\begin{tikzcd}[column sep=small]
		G \ar{rr}{\pi_N} \ar{rd}[below,left]{\pi_M} & & K\\
		& \nicefrac{G}{N} \ar{ru}[below,right]{\overline{\pi_N}\leftarrow \text{surjektiv}} &
	\end{tikzcd}
	\captionof{figure}{2. Isomorphiesatz}
\end{center}
Nach dem vorigen Satz gilt:\\
\[\nicefrac{\nicefrac{G}{M}}{ker(\overline{\pi_N})}\stackrel{\cong}{\to} \nicefrac{G}{N}\]
\[ker(\overline{\pi_N})\stackrel{\hyperref[sub:der_homomorphiesatz]{1.20}}{=}\pi_M(N)=\nicefrac{N}{M} \]
\hfill $\square$
%sub end

\subsection{Produkte von Gruppen}
\label{sub:produkte}
Seien $G,K$ zwei Gruppen. Dann ist das Produkt $G\times K$ wieder eine Gruppe das \Index{direkte Produkt}, mit Verknüpfung \[(g_1,k_1)\cdot(g_2,k_2)=(g_1g_2,k_1k_2) \]
\[\text{Neutralelement } e=(e_G,e_K) \]
\[\text{Das Inverse zu }(g,k)\in G\times K \text{ ist }(g,k)^{-1}=(g^{-1},k^{-1})\]
Den Beweis lassen wir weg, die Gruppenaxiome (G1)-(G3) sind leicht zu prüfen.\\
Wir haben kanonische Homomorphismen:\\
\begin{minipage}[c]{8cm}
	\begin{equation*}
	\begin{aligned}
		i_G: G &\to G\times K\\
		g &\mapsto (g,e_K)
	\end{aligned}
	\end{equation*}
\end{minipage}
\begin{minipage}[c]{8cm}
	\begin{equation*}
	\begin{aligned}
	i_K: K &\to G\times K\\
	k &\mapsto (e_G,k)
	\end{aligned}
	\end{equation*}
\end{minipage}
sowie \[pr_G:G\times K \to G,\quad (g,k)\mapsto g \]
\[pr_K:G\times K \to K,\quad (g,k)\mapsto k \]
mit \[pr_G\circ i_G=\id_G \qquad\qquad pr_K\circ i_K=\id_K \]
\[ ker(pr_G)=\{e_G\}\times K\cong K \qquad\qquad ker(pr_K)=G\times \{e_K\}\cong G \]

Das geht auch mit Familien von (endliche vielen) Gruppen: ist $(G_i)_{i\in I}$ eine Familie von Gruppen, so ist $\prod\limits_{i\in I}G_i$ wieder eine Gruppe, das \bet{direkte Produkt} der $G_i$. Die Elemente sind Folgen $(g_i)_{i\in I},~g_i\in G_i$ mit Verknüpfung $(g_i)_{i\in I}\cdot (g_i')_{i\in I}=(g_ig_i')_{i\in I}$ usw.\\
\newpage

\minisec{Satz}
Sei $G$ eine Gruppe mit Untergruppe $H,K\subseteq G$.Angenommen, es gilt folgendes
\begin{enumerate}[(i)]
	\item $G=HK$
	\item $H\cap K=\{e\}$
	\item $hk=kh\quad \forall h\in H,~k\in K$
\end{enumerate}
Dann ist die Abbildung $H\times K \stackrel{\varphi}{\to} G$, $(h,k)\mapsto hk$ ein Isomorphismus, d.h. $G$ 'ist' das direkte Produkt aus $H$ und $K$.

\bet{Beweis:}\\
Wegen (iii) gilt
\begin{equation*}
\begin{aligned}
	\varphi((h_1,k_1)(h_2,k_2)) &= \varphi(h_1h_2,k_1k_2)=h_1h_2k_1k_2\\
	\varphi(h_1,k_1)\varphi(h_2,k_2) &= h_1k_1h_2k_2=h_1h_2k_1k_2
\end{aligned}
\end{equation*}
also ist $\varphi$ ein Homomorphismus. Wegen (i) ist $\varphi$ surjektiv.\[ (h,k)\in ker(\varphi) \Leftrightarrow hk=e \Leftrightarrow \underbracket[.7pt]{h}_{\in H}=\underbracket[.7pt]{k^{-1}}_{\in K} \Leftrightarrow h=k=e \text{ wegen (ii)}\]
\hfill $\square$

\minisec{Beispiel}
$G=\nicefrac{\mathds{Z}}{6\mathds{Z}}=\{\overline{0},\dots,\overline{5}\}$ vgl. \hyperref[sub:satz_eigenschaften]{1.22}.Dann sind $H=\{\overline{0},\overline{3}\}$ sowie $K=\{\overline{0},\overline{2},\overline{4}\}$ Untergruppen (nachrechnen!), $H\cong \nicefrac{\mathds{Z}}{2\mathds{Z}},~K\cong \nicefrac{\mathds{Z}}{3\mathds{Z}}$ und (i),(ii),(iii) aus dem vorigen Satz sind erfüllt. Es folgt
\[\nicefrac{\mathds{Z}}{6\mathds{Z}}\cong \nicefrac{\mathds{Z}}{3\mathds{Z}}\times \nicefrac{\mathds{Z}}{2\mathds{Z}} \]

%sec end
\newpage

\section{Gruppenwirkungen und Sylow-Sätze}
\label{sec:gruppen_sylow}

\subsection{Gruppenwirkungen}
\label{sub:gruppenwirkungen}
Sei $G$ eine Gruppe und $X$ eine nicht leere Menge. Eine \Index{Wirkung} von G auf $X$ (auch: \bet{$G$-Wirkung}, \bet{'$G$-Aktion'}) ist ein Homomorphismus $\alpha:G\to \sym(X)$. Für $g\in G$ und $x\in X$ schreibe kurz \[g(x)=\alpha(g)(x) \] (wenn klar ist welches $\alpha$ gemeint ist). Die Abbildung $G\times X\to X,\quad (g,x)\mapsto g(x)$ erfüllt folgende Eigenschaften:
\begin{enumerate}[(W1)]
	\item $e(x)=x~\forall x\in X$ ($e\in G$ Neutralelement)
	\item $(a\circ b)(x)=a(b(x))~\forall a,b\in G,~x\in X$
\end{enumerate}
Ist umgekehrt eine Abbildung $G\times X\to X$ gegeben die (W1) und (W2) erfüllt, so erhalten wir eine Wirkung $\alpha: G\to Sym(X)$ durch \[\alpha(g)=[x\mapsto g(x)] \]
denn aus (W2) folgt: $\alpha(g^{-1})$ ist Inverse zu $\alpha(g)$, also ist die Abbildung $\alpha(g):X\to X$ bijektiv und $\alpha:G\to Sym(X)$ ist ein Homomorphismus nach (W2).
%sub end

\subsection{Mehrere Definitionen}
\label{sub:mehrere_def}
Gegeben sei eine $G$-Wirkung $G\times X\to X$. Für $x\in X$ ist der \Index{Stabilisator} (die \Index{Standgruppe})
\[G_x=\{g\in G~|~g(x)=x \}\subseteq G \]
Die \Index{Bahn} (der \Index{Orbit}) von $x$ ist 
\[G(x)=\{g(x)~|~g\in G \}\subseteq X \]
Der \uline{Kern} der Wirkung ist $\bigcap\limits_{x\in X}G_x\subseteq G$.

\subsubsection*{Satz}
Der Stabilisator $G_x$ ist eine Untergruppe und der Kern ist ein Normalteiler.\\

\bet{Beweis:}\\
Es gilt $e(x)=x \rightsquigarrow e\in G_x$. Für $a,b\in G_x$ gilt \[(ab)(x)=a(\underbracket{b(x)}_{=x})=a(x)=x \rightsquigarrow ab\in G_x\]
\[a^{-1}(x)=a^{-1}(\underbracket{a(x)}_{=x})=(a^{-1}a)(x)=e(x)=x \rightsquigarrow a^{-1}\in G_x \]
Also ist $G_x\subseteq G$ Untergruppe.\\
Es gilt: \[\bigcap\limits_{x\in X}G_x=\{g(x)=x~|~\forall x\in X \} \]
Das ist genau der Kern der zugehörigen Homomorphie $\alpha:G\to \sym(X)$, also ein Normalteiler.
\hfill $\square$
%sub end

\subsection{Beispiele Wirkungen}
\label{sub:bsp_wirkungen}
\begin{enumerate}[(a)]
	\item Sei $G$ eine Gruppe. Für $g\in G$ definiere eine Abbildung $\lambda_g:G\to G$ durch $\lambda_g(x)=gx$. Es folgt
	\[\lambda_g\circ\lambda_h=\lambda_{gh} \quad \lambda_e=\id_G \rightsquigarrow \lambda_g\lambda_{g^{-1}}=\id_G=\lambda_{g^{-1}}\lambda_g \]
	also $\lambda_g\in \sym(G)$. Die Gruppe $G$ wirkt also auf der Menge $G=X$. Es gilt für die Wirkung:
	\[G_x=\{g\in G~|~\lambda_g(x)=x \}=\{g\in G~|~gx=x \}=\{e\} \]
	Zu $x,y\in G$ gibt es genau ein $g\in G$ mit $\lambda_g(x)=y$, nämlich $g=yx^{-1}$.\\
	Man nennt das die \bet{Linksreguläre Wirkung}\index{Wirkung!Linksregulär} von $G$ auf sich.
	\item Sei $G$ eine Gruppe und $H\subseteq G$ Untergruppe. Sei $X=\nicefrac{G}{H}=\{aH~|~a\in G \}$. Die Gruppe $G$ wirkt auf $X$ durch \[\lambda_g:\nicefrac{G}{H}\to \nicefrac{G}{H},~aH\mapsto gaH \]
	Es gilt wieder $\lambda_g\lambda_h=\lambda_{gh},~\lambda_e=\id_{\nicefrac{G}{H}}$.\\
	Der Stabilisator von $x=H\in X$ ist \[G_x=\{g\in G~|~gH=H \}=H \]
	Zu $x=aH,y=bH\in X$ gibt es wieder $g\in G$ mit $g(x)=y$, nämlich $g=ba^{-1}$. Anders als im Bsp(a) ist $g$ nicht eindeutig, falls $H\not= \{e\}$ gilt (für $H=\{e\}$ erhalten wir wieder Bsp(a)). 
\end{enumerate}
%sub end

\subsection{Satz 4, Satz von Cayley}
\label{sub:satz_von_cayley}
Zu jeder Gruppe $G$ gibt es eine Menge $X$ und ein injektiven Homomorphismus $\alpha:G\to \sym(X)$.\\

\bet{Beweis:}\\
Setze $G=X$ und $\lambda:G\to \sym(X)$ wie in \hyperref[sub:bsp_wirkungen]{Beispiel 2.3(a)}
\hfill $\square$

Eine Untergruppe von $\sym(X)$ nennt man auch eine \Index{Permutationsgruppe}. Der Satz von Cayley wird auch so formuliert:\\
\uline{Jede} Gruppe 'ist' (bis auf Isomorphie) eine Permutationsgruppe.
%sub end

\subsection{Definition transitiv}
\label{sub:def_transitiv}
Eine $G$-Wirkung $G\times X\to X$ heißt \Index{transitiv}, wenn es für alle $x,y\in G$ ein $g\in G$ gibt mit $g(x)=y$.\\
Die in \hyperref[sub:bsp_wirkungen]{Bsp. 2.3(a)(b)} betrachteten Wirkungen sind also transitiv.

\subsubsection*{Satz}
Gegeben sei ein transitive $G$-Wirkung $G\times X\to X$. Sei $x\in X$ und $H=G_x$. Dann ist die Abbildung $\nicefrac{G}{H}\to X,~gH\mapsto g(x)$ wohldefiniert und bijektiv. Für jedes $y\in X$ mit $y=g(x)$ gilt $G_y=gG_xg^{-1}$.\\

\bet{Beweis:}\\
Betrachte die Abbildung $\epsilon:G\to X,\epsilon(g)=g(x)$. Es gilt \[\epsilon(g)=\epsilon(g') \Leftrightarrow g(x)=g'(x) \Leftrightarrow g^{-1}g'=x \Leftrightarrow g^{-1}g'\in G_x=H \stackrel{\hyperref[sub:nebenklassen]{1.13}}{\Leftrightarrow} g'H=gH\]
Damit ist die erste Behauptung gezeigt.\\
Für $y=g(x)$ gilt \[a(y)=y \Leftrightarrow ag(x)=g(x) \Leftrightarrow g^{-1}ag(x)=x \Leftrightarrow g^{-1}ag\in G_x \Leftrightarrow a\in gG_xg^{-1}\]
\hfill $\square$
%sub end

\subsection{Bahnen}
\label{sub:bahnen}
Gegeben sei eine $G$-Wirkung $G\times X\to X$.

\subsubsection*{Lemma}
Für \Index{Bahnen} $G(x),~G(y)\subseteq X$ gilt stets:\[ \text{Ist } G(x)\cap G(y)\not=\emptyset, \text{ so gilt } G(x)=G(y)\]
Bahnen sind entweder disjunkt oder gleich.\\

\bet{Beweis:}\\
Angenommen, $z\in G(x)\cap G(y)$, also $z=a(x)=b(y)$ für $a,b\in G$. Es folgt $b^{-1}a(x)=y$, also $y\in G(x)$, also $G(y)\subseteq G(x)$. Genauso folgt auch $G(y)\supseteq G(x)$, also $G(x)=G(y)$.
\hfill $\square$

\subsubsection*{Bemerkung}
Für jedes $x\in X$ wirkt $G$ transitiv auf der Bahn $G(x)\subseteq X$. Denn: $y,z\in G(x),~y=a(x)$ und $z=b(x)\rightsquigarrow x=a^{-1}(y) \rightsquigarrow z=ba^{-1}(x)$. Weiter gilt $g(y)=ga(x)\in G(x)$.
\subsubsection*{Definition Bahnenraum}
Die Menge der Bahnen bezeichnen wir mit $\xfrac{G}{X}=\{G(x)~|~x\in X \}$ 'Bahnenraum'
\subsubsection*{Bemerkung}
Das passt zur Notation für Nebenklassen: Gegeben sei eine Untergruppe $H\subseteq G$. Setze $X=G$, dann wirkt $H$ auf $G=X$ durch $H\times X\to X,~(h,x)\mapsto hx$\\
Die \bet{Länge}\index{Bahnen!Länge} einer Bahn $G(x)$ ist $\#G(x)$. Ist $\{x\}=\{G\}$ (Bahn der Länge 1), so sagt man,dass $x\in X$ ein \Index{Fixpunkt} der $G$-Wirkung auf $X$ ist. Für alle $g\in G$ gilt dann $g(x)=x$.\\
Die Bahnen der Wirkung von $H$ auf $G$ sind dann genau die Rechtsnebenklassen, $H(x)=Hx$ für $x\in X=G$, die Bahnenmenge ist also $\xfrac{H}{G}=\{Hx~|~x\in G\}$

%sub end

\subsection{Satz 5, Die Bahnengleichung}
\label{sub:bahnengleichung}
Gegeben sei eine $G$-Wirkung $G\times X\to X$. Ein \Index{Schnitt} (ein \Index{Transversale}) ist eine Teilmenge $S\subseteq X$ mit folgender Eigenschaft: für jedes $x\in X$ gilt $\#(s\cap G(x))=1$, jede Bahn trifft $S$ genau einmal. Es folgt $\#S=\#\enbrace{\xfrac{G}{X}}$. Mit Hilfe des Auswahlaxioms sieht man, dass Schnitte stets existieren.
\begin{center}
\begin{tikzpicture}[line cap=round,line join=round,>=triangle 45,x=1.0cm,y=1.0cm]
	\draw(0.02,0.2) -- (3.88,0.18) -- (3.8,-3.56) -- (0.02,-3.54) -- cycle;
	\draw (0.02,0.2)-- (3.88,0.18);
	\draw (3.88,0.18)-- (3.8,-3.56);
	\draw (3.8,-3.56)-- (0.02,-3.54);
	\draw (0.02,-3.54)-- (0.02,0.2);
	\draw (0.5,-0.42)-- (3.48,-0.14);
	\draw (3.46,-0.76)-- (0.44,-0.9);
	\draw (0.44,-1.36)-- (3.42,-1.26);
	\draw (3.38,-1.8)-- (0.48,-2.36);
	\draw (0.6,-2.74)-- (3.36,-2.2);
	\draw (3.32,-2.74)-- (0.36,-3.32);
	\draw [rotate around={-89.301305617:(1.08,-1.72)},color=green] (1.08,-1.72) ellipse (1.65838338606cm and 0.245429124511cm);
	\draw (4,-3.5) node[anchor=north west] {X};
	\draw [color=green](0.4,0.1) node[anchor=north west] {S};
	\draw [->] (1.8,-3.7) -- (1.8,-3.1);
	\draw (1.4,-4) node[anchor=north west] {Bahnen};
	\draw [color=green] (1.07864981917,-0.365630218333) circle (2.5pt);
	\draw [color=green] (1.08140306346,-0.870266083151) circle (2.5pt);
	\draw [color=green] (1.06131377666,-1.33915054441) circle (2.5pt);
	\draw [color=green] (1.10005135495,-2.24026594525) circle (2.5pt);
	\draw [color=green] (1.10836595357,-2.64053709604) circle (2.5pt);
	\draw [color=green] (1.08687330088,-3.17757212348) circle (2.5pt);
\end{tikzpicture}
\captionof{figure}{Die Bahnengleichung}
\end{center}
\subsubsection*{Satz}
Sei $S\subseteq X$ ein Schnitt der $G$-Wirkung $G\times X\to X$. Wenn $X$ endlich ist, dann gilt \[\#X=\sum_{s\in S}[G:G_s] \]
\bet{Beweis:}\\
Sei $\#S=m$, $S=\{s_1,\dots,s_m\} \rightsquigarrow X=G(s_1)\stackrel{.}{\cup} G(s_2)\stackrel{.}{\cup} \dots \stackrel{.}{\cup} G(s_m)$
\[\#G(s_i)\stackrel{\hyperref[sub:def_trans]{2.5}}{=}\#\nicefrac{G}{G_{s_i}}\stackrel{\hyperref[sub:satz_von_lagrange]{1.14}}{=}[G:G_{s_i}] \]

%sub end

\subsection{Automorphismen und Konjugationswirkungen}
\label{sub:automor}
Sei $G$ Gruppe. Ein bijektiver Homomorphismus $\alpha:G\to G$ heißt \Index{Automorphismus} von $G$. Die Menge \[\aut(G)=\{\alpha:G\to G~|~\alpha \text{ Automorphismus}\} \] 
ist eine Gruppe, mit der Komposition von Automorphismus als Verknüpfung und $\id_G$ als Neutralelement.

\subsubsection*{Beispiel}
Sei $a\in G$. Dann ist die Abbildung $\gamma_a:G\to G,~g\mapsto aga^{-1}$ ein Automorphismus. Denn:
\begin{equation*}
\begin{aligned}
	&\gamma_a(gh)=agha^{-1}=aga^{-1}aha^{-1}=\gamma_a(g)\gamma_a(h)\\
	&\rightsquigarrow \gamma_a \text{ Homomorphismus}\\
	&\gamma_a(g)=e \Leftrightarrow aga^{-1}=e \Leftrightarrow g=a^{-1}ea=e\\
	\marginnote{oder: $\gamma_a\circ\gamma_a=\id_G=\gamma_{a^{-1}}\circ\gamma_a$}
	&\rightsquigarrow \gamma_a \text{ Monomorphismus},~ker(\gamma_a)=\{e\}\\
	&\text{Gegeben } g\in G \text{ folgt } \gamma_a(aga^{-1})=g\\
	&\rightsquigarrow \gamma_a \text{ Epimorphismus}\\
	&\Rightarrow \gamma_a \text{ Automorphismus}
\end{aligned}
\end{equation*}

\subsubsection*{Satz}
Die Abbildung $G\stackrel{\gamma}{\to} \aut(G),~a\mapsto \gamma_a$ ist ein Homomorphismus.\\

\bet{Beweis:}\\
Es gilt \[\gamma_a\circ\gamma_b(g)=abgb^{-1}a^{-1}=abg(ab)^{-1}=\gamma_{ab}(g) \]
also $\gamma_a\circ\gamma_b=\gamma_{ab}$,
\hfill $\square$\\

Weil $\aut(G)\subseteq \sym(G)$ eine Untergruppe ist, ist $\gamma:G\to \aut(G)$ eine Wirkungvon $G$ auf $G$, die \Index{Konjugationswirkung}.\\
Beachte den Unterschied zu \hyperref[sub:bsp_wirkungen]{2.3(a)}:
\[\lambda_a(g)=ag\qquad\qquad \gamma_a(g)=aga^{-1}\]
$\lambda_a$ ist \uline{kein} Homomorphismus (für $a\not=e $) \[\lambda_a(gh)=agh\not= \lambda_a(g)\lambda_a(h)=agah \]
Der \uline{Kern} von $\gamma:G\to\aut(G)$ ist
\begin{equation*}
\begin{aligned}
	Z(G) &= \{a\in G~|~\forall g\in G\text{ gilt }aga^{-1}=g\}\\
	&= \{a\in G~|~\forall g\in G\text{ gilt }ag=ga \}
\end{aligned}
\end{equation*}
Man nennt diesen Normalteiler das \Index{Zentrum} von $G$. Das Zentrumvon $G$ ist also abelsch (und $G$ ist genau dann abelsch, wenn $Z(G)=G$ gilt).


\subsubsection*{Bemerkung}
Im Allgemeinen ist die Abbildung $\gamma:G\to\aut(G)$ weedr injektiv und surjektiv. Das Bild $\gamma(G)\subseteq\aut(G)$ ist die Gruppe der \Index{inneren Automorphismen}, $\gamma(G)=\inn(G)\subseteq \aut(G)$.\\
Mit dem \hyperref[sud:der_homomorphiesatz]{Homomorphiesatz} also: \[\nicefrac{G}{Z(G)}\cong\inn(G)\]
Wie sehen die Stabilisatoren in der Konjugationswirkung aus? Der Stabilisator von $g\in G$ ist der \Index{Zentralisator} von $g$ (vgl. \hyperref[sub:def_zentralisieren]{1.6})
\begin{equation*}
\begin{aligned}
	Z_G(g) &= \{a\in G~|~aga^{-1}=g \}\\
	&= \{a\in G~|~ag=ga \}
\end{aligned}
\end{equation*}
Beachte: es gilt stets $\lh{g}\subseteq Z_G(g)$,denn \[ggg^{-1}=g \rightsquigarrow g\in Z_G(g) \rightsquigarrow \lh{g}\subseteq Z_G(g) \]
Die Bahnen $G(g)=\{aga^{-1}~|~a\in G\}$ nennt man \Index{Klassen} oder \Index{Konjugiertenklassen} in $G$.

%sub end

\subsection{Satz 6, Die Klassengleichung}
\label{sub:klassengleichung}
Sei $G$ eine endliche Gruppe, sei $S\subseteq G$ ein Schnitt der Konjugationswirkung $\gamma$. Sei $\mathcal{K}=S-Z(G)$. Dann gilt \[\#G=\#Z(G)+\sum_{s\in \mathcal{K}}[G:Z_G(s)] \]

\bet{Beweis:}\\
Nach der Bahnengleichung gilt \[\#G=\sum_{s\in S}[G:Z_G(s)] \]
Für jedes $z\in Z(G)$ gilt $G(z)=\{aza^{-1}~|~a\in G \}=\{z\}$, also $Z(G\subseteq S$ und $\#G(z)=1~\forall z\in Z$.
\hfill $\square$

%sub end

\subsection{Korollar über das Zentrum}
\label{sub:kor_zentrum}
Sei $p$ eine Primzahl und $G$ eine endliche Gruppe mit $\#G=p^m,~m\ge 1$.\\
Dann gilt $Z(G)\not= \{e\}$.\\

\bet{Beweis:}\\
Für $g\in G\backslash Z(G)$ ist $Z_G(g)\not= G$. Nach dem Satz von \hyperref[sub:satz_von_lagrange]{Lagrange 1.14} folgt $\#Z_G(g)=p^l,~l<m$. Insbesondere ist dann $p$ ein Teiler von $[G:Z_G(g)]=p^{m-l}\not=1$. Folglich ist $p$ ein Teiler von $\#Z(G)$, also $\#Z(G)\ge p$.
\hfill $\square$

Wenn $G$ eine endliche Gruppe ist, dann nennt man ihre Kardinalität $\#G$ die \Index{Ordnung} von $G$. Das passt zu \hyperref[sub:def_zyklische_gruppen]{1.11}: die Ordnung eines Elements $g\in G$ ist die Ordnung der von $g$ erzeugten zyklischen Gruppe, $o(g)=\#\lh{g}$, vgl. \hyperref[sub:zyklische_gruppen]{1.12}.

\subsubsection*{Definition p-Gruppe}
Eine endliche Gruppe $G$ heißt \Index{p-Gruppe}, für eine Primzahl $p$, wenn gilt $\#G=p^m$ für ein $m\ge 1$. Das vorige Korollar besagt also: jede p-Gruppe hat ein nicht-triviales Zentrum.

\minisec{Beispiel}
$G=\penbrace{\begin{pmatrix}
1&x&z\\ 0&1&y\\ 0&0&1 \end{pmatrix} \in K^{3\times 3}}$ mit $K=\mathds{F}_p$ (Körper mit p Elementen)\\
$\#G=p^3 \rightsquigarrow G$ ist p-Gruppe. Das Zentrum ist $\penbrace{\begin{pmatrix}1&0&z\\&1&0\\&&1\end{pmatrix} \in K^{3\times 3}}$\\
Unser nächstes Ziel ist der Beweis der Sylow-Sätze. Das braucht etwas Vorbereitung.
%sub end

\subsection{Definition Normalisator}
\label{sub:def_normalisator}
Sei $G$ eine Gruppe und $H\subseteq G$ eine Untergruppe. Der \Index{Normalisator} von $H$ in $G$ ist \[N_G(H)=\{n\in G ~|~nHn^{-1}=H\} \]

\subsubsection*{Satz}
Der Normalisator $N_G(H)$ ist eine Untergruppe von $G$ und es gilt \[H\nt N_G(H) \] Insbesondere gilt $H\subseteq N_G(H)$\\

\bet{Beweis:}\\
Setze $X=\{aHa^{-1}~|~a\in G \}$. Dann wirkt $G$ auf der Menge $X$ durch Konjugation,
\begin{equation*}
\begin{aligned}
	G\times X &\to X\\
	(g,aHa^{-1}) &\mapsto gaHa^{-1}g^{-1}=(ga)H(ga)^{-1}
\end{aligned}
\end{equation*}
Der Stabilisator von $H\in G$ ist genau $N_G(H)$, also eine Untergruppe.\\
Weiter gilt $H\subseteq N_G(H)$ (klar) und nach Definition gilt für alle $n\in N_G(H)$, dass $nHn^{-1}=H$, also $H\nt N_G(H)$.
\hfill $\square$

Die Menge $X=\{aHa^{-1}~|~a\in G \}$ nennt man auch die \Index{Konjugationsklasse} der Untergruppe $H$ in $G$.\\
Folgerung aus dem Satz: Ist $K\subseteq N_G(H)$ eine Untergruppe, dann ist $KH\subseteq N_G(H)$ eine Untergruppe, denn $H\nt N_G(H)$, das folgt aus \hyperref[sub:isomorphiesätze]{1.23 Lemma}.
%sub end

\subsection{Satz 7, Cauchys Satz}
\label{sub:cauchys_satz}
Sei $G$ eine endliche Gruppe und sei p eine Primzahl. Wenn p ein Teiler von $\#G$ ist , dann enthält $G$ (mindestens) ein Element der Ordnung p.\\

\bet{Beweis:}\\
Setze $X=\{ (g_1,\dots,g_p)\in G^p~|~g_1\cdots g_p=e \}$. Da $g_1,\dots,g_{p-1}\in G$ frei gewählt werden können und $g_p=(g_1,\cdots,g_{p-1})^{-1}$, gilt, $\#X=(\#G)^{p-1}$ und p teilt $\#X$. Gesucht ist ein Element $g\in G$ mit $g\not= e$ und $(g,\dots,g)\in X$ (d.h. $g^p=e\not=g$).\\
Setze $K=\nicefrac{\Z}{p\Z}$. Diese Gruppe $K$ wirkt auf $X$ wie folgt: sei $\overline{k}\in K$, setze $\overline{k}(g_{\overline{1}},\dots,g_{\overline{p}})=(g_{\overline{1+k}},\dots,g_{\overline{p+k}})$\\
Das ist wirklich eine $K$-Wirkung: $0<k\le p$wirkt durch \[\overline{k}:(g_{\overline{1}},\dots,g_{\overline{p}}) \mapsto (g_{\overline{1+k}},\dots,g_{\overline{p}},g_{\overline{1}},\dots,g_{\overline{k}}) \] 
$g_{\overline{1}}\cdots g_{\overline{k}}=a\quad g_{\overline{k+1}}\cdots g_{\overline{p}}=b\qquad ab=e$ nach Voraussetzung $\Rightarrow b=a^{-1}$
\[g_{\overline{1+k}}\cdots g_{\overline{p}}\cdot g_{\overline{1}}\cdots g_{\overline{k}} =ba=e \Rightarrow (g_{\overline{1+k}},\dots,g_{\overline{p}})\in X \]
Die Fixpunkte dieser $K$-Wirkung sind genau die Tupel $(g,\dots,g)\in X$. Also ist $(e,\dots,e)$ ein Fixpunkt. Da $\#K=p$ hat jede $K$-Bahn $K(x)$ Länge $\#K(x)=[K:K_x]\in \{1,p\}$ und die der Länge 1 sind die Fixpunkte. Nach der Bahnengleichung gilt (für ein Schnitt $S\subseteq X$) \[\#X=\#G^{p-1}=\sum_{s\in S}[K:K_s] \]
Die Primzahl p teilt beide Seiten, es gilt $[K:K_s]\in \{1,p\}$ und für $s=(e,\dots,e)$ gilt $[K:K_s]=1$. Also gibt es ein $s\not=(e,\dots,e)$ mit $[K:K_s]=1$
\hfill $\square$
\newpage

Wir brauchen noch das folgende technische Hilfsmittel.
%sub end

\subsection{Lemma 3}
\label{sub:lemma_3}
Sei $G\times X\to X$ eine Wirkung einer endlichen Gruppe $G$ auf einer endlichen Menge $X$. Sei $p$ eine Primzahl. Angenommen, es gilt folgendes:\\
(i) zu jedem $x\in X$ gibt es eine $p$-Gruppe $P\subseteq G$ mit $P(x)=\{x\}$.\\
Dann gilt $\#X=kp+1$ für ein $k\ge0$ und $G$ wirkt transitiv auf $X$.\\

\bet{Beweis:}\\
Sei $S\subseteq X$ ein Schnitt. Für jedes $s\in S$ wirkt $G$ also transitiv auf $G(s)$. Sei $s\in S$. Sei $P\subseteq G$ p-Gruppe mit $P(s)=\{s\}$. Für jedes $x\in X\backslash\{s\}$ teilt p die Länge der Bahn $P(x)$ \big(weil $P$ p-Gruppe ist und $P(x)\not=\{x\}$ nach (i)\big). Es folgt $\#G(s)=kp+1$.\\
Angenommen, $S\not=\{s\}$. Für $t\in S\backslash\{s\}$ folgt $\#G(t)=lp$, weil $P$ in $G(t)$ kein Fixpunkt hat. Anderseits zeigt das gleiche Argument, dass $G(t)=mp+1\lightning$\\
Es folgt $S=\{s\}$ und $X=G(s)$
\hfill $\square$
\\

%sub end
Jetzt beweisen wir Sylows Sätze. Peter Sylow war ein norwegischer Mathematiker und Lehrer. Seine Sätze sind in der endlichen Gruppentheorie ganz wesentlich.

\subsection{Definition Sylow-Gruppe}
\label{sub:def_sylow_gruppe}
Sei $G$ eine endliche Gruppe, sei $p$ eine Primzahl mit $\#G=p^m\cdot r$, wobei $m\ge 1$ sei und $p$ kein Teiler von $r$ ist. Eine Untergruppe $U\subseteq G$ heißt \Index{Sylow-p-Gruppe} in $G$, wenn gilt $\#U=p^m$.\\
Die Menge aller Sylow-p-Gruppen in $G$ wird mit $\syl_p(G)$ bezeichnet.\\
(Im Moment ist nicht klar, dass $\syl_p(G)\not=\emptyset$, aber das beweisen wir gleich.)

\subsubsection*{Sylows Sätze}
\label{ssub:sylows_sätze}
Sei $G$ eine endliche Gruppe, sei $p$ eine Primzahl mit $\#G=p^m\cdot r,~m\ge 1$, $p$ kein Teiler von $r$. Dann gilt folgendes:\\
\begin{enumerate}[(1)]
	\item $\syl_p(G)\not=\emptyset$
	\item $G$ wirkt transitiv auf $\syl_p(G)$: zu $U,V\in \syl_p(G)$ gibt es stets $g\in G$ mit $gUg^{-1}=V$
	\item $\#\syl_p(G)=kp+1$ für ein $k\ge 0$
	\item Ist $P\subseteq G$ ein $p$-Gruppe, so gibt es $U\in \syl_p(G)$ mit $P\subseteq U$.
\end{enumerate}

\bet{Beweis:}\\
Sei $\Gamma$ die Menge aller $p$-Gruppen in $G$. Nach Cauchys Satz ist $\Gamma\not=\emptyset$. Sei $\Omega\subseteq \Gamma$ die Menge aller maximalen $p$-Gruppen in $\Gamma$ (weil $G$ endlich ist, ist jede $p$-Gruppe $P\subseteq G$ ein einer maximalen $p$-Gruppe enthalten).\\
Die Gruppe $G$ wirkt durch Konjugation auf der Menge $\Gamma$ und $\Omega$. Nach Definition gilt $\syl_p(G)\subseteq \Omega$.\\

\uline{1. Schritt:} $G$ wirkt transitiv auf $\Omega$ und es gilt $\#\Omega=kp+1$ für ein $k\ge0$.\\
\uline{Beweis 1. Schritt:} Wir benutzen das Lemma \hyperref[sub:lemma_3]{2.13}. Für $U\in \Omega$ ist $U$ der einzige Fixpunkt der Wirkung von $U$ auf der Menge $\Omega$. Denn: wenn $U$ das Element $V\in \Omega$ fixiert, so folgt $U\subseteq N_G(V)\stackrel{\hyperref[sub:def_normalisator]{2.11}}{=} UV\subseteq G$ Untergruppe, $V\nt UV$. Es gilt \[\#UV\stackrel{\hyperref[sub:satz_von_lagrange]{1.14}}{=}\#V\cdot[UV:V]=\#V\cdot\#\nicefrac{UV}{V}\]
sowie \[ \nicefrac{UV}{V} \stackrel{\hyperref[sub:isomorphiesaetze]{1.23}}{\cong} \nicefrac{U}{U\cap V}= \frac{\#U}{\#(U\cap V)}\text{ also ist }\#\nicefrac{UV}{V} \text{ eine $p$-Potenz} 
\]
denn $\#U$ und $\#U\cap V$ sind $p$-Potenzen. Folglich ist $UV\subseteq G$ eine $p$-Gruppe. Da $U$ und $V$ maximale $p$-Gruppen sind und $U,V\subseteq UV$ folgt \[U=UV=V \]
Mit Lemma \hyperref[sub:lemma_3]{2.13} folgt nun: $G$ wirkt transitiv auf $\Omega$ und $\#\Omega=kp+1$
\hfill $\square$

\uline{2. Schritt:} Es gilt $\Omega=\syl_p(G)$\\
\uline{Beweis 2. Schritt:} Sei $U\in \Omega,~\#U=p^l$. Wir müssen zeigen, dass $p^l=p^m$ gilt.\\
Wegen Schritt 1 gilt jedenfalls 
\[\#G=p^m\cdot r= \#N_G(U)\cdot \#\Omega= \#N_G(U)(kp+1) \tag*{$(\ast)$} \]
und folglich 
\[\#N_G(U)=p^m\cdot s \quad \text{ für ein } s\ge 1 \tag*{$(\ast\ast)$} \]
Angenommen, es gilt $l<m$. Betrachte \[N_G(U)\stackrel{\pi_U}{\to}\nicefrac{N_G(U)}{U}=K \]
Es folgt $\#N_G(U)=p^m\cdot s=\underbracket{\#U}_{=p^e}$, also ist p ein Teiler von $\#K$. Nach Cauchys Satz \hyperref[sub:cauchys_satz]{2.12} gibt es eine $p$-Gruppe $P\subseteq K$. Setze $V=\pi_U^{-1}(P)\subseteq N_G(U)$. Es folgt mit $P=\nicefrac{V}{U}$, dass \[\#V=\#U\cdot \#P \]
also ist $V$ eine $p$-Gruppe.\\
Da $p$ ein Teiler von $\#P$ ist, folgt $V\nsupseteq U$, ein Widerspruch zur Maximalität von $U$.\\
Folglich gilt $\#U=p^m$ für alle $U\in \Omega$ und damit $\Omega=\syl_p(G)$.
\hfill $\square$
\\

Damit sind (1),(2) und (3) bewiesen. Wegen $\syl_p(G)=\Omega$ folgt (4).
\hfill $\square$

\subsubsection*{Addendum zu Sylows Theorem}
Es gilt (mit den Bezeichnungen von oben) \[r=s\cdot(kp+1) \]
Das folgt aus $(\ast)$ und $(\ast\ast)$.
%sub end

\subsection{Beispiel einer Anwendung}
\label{sub:bsp_anwendung}
\subsubsection*{Lemma}
Seien $p,q$ Primzahlen mit $p<q$. Wenn $G$ eine Gruppe ist mit $\#G=p\cdot q$ und wenn $p$ kein Teiler von $q-1$ ist, dann ist $G$ abelsch.\\

\bet{Beweis:}\\
Setze $\#\syl_p(G)=kp+1$ und $\#\syl_q(G)=lq+1$, dann folgt $q=s(kp+1)$.\\
1.Fall: $s=1 \rightsquigarrow q=kp+1$ Widerspruch zur Annahme, dass $p$ kein Teiler von $q-1$ ist.\\
2.Fall: $kp+1=1 \rightsquigarrow$ es gibt genau eine Sylow-$p$-Gruppe $U\subseteq G \rightsquigarrow G=N_G(U)$, d.h. $U\nt G$.\\
Jetzt $p=s'\cdot(lq+1)$ wegen $q>p$ folgt $s'=p$ und $lq+1=1 \rightsquigarrow$ es gibt genau eine Sylow-$q$-Gruppe $Q\subseteq G \rightsquigarrow Q\nt G$.\\
Weiter gilt: $\#P=p\qquad \#Q=q$ und $\#(P\cap Q)$ teilt nach Lagrange $p$ und $q$ $\Rightarrow P\cap Q=\{e\}$. Weil $P\nt G$ und $Q\nt G$ gilt für $a\in P$ und $b\in Q$, dass
\[ \underbracket{\underbracket{aba^{-1}}_{\in Q}\underbracket{b^{-1}}_{\in Q}}_{\in P}\in Q\cap P \text{ d.h. }ab=ba \]
Nach \hyperref[sub:isomorphiesaetze]{1.23} haben wir ein Monomorphismus $P\times Q\stackrel{\varphi}{\to} G,~(a,b)\mapsto ab$. Wegen $\#(P\times Q)=p\cdot q=\#G$ ist $\varphi$ surjektiv, also ein Isomorphismus.\\
Wegen $\#P=p$ und $\#Q=q$ sind $P$ und $Q$ abelsch: ist $a\in P,~a\not=e$, so gilt $o(a)>1$ und $o(a)$ teilt $p$ 
\[\Rightarrow o(a)=p \Rightarrow \lh{a}=P \Rightarrow P \text{ zyklisch } \Rightarrow P \text{ abelsch, vgl. \hyperref[sub:zyklische_gruppen]{1.12}.} \]
Gleiches gilt für $Q$ (mit ÜA \hyperref[sub:a_4_3]{4.3 einfügen} folgt jetzt sogar: $G$ ist \uline{zyklisch}) 
\hfill $\square$

\subsubsection*{Beispiel}
Die Gruppe $\sym(3)$ ist nicht abelsch, vgl \hyperref[sub:beispiel_3]{1.7}. Es gilt $\#\sym(3)=2\cdot 3$ (aber 2 teilt 3-1 !). Was sind die Sylowgruppen in $\sym(3)$? (ÜA)\\

\subsubsection*{Bemerkung}
Im Beweis vom obigen Lemma haben wir einige \uline{nützliche Fakten} bewiesen, die auch sonst hilfreich sein können:
\begin{enumerate}[(1)]
	\item Jede endliche Gruppe, deren Ordnung eine Primzahl ist, ist ablesch.
	\item Wenn $\varphi:K\to G$ ein Monomorphismus von endlichen Gruppen ist und wenn gilt $\#K=\#G$, dann ist $\varphi$ ein Isomorphismus.
	\item Wenn $N,M\subseteq G$ Normalteiler sind und wenn gilt $N\cap M=\{e\}$, dann ist die Abbildung $N\times M\to G,~(n,m)\mapsto n\cdot m$ ein Monomorphismus.
	\item Wenn $G$ endlich ist und $p$ eine Primzahl und wenn $p$ ein Teiler von $\#G$ ist mit $\syl_p(G)=1$, dann ist die (eindeutige) Sylow-$p$-Gruppe $U\in \syl_p(G)$ ein Normalteiler in $G$, $U\nt G$.
\end{enumerate}

%sub end

\subsection{Satz 8}
\label{sub:satz_8}
Sei $G$ eine endliche Gruppe mit $\#G=pq,~p\not=q$ Primzahlen. Dann gilt es gibt einen Normalteiler $N\nt G,~\{e\}\not=N\not=G$.\\

\bet{Beweis:}\\
\OE $p<q,~\#\syl_q(G)=lq+1$
\begin{equation*}
\begin{aligned}
	&\stackrel{\hyperref[sub:def_sylow_gruppe]{2.14}}{\Rightarrow} p=s(lq+1) \Rightarrow lq+1=1 \text{ wegen } p<q\\
	&\Rightarrow \text{ es gibt genau eine Sylow-$q$-Gruppe } U\subseteq G\\
	&\Rightarrow U\nt G\text{ und } \#U=p
\end{aligned}
\end{equation*}
\hfill $\square$
%sub end

Wir betrachten als nächstes $p$-Gruppen genauer.\\

\subsection{Lemma 4}
\label{sub:lemma_4}
Sei $G$ eine Gruppe. Dann ist jede Untergruppe $H\subseteq Z(G)$ Normalteiler in $G$.\\

\bet{Beweis:}\\
Sei $g\in G$ und $h\in H\subseteq Z(G)$. Es folgt $ghg^{-1}=h$, also $gHg^{-1}=H$.
\hfill $\square$

\subsubsection*{Satz}
Sei $p$ Primzahl und $G$ eine $p$-Gruppe, $\#G=p^m$, $m\ge 1$. Dann gibt es Normalteiler $G_k\nt G$ mit $\#G_k=p^k$ für $0\le k\le m$ und mit \[ G_m\nt G_{m-1}\nt \dots \nt G_1\nt G_0=\{e\} \]

\bet{Beweis:}\\
Induktion nach $m$. Für $m=1$ ist nichts zu zeigen. Sei jetzt $\#G=p^m$, $m\ge 1$.\\
Nach \hyperref[sub:kor_zentrum]{2.10} ist $Z(G)\not=\{e\}$, also $Z(G)=p^s$ für ein $s>1$ (Lagrange). Nach Cauchys Satz \hyperref[sub:cauchys_satz]{2.12} gibt es $g\in Z(G)$ mit $o(g)=p$. Setze $G_1=\lh{g}$ und $G\stackrel{\pi}{\to}\tilde{G}=\nicefrac{G}{G_1}$ (nach dem Lemma gilt $G_1\nt G$).\\
Es folgt $\#\tilde{G}=p^{m-1}$ nach Induktionannahme gibt es $\tilde{G}_k\nt \tilde{G}$ mit $\#\tilde{G}_k=p^k,~\tilde{G}\supseteq \tilde{G}_{m-2}\supseteq \dots \supseteq \tilde{G}_0$. Setze $G_{k+1}=\pi^{-1}(G_k)$, es folgt nach \hyperref[sub:satz_eigenschaften]{1.22}, dass $G_{k+1}\nt G$, sowie $G_m\supseteq G_{m-1}\supseteq \dots \supseteq G_0=\{e\}$. Wegen $G_1\subseteq G_{k+1}$ folgt $\tilde{G}_k\cong \nicefrac{G_{k+1}}{G_1}$, also \[\#G_{k+1}=p\cdot \#\tilde{G}_k=p^{k+1} \]
\hfill $\square$

\subsubsection*{Folgerung}
Ist $G$ eine endliche Gruppe, $p$ eine Primzahl und ist $p^k$ ein Teiler von $\#G$, dann hat $G$ eine Untergruppe der Ordnung $p^k$.\\

\bet{Beweis:}\\
Sei $U\in \syl_p(G),~\#U=p^m$.\\
Dann gilt $k\le m$ und nach dem vorigen Satz gibt es eine Untergruppe $H\subseteq U$ mit $\#H=p^k$
\hfill $\square$
%sub end

\subsection{Definition Normalreihe}
\label{sub:def_normalreihe}
Sei $G$ eine Gruppe, sei $G=G_m\supseteq G_{m-1}\supseteq \dots\supseteq G_0=\{e\}$ Untergruppen. Wenn gilt \[G_{k-1}\nt \G_k, \]
dann heißt $G_m\supseteq \dots\supseteq G_0$ \Index{Normalreihe} in $G$. Die Quotienten $\nicefrac{G_k}{G_{k+1}}$ heißen \Index{Faktoren} der Normalreihe.\\
Eine Gruppe, die eine Normalreihe mit ableschen Faktoren hat, heißt \Index{auflösbare Gruppe}.

\subsubsection*{Beispiele}
\begin{enumerate}[(a)]
	\item $G$ abelsch $\Rightarrow G$ auflösbar, setze $G_1=GG\supseteq G_0=\{e\}$
	\item $G=\sym(3),~\#G=6$, $\tau:\{1,2,3\}\to \{1,2,3\}~\tau:\begin{array}{c} 1\mapsto 2\\ 2\mapsto 3\\ 3\mapsto 1    \end{array}$\\
	$o(\tau)=3,~G_1=\lh{\tau}\nt G$ (weil $[G:G_1]=2$,ÜA 3.2 oder \hyperref[sub:satz_8]{2.16})\\
	$\#\nicefrac{G}{G_1}=2\rightsquigarrow$ abelsch, also ist $\sym(3)$ auflösbar.
	\item Nach Satz \hyperref[sub:lemma_4]{2.17} ist jede $p$-Gruppe auflösbar.
\end{enumerate}

Wir betrachten jetzt \uline{abelsche} $p$-Gruppen.

\subsection{Lemmata 5,6,7}
\label{sub:lemmata}
\subsubsection*{Lemma A}
Sei $G$ abelsche $p$-Gruppe. Wenn $G$ genau eine Untergruppe $H\subseteq G$ der Ordnung $p$ hat, dann ist $G$ zyklisch.\\

\bet{Beweis:}\\
Setze $\#G=p^m,~m\ge 1$. Induktion nach $m$. Für $m=1$ ist nichts zu zeigen. Sei jetzt $m>1$. Betrachte den Homomorphismus $\varphi: G\to G,~g\mapsto g^p$ (das ist ein Homomorphismus, weil $G$ abelsch ist: $(gh)^p=g^ph^p$).\\
Es gilt $\ker(\varphi)=\big\{g\in G~|~g^p=e \big\}=\big\{g\in G~|~o(g)\in \{1,p\} \big\}$.Ist $o(g)=p$,so folgt aus der Annahme $g\in H$, also $H=\ker(\varphi)$, denn $h\in H\rightsquigarrow o(h)\in \{1,p\}$.\\
Setze $K=\varphi(G)$. Nach dem Homomorphiesatz \hyperref[sub:der_homomorphiesatz]{1.20} gilt $K\cong \nicefrac{G}{H}$, also $\#K=p^{m-1}$. Wegen $m>1$ folgt aus Cauchys Satz \hyperref[sub:cauchys_satz]{2.12}, dass $K$ ein Element der Ordnung $p$ enthält. Folglich gilt $H\subseteq K$. Also hat $K$ genau eine Untergruppe der Ordnung $p$ und ist deswegen nach Induktionsannahme zyklisch, $K=\lh{k}$ für ein $k\in K=\varphi(G)$. Wähle $g\in G$ mit $\varphi(g)=g^p=k$. Wegen $o(g)=p\cdot r$ folgt $o(g^r)=p\rightsquigarrow H\subseteq \lh{g}$ (wegen der Eindeutigkeit von $H$), also \[\nicefrac{\lh{g}}{H}\cong K\Rightarrow\#\lh{g}=\#K\cdot\#H=\#G\Rightarrow G=\lh{g}\]
\hfill $\square$

\subsubsection*{Lemma B}
Sei $G$zyklisch mit $\#G=k\cdot l$. Dann hat $G$ genau eine Untergruppe $H\subseteq G$ mit $\#H=k$ (ÜA 4.1).\\

\bet{Beweis:}\\
Betrachte $\varphi:G\to G,~g\mapsto g^k$, das ist ein Homomorphismus. Der Kern ist $K=\{g\in G~|~g^k=e \}$. Ist $H\subseteq G$ Untergruppe mit $\#H=k$,so folgt $H\subseteq K$. Sei $u\in G$ Erzeuger, $G=\lh{u}$. Das Bildvon $\varphi$ ist dann $\varphi(G)=\lh{u^k}$ und $o(u^k)=l$. Also folgt \[ l=\#\varphi(G)= \frac{\#G}{\#K} \Rightarrow \#K=k \Rightarrow H=K. \]
\hfill $\square$

\subsubsection*{Lemma C}
Sei $G$ eine abelsche $p$-Gruppe, sei $u\in G$ eine Element maximaler Ordnung in $G$ und sei $U=\lh{u}$. Dann gibt es eine Untergruppe $H\subseteq G$ mit
\[ H\cap U=\{e\} \text{ und } G=HUs, \text{ d.h. } H\times UU\cong G. \]

\bet{Beweis:}\\
Setze $\#G=p^m$. Für $m=1$ist $G$ zyklisch, setze $U=G$ und $H=\{e\} \rightsquigarrow$ fertig.\\
Sei jetzt $m>1$, Induktion nach $m$.\\
1. Fall: $G$ zyklisch, $G=U,~H=\{e\}\rightsquigarrow$ fertig.\\
2. Fall: $G$ nicht zyklisch. Da $U$ genau eine Untergruppe der Ordnung $p$ hat (Lemma B) gibt es nach Lemma A und Cauchys Satz \hyperref[sub:cauchys_satz]{2.12} ein Element $w\in G\backslash U$ mit $o(w)=p$. Setze $W=\lh{w}$.\\
Es folgt $U\cap W=\{e\}$,weil $w\notin U$ ($\#U\cap W$ ist $p$-Potenz). Betrachte $\pi:G\to \nicefrac{G}{W}$. Wegen $\ker(\pi)=W$ ist die Einschränkung von $\pi$ auf $U$ injektiv, d.h. $o(\pi(u))=o(u)$. Folglich ist $\pi(u)$ ein Element maximaler Ordnung in $L=\nicefrac{G}{H}$, und $\#\nicefrac{G}{W}=p^{m-1}$.\\

Nach Induktionsannahme gibt es eine Untergruppe $H'\subseteq L$ mit $H'\cap \pi(U)=\{e_L\}$ und $L=\pi(U)H'\cong \pi(U)\times H'$.\\
Setze $H=\pi^{-1}(H')$. Es folgt $H\cap U=\{e\}$, denn:
\[ h\in H,~\pi(h)\in \pi(U)\rightsquigarrow \pi(h)=e_L \rightsquigarrow h\in W. \]
Weiter gilt für $g\in G$, dass 
\begin{equation*}
\begin{aligned}
	&\pi(g)=\pi(u^k)\pi(h)\qquad \text{für ein } k\ge 0,~h\in H\\
	&\rightsquigarrow g=u^k(h\cdot w^l)\qquad\text{für ein } l\ge 0,\text{ aber } w\in H\\
	&\Rightarrow G=UH
\end{aligned}
\end{equation*}
\hfill $\square$

\subsubsection*{Korollar}
Sei $G$ eine abelsche $p$-Gruppe, $\#G=p^m$ mit $m\ge 1$. Dann gibt es Zahlen $n_1\ge \dots \ge n_r\ge 1$ mit $m=n_1+\dots+n_r$ und \[ G\cong \nicefrac{\Z}{p^{n_1}\Z} \times \nicefrac{\Z}{p^{n_2}\Z} \times \dots \times \nicefrac{\Z}{p^{n_r}\Z} \]

\bet{Beweis:}\\
Wähle $u_1\in G$ mit maximaler Ordnung $o(u_1)=p^{n_1}$, $U_1=\lh{u_1}\cong \nicefrac{\Z}{p^{n_1}\Z}$ und eine Untergruppe $G_1\subseteq G$ wie in Lemma C mit $U_1\cap G_1=\{e\},~G=U_1G_1\cong U_1\times G_1$. Wähle $u_2\in G_1$ mit maximaler Ordung $o(u_2)=p^{n_2}, ~U_2=\lh{u_2}\cong \nicefrac{\Z}{p^{n_2}\Z},~ G_1=U_2G_2$ usw. Nach endlich vielen Schritten \[G=U_1U_2\cdots U_r\cong U_1\times \dots \times U_r \]
Zur \uline{Eindeutigkeit} der Zahlen $n_1,\dots,n_r$:\\
Für $l\ge 1$ sei $\varphi_l:G\to G,~g\mapsto g^{p^l}$.\\
Da $G$ abelsch ist, ist $\varphi_l$ ein Homomorphismus mit \[\ker(\varphi_l)=\{g\in G~|~o(g)\text{ teilt }p^l\}, \]
insbesondere
\[ \left.\begin{array}{cl} \varphi_l(u_i)=e & \text{für } l\ge n_i\\ \varphi_l(u_i)\not= & \text{sonst}    \end{array}\right\}\Rightarrow \#\varphi_l(U_i)=\left\{\begin{array}{cl} \{1\} & l\ge n_i\\ \nicefrac{\Z}{p^{n_i-l}\Z} & l<n_i    \end{array}\right. \]
$\Rightarrow \#\varphi_l(G)=\prod_{n_i>l}p^{n_i-l}=p^{N_l}$,aus den Zahlen $N_1,N_2,\dots$ lassen sich die $n_i$ berechnen, $N_l=\sum_{n_i>l}(n_i-l)$
\hfill $\square$
% sub end

\subsection{Satz 9}
\label{sub:satz_9}
Sei $G$ eine endliche abelsche Gruppe,$\#G=p_1^{l_1}\cdots p_s^{l_s},~ 2\le p_1<p_2<\dots<p_s$ Primzahlen, $l_1,\dots,l_s\ge 1$. Dann gilt \[G\cong P_1\times \dots\times P_s \]
wobei $P_j$eine abelsche $p_j$-Gruppe der Ordnung $p_j^{l_j}$ ist wie im vorigen Korollar.\\
Insbesondere ist jede endliche abelsche Gruppe eine Produkt von zyklischen Gruppen.\\
\newpage

\bet{Beweis:}\\
Da $G$ abelsch ist,ist jede Sylow-$p_j$-gruppe in $G$ normal, also gibt es (wegen \hyperref[sub:def_sylow_gruppe]{2.14(2)}) genau eine Sylow-$p_j$-Gruppe$P_j\subseteq G$, und $P_j$ enthält alle Elemente $g\in G$, deren Ordnung eine $p_j$-Potenz ist.\\
Betrachte
\begin{equation*}
\begin{aligned}
	\varphi:P_1\times\dots\times P_s&\to G\\
	(g_1,\dots,g_s)&\mapsto g_1g_2\cdots g_s
\end{aligned}
\end{equation*}
Weil $G$ abelsch ist, ist $\varphi$ ein Homomorphismus (oder: weil für alle $i<j$ gilt $P_i\cap P_j=\{e\}\rightsquigarrow$ B6 A($\ast$)). Es genügt zu zeigen, dass $\varphi$ injektiv ist, dann folgt aus Kardinalitätsgründen, dass $\varphi$ bijektiv ist.\\
\zz  $\ker(\varphi)=\{e\}$.\\
Angenommen, $g_1\cdots g_s=e,~ g_i\in P_i$.\\
Setze $r_i=\frac{\#G}{p_i^{l_i}}$.Für $i\neq j$ folgt $g_j^{r_i}=e$, weil $\#P_j$ ein Teiler von $r_i$ ist. Also gilt \[ (g_1\cdot g_s)^{r_i}=g_1^{r_i}\cdot g_s^{r_i}=g_i^{r_1}=e^{r_i}=e \]
Also ist $o(g_i)$ ein Teiler von $r_i$. Weil $o(g_i)$ eine $p_i$-Potenz ist, folgt $o(g_i)=1$, d.h. $g_i=1$.\\
Es folgt $\ker(\varphi)=\{(e,\dots,e)\}$.
\hfill $\square$
%sub end

\subsection{Satz 10}
\label{sub:satz_10}
Sei $G$ eine endliche auflösbare Gruppe mit einer Normalreihe $G=G_m\nt\dots\nt G_0$ mit abelschen Faktoren. Dann gibt es für jedes $1\le k\le m$ Untergruppe $H_j$ mit
\[
G_k\nt H_l\nt\dots\nt H_0=\G_{k-1}
\]
mit $\nicefrac{H_j}{H_{j-1}}\cong \nicefrac{\Z}{p_j\Z},\quad p_j$ Primzahl.\\
Insbesondere hat jede endliche auflösbare Gruppe eine Normalreihe, in der alle Faktoren zyklisch von Primzahlordnung sind.\\

\bet{Beweis:}\\
Betrachte die abelsche Gruppe $A=\nicefrac{G_k}{G_{k-1}}$.\\
Nach Satz \hyperref[sub:satz_9]{2.20} und \hyperref[sub:lemma_4]{2.17}, angewandt auf die Sylowgruppen von $A$, gibt es Untergruppen
\[
A=A_l\supseteq \dots\supseteq A_0=\{e\}\text{ mit } \nicefrac{A_j}{A_{j-1}}\cong\nicefrac{\Z}{p_j\Z},~p_j\text{ Primzahl}
\] 
Setze $\pi:\G_k\to\nicefrac{G_k}{G_{k-1}}=A$ kanonischee Epimorphismus und $H_j=\pi^{-1}(A_j)\rightsquigarrow H_j\nt G_k$ und
\[
G_k\nt H_l\nt \dots H_0=G_{k-1}
\]
\[
\nicefrac{H_j}{H_{j-1}}\stackrel{\text{2.Iso-satz}}{\cong} \nicefrac{A_j}{A_{j-1}} \cong \nicefrac{\Z}{p_j\Z}
\]
\hfill $\square$
%sub end

\subsection{Komutatoren}
\label{sub:komutatoren}
Sei $G$ eine Gruppe, $a,b\in G$. Der \Index{Komutator} von $a$ und $b$ ist 
\[ 
[a,b]=aba^{-1}b^{-1}=ab(ba)^{-1}\rightsquigarrow ab=[a,b]ba
\]
Offensichtlich gilt $[a,b]^{-1}=[b,a]$ und 
\[
[a,b]=e\Leftrightarrow a\text{ zentralisiert }b\Leftrightarrow b\text{ zentralisiert }a\Leftrightarrow a\text{ und }b\text{ vertauschen}
\]
Die \Index{Kommutatorengruppe} von $G$ ist
\[
\D G=\sprod{[a,b]}{a,b\in G},
\]
die von allen Komutatoren erzeugte Gruppe.

\subsubsection*{Satz}
Sei $G$ eine Gruppe. Dann gilt
\begin{enumerate}[(i)]
	\item $\D G\nt G$
	\item $\nicefrac{G}{\D G}$ ist abelsch
	\item Ist $A$ abelsche Gruppe und $\varphi:G\to A$ ein Homomorphismus, so gilt $\D G\subseteq \ker(\varphi)$.
\end{enumerate}

\bet{Beweis:}\\
\begin{enumerate}[(i)]
	\item Für $g,a,b \in G$ gilt $g[a,b]g^{-1}=[gag^{-1},gbg^{-1}]$ (nachrechnen), also gilt für alle $g\in G,~a_1,\dots,a_s,b_1,\dots,b_s\in G$, dass
	\[
	g[a_1,b_1]\cdots[a_s,b_s]g^{-1}\in \D G
	\]
	also $g\D G g^{-1}\subseteq \D G$ für alle $g\in G\Rightarrow \D G\nt G$.
	\item Sei $g,h\in G$. Es folgt wegen $gh=[g,h]hg$, dass
	\[
	gh\D G=\underbracket{[g,h]}_{\in \D G}hg \D G=hg\D G
	\]
	und damit, dass $\nicefrac{G}{\D G}$ abelsch ist.
	\item Für alle $g,h\in G$ gilt
	\[
	\varphi([g,h])=[\varphi(g),\varphi(h)]=e_A, \text{ weil $A$ abelsch ist,}
	\]
	also 
	\[
	\{[g,h]~|~g,h\in G \}\subseteq \ker(\varphi)\Rightarrow\D G\subseteq \ker(\varphi)
	\]
	\hfill $\square$
\end{enumerate}
Man definiert rekursiv
\[
\D^0 G=G,~\D^1 G=G,~\D^{k+1}G=\D(\D^kG)
\]
Es folgt $D^{k+1}G\nt G$.\\
Genauer: $D^{k+1}G\nt G$ mit Induktion
\[
a,b\in \D^kG\Rightarrow g[a,b]g^{-1}=[\underbracket{gag^{-1}}_{\in \D^kG},\underbracket{gbg^{-1}}_{\in \D^kG}]\in D^{k+1}G
\]
also $g(D^{k+1}G)g^{-1}\subseteq D^{k+1}G$.
%sub end

\subsection{Satz 11}
\label{sub:satz_11}
Eine Gruppe $G$ ist auflösbar genau dann, wenn gilt $D^mG=\{e\}$ für ein $m\ge 0$.\\

\bet{Beweis:}\\
Angenommmen, $D^mG=\{e\}$ für ein $m\ge 0$. Dann ist $\D^0G\supseteq \D^1G\supseteq \dots\supseteq \D^mG=\{e\}$ eine Normalreihe und $\nicefrac{\D^kG}{\D^{k+1}G}=\nicefrac{\D^kG}{\D(\D^kG)}$ ist abelsch nach \hyperref[sub:komutatoren]{2.22(ii)}, also ist $G$ auflösbar.\\
Ist umgekehrt $G$ auflösbar und $G=G_m\nt\dots\nt G_0=\{e\}$ eine Normalreihe mit abelschen Faktoren, so folgt aus \hyperref[sub:komutatoren]{2.22(iii)}, dass $\D G_k\subseteq G_{k+1}$, also iteriert auch
\[
D^{l+1}G_k\subseteq D^lG_{k-1}
\]
\[
\Rightarrow \D^mG=\D^mG_m\subseteq \D^{m-1}G_{m-1}\subseteq\dots\subseteq \D^0G_0=\{e\}
\]
\hfill $\square$

\subsubsection*{Korollar}
Bilder und Untergruppen von auflösbaren Gruppen sind wieder auflösbar.\\

\bet{Beweis:}\\
Sei $\varphi:G\to K$ Homomorphismus und $G$ auflösbar, $D^mG=\{e\}$. Wegen
\[
\varphi([a,b])=\varphi(aba^{-1}b^{-1})=[\varphi(a),\varphi(b)]
\]
folgt
\[
\D^m(\varphi(G))= \varphi(\D^mG)=\varphi(e_G)=\{e_K\} \marginnote{Bilder von Komutatoren sind Komutatoren}
\]
Ist $H\subseteq G$, so folgt $\D^kH\subseteq \D^kG$ für alle $k\ge 0$, also 
\[
\D^mG\{e_g\}\Rightarrow D^mH=\{e_G\}
\]
Also folgt mit dem Satz von oben, dass $H$ auflösbar ist.
\hfill $\square$
%sub end

\subsection{Definition perfekt}
\label{sub:def_perfekt}
Eine Gruppe $G$ heißt \Index{perfekt}, wen gilt $\D G=G$.\\
Eine Gruppe, die gleichzeitig perfekt und auflösbar ist, ist trivial.
%sub end

\subsection{Die symmetrischen und alternierenden Gruppen}
\label{sub:sym_alt_gruppen}
Sei $\sym(n)$ die Gruppe aller Permutationen der Menge $\{1,\dots,n\}$. Es gilt $\#\sym(n)=n!=n(n-1)(n-2)\cdots 2 \cdot 1$, denn $\sym(n)$ wirkt transitiv auf der $n$-elementigen Menge $\{1,\dots,n\}$.\\
Der Stabilisator von $n$ ist isomorph zu $\sym(n-1)$.
\[
\stackrel{\text{Bahnengl.}}{\Rightarrow} \#\sym(n)=n\cdot \sym(n-1)\text{ und }\#\sym(1)=1
\]
Erinnerung an LA II, Kapitel über Determinanten, 4.6.\\
Für $\pi\in \sym(n)$ setze $\sign(\pi)=\prod_{i<j}\frac{\pi(i)-\pi(j)}{i-j}\in \{\pm 1\}=C_2$. \marginnote{abelsche Gruppe der Ordnung 2 bzgl. Multiplikation}\\
$\sign:\sym(n)\to C_2$ ist ein Homomorphismus.\\
Der Kern von $\sign$ ist die alternierende Gruppe 
\[
\alt(n)=\{\pi\in\sym(n)~|~\sign(\pi)=1\}
\]
Aus \hyperref[sub:komutatoren]{2.22} folgt $\D\sym(n)\subseteq \alt(n)$, weil $C_2$ abelsch ist.

\subsubsection*{Satz}
Es gilt $\D\sym(n)=\alt(n)$. Für $n\ge 5$ ist $\alt(n)$ perfekt.\\

\bet{Beweis:}\\
Seien $i_1,\dots,i_k$ $k$ paarweise verschiedene Zahlen in $\{1,\dots,n\}$. Die Permutation $i_1\stackrel{\pi}{\to} i_2\stackrel{\pi}{\to} i_3\stackrel{\pi}{\to}\dots,\stackrel{\pi}{\to} i_k\stackrel{\pi}{\to} i_1$, also $\pi(i_l)=i_{l+1}$ für $l=1,\dots,k$, $\pi(i_k)=i_1$ und $\pi(j)=j$ sonst. Diese Permutation nennt man ein \Index{$k$-Zykel} und schreibt sich kurz mit $\pi=(i_1,\dots,i_k)$.\\
Die 2-Zykel vertauschen zwei Zahlen $i_1,i_2$, man nennt sie \Index{Transpositionen}. Nach LA II Übungsaufgabe 4.3 ist jede Permutation ein Produkt von 2-Zykeln. Weiter gilt $\sign((i_1,i_2))=-1$. Also besteht $\alt(n)$ aus allen Permutationen, die sich schreiben lassen als Produkt einer \uline{geraden} Anzahl von 2-Zykeln.\\
\newpage
\uline{Behauptung:} $\alt(n)$ wird von den 3-Zykeln erzeugt.\\
\uline{Beweis:} Seien $a,b,c,d\in \{1,\dots,n\}$ paarweise verschieden. Es gilt
\[
(a,c)\circ(a,b)=(a,b,c) \text{ sowie } (a,b)\circ (c,d)=(a,d,c)\circ (a,b,c)
\] 
\hfill $\square$

\uline{Zum Satz:} 
\[
[(a,b,c),(b,c)]=(b,a,c)\in \D \sym(n)\Rightarrow \D\sym(n)=\alt(n)
\]
Seien $a,b,c,d,e$ paarweise verschieden
\[
[(a,b,c),(c,d,e)]=(d,c,a)\Rightarrow \D \alt(n)=\alt(n)\text{ für }n\ge 5
\]
\hfill $\square$

\subsubsection*{Folgerung}
Für $n\ge 5$ ist $\sym(n)$ \uline{nicht} auflösbar.\\
Für $n=1,2,3,4$ ist $\sym(n)$ auflösbar. (ÜA)

\subsubsection*{Ausblick}
\begin{enumerate}[(1)]
	\item Jede endliche Gruppe $G$ mit ungerader Ordnung ist auflösbar. (Feit-Thompson-Theorem, viele hundert Seiten langer Beweis)
	\item Eine Gruppe $G$ heißt \Index{einfach}, wenn $G\neq \{e\}$ und wenn $G,\{e\}$ die einzigen Normalteiler in $G$ sind.
\end{enumerate}

\subsubsection*{Theorem (Klassifikation der endlichen einfachen Gruppen)}
Sei $G$ eine endliche einfache Gruppe. Dann kommt $G$ in folgender Liste vor:
\begin{itemize}
	\item abelsche einfache Gruppe $\nicefrac{\Z}{p\Z},~p$ Primzahl\marginnote{Die größte sporadische Gruppe, das "Monster", hat mehr Elemente als es Elemntarteilchen gibt.}
	\item $\alt(n),~n\ge 5$
	\item Matrizengruppen wie $\Sl_n(F),~F$ endlicher Körper, "Gruppenvom Lie-Typ"
	\item 26 sogenannte sporadische einfache endliche Gruppen.\\
	Der Beweis ist ca. 10000 Seiten in vielen Arbeiten lang, ca. 1980er Jahre.
\end{itemize} 
%sub end 
%sec end
\newpage

\section{Kommutative Ringe}
\label{sec:komm_ringe}
\subsection{Erinnerung / Definiton}
\label{sub:erinnerung_def}
Sei $(R,+)$ eine abelsche Gruppe mit Neutralelement $0\in R$. Angenommen, es gibt eine weitere assoziative Verknüpfung auf $R$, die \uline{Multiplikation} $R\times R\to R,~ (a,b)\mapsto a\cdot b=ab$.\\
Weiter gilt: 
\begin{enumerate}[(R1)]
	\item es gelten die \uline{Distributivgesetze},
	\begin{equation*}
	\begin{aligned}
		a(x+y)&= ax+ay\\
		(x+y)a&= xa+ya
	\end{aligned}
	\end{equation*}
	\item Es gibt ein \uline{Einselement} $1\in R$, d.h. 
	\[
	1\cdot x=x=x\cdot 1~\forall x\in R
	\]
	\item $ab=ba$ für alle $a,b\in R$
\end{enumerate}
dann heißt $(R,+,\cdot)$ ein \Index{kommutativer Ring}. Verlangt man nur (R1)\& (R2), spricht man von einem \uline{nicht kommutativem Ring}. Wenn man nur (R1) fordert, spricht man von einem \uline{Ring ohne Eins} oder \Index{Rng} (Jacobsen).

\subsubsection*{Beispiele}
\begin{enumerate}[(a)]
	\item Jeder Körper ist ein Ring, z.B. $\Q,\R,\C$
	\item $\Z$ ist ein Ring (kommutativ).
	\item $V$ ein $K$-Vektorraum, $\End(V)=\{\varphi:V\to V~|~\varphi \text{ linear}\}$\\
	\begin{equation*}
	\begin{aligned}
		\varphi,\psi\in \End(V): &(\varphi+\psi)(v)=\varphi(v)+\psi(v),~v\in V\\
		&(\varphi\circ\psi)(v)=\varphi(\psi(v))
	\end{aligned}
	\end{equation*}
	$\Rightarrow \End(V)$ Ring, nicht kommutativ, falls $\dim(V)\ge 2$.
	\item $m\Z=\{mk~|~k\in \Z \}$ für ein $m\ge 1$\\
	Rng, wenn $m\ge 2$.
	\item $R=\{0\}$ mit $0\cdot 0=0=0+0$ der \uline{Nullring}. Im Nullring gilt $0=1$. 
\end{enumerate}
%sub end

\subsection{Rechenregeln in Ringen}
\label{sub:rechenregeln_ringen}
\begin{enumerate}[(a)]
	\item \uline{Additiv} darf man kürzen:
	\[
	a+x=a+y\Rightarrow x=y
	\]
	(addieren von $-a$ auf beiden Seiten)
	\item Es gilt stets 
	\[
	0\cdot a=a\cdot 0=0
	\]
	\item Es gilt 
	\[
	a(-b)=-(ab)=(-a)b,~ (-a)(-b)=ab \text{ und } (-1)a=-a=a(-1)
	\]
\end{enumerate}

\bet{Beweis:}\\
(b):
\[
0\cdot a =(0+0)a\stackrel{\text{R1}}{=}0a+0a\stackrel{\text{Kürzen}}{\Rightarrow} 0a=0
\]
genauso $a\cdot 0=0$.\\
(c): 
\[
a(-b)+ab\bgl{\text{R1}} a(b-b)=a0=0\Rightarrow a(-b)=-(ab)
\]
genauso
\[
(-a)b+ab=(-a+a)b=0b=0\Rightarrow (-a)b=-(ab)
\]
\[
(-a)(-b)=-(a(-b))=-(-(ab))=ab
\]
sowie
\[
(-1)a=-(1a)=-a=a(-1)
\]
\hfill $\square$\\
\uline{Vorsicht!} Beim Multiplizieren darf man nicht immer einfach kürzen. Beispiel:
\[
a=\begin{pmatrix}1&0\\0&0\end{pmatrix},~x=\begin{pmatrix}1&0\\0&2\end{pmatrix},~y=\begin{pmatrix}1&0\\0&3\end{pmatrix}
\]
$a,x,y\in \R^{2\times 2},~ ax=ay$, aber $x\neq y$.
%sub end

\subsection{Definition Einheiten}
\label{sub:def_einheiten}
Sei $R$ ein Ring. Ein Element $a\in R$ heißt \Index{Einheit}, wenn es $b\in R$ gibt mit 
\[
ab=1=ba
\]
Die Menge aller Einheiten ist die \Index{Einheitengruppe}
\[
R^*=\{a\in R~|~a\text{ Einheit}\}
\]
Offensichtlich ist $(R^*,\cdot)$ eine Gruppe, mit $1$ als Neutralelement.\\

\uline{Beispiel:}
\begin{enumerate}[(a)]
	\item $K$ Körper, $K^*=K\backslash \{0\}$
	\item $\Z^*=\{\pm 1\}$
	\item $\End(V)^*=\Gl(V)=\{\varphi:V\to V~|~\varphi\text{ linear + bijektiv}\}$
	\item $R=\{0\},~R^*=R$
\end{enumerate}
% sub end

\subsection{Homomorphismen und Ideale}
\label{sub:homomor_ideale}
Seien $R$ und $S$ Ringe. Eine Abbildung $\varphi:R\to S$ heißt \Index{Ringhomomorphismus}, wenn für alle $x,y\in R$ gilt:
\begin{enumerate}[(H1)]
	\item $\varphi(x+y)=\varphi(x)+\varphi(y)$
	\item $\varphi(xy)=\varphi(x)\varphi(y)$
	\item $\varphi(1_R)=1_S$
\end{enumerate}
(H1) sagt, dass $\varphi$ ein Homomorphismus der additven Gruppe $(R,+)$ und $(S,+)$ ist.\\
Der \uline{Kern} eines Ringhomomorphismus $\varphi$ ist
\[
\ker(\varphi)=\{x\in R~|~\varphi(x)=0 \}
\]
Ist $R$ ein Ring und $S\subseteq R$ eine Teilmenge mit folgenden Eigenschaften, so heißt $S$ \Index{Teilring} oder \Index{Unterring}
\begin{enumerate}[(TR1)]
	\item $0\in S$ und $x\pm y\in S$ für alle $x,y\in S$
	\item $x\cdot y\in S$ für alle $x,y\in S$
	\item $1\in S$
\end{enumerate}
Wenn nur (TR1) und (TR2) verlangt wird, spricht man von einem "Teilrng".\\
Sei $R$ ein Ring. Ein Teilrng $I\in R$ heißt \Index{Ideal}, wenn für alle $r\in R$ und $i\in I$ gilt
\[
ir\in I \text{ und } ri\in I
\]
Man schreibt $I\trianglelefteq R$.
Für ein Ideal $I\trianglelefteq R$ gilt offensichtlich
\[
I=R\Leftrightarrow 1\in I
\]
(denn: $1\in I\Rightarrow r=r\cdot 1\in I$ für alle$r\in R$.)

\subsubsection*{Konstruktion}
Sei $R$ ein Ring und $I\trianglelefteq R$ Ideal. Dann ist 
\[
\nicefrac{R}{I}=\{x+I~|~x\in R\}
\]
ein Ring mit Multiplikation
\[
(x+I)(y+I)=xy+I
\]
\uline{Denn:} Das ist eine wohldefinierte Verknüpfung,\\
\begin{equation*}
\begin{aligned}
&\begin{array}{c} x+I=x'+I\\ y+I=y'+I    \end{array} \Rightarrow \begin{array}{c} x'=x+i\\y'=y+j \end{array} \text{ für } i,j\in I \Rightarrow x'y'+I=(x+i)(y+j)+I\\
&= xy+\underbracket{iy+xj+ij}_{\in I}+I=xy+I
\end{aligned}
\end{equation*}
Es gilt weiter
\[
(1+I)(x+I)=(x+I)=(x+I)(1+I)
\]
\hfill $\square$

\subsubsection*{Satz}
Sei $R$ ein Ring und $I\subseteq R$. Dann sind äquivalent:
\begin{enumerate}[(i)]
	\item $I\trianglelefteq R$
	\item Es gibt ein Ring $S$ und einen Homomorphismus $R\stackrel{\tiny \varphi}{\to} S$ mit $\ker(\varphi)=I$.
\end{enumerate}

\bet{Beweis:}\\
\uline{(i)$\Rightarrow$(ii):} Setze $S=\nicefrac{R}{I},~\pi_I:R\to S,~x\mapsto x+I$\\
Nach obiger Konstruktion ist $\nicefrac{R}{I}$ ein Ring.
Es gilt 
\[
\ker(\pi_I)=\{x\in R~|~x+I=I \}=I
\]
\uline{(ii)$\Rightarrow$(i):} Sei $\varphi:R\to S$ ein Ringhomomorphismus mit $I=\ker(\varphi)$. Dann ist $(I,+)$ Untergruppe von $(R,+)$. Für alle $i\in I,~r\in R$ gilt
\[
\left.\begin{array}{c} \varphi(ir)=\varphi(i)\varphi(r)=0_S\cdot \varphi(r)=0_S\\ \text{und } \varphi(ri)=\dots=0_S    \end{array}\right\} \Rightarrow ir,ri\in I
\]
\hfill $\square$
%sub end

\subsection{Homomorphiesatz für Ringe, Isomorphiesätze}
\label{sub:homosatz_isosatz}
\subsubsection*{Satz (Homomorphiesatz)}
Sei $R\stackrel{\varphi}{\to}S$ ein Ringhomomorphismus, sei $I\trianglelefteq R$ Ideal mit $I\subseteq \ker(\varphi)$. 
Dann gibt es genau ein Ringhomomorphismus $\overline{\varphi}:\nicefrac{R}{I}\to S$ mit $\overline{\varphi}\circ \pi_I=\varphi$
\begin{center}
	\begin{tikzcd}[column sep=small]
		R \ar{rr}{\varphi} \ar{rd}[below,left]{\pi_I} & & S\\
		& \nicefrac{R}{I} \ar{ru}[below,right]{\overline{\varphi}} &
	\end{tikzcd}
	\captionof{figure}{Homomorphiesatz für Ringe}
\end{center}

\bet{Beweis:}\\
Aus dem Isomorphiesatz für Gruppen \hyperref[sub:der_homomorphiesatz]{1.20} angewandt auf den Gruppenhomomorphismus $(R,+)\stackrel{\varphi}{\to} (S,+)$ erhalten wir die Existenz und Eindeutigkeit des Gruppenhomomorphismus $\overline{\varphi}$. Zu zeigen bleibt, dass$\overline{\varphi}$ ein Ringhomomorphismus ist. Für $x\in R$ gilt
\[
\overline{\varphi}(x+I)=\varphi(x)\qquad \text{vgl. \hyperref[sub:der_homomorphiesatz]{1.20}}
\]
\[
\overline{\varphi}(xy*I)=\varphi(xy)\bgl{\varphi\text{ Ringhom.}}\varphi(x)\varphi(y)=\overline{\varphi}(x+I)\overline{\varphi}(y+I)
\]
sowie
\[
\overline{\varphi}(1_R+I)=\varphi(1_R)=1_S
\]
\hfill $\square$

\subsubsection*{Satz (1. Isomorphiesatz für Ringe)}
Sei $R$ ein Ring, $S\subseteq R$ Teilring und $I\trianglelefteq R$ ein Ideal. Dann ist $S+I=\{s+i~|~s\in S,~i\in I\}\subseteq R$ Teilring und $S\cap I\trianglelefteq S$ Ideal. Die Abbildung
\[
\nicefrac{s}{s\cap I}\stackrel{\varphi}{\to}\nicefrac{S+I}{I},~s+S\cap I\mapsto s+I
\]
ist ein Ringisomorphismus (bijektiver Ringhomomorphismus).\\

\bet{Beweis:}\\
Klar: $S+I$ und $S\cap I$ sind Untergruppen in $(R,+)$.
Für $s,s'\in S,~i,i'\in I$ gilt
\[
(s+i)(s'+i')=ss'+\underbracket{is'+si+ii'}_{\in I}\in S+I
\]
sowie $1\in S\subseteq S+I\Rightarrow S+I\subseteq R$ ist Teilring.
Für $s\in S,~i\in I\cap S$ gilt $\left.\begin{array}{c} is\in I\cap S\\ si\in I\cap S \end{array}\right\} \Rightarrow I\cap S \trianglelefteq S$.\\
Die Abbildung $\varphi:s+S\cap I \mapsto s+I$ ist nach \hyperref[sub:isomorphisaetze]{1.23} ein Gruppenisomorphismus bzgl. der Addition. Es gilt $\varphi(1+S\cap I)=1+I$ sowie für $s,t\in S$
\[
\varphi(st+I\cap S)=st+I=(s+I)(t+I)=\varphi(s+I\cap S)\varphi(t+I\cap s)
\]
\hfill $\square$

\subsubsection*{Satz (2. Isomorphiesatz für Ringe)}
Sei $R$ ein Ring, $I,J\trianglelefteq R$ Ideale mit $I\subseteq J$. Dann ist 
\[
\nicefrac{J}{I}=\{j+I~|~j\in J \}\subseteq \nicefrac{R}{I}
\]
ein Ideal un es gibt 
\[
\nicefrac{\nicefrac{R}{I}}{\nicefrac{J}{I}} \stackrel{\cong}{\to} \nicefrac{R}{J}
\]
einen Ringisomorphismus.\\

\bet{Beweis:}\\
Genau wie in \hyperref[sub:isomorphiesaetze]{1.23}. Betrachte $\psi:R\to\nicefrac{R}{J},~x\mapsto x+J \rightsquigarrow \text{ Homomorphismus }\overline{\psi}:\nicefrac{R}{I}\to\nicefrac{R}{J}$ (Homomorphiesatz). $\ker(\overline{\psi}=\nicefrac{J}{I})$, also existiert der Ringisomorphismus.
\hfill $\square$

\subsubsection*{Bemerkung}
Ein \Index{Ringisomorphismus} ist also ein bijektiver Ringhomomorphismus $\varphi:R\to S$. 
Die Umkehrabbildung $\psi$ von $\varphi$, $\psi:S\to R$ ist dann ebenfalls ein Ringhomomorphismus (Ringisomorphismus).

%sub end

\subsection{Rechnen mit Idealen}
\label{sub:rechnen_ideale}
Sei $R$ ein Ring mit Idealen $I,J\trianglelefteq R$. Dann sind auch die folgenden Mengen Ideale:
\begin{enumerate}[(a)]
	\item $I+J=\{i+j~|~i\in I,~j\in J\}$
	\item $I\cap J$
	\item $IJ=\{i_1j_1+i_2j_2+\dots+i_lj_l~|~l\ge 1,~i_1,\dots,i_l\in I,~j_1,\dots,j_l\in J \}$
\end{enumerate}
Es gilt
\[
IJ\subseteq I\cap J\subseteq I,J\subseteq I+J
\]

\bet{Beweis:}\\
Klar: $I+J,~I\cap J$ und $IJ$ sind additive Gruppen.Sei $r\in R,~i\in I,~j\in J$. 
Es folgt
\[
r(i+j)=\underbracket{ri+rj}_{\in I+J}
\]
\[
(i+j)r=\underbracket{ir+jr}_{\in I+J}\Rightarrow I+J\trianglelefteq R
\]
\[
i\in I\cap J \Rightarrow ri\in I\cap J\Rightarrow I\cap J\trianglelefteq R
\]
\[
r(ij)=\underbracket{ri}_{\in I}\cdot j\in J \text{ genauso } r(ij)\in I
\]
also $IJ\trianglelefteq R$ und $IJ\subseteq I\cap J$.
\hfill $\square$
%sub end

\subsection{Beispiele Ideale}
\label{sub:bsp_ideale}
\begin{enumerate}[(a)]
	\item $K$ ein Körper. Ist $I\trianglelefteq K$ Ideal und $I\neq \{0\}$, so folgt $1\in I$, denn:
	\[
	i\in I\backslash\{0\}\Rightarrow i^{-1}i=1\in I \Rightarrow I=K.
	\]
	Also sind $\{0\}$ und $K$ die einzigen Ideale in $K$.
	\item $V\neq \{0\}$ ein $K$-Vektorraum, $R=\End(V)$. Die einzigen Ideale in $R$ sind $\{0\},R$ ($\rightsquigarrow$ höhere Algebra?)
	\item $R$ kommutativer Ring, $a\in R$. Setze $(a)=Ra=\{ra~|~r\in R\}$. Dann gilt $(a)\trianglelefteq R$ (später genauer).
	\item $R=\Z$. Wir zeigen gleich: jedes Ideal $I\trianglelefteq \Z$ ist von der Form $I=m\Z=\{mk~|~k\in \Z\}$ für ein $m\in \N$. Als Quotient erhältman für $m\ge 1$
	\[
	\nicefrac{\Z}{m}=\nicefrac{\Z}{m\Z}=\{\overline{0},\dots,\overline{m-1},\overline{m}=\overline{0} \}
	\]
	$\overline{k}=k+m\Z$ (die Bedeutung des Querstrichs hängt also vom $m$ ab!)\\
	$\overline{k}\cdot\overline{l}=\overline{kl}$ nach \hyperref[sub:homomor_ideale]{3.4}. Also ist für $m\ge 1$ $\nicefrac{\Z}{m}$ ein kommutativer Ring mit $m$ Elementen.
\end{enumerate}


















\cleardoubleoddemptypage
\pagenumbering{Alph}
\setcounter{page}{1}




\printindex
\listoffigures
\end{document}