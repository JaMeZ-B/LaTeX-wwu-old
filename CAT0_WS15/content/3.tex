\chapter{Gruppenwirkungen auf kubischen Komplexen} % (fold)
\label{cha:3}

\subsection*{Motivation}
\begin{enumerate}[(i)]
	\item Spezieller Fall des Satzes von \textsc{Cartan-Hadamard}: \marginnote{15.01. \\ \ [16]}
	
	Sei $X$ ein vollständiger einfach-zusammenhängender lokaler $\CAT$-Raum.
	Dann ist $X$ $\CAT$. Also:
	
	\[
		\left. \begin{array}{l}
			\text{lokale Eigenschaft: lokal } \CAT \\
			\text{globale Eigenschaft: einfach zusammenhängend}
		\end{array} \right\} \quad \Rightarrow \quad \text{global } \CAT
	\]
	Frage: Wie kann man die lokale $\CAT$-Eigenschaft testen?
	
	Für eine spezielle Klasse von Räumen -- kubische Komplexe -- kann man diese lokale Eigenschaft in eine kombinatorische Eigenschaft übersetzen und somit leichter testen ($\Rightarrow$ \textsc{Gromov}'s Link Condition).
	\item Fragestellung: Wann besitzt die simpliziale Wirkung einer endlich erzeugten Gruppe $G$ auf einen $\CAT$ kubischen Komplex einen Fixpunkt?
	
	Für $\CAT$ kubische Komplexe werden wir eine Methode sehen, um diese Fragestellung anzugehen.
\end{enumerate}

\section{Kubische Komplexe}
\label{sec:3.1}
	Grobe Idee: Wir bauen einen topologischen Raum aus Würfeln.\\
	Als Konvention legen wir fest: $[0,1]^0 = \{0\}$.
	
\begin{definition}[Würfel, Seite]
\label{def:3.1}
	Sei $W:= [0,1]^n \subseteq (\RR^n,d_2)$ ein \Index{Würfel}.
	Eine \Index{Seite} $S \subseteq W$ ist gegeben durch
	\[
		S = S_1 \times \dots \times S_n \text{ mit } S_i \in \{ \{0\}, \{1\}, [0,1]\}.
	\]
	Die eingeschränkte euklidische Metrik auf $W$ bezeichnen wir mit $d_W$.
	Weiter definieren wir die \Index{Dimension} von $W$ wie folgt:
	\[
		\dim(W) := \dim(\sprod{W}) \text{ mit } \sprod{W} \text{ als } \RR-\text{Vektorraum}.
	\]
\end{definition}	
\newpage
\begin{beispiel}
\label{bsp:3.2}
	\mbox{} \\[-1cm]
	\begin{figure}[h]
		\centering
		\begin{tikzpicture}[scale=2,>=Latex]
			\draw [gray,thick,->] (-.1,0) -- (1.2,0);
			\draw [gray,thick,->] (0,-.1) -- (0,1.2);
			\draw [schraffiert=teal,thick] (0,0) -- (1,0) -- (1,1) -- (0,1) -- cycle;
			
			\draw [color=red,very thick] (0,1) -- (1,1);
			\draw [color=red] (1,0) node[fill,circle,inner sep=1.5pt]{};
			\draw [color=red] (1,1) node[right]{$[0,1] \times \{1\}$};
			\draw [color=red] (1,0) node[below]{$\{1\} \times \{0\}$};
		\end{tikzpicture}
		\caption{Zwei Seiten des Würfels $W = [0,1]^2$.}
	\end{figure}
\end{beispiel}

\begin{definition}[Verklebung, kubischer Komplex]
\label{def:3.3}
	Seien $W_1,W_2$ zwei Würfel und $S_1 \subseteq W_1$ und $S_2 \subseteq W_2$ Seiten.
	Eine Isometrie $\varphi \colon S_1 \rightarrow S_2$ heißt \Index{Verklebung} von $W_1$ und $W_2$.
	Sei nun $\mathcal{W}$ eine Familie von Würfeln und $\mathcal{V}$ eine Familie von Verklebungen von Würfeln aus $\mathcal{W}$ mit folgenden Eigenschaften:
	\begin{enumerate}[(i)]
		\item Kein Würfel ist mit sich selbst verklebt.
		\item Je zwei Würfel aus $\mathcal{W}$ sind höchstens einmal miteinander verklebt.
	\end{enumerate}
	Sei $\sim$ die durch
	\[
		x \sim y \quad :\Leftrightarrow \quad \text{es existiert ein } \varphi \in \mathcal{V} \text{ mit } x \in \dom(\varphi) \text{ und } \varphi(x) = y
	\]
	erzeugte Äquivalenzrelation auf $\bigcup_{W \in \mathcal{W}} W$. Die Menge
	\[
		X := \faktor{\enbrace*{ \bigcup_{W \in \mathcal{W}} W}}{\sim}
	\]
	heißt kubischer Komplex definiert durch $(\mathcal{W},\mathcal{V})$.
	Die Dimension von $X$ ist definiert durch
	\[
		\dim(X) := \sup\{ \dim(W) : W \in \mathcal{W}\}.
	\]
\end{definition}

\begin{beispiel}
\label{bsp:3.4}
	\mbox{} \\[-1.4cm]
	\begin{align*}
		\mathcal{W} := \ &\{W_1 = [0,1]^2, W_2 = [0,1]^2, W_3 = [0,1]\} \\
		\mathcal{V} := \ &\{\varphi_1 \colon \{0\} \times \{0\} \subseteq W_1 \rightarrow \{1\} \times \{1\} \subseteq W_2, \varphi_2 \colon [0,1] \times \{1\} \subseteq W_2 \rightarrow [0,1] = W_3\}
	\end{align*}
\end{beispiel}
\newpage
\begin{figure}[h]
	\centering
	\begin{tikzpicture}[scale=1.5,>=Latex]
		\draw [thick,schraffiert=teal] (0,0) -- (1,0) -- (1,1) -- (0,1) -- cycle;
		\draw [thick,schraffiert=teal] (2,0) -- (3,0) -- (3,1) -- (2,1) -- cycle;
		\draw [very thick,color=red] (4,0) -- (5,0);
		\draw [very thick,color=red] (2,1) -- (3,1);
		
		\draw (.5,0) node[below]{$W_1$};
		\draw (2.5,0) node[below]{$W_2$};
		\draw (4.5,0) node[below]{$W_3$};
		
		\draw [color=Green4,->] (0,0) .. controls (-1,1) and (1,2.5) ..   (3,1);
		\draw [color=Green4,->] (2.5,1) to[out=90,in=90] (4.5,0);
		
		\draw [color=RoyalBlue2] (0,0) node[fill,circle,inner sep=2pt]{};
		\draw [color=RoyalBlue2] (3,1) node[fill,circle,inner sep=2pt]{};
		\draw [color=Green4] (4,1) node[right]{$\varphi_2$};
		\draw [color=Green4] (1,1.6) node[above]{$\varphi_1$};
	\end{tikzpicture} \hspace{1.5cm}
	\begin{tikzpicture}[scale=1.5,>=Latex]
		\draw (0,1.8) node[right]{$\mathbf{X}$};
		\draw [thick,schraffiert=teal] (0,0) -- (1,0) -- (1,1) -- (0,1) -- cycle;
		\draw [thick,schraffiert=teal] (1,1) -- (2,1) -- (2,2) -- (1,2) -- cycle;
		\draw [very thick,red] (0,1) -- (1,1);
		\draw [color=RoyalBlue2] (1,1) node[fill,circle,inner sep=2pt]{};
	\end{tikzpicture}
\end{figure}

\begin{definition}
\label{def:3.5}
	Seien $x,y \in X$ beliebig.
	Ein Weg $c$ von $x$ nach $y$ in $X$ ist definiert als eine endliche Folge $c=(x_0,\dots,x_m), x_i \in X$ mit $x_0 = x$, $x_m = y$ und folgender Eigenschaft:
	Für alle $i \in \{0,\dots,m-1\}$ existiert ein Würfel $W_i \in \mathcal{W}$ mit $x_i,x_{i+1} \in W_i$.
	
	Weiter definieren wir die Länge von $c$ als
	\[
		\ell(c) = \ell((x_0,\dots,x_m)) = \sum\limits_{i=0}^{m-1} d_{W_i} (x_i,x_{i+1}).
	\]
	$X$ heißt \textbf{wegzusammenhängend}, wenn es für je zwei Elemente aus $X$ ein solcher Weg existiert. \index{zusammenhängend!wegzusammenhängend}
	
	Nun definieren wir auf $X$ wegzusammenhängend eine Metrik:
	\begin{align*}
		d_X \colon X \times X &\longrightarrow \RR_{\geq 0} \\
		(x,y) &\longmapsto \inf\{ \ell(c) : c \text{ ist ein Weg von } x \text{ nach } y\}
	\end{align*}
\end{definition}

Ab jetzt betrachten wir eine wegzusammenhängenden kubischen Komplex immer mit der Metrik $d_X$.

\begin{beispiel}
\label{bsp:3.6}
	$c=(x,x_1,x_2,y)$ ist ein Weg von $x$ nach $y$.
	\begin{figure}[h]
		\centering
		\begin{tikzpicture}[scale=1,>=Latex]
			\draw [thick] (.5,3) node[fill,circle,inner sep=1.5pt]{}
			-- (1.5,2.5) node[fill,circle,inner sep=1.5pt]{}
			-- (2.4,3.5) node[fill,circle,inner sep=1.5pt]{}
			-- (1.5,2.5)
			to[pos=.6] coordinate (X) (2,1.5) node[fill,circle,inner sep=1.5pt,color=RoyalBlue2]{}
			-- (1,1) node[fill,circle,inner sep=1.5pt]{}
			-- (2,1.5)
			-- (3.5,1) node[fill,circle,inner sep=1.5pt,color=RoyalBlue2]{}
			-- (3,0) node[fill,circle,inner sep=1.5pt]{}
			-- (3.5,1)
			to[pos=.6] coordinate (Y) (4.5,.25) node[fill,circle,inner sep=1.5pt]{};
			
			\draw [very thick,color=RoyalBlue2] (X) -- (2,1.5) -- (3.5,1) -- (Y);
			\draw [color=RoyalBlue2] (X) node[left]{$x$};
			\draw [color=RoyalBlue2] (2,1.5) node[anchor=south west]{$x_1$};
			\draw [color=RoyalBlue2] (3.5,1) node[above]{$x_2$};
			\draw [color=RoyalBlue2] (Y) node[anchor=south west]{$y$};
			\draw (X) node[fill,circle,inner sep=1.5pt,color=RoyalBlue2]{};
			\draw (Y) node[fill,circle,inner sep=1.5pt,color=RoyalBlue2]{};
		\end{tikzpicture}
	\end{figure}
\end{beispiel}
\newpage
\subsection{Simpliziale Bäume}
\label{subsec:3.1.1}

Erinnerung:
Sei $\Gamma = (V,E)$ ein ungerichteter Graph.
$\Gamma$ heißt \textbf{(simplizialer) Baum}, wenn $\Gamma$ wegzusammenhängend ist und $\Gamma$ keine Kreise hat. \index{Baum}

Bäume haben eine natürliche Struktur als $1$-dimensionale kubische Komplexe.
Wir betrachten nun Bäume mit dieser kubischen Struktur.

\begin{satz}
\label{satz:3.7}
	Bäume sind vollständige $\CAT$-Räume. (\autoref{aufg:7.3})
\end{satz}

Fragestellung:
Sei $G$ eine endlich erzeugte Gruppe und $\Phi\colon G \rightarrow \Isom(X)$ eine isometrische Wirkung auf einem Baum. \marginnote{20.01. \\ \ [17]}
Hat $\Phi$ einen globalen Fixpunkt, das heißt existiert ein $x \in X$, sodass für alle $g \in G$ gilt: $\Phi(g)(x)=x$?

Wir betrachten simpliziale Wirkungen, das heißt
\[
	\Phi(G) \subseteq \SIsom(X) := \{f\colon X \rightarrow X : f \text{ ist Isometrie und } f(\text{Ecke}) = \text{ Ecke' für alle Ecken in } X\}
\]
Dies ist tatsächlich eine Einschränkung: Für den Baum $X=\RR$ ist $f\colon X \rightarrow X, x \mapsto x + \frac{1}{2}$ eine Isometrie, aber keine simpliziale Isometrie.

Wir schränken das Bild von $\Phi$ noch ein wenig ein:
Wir betrachten nur simpliziale Isometrien ohne Inversionen.
Eine simpliziale Isometrie $f \colon X \rightarrow X$ hat eine Inversion, wenn folgendes passiert: \begin{figure}[h]
	\centering
	\begin{tikzpicture}[scale=1.5,>=Latex]
		\draw [ultra thick] (0,0) node[below]{$v_1$} -- (1.7,0) node[below]{$v_2$};
		\draw (0,0) node[fill,circle,inner sep=2pt]{};
		\draw (1.7,0) node[fill,circle,inner sep=2pt]{};
		
		\draw [->,thick] (2.2,0) to [bend angle=30,bend left] coordinate[pos=.5] (A) (3.8,0);
		\draw (A) node[above]{$f$};
		
		\draw [ultra thick] (4.3,0) node[below]{$v_2$} -- (6,0) node[below]{$v_1$};
		\draw (4.3,0) node[fill,circle,inner sep=2pt]{};
		\draw (6,0) node[fill,circle,inner sep=2pt]{};
	\end{tikzpicture}
\end{figure}

Eine Methode, um die Frage zu beantworten, ist das Theorem von \textsc{Helly} für simpliziale Bäume.

\begin{no-satz}[\textsc{Helly}, 1923]
	Sei $\mathcal{S} := \{S_1, \dots, S_l : S_i \subseteq \RR^d, S_i \neq \emptyset, S_i \text{ ist abgeschlossen und konvex für } i = 1, \dots, l\}$ eine endliche Familie.
	Wenn sich jeweils $(d+1)$ Elemente aus $\mathcal{S}$ nichttrivial schneiden, dann ist $\cap \mathcal{S}$ nichtleer.
\end{no-satz}

\begin{no-bem}
	$d+1$ ist eine minimale Schranke, denn zum Beispiel im $\RR^2$ ist $S_1 \cap S_2, S_2 \cap S_3, S_1 \cap S_3 \neq \emptyset$ und $S_1 \cap S_2 \cap S_3 = \emptyset$ möglich.
	
	\begin{figure}[h]
		\centering
		\begin{tikzpicture}[scale=.5,>=Latex]
			\draw [->,very thick] (-1,0) -- (4.5,0);
			\draw [->,very thick] (0,-1) -- (0,3.5);
			
			\draw (-.8,2) -- (4,2) node[right]{$S_3$};
			\draw (-1,2.5) -- (2.5,-1) node[right]{$S_2$};
			\draw (-.5,-1) -- (3.5,3) node[right]{$S_1$};
		\end{tikzpicture}
	\end{figure}
\end{no-bem}

Gibt es eine Verallgemeinerung des Theorems für $\CAT$-Räume?
Ja, zum Beispiel für Bäume:

\begin{no-satz}[\textsc{Helly}s Theorem für Bäume]
	Sei $X$ ein Baum und $\mathcal{S} = \{S_1,\dots,S_l\}$ eine endliche Familie von Teilbäumen.
	Wenn sich jeweils zwei Elemente aus $\mathcal{S}$ nichttrivial schneiden, dann ist $\cap \mathcal{S}$ nicht leer. (\autoref{aufg:7.2})
\end{no-satz}

\begin{satz}
\label{satz:3.8}
	Sei $G = \sprod{g_1,\dots,g_k}$ eine endlich erzeugte Gruppe.
	Sei $X$ ein Baum und $\Phi\colon G \rightarrow \SIsom(X)$ eine simpliziale Wirkung ohne Inversionen.
	Wenn $\Fix_\Phi(\sprod{g_i}) \neq \emptyset$ und $\Fix_\Phi(\sprod{g_i}) \cap \Fix_\Phi(\sprod{g_j}) \neq \emptyset$ für alle $i,j \in \{1,\dots,k\}$, dann gilt $\Fix_\Phi(G) \neq \emptyset$.
\end{satz}

\begin{beweis}
	$\Fix_\Phi(\sprod{g_i}) \subseteq X$ ist ein Teilbaum von $X$.
	Mit \textsc{Helly}s Theorem für Bäume folgt $\emptyset \neq \cap \Fix_\Phi(\sprod{g_i}) = \Fix_\Phi(\sprod{g_1,\dots,g_k}) = \Fix_\Phi(G)$. \qedhere
\end{beweis}

\begin{satz}[Verallgemeinerung]
\label{satz:3.9}
	Sei $G = \sprod{g_1,\dots,g_k}$ eine endlich erzeugte Gruppe und $X$ ein Baum.
	Sei $\Phi\colon G \rightarrow \Isom(X)$ eine isometrische Wirkung.
	Wenn $\Fix_\Phi(\sprod{g_i}) \neq \emptyset$ und $\Fix_\Phi(\sprod{g_i}) \cap \Fix_\Phi(\sprod{g_j}) \neq \emptyset$ für alle $i,j \in \{1,\dots,k\}$, dann ist $\Fix_\Phi(G) \neq \emptyset$. \autoref{aufg:7.2}
\end{satz}

\begin{beispiel}
\label{bsp:3.10}
	Für $n \geq 3$ sei $F_n = \sprod{x_1, \dots, x_n}$ die freie Gruppe vom Rang $n$ und $\Aut(F_n)$ der Automorphismengruppe von $F_n$. \index{Automorphismus}
	Für zwei Automorphismen $\alpha, \beta \in \Aut(F_n)$ ist $\alpha \beta := \beta \circ \alpha$.
	Wir definieren Rechtsnielsenautomorphismen für $i,j \in \{1, \dots, n\}, i \neq j$.
	\begin{align*}
		\rho_{i,j} \colon F_n &\longrightarrow F_n \\
		x_k &\longmapsto \begin{cases}
			x_ix_j, & \text{falls } k=i \\
			x_k, & \text{falls } k \neq i
		\end{cases}
	\end{align*}
	Weiter definieren wir die Involution
	\begin{align*}
		(x_i,x_j)\colon F_n &\longrightarrow F_n \\
			s_k &\longmapsto \begin{cases}
				x_j, & \text{falls } k=i \\
				x_i, & \text{falls } k=j \\
				x_k, & \text{sonst}
			\end{cases}
	\end{align*}
	und
	\begin{align*}
		e_i \colon F_n &\longrightarrow F_n \\
		x_k &\longmapsto \begin{cases}
			x_i^{-1}, & \text{falls } k = i \\
			x_k, &\text{falls } k \neq i.
		\end{cases}
	\end{align*}
	Es gilt $\rho_{i,j}, (x_i,x_j), e_i \in \Aut(F_n)$, und es ist
	\[
		\Aut(F_n) = \sprod{\Underbrace{(x_1,x_2)e_1e_2, (x_2,x_3)e_1, (x_i,x_{i+1}), e_2\rho_{1,2},e_n : i = 3, \dots n-1}{=:Y}}
	\]
	Es ist $\ord(\alpha) = 2$ für $\alpha \in Y$ und $\abs{\sprod{\alpha,\beta}} < \infty$ für $\alpha,\beta \in Y$.
	
	Sei nun $X$ ein Baum und $\Phi\colon \Aut(F_n) \rightarrow \Isom(X)$ eine isometrische Wirkung.
	Dann ist $\Fix_\Phi(\Aut(F_n)) \neq \emptyset$, denn:
	\begin{itemize}
		\item Nach \textsc{Bruhat-Tits} (\autoref{BTFT}) ist $\Fix_\Phi(\sprod{\alpha}) \neq \emptyset$, da Bäume $\CAT$ und vollständig sind.
		\item Ebenso ist $\Fix_\Phi(\sprod{\alpha}) \cap \Fix_\Phi(\sprod{\beta}) \neq \emptyset$ für $\alpha, \beta \in Y$ beliebig.
		\item Mit \autoref{satz:3.9} folgt somit $\Fix_\Phi(\Aut(F_n)) \neq \emptyset$.
	\end{itemize}
\end{beispiel}

\begin{definition}[Eigenschaft $\prF{X}$]
\label{def:3.11}
	Sei  $\mathcal{X}$ eine Klasse von metrischen Räumen. \index{Eigenschaft $\prF{X}$}
	Eine Gruppe $G$ hat die Eigenschaft $\prF{X}$, wenn jede isometrische Wirkung auf jedem Raum aus $\mathcal{X}$ einen globalen Fixpunkt hat.
\end{definition}

\begin{beispiel}[\textsc{Bogopolski}, 1973]
\label{bsp:3.12}
	Sei $\mathcal{A}$ die Klasse der Bäume.
	Für $n \geq 3$ hat $\Aut(F_n)$ die Eigenschaft $\prF{A}$. 
\end{beispiel}

Frage:
Sei $G \leq \Aut(F_n), n \geq 3$ eine Untergruppe mit $[\Aut(F_n) : G] < \infty$.
Hat $G$ die Eigenschaft $\text{F}\mathcal{A}$?
Für $n = 3$ lautet die Antwort nein, für $n \geq 4$ ist die Frage bislang ungeklärt.

Eine Anwendung der Eigenschaft $\prF{A}$ ist die Strukturtheorie:
Sei $G$ eine endlich erzeugte Gruppe.
Kann man $G$ in \enquote{einfache} Bestandteile zerlegen?
\begin{itemize}
	\item Zum Beispiel in ein kartesisches Produkt $G \simeq G_1 \times G_2$.
	\item Zum Beispiel in ein amalgamiertes Produkt $G \simeq G_1 *_H G_2$.
\end{itemize}

\begin{no-satz}[\textsc{Serre}]
	Sei $G$ eine endlich erzeugte Gruppe.
	Wenn $G$ die Eigenschaft $\text{F}\mathcal{A}$ hat, dann ist $G$ kein nichttriviales amalgamiertes Produkt.
\end{no-satz}

Im Seminar im Sommersemester 2016 werden wir sehen, dass $\Aut(F_n)$ für $n \geq 3$ kein nichttriviales amalgamiertes Produkt ist.

\subsection{\textsc{Helly}s Theorem für CAT(0) kubische Komplexe}
\label{subsec:3.1.2}

\begin{definition}
\label{def:3.13}
	Sei $X$ ein kubischer Komplex definiert durch $(\mathcal{W},\mathcal{V})$.
	Eine Teilmenge $Y \subseteq X$ heißt \textbf{kubischer Unterkomplex}, wenn es eine Teilfamilie $\mathcal{W'} \subseteq \mathcal{W}$ existiert, sodass
	\[
		Y = \bigsqcup_{W \in \mathcal{W}'} W \diagup \sim.
	\]
	Dabei ist $\sim$ erzeugt durch
	\[
		x \sim y \quad :\Leftrightarrow \quad \text{es existiert ein } \varphi \colon S_1 \rightarrow S_2 \in \mathcal{V} \text{ mit } S_1,S_2 \text{ sind Seiten von } W_1,W_2 \in \mathcal{W'} \text{ und } \varphi(x) = y
	\]
\end{definition}

\begin{bemerkung}
\label{bem:3.14}
	Wir setzen nicht voraus, dass kubische Unterkomplexe wegzusammenhängend sind.
	\begin{figure}[h]
		\centering
		\begin{tikzpicture}[scale=1,>=Latex]
			\draw (0,2.5) node[right]{$\mathbf{X}$};
			\draw [very thick] (2,0) -- (1,0) -- (1,1) -- (0,1) -- (0,2) -- (1,2) -- (1,1) -- (2,1) -- (2,2) -- (1,2) -- (1.5,2.5) -- (2.5,2.5) -- (2,2);
			\draw [thick] (2.5,2.5) -- (2.5,1.5) -- (2,1) -- (2,0) -- (3,0);
			
			\draw (6,2.5) node[right]{$\mathbf{Y}$};
			\draw [very thick] (6,1) -- (7,1) -- (7,2) -- (6,2) -- cycle;
			\draw [very thick] (8,0) -- (9,0);
		\end{tikzpicture}
	\end{figure}
\end{bemerkung}

\begin{theorem}[\textsc{Helly}s Theorem für $\CAT$ kubische Komplexe]
\label{thm:3.15}
	Sei $X$ ein $\CAT$ kubischer Komplex und $\mathcal{S}$ eine endliche Familie von nichtleeren konvexen kubischen Unterkomplexen.
	Wenn der Schnitt von jeweils zwei Elementen aus $\mathcal{S}$ nicht leer ist, dann ist $\cap \mathcal{S}$ nicht leer.
\end{theorem}