\chapter{Übungsaufgaben}
\label{cha:aufg}
	\begin{aufgabe}
	\label{aufg:1.1}
		Zeigen Sie für metrische Räume $X_1, X_2$:
		\begin{enumerate}[(i)]
			\item $X_1 \times X_2$ ist geodätisch genau dann, wenn $X_1$ und $X_2$ geodätisch sind.
			\item $X_1 \times X_2$ ist $\CAT$ genau dann, wenn $X_1$ und $X_2$ $\CAT$ sind.
		\end{enumerate}
	\end{aufgabe}
	
	\begin{aufgabe}
	\label{aufg:1.2}	
		Sei $(X,d)$ ein geodätischer Raum.
		Die Metrik $d$ heißt konvex, wenn für je zwei Geodäten $\alpha,\beta \colon [0,1] \rightarrow X$ die Ungleichung
		\[
			d(\alpha(t),\beta(t)) \leq (1-t) \cdot d(\alpha(0),\beta(0)) + t \cdot d(\alpha(1),\beta(1))
		\]
		für alle $t \in [0,1]$ gilt.
		Zeigen Sie, dass die Metrik eines $\CAT$-Raumes konvex ist. \\
		\textit{Hinweis}: Zeigen Sie die Aussage zunächst für den Spezialfall $\alpha(0) = \beta(0)$.
	\end{aufgabe}
	
	\begin{aufgabe}
	\label{aufg:1.3}	
		Sei $(X,d)$ ein vollständiger metrischer Raum.
		Für alle $x,y \in X$ existiere ein Mittelpunkt, das heißt ein $m_{xy} \in X$, sodass
		\[
			d(x,m_{xy}) = d(m_{xy},y) = \frac{1}{2} d(x,y)
		\]
		gilt. Zeigen Sie, dass $(X,d)$ ein geodätischer Raum ist.
	\end{aufgabe}
	
	\begin{aufgabe}
	\label{aufg:1.4}	
		Sei $(X,d)$ ein $\CAT$-Raum, $C \subseteq X$ eine konvexe vollständige Teilmenge und sei $\pi_C\colon X \rightarrow C$ die Projektion auf $C$.
		Zeigen Sie, dass für alle $x,y \in X$ gilt:
		\[
			d(\pi_C(x),\pi_C(y)) \leq d(x,y).
		\]
	\end{aufgabe}
	\begin{aufgabe}
	\label{aufg:2.1}	
		Sei $X$ ein metrischer Raum.
		Eine Abbildung $\gamma \colon [0,l] \rightarrow X$ heißt \textbf{lokale Geodäte}, wenn gilt:
		Für alle $t \in [0,l]$ existiert ein $\varepsilon > 0$, sodass $\gamma \big|_{[t-\varepsilon,t+\varepsilon]}$ eine Geodäte ist. \index{Geodäte!lokal}
		
		Sei $X$ ein $\CAT$-Raum und $\gamma\colon [0,l] \rightarrow X$ eine lokale Geodäte.
		Zeigen Sie, dass $\gamma$ eine Geodäte ist. \\
		\textit{Hinweis:} Zeigen Sie, dass folgende Menge offen und abgeschlossen in $[0,l]$ ist.
		\[
			S := \{t \in [0,l] : \gamma\big|_{[0,t]} \text{ ist Geodäte}\}.
		\]
	\end{aufgabe}
	
	\begin{aufgabe}
	\label{aufg:2.2}	
		Sei $X$ ein $\CAT$-Raum und $\Phi\colon G \rightarrow \Isom(X)$ eine isometrische Wirkung.
		Zeigen Sie:
		\begin{enumerate}[(i)]
			\item Die Teilmenge $\Fix_\Phi(G) \subseteq X$ ist abgeschlossen.
			\item Wenn $\Fix_\Phi(G) \neq \emptyset$, dann ist $\Fix_\Phi(G)$ abgeschlossen.
		\end{enumerate}
	\end{aufgabe}
	
	\begin{aufgabe}
	\label{aufg:2.3}	
		Sei $X$ ein vollständiger $\CAT$-Raum, seien $A_1,A_2 \subseteq X$ beschränkte Teilmengen mit
		\[
			A_1 \subseteq \ol{B_{\rad(A_1)}(c_{A_1})}, \qquad A_2 \subseteq \ol{B_{\rad(A_2)}(c_{A_2})},
		\]
		wobei $c_{A_1}, c_{A_2} \in X$.
		Zeigen Sie: Wenn $A_1 \subseteq A_2$ gilt, dann ist
		\[
			d(c_{A_1},c_{A_2}) \leq \sqrt{2 \cdot (\rad(A_2)^2 - \rad(A_1)^2)}.
		\]
	\end{aufgabe}
	
	\begin{aufgabe}
		\label{aufg:2.4}	
		Sei $X$ ein metrischer Raum.
		Der Raum $X$ heißt \Index{sphärisch vollständig}, wenn für alle Folgen $(x_n)_{n \in \NN}$ in $X$ und $(r_n)_{n \in \NN}$ in $\RR_{>0}$ mit $\ol{B_{r_{n+1}}(x_{n+1})} \subseteq \ol{B_{r_n}(x_n)}$ gilt:
		\[
			\bigcap_{n \in \NN} \ol{B_{r_n}(x_n)} \neq \emptyset.
		\]
		Sei $X$ ein $\CAT$-Raum.
		Zeigen Sie, dass $X$ genau dann vollständig ist, wenn er sphärisch vollständig ist.
	\end{aufgabe}
	
	\begin{aufgabe}
		\label{aufg:3.1}	
		Sei $X$ ein $\CAT$-Raum und $C \subseteq X$ eine konvexe vollständige Teilmenge.
		Sei weiter $f \in \Isom(X)$ beliebig.
		Zeigen Sie, dass gilt:
		\begin{enumerate}[(i)]
			\item Die Teilmenge $f(C)$ ist konvex und vollständig.
			\item Es gilt $\pi_{f(C)} \circ f = f \circ \pi_C$.
		\end{enumerate}
	\end{aufgabe}

	\begin{aufgabe}
		\label{aufg:3.2}	
		Sei $(X,d)$ ein geodätischer Raum.
		Zeigen Sie, dass $X$ genau dann $\CAT$ ist, wenn für jedes geodätische Dreieck $\Delta(x,y,z,\alpha,\beta,\gamma)$ mit Vergleichsdreieck $\ol{\Delta}(\ol{x},\ol{y},\ol{z},\ol{\alpha},\ol{\beta},\ol{\gamma})$ gilt:
		\[
			d(x,\beta(s)) \leq d_2(\ol{x},\ol{\beta}(s)) \text{ für alle } s \in [0,d(y,z)].
		\]
	\end{aufgabe}
	\newpage
	\begin{aufgabe}
		\label{aufg:3.3}	
		Sei $X$ ein vollständiger $\CAT$-Raum und $G_1, G_2$ Gruppen. 
		Seien weiter $\Phi_1\colon G_1 \rightarrow \Isom(X)$, $\Phi_2\colon G_2 \rightarrow \Isom(X)$ isometrische Wirkungen.
		Weiter gelte für alle $g_1 \in G_1, g_2 \in G_2$:
		\[
			\Phi_1(g_1) \circ \Phi_2(g_2) = \Phi_2(g_2) \circ \Phi_1(g_1).
		\]
		Wir definieren die Abbildung
		\begin{align*}
			\Phi_1 \times \Phi_2 \colon G_1 \times G_2 &\longrightarrow \Isom(X) \\
			(g_1,g_2) &\longmapsto \Phi_1(g_1) \circ \Phi_2(g_2).
		\end{align*}
		\begin{enumerate}[(i)]
			\item Zeigen Sie, dass $\Phi_1 \times \Phi_2$ eine isometrische Wirkung ist.
			\item Wenn $\Fix_{\Phi_1}, \Fix_{\Phi_2} \neq \emptyset$, dann gilt $\Fix_{\Phi_1 \times \Phi_2}(G_1 \times G_2) \neq \emptyset$.
		\end{enumerate}
	\end{aufgabe}
	
	\begin{aufgabe}
		\label{aufg:3.4}
		Sei $(X,d)$ ein metrischer Raum und $c \colon [a,b] \rightarrow X$ eine stetige Abbildung mit $d(c(a),c(b)) = b-a$.
		Zeigen Sie, dass $c$ genau dann eine Geodäte ist, wenn gilt:
		\[
			d(c(s),c(t)) = 2 \cdot d\enbrace*{c(s),c \enbrace*{\frac{s+t}{2}}} \text{ für alle } s,t \in [a,b].
		\]
	\end{aufgabe}
	
	\begin{aufgabe}
		\label{aufg:4.1}	
		Sei $(X,d)$ ein $\CAT$-Raum.
		Weiter sei $x \in X$ und $\varepsilon > 0$ beliebig.
		Zeigen Sie, dass die Teilmenge $B_\varepsilon(x) = \{y \in X : d(x,y) < \varepsilon\} \subseteq X$ konvex ist.
	\end{aufgabe}
	
	\begin{aufgabe}
		\label{aufg:4.2}	
		Sei $(X,d)$ ein metrischer Raum und $c \colon [a,b] \rightarrow X$ ein Weg.
		Zeigen Sie:
		\begin{enumerate}[(i)]
			\item Es gilt $\ell(c) \geq d(c(a),c(b))$.
			Weiter gilt $\ell(c) = 0$ genau dann, wenn $c$ konstant ist.
			\item Sei $\varphi\colon [a',b'] \rightarrow [a,b]$ eine stetige schwach monotone surjektive Funktion.
			Dann gilt $\ell(c) = \ell(c \circ \varphi)$.
			\item Seien $c_1\colon [a_1,b_1] \rightarrow X, c_2\colon [a_2,b_2] \rightarrow X$ zwei Wege mit $c_1(b_1) = c_2(a_2)$.
			Dann gilt $\ell(c_1 * c_2) = \ell(c_1) + \ell(c_2)$.
			\item Es gilt $\ell(\ol{c}) = \ell(c)$.
		\end{enumerate}
	\end{aufgabe}
	
	\begin{aufgabe}
		\label{aufg:4.3}	
		Sei $(X,d)$ ein metrischer Raum und $c \colon [a,b] \rightarrow X$ ein rektifizierbarer Weg.
		Wir definieren
		\begin{align*}
			\lambda_c \colon [a,b] &\longrightarrow [0,\ell(c)] \\
			t &\longmapsto \ell\enbrace*{c \big|_{[a,t]}}.
		\end{align*}
		Zeigen Sie, dass ein eindeutiger Weg $\tilde{c}\colon [0,\ell(c)] \rightarrow X$ mit $\tilde{c} \circ \lambda_c = c$ und $\ell\enbrace*{\tilde{c}\big|_{[0,t]}} = t$ existiert.
	\end{aufgabe}
	
	\begin{aufgabe}
		\label{aufg:4.4}	
		Sei $(X,d)$ ein Längenraum, in dem alle abgeschlossenen Bälle kompakt sind.
		Zeigen Sie, ohne \autoref{lemma:2.15} zu benutzen, dass $X$ geodätisch ist. \\
		Hinweise: Zeigen Sie, dass $(X,d)$ Mittelpunkte hat.
		Gehen Sie dabei wie folgt vor:
		\begin{enumerate}[(i)]
			\item Seien $x,y \in X$ beliebig.
			Da $X$ ein Längenraum ist, existiert für jedes $n \in \NN$ ein Weg $c_n$ von $x$ nach $y$ mit $\ell(c_n) \leq d(x,y) + \frac{1}{n}$.
			Zeigen Sie, dass ein Punkt $m_n \in \im(c_n)$ existiert mit $d(x,m_n) = d(m_n,y)$.
			\item Zeigen Sie weiter, dass die Folge $(m_n)_{n \in \NN}$ eine konvergente Teilfolge $(m_{n_k})_k$ besitzt.
			Sei $m$ der Limes dieser Folge.
			\item Zeigen Sie, dass $m$ Mittelpunkt von $x$ und $y$ ist.
		\end{enumerate}
	\end{aufgabe}
	
	\begin{aufgabe}
		\label{aufg:5.1}	
		\mbox{} \\[-1.4cm]
		\begin{enumerate}[(i)]
			\item Sei $(X,d)$ ein Längenraum und $x,y \in X, x \neq y$.
			Zeigen Sie, dass für alle $r < d(x,y)$ gilt:
			\[
				d(x,B_r(y)) = d(x,y) - r
			\]
			\item Gilt (i) auch für beliebige metrische Räume?
		\end{enumerate}
	\end{aufgabe}
	
	\begin{aufgabe}
		\label{aufg:5.2}	
		Seien $X,Y$ metrische Räume und $p \colon Y \rightarrow X$ ein lokaler Homöomorphismus.
		Zeigen Sie, dass $p$ die eindeutige Liftungseigenschaft hat:
		Für jede stetige Abbildung $c \colon [0,1] \rightarrow X$ und stetige Abbildungen $f,g \colon [0,1] \rightarrow Y$ mit $p \circ f = p \circ g = c$ und $f(0) = g(0)$ gilt $f(t) = g(t)$ für alle $t \in [0,1]$. \\
		\textit{Hinweis:} Betrachten Sie die Menge $S:= \{t \in [0,1] : f(t) = g(t)\}$ und zeigen Sie, dass $S$ nichtleer, offen und abgeschlossen ist.
	\end{aufgabe}
	
	\begin{aufgabe}
		\label{aufg:5.3}	
		Sei $X$ ein eindeutig geodätischer Raum und $x,y \in X$ belieibg.
		Seien weiter $(x_n)_{n \in \NN}, (y_n)_{n \in \NN}$ Folgen in $X$ mit $x_n \rightarrow x$ und $y_n \rightarrow y$.
		Sei weiter $\gamma_n$ die Geodäte von $x_n$ nach $y_n$ und $\gamma$ die Geodäte von $x$ nach $y$.
		Wir definieren
		\begin{align*}
			\gamma_n'\colon [0,1] &\longrightarrow X \\
			t &\longmapsto \gamma_n(t \cdot d(x_n,y_n))
		\end{align*}
		und
		\begin{align*}
			\gamma'\colon [0,1] &\longrightarrow X \\
			t &\longmapsto \gamma(t \cdot d(x,y)).
		\end{align*}
		Zeigen Sie, dass gilt:
		$\gamma_n'$ konvergiert gleichmäßig gegen $\gamma'$ genau dann, wenn $\gamma_n'$ punktweise gegen $\gamma'$ konvergiert. \\
		\textit{Hinweis:} Beweis zu \autoref{lemma:2.15}.
	\end{aufgabe}

	\begin{aufgabe}
		\label{aufg:5.4}	
		Sei $X$ ein vollständiger metrischer Raum.
		Zeigen Sie, dass $X$ genau dann ein Längenraum ist, wenn $X$ ungefähre Mittelpunkte hat.
	\end{aufgabe}
	
	\begin{aufgabe}
		\label{aufg:5.5}	
		Zeigen Sie, dass die Vervollständigung von einem Längenraum wieder ein Längenraum ist.
	\end{aufgabe}
	
	\begin{aufgabe}
		\label{aufg:6.1}
		Zeigen Sie, dass die linear umparametrisierten lokalen Geodäten in $\tilde{X}_{x_0}$ stetig von ihren Endpunkten abhängen. \\
		\textit{Hinweis:} \autoref{lemma:2.32}.
	\end{aufgabe}
	
	\begin{aufgabe}
		\label{aufg:6.2}
		Wir betrachten den metrischen Raum
		\[
			X := \RR^2 \setminus \{(x,y) \in \RR^2 : x > 0, y>0\}
		\]
		mit der euklidischen Metrik.
		Zeigen Sie, dass $X$ versehen mit der induzierten Längenmetrik ein $\CAT$-Raum ist. \\
		\textit{Hinweis:} \autoref{lemma:2.44} und \autoref{satz:2.46}.
	\end{aufgabe}
	
	\begin{aufgabe}
		\label{aufg:6.3}
		Wir betrachten den metrischen Raum
		\[
			Y := \RR^3 \setminus \{(x,y,z) \in \RR^3 : x > 0, y>0, z >0\}
		\]
		mit der euklidischen Metrik.
		Zeigen Sie, dass $Y$ versehen mit der induzierten Längenmetrik kein $\CAT$-Raum ist. \\
		\textit{Hinweis:} Betrachten Sie das Dreieck $\Delta((1,0,0),(0,1,0),(0,0,1))$ in $Y$.
	\end{aufgabe}
	
	\begin{aufgabe}
		\label{aufg:6.4}
		Sei $(X,d_X)$ ein metrischer Raum und $c_1,c_2,c_3$ Geodäten in $X$ mit $c_1(0) = c_2(0) = c_3(0)$.
		Zeigen Sie, dass gilt:
		\[
			\angle(c_1,c_3) \leq \angle(c_1,c_2) + \angle(c_2,c_3).
		\]
	\end{aufgabe}
	
	\begin{aufgabe}
		\label{aufg:7.1}
		Gegeben sei eine lokal konvexe Funktion $\Phi\colon [0,1] \rightarrow \RR$.
		Zeigen Sie, dass $\Phi$ konvex ist.
	\end{aufgabe}
	
	\begin{aufgabe}
		\label{aufg:7.2}
		\mbox{} \\[-1.4cm]
		\begin{enumerate}[(i)]
			\item Sei $X$ ein Baum und $\mathcal{S}$ eine endliche Familie von nichtleeren abgeschlossenen konvexen Teilmengen aus $X$.
			Zeigen Sie:
			Wenn sich jeweils zwei Elemente aus $\mathcal{S}$ nichttrivial schneiden, dann ist $\cap \mathcal{S}$ nichtleer.
			\item Sei $G:= \sprod{g_1,\dots,g_k}$ eine endlich erzeugte Gruppe.
			Sei weiter $X$ ein Baum und $\Phi \colon G \rightarrow \Isom(X)$ eine isometrische Wirkung.
			Zeigen Sie:
			Wenn $\Fix_\Phi(\sprod{g_i}) \neq \emptyset$ und $\Fix_\Phi(\sprod{g_i}) \cap \Fix_\Phi(\sprod{g_j}) \neq \emptyset$ für alle $i,j = 1,\dots,k$, dann ist $\Fix_\Phi(G)$ nichtleer.
		\end{enumerate}
	\end{aufgabe}
	
	\begin{aufgabe}
		\label{aufg:7.3}
		Sei $X$ ein Baum.
		Zeigen Sie, dass $X$ ein vollständiger $\CAT$-Raum ist.
	\end{aufgabe}
	
	\begin{aufgabe}
		\label{aufg:7.4}
		Sei $G$ eine Gruppe und $N \subseteq G$ ein Normalteiler mit $[G : N] < \infty$.
		Zeigen Sie:
		Wenn $N$ die Eigenschaft $\prF{A}$ hat, dann hat auch $G$ die Eigenschaft $\prF{A}$. \\
		\textit{Hinweis:} Sei $\Phi\colon G \rightarrow \Isom(X)$ eine beliebige isometrische Wirkung auf einen Baum $X$.
		Betrachten Sie die induzierte Wirkung der Faktorgruppe $G/N$ auf $\Fix_\Phi(N)$.
	\end{aufgabe}