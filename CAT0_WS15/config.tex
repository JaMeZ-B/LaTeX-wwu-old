%!TEX root = AutF3.tex
% -- Author: Phil Steinhorst, p.st@wwu.de
\documentclass[a4paper,twoside,index=totoc,toc=bibliography,fontsize=10,DIV=12,headinclude,BCOR=12mm,cleardoublepage=empty]{scrreprt}


%% Grafikpakete
\usepackage[usenames,x11names]{xcolor} % Die Optionen definieren zusätzliche Farben (siehe Dokumentation)
\usepackage[final]{graphicx}

\usepackage{lmodern}
\usepackage{mathtools,amssymb,amsthm} % Verbesserung von amsmath (die amsmath selbst lädt)
\mathtoolsset{showonlyrefs}
\usepackage[libertine,cmintegrals,bigdelims,varbb]{newtxmath}
\usepackage[no-math]{fontspec}
\usepackage{polyglossia} % moderner babel-ersatz
\setmainlanguage[spelling=new,babelshorthands=true,latesthyphen]{german}
\shorthandoff{"}
\defaultfontfeatures{Mapping=tex-text, Ligatures={Required,Common,Contextual}}
\setmainfont{LinLibertine}[Extension=.otf,UprightFont=*_R,BoldFont=*_RZ,ItalicFont=*_RI,BoldItalicFont=*_RZI]
%\setmainfont{LinLibertine}[Extension=.otf,UprightFont=*_R,BoldFont=*_RZ,ItalicFont=*_RI,BoldItalicFont=*_RZI,ItalicFeatures={Ligatures=Historical}]
\setsansfont{LinBiolinum}[Scale=MatchUppercase, Extension=.otf, UprightFont=*_R, BoldFont=*_RB, ItalicFont=*_RI,BoldItalicFont=*_RBO]
\setmonofont{Inconsolatazi4}[Scale=MatchUppercase,Extension=.otf,UprightFont=*-Regular,BoldFont=*-Bold,StylisticSet=1]
%\setmainfont{LinLibertine}[Extension=.otf,UprightFont=*_R,BoldFont=*_RZ,ItalicFont=*_RI,BoldItalicFont=*_RZI,ItalicFeatures={Ligatures=Historical}]
%\setsansfont{LinBiolinum}[Scale=MatchUppercase, Extension=.otf, UprightFont=*_R, BoldFont=*_RB, ItalicFont=*_RI,BoldItalicFont=*_RBO]
%\setmonofont{Inconsolatazi4}[Scale=MatchUppercase,Extension=.otf,UprightFont=*-Regular,BoldFont=*-Bold,StylisticSet=1]
\usepackage[final]{microtype}

%%% Mathematikpakete und Einstellungen
\mathtoolsset{centercolon} % sorgt dafür dass := und =: besser aussehen
\usepackage{mathdots} % sorgt dafür, dass Punte wie zB \ddots besser aussehen

%%% kommutative Diagramme
\usepackage{tikz-cd} %-- meiner Meinung nach das beste Paket für kommutative Diagramme
\tikzset{% um Kompatibilität mit Babel herzustellen und die angenehme "<label>"-Syntax zu nutzen
  every picture/.append style={
    execute at begin picture={\shorthandoff{"}},
    execute at end picture={\shorthandon{"}}
  }
}
\usetikzlibrary{quotes,babel,angles}
\usetikzlibrary{patterns}
\usetikzlibrary{arrows.meta}
\tikzset{
	schraffiert/.style={pattern=north west lines,pattern color=#1},
	schraffiert/.default=black
}
\tikzset{
	schraffiert2/.style={pattern=north east lines,pattern color=#1},
	schraffiert2/.default=black
}
\tikzset{
	schraffiert3/.style={pattern=horizontal lines,pattern color=#1},
	schraffiert3/.default=black
}

%%% Biber-Settings
\usepackage[%
	backend=biber,
	sortlocale=auto,
	natbib,
	hyperref,
	backref,
	backrefstyle=three+,
	style=alphabetic % eine unvollständige Auswahl von Styles: ieee, numeric, apa
	]%
{biblatex}
\setlength{\bibitemsep}{1em}     % Abstand zwischen den Literaturangaben
\setlength{\bibhang}{2em}        % Einzug nach jeweils erster Zeile
\addbibresource{literature.bib} % Literaturdatei einlesen

%%% Hyperref-Konfigration
\usepackage[hidelinks, pdfpagelabels, bookmarksopen=true, bookmarksnumbered=true, linkcolor=black, urlcolor=SkyBlue2, plainpages=false,pagebackref, citecolor=black, hypertexnames=true, pdfauthor={Phil Steinhorst}, pdfborderstyle={/S/U}, linkbordercolor=SkyBlue2, colorlinks=false,backref=false]{hyperref}
\hypersetup{final}

%%% Aufzählung und Zitate
\usepackage[shortlabels]{enumitem}
\setlist[enumerate,description]{font=\sffamily\bfseries}
\usepackage[german=quotes]{csquotes}

\usepackage[textsize=small]{todonotes}
\usepackage{marginnote}
\renewcommand*{\marginfont}{\color{gray} \footnotesize }
\setlength{\parindent}{0em} 

%%% Kopf-/Fußzeilen
\usepackage{scrpage2}
\pagestyle{scrheadings}
\clearscrheadfoot 
\setheadsepline{1pt} 
\automark[chapter]{chapter} % definiert, welcher Text in den Kolumnentiteln erscheinen soll
\rohead{\rightmark} 
\lehead{\rightmark} 
\ofoot[\pagemark]{\pagemark} 
%\ifoot[Titel der Bachelorarbeit]{Titel der Bachelorarbeit}
%\renewcommand*{\pnumfont}{\LARGE\sffamily} % Seitenzahlen in groß und serifenlos
%\renewcommand*{\footfont}{\large\sffamily\color{gray}}
% \renewcommand*{\headfont}{\normalfont}

% Indexerstellung
\usepackage{makeidx}
\newcommand{\Index}[1]{\textbf{#1}\index{#1}}
\makeindex
\renewcommand{\indexpagestyle}{scrheadings}


%-- Theorem-Pakete und Konfiguration
\usepackage{thmtools}

\declaretheoremstyle[%
	headfont=\sffamily\bfseries,
	notefont=\normalfont\sffamily,
	bodyfont=\normalfont,
	headformat=\NUMBER \ \NAME \NOTE,
	headpunct={\\},
	postheadspace=1ex,
	spaceabove=15pt,spacebelow=10pt]%
{mainstyle}
\declaretheoremstyle[%
	headfont=\sffamily\bfseries,
	notefont=\normalfont\sffamily,
	bodyfont=\normalfont,
	headformat=\NAME \NOTE,
	headpunct={\\},
	postheadspace=1ex,
	spaceabove=15pt,spacebelow=10pt]%
{nonumber}
\declaretheoremstyle[%
	headfont=\sffamily\bfseries,
	notefont=\normalfont\sffamily,
	bodyfont=\normalfont,
	headformat=\NAME \ \NOTE,
	headpunct={\\},
	postheadspace=1ex,
	spaceabove=15pt,spacebelow=10pt]%
{miscstyle}
\declaretheoremstyle[%
	headfont=\bfseries\scshape,
	bodyfont=\normalfont,
	headpunct=:,
	postheadspace=1ex,
	spacebelow=12pt,spaceabove=2pt,
	qed=\qedsymbol]%
{beweise}


\declaretheorem[name=Definition,parent=chapter,style=mainstyle]{definition}
\declaretheorem[name=Satz,sharenumber=definition,style=mainstyle]{satz}
\declaretheorem[name=Korollar,sharenumber=definition,style=mainstyle]{korollar}
\declaretheorem[name=Lemma,sharenumber=definition,style=mainstyle]{lemma}
\declaretheorem[name=Proposition,sharenumber=definition,style=mainstyle]{proposition}
\declaretheorem[name=Bemerkung,sharenumber=definition,style=mainstyle]{bemerkung}
\declaretheorem[name=Beispiel,sharenumber=definition,style=mainstyle]{beispiel}
\declaretheorem[name=Erinnerung,sharenumber=definition,style=mainstyle]{erinnerung}
\declaretheorem[name=Beobachtung,sharenumber=definition,style=mainstyle]{beobachtung}
\declaretheorem[name=Theorem,sharenumber=definition,style=mainstyle]{theorem}
\declaretheorem[name=Aufgabe,parent=chapter,style=mainstyle]{aufgabe}

\declaretheorem[name=Beweis,numbered=no,style=beweise]{beweis}

\declaretheorem[name=Erinnerung,numbered=no,style=nonumber]{no-erinnerung}
\declaretheorem[name=Definition,numbered=no,style=nonumber]{no-def}
\declaretheorem[name=Satz,numbered=no,style=nonumber]{no-satz}
\declaretheorem[name=Bemerkung,numbered=no,style=nonumber]{no-bem}
\declaretheorem[name=Beispiel,numbered=no,style=nonumber]{no-bsp}

% % % % % % % % % % % % % % % % % % % % % % % % % % % % % % % % %
%		PhistMath.tex											%
%		Weitere Mathe-Befehle									%
%																%
%		Author: Phil Steinhorst									%
% % % % % % % % % % % % % % % % % % % % % % % % % % % % % % % % %

% % % Buchstaben und Zahlen
\newcommand{\aff}{\mathbb{A}}
\newcommand{\CC}{\mathbb{C}}
\newcommand{\EE}{\mathbb{E}}
\newcommand{\FF}{\mathbb{F}}
\newcommand{\HH}{\mathcal{H}}
\newcommand{\KK}{\mathbb{K}}
\newcommand{\MM}{\mathcal{M}}
\newcommand{\NN}{\mathbb{N}}
\newcommand{\OO}{\mathbb{O}}
\newcommand{\QQ}{\mathbb{Q}}
\newcommand{\RR}{\mathbb{R}}
\newcommand{\UU}{\mathcal{U}}
\newcommand{\ZZ}{\mathbb{Z}}
\newcommand{\oh}{\mathcal{O}}					% Landau-O
\newcommand{\ind}{1\hspace{-0,8ex}1} 			% Indikatorfunktion (Doppeleins)

% % % Abk�rzungen
\newcommand{\bewrueck}{"$\Leftarrow$":} 	% Beweis R�ckrichtung
\newcommand{\bewhin}{"$\Rightarrow$":}		% Beweis Hinrichtung
\newcommand{\setone}{\{1\}}							% Einsmenge
\newcommand{\NT}{\trianglelefteq}					% Normalteiler
\newcommand{\setzero}{\{0\}}						% Nullmenge

% % % Operatoren
\DeclareMathOperator{\Aut}{Aut} 					% Automorphismen
\DeclareMathOperator{\Char}{char} 					% Charakteristik
\DeclareMathOperator{\GL}{GL}						% allgemeine lineare Gruppe
\DeclareMathOperator{\Hom}{Hom} 					% Homomorphismen
\DeclareMathOperator{\id}{id} 						% Identit�t
\DeclareMathOperator{\im}{im} 						% image
\DeclareMathOperator{\Isom}{Isom}					% Isometrien
\DeclareMathOperator{\Kern}{Kern}					% Kern
\DeclareMathOperator{\ord}{ord} 					% Ordnung
\DeclareMathOperator{\pot}{\mathcal{P}}				% Potenzmenge
\DeclareMathOperator{\SAut}{SAut}
\DeclareMathOperator{\sgn}{sgn} 					% Signum
\DeclareMathOperator{\SL}{SL} 						% Spezielle lineare Gruppe
\DeclareMathOperator{\SO}{SO} 						% Spezielle orthogonale Gruppe
\DeclareMathOperator{\SU}{SU} 						% Spezielle unit�re Gruppe
\DeclareMathOperator{\Sym}{Sym} 					% Symmetrische Gruppe

% % % Klammerungen
\DeclarePairedDelimiter{\abs}{\lvert}{\rvert}			% Betrag
\DeclarePairedDelimiter{\ceil}{\lceil}{\rceil}			% aufrunden
\DeclarePairedDelimiter{\floor}{\lfloor}{\rfloor}		% aufrunden
\DeclarePairedDelimiter{\Norm}{\lVert}{\rVert}			% Norm
\DeclarePairedDelimiter{\sprod}{\langle}{\rangle}		% spitze Klammern
\DeclarePairedDelimiter{\enbrace}{(}{)}					% runde Klammern
\DeclarePairedDelimiter{\benbrace}{\lbrack}{\rbrack}	% eckige Klammern
\DeclarePairedDelimiter{\penbrace}{\{}{\}}				% geschweifte Klammern
\newcommand{\Underbrace}[2]{{\underbrace{#1}_{#2}}} % Underbrace als Befehl in LaTeX-Syntax (und ohne Spacing-Probleme mit nachfolgenden Operatoren...)

% % % CAT(0) kubische Komplexe
\DeclareMathOperator{\prT}{(T)}		% Eigenschaft (T)
\DeclareMathOperator{\prFH}{(FH)}	% Eigenschaft (FH)
\DeclareMathOperator{\prFC}{(FC)}	% Eigenschaft (FC)
\DeclareMathOperator{\prFA}{(FA)}	% Eigenschaft (FA)
\DeclareMathOperator{\CAT}{CAT(0)}	% CAT(0)-Raum
\DeclareMathOperator{\Fix}{Fix}		% Fixpunktmenge
\DeclareMathOperator{\Eig}{Eig}		% Eigenraum
\newcommand{\geo}{\rightsquigarrow}	% Geod�te
\newcommand{\ol}[1]{\overline{#1}}	% Overline
\DeclareMathOperator{\rel}{rel}		% Relative Homotopie

\newcommand{\stack}[2]{\makebox[1cm][c]{$\stackrel{#1}{#2}$}}

\usepackage{setspace}
\usepackage{hyphenat}
\usepackage{bm}
\allowdisplaybreaks
\raggedbottom
\renewcommand{\phi}{\varphi}
\usepackage{chngcntr}
\counterwithout{equation}{chapter} % undo numbering system provided by phstyle.cls
\usepackage{faktor}