\documentclass[a4paper, pagesize=pdftex, pdftex, twoside, headsepline, index=totoc,toc=listof, fontsize=10pt, cleardoublepage=empty, headinclude, DIV=13, BCOR=13mm]{scrartcl}

\usepackage[ngerman]{babel}
\usepackage{scrtime} % Bestandteil von KOMA-Skript, ermoeglicht Zugriff auf Uhrzeit des Kompilierens 
\usepackage{scrpage2} % ermöglicht Bearbeiten von Kopf- und Fusszeilen (wie fancyhdr, nur optimiert auf KOMA-Skript, leich andere Syntax)
\usepackage[utf8]{inputenc} % Gibt an in welcher Textcodierung der Code verstandne werden soll
\usepackage{etex} % sehr technisch, ermöglicht LaTeX mehr Speicher zu belegen
\usepackage[T1]{fontenc} % auch sehr technisch; ist wichtig, um die Schriftarten richtig zu behandeln
\usepackage{textcomp} %verhindert ein paar Fehler bei den Fonts
\usepackage{mathtools} % Packet der American Mathematical Society, das viele Mathematik-Umgebungen und -Befehle definiert
\usepackage{amssymb} %zusätzliche Symbole
\usepackage{latexsym} % nochmal zusätzliche Symbole
\usepackage{stmaryrd} % nochmal mehr zusätzliche Symbole, u.a. Blitz für Widerspruchsbeweise ;)
\usepackage{nicefrac} % schräge Brüche, benutzte ich für Quotienvektorräume
\usepackage{paralist} % redefiniert alle Listenbefehle, sodass diese einen optionalen Parameter haben, der die Nummerierung angibt
\usepackage{dsfont} % Schriftart für N,Z,Q,R die ich momentan benutze (mittels \mathds{R} z.B)
\usepackage[pdftex]{graphicx} % Packet, dass das Einbinden von Grafiken aus Dateien ermöglicht
\usepackage{makeidx}% ermöglicht das automatische Anlegen eines Index 
\usepackage{extarrows}
\usepackage{bbold}
\usepackage[hyphens]{url}
\usepackage{algorithmicx}
\usepackage{algpseudocode}


%\usepackage{MnSymbol}
\flushbottom
\usepackage[normalem]{ulem}
\setlength{\ULdepth}{1.8pt}

%--Indexverarbeitung
\newcommand{\bet}[1]{\textbf{#1}} %Betonung von Text
\newcommand{\Index}[1]{\textbf{#1}\index{#1}} % Befehl, der gleichzeitg das Argument hervorhebt und in den Index mitaufnimmt
\makeindex % startet das automatische Sammeln der Index-Einträge
% Ein kleiner Text am Anfang des Index
\setindexpreamble{{\noindent \itshape Die \emph{Seitenzahlen} sind mit Hyperlinks zu den entsprechenden Seiten versehen, also anklickbar!} \par \bigskip}
\renewcommand{\indexpagestyle}{scrheadings} % Seitenstil für den Index festlegen

%--Farbdefinitionen
\usepackage[usenames, table, x11names]{xcolor} %usenames und x11names, aktivieren viele Farben; siehe Dokumentation von xcolor
% Es lassen sich natürlich auch eigene Farben definieren (hier nur Graustufen)
\definecolor{dark_gray}{gray}{0.45}
\definecolor{light_gray}{gray}{0.7}

%--Zum Zeichnen (ich habe es jetzt mal mit aufgenommen, aber es ist eigentlich nochmal ein ganz anderes Thema, sodass ich da jetzt nicht viel zu sagen werde)
\usepackage{tikz} % TikZ steht übrigens für "TikZ ist kein Zeichenprogramm", ein rekursives Akronym ...
\tikzset{>=latex}
\usetikzlibrary{shapes,arrows}
\usetikzlibrary{calc}
\usetikzlibrary{decorations.pathreplacing}
% Hiermit kann man ganz leicht kommutative Diagramme zeichnen (deswegen auch "cd")
\usepackage{tikz-cd}

%--Marginnote, ermöglicht es kleine Notizen an neben den eigentlichen Textkörper zu setzten
\usepackage{marginnote}
\renewcommand*{\marginfont}{\color{Honeydew4} \footnotesize }

%--Schriftarten
\usepackage{lmodern} % neuere Version der Standard-LaTeX-Schriftarten
\renewcommand{\familydefault}{\sfdefault} %Standardschriftart auf die serifenlose Schriftart setzen

%--Hyperref; aktiviert Hyperlinks in der erzeugten PDF-Datei und definiert deren Aussehen
\usepackage[hidelinks, pdfpagelabels,  bookmarksopen=true, bookmarksnumbered=true, linkcolor=black, urlcolor=SkyBlue2, plainpages=false,pagebackref, citecolor=black, hypertexnames=true, pdfauthor={Tobias Wedemeier}, pdfborderstyle={/S/U}, linkbordercolor=SkyBlue2, colorlinks=false]{hyperref}
%--Römische Zahlen
\newcommand{\RM}[1]{\MakeUppercase{\romannumeral #1{}}}



%-- Definitionen von weiteren Mathe-Befehlen, die dann das "richtige" Aussehen haben. Hier sind der Phantasie keine Grenzen gesetzt
\DeclareMathOperator{\id}{id} %identische Abbildung
\DeclareMathOperator{\End}{End} %Endomorphismen
\DeclareMathOperator{\rg}{rg} %Rang
\DeclareMathOperator{\diam}{diam} %Durchmesser
\DeclareMathOperator{\dist}{dist} %Distanz
\DeclareMathOperator{\grad}{grad} %Gradient
\DeclareMathOperator{\rot}{rot} %Rotation
\DeclareMathOperator{\hess}{Hess} %Hesse-Matrix
\DeclareMathOperator{\supp}{supp}
\DeclareMathOperator{\aut}{Aut}
\DeclareMathOperator{\inn}{Inn}
\DeclareMathOperator{\sym}{Sym}
\DeclareMathOperator{\syl}{Syl}
\DeclareMathOperator{\alt}{Alt}
\DeclareMathOperator{\sign}{sign}
\DeclareMathOperator{\Sl}{Sl}
\DeclareMathOperator{\Gl}{Gl}
\DeclareMathOperator{\Quot}{Quot}
\DeclareMathOperator{\ggT}{ggT}
\DeclareMathOperator{\cone}{cone}
\DeclareMathOperator{\Char}{char}
\DeclareMathOperator{\im}{Im}
\DeclareMathOperator{\re}{Re}
\DeclareMathOperator{\Bin}{Bin}
\DeclareMathOperator{\acl}{acl}
\DeclareMathOperator{\cov}{cov}
\DeclareMathOperator{\argmin}{argmin}
\DeclareMathOperator{\argmax}{argmax}
\DeclareMathOperator{\Tol}{TOL}
\DeclareMathOperator{\RK}{RK}
\DeclareMathOperator{\divv}{div}
\DeclareMathOperator{\tr}{tr}
\DeclareMathOperator{\spann}{span}
\DeclareMathOperator{\esssup}{ess sup}
\DeclareMathOperator{\bild}{bild}
\DeclareMathOperator{\cond}{cond}
\DeclareMathOperator{\train}{train}
\DeclareMathOperator{\spur}{spur}
\DeclareMathOperator{\diag}{diag}
%--Skalarprodukt (cooler Befehl, den ich im Internet gefunden habe; benutzt TeX-Befehle)
\makeatletter
\newcommand{\sprod}[2]{\ensuremath{%
  \setbox0=\hbox{\ensuremath{#2}}
  \dimen@\ht0
  \advance\dimen@ by \dp0
  \left\langle \left.#1 \,\rule[-\dp0]{0pt}{\dimen@}\right|#2\right\rangle}}
\makeatother

%--Norm (auch aus dem Internet, wird auch auf der Beispielseite verwandt)
\newcommand{\norm}[1]{
	\ensuremath{\left\Vert#1\right\Vert}
}
\newcommand{\ener}[1]{
	\ensuremath{\left|\left|\left|#1\right|\right|\right|_\mu}
}

%--selbstgeschriebenen Befehle
%--Betrag
\newcommand{\abs}[1]{\ensuremath{\left\vert#1\right\vert}}

%--Umklammern mit passender Größe der Klammern
\newcommand{\enbrace}[1]{\ensuremath{\left( #1\right)}}

%--Mengen
\newcommand{\penbrace}[1]{\ensuremath{\left\{#1\right\}}}
\newcommand{\cenbrace}[1]{\ensuremath{\left[#1\right]}}

%--Differential
\newcommand{\diff}[2]{\ensuremath{\frac{\partial #1}{\partial #2} }}
\newcommand{\difff}[2]{\ensuremath{\frac{\dint #1}{\dint #2} }}

\newcommand{\zz}{\ensuremath{\mathrm{Z\kern-.3em\raise-0.5ex\hbox{$Z$}}}} % zu zeigen ZZ aus dem inet
\setlength{\parindent}{0pt}%absatz nicht einrücken
\newcommand{\lh}[1]{\langle #1 \rangle} %lineare Hülle
\newcommand{\nt}{\trianglelefteqslant} %normalteiler
\newcommand{\pfs}{\mathds{P}-\text{f.s.}} %P fast sicher Konvergenz
\newcommand{\dint}{\mathrm{d}} % d des integrals

\newcommand{\xfrac}[2]{%
	\mbox{\raisebox{-0.4ex}{\ensuremath{\displaystyle #1}\hspace{0.2ex}}%
		{\raisebox{-0.1ex}{\big \backslash}}%
		\raisebox{0.6ex}{\ensuremath{\displaystyle #2}}%
	}%
}
\newcommand{\Pw}{\mathds{P}}
\newcommand{\V}{\mathds{V}}
\newcommand{\E}{\mathds{E}}
\newcommand{\R}{\mathds{R}}
\newcommand{\N}{\mathds{N}}
\newcommand{\Z}{\mathds{Z}}
\newcommand{\Q}{\mathds{Q}}
\newcommand{\G}{\mathcal{G}}
\newcommand{\F}{\mathcal{F}}
\newcommand{\D}{\mathcal{D}}
\newcommand{\C}{\mathds{C}}
\newcommand{\A}{\mathcal{A}}
\newcommand{\mc}[1]{\mathcal{#1}}
\newcommand{\Pfs}[1][\relax]{\ensuremath{
	\ifx#1\relax \Pw-\text{f.s.}
	\else \Pw^{#1}-\text{f.s.}
	\fi}}
\newcommand{\bgl}[1]{\stackrel{#1}{=}}
\newcommand{\dt}[1]{#1 \dots #1}
\newcommand{\ablim}[1]{\limits_{\mathclap{#1}}}
\newcommand{\degree}{\ensuremath{^\circ}}
\newcommand{\adj}{\ensuremath{^\ast}}



\newcommand{\sect}[1]{\section*{#1}\addcontentsline{toc}{section}{#1}}
\newcommand{\ssect}[2]{ \subsubsection*{#1 #2}\addcontentsline{toc}{subsubsection}{#1}}


\usepackage{listings}
\definecolor{light_green}{rgb}{0,0.5,0}
\definecolor{grey}{rgb}{.5,.5,.5}
\lstset{language=Python, commentstyle=\color{light_green}\bfseries,
	keywordstyle=\color{blue}\bfseries,
	stringstyle=\ttfamily\color{orange},
	morekeywords={as},
	deletendkeywords={range,abs,set},
	escapeinside={\%*}{\&*}
}

\lstset{showspaces=false,
	showstringspaces=false, tabsize=4, breaklines=true, rulecolor=\color{black}}
\lstset{numbers=left,basicstyle=\ttfamily\footnotesize, numberstyle=\tiny\color{grey}, numbersep=10pt}

\renewcommand{\lstlistlistingname}{Programmcodeverzeichnis}
\renewcommand{\lstlistingname}{Programmcode}








\newcommand{\vorlesung}{Numerik partieller Differentialgleichungen}
\newcommand{\Prof}{Dr. Schindler}
\newcommand{\subt}{Mitschrift der Tafelnotizen}

\input{../!config/Tazdr/extra_files/headings.tex}

\numberwithin{equation}{section}


\begin{document}
\maketitle
\thispagestyle{empty}
\cleardoubleoddemptypage

\thispagestyle{empty}
\vspace*{\fill}
\begin{center}
	Hierbei handelt es sich um eine \subt von \textbf{\Prof}, WWU Münster, aus der Vorlesung \textbf{\vorlesung} im Wintersemester 2015/16. 
	Dies ist kein Skript der Vorlesung und keine eigene Arbeit des Autors.\\
	\vspace{2cm}
	Für Fehler in der Mitschrift wird keine Haftung übernommen. 
	Hinweise auf Fehler sind gerne gesehen, hierfür kann man mich in der Uni ansprechen oder alternativ eine e-Mail an: \textit{tobias.wedemeier@gmx.de}\\
	Auch ist eine Mitarbeit über Github möglich.\\
	\vspace{2cm}
	Wenn Teile aus der Vorlesung selber fehlen, können diese gerne an meine e-Mail versandt werden. 
	Ich werde diese dann einarbeiten.\\
\end{center}
\vspace*{\fill}
\cleardoubleoddemptypage

\pagenumbering{Roman}

\tableofcontents
\cleardoubleoddemptypage %sorgt dafür, dass alles folgende erst auf der nächsten freien "rechten" Seite steht

\cleardoubleoddemptypage
\pagenumbering{Alph}
\setcounter{page}{1}
\setcounter{section}{0}

\section{Einleitung}

\minisec{Beispiele in einer Raumdimension}
\begin{enumerate}[(i)]
	\item Poisson-Gleichung: 
	\[
	-\partial_{xx} u(x) = f(x), ~ x\in \Omega \subseteq \R
	\]
	\item Wärmeleitungsgleichung (parabolisch):
	\[
	\partial_t u(x,t)-\partial_{xx} u(x,t)= f(x,t), ~(x,t)\in \Omega_T\subseteq \R \times \R^+
	\]
	\item Wellengleichung (hyperbolisch):
	\[
	\partial_{tt} u(x,t) = c \cdot \partial_{xx} u(x,t),~ (x,t)\in \Omega_T \subseteq \R \times \R^+
	\]
\end{enumerate}

\subsection{Modellierung mit partiellen Differentialgleichungen}

\subsubsection{Erhaltungsgleichungen:}
Beispiel: Ausbreitung eines Tintenkleckses in Wasser\\
$\Omega \subseteq \R^d,~d=1,2,3$, Gebiet (offen, zusammenhängend, beschränkt)\\
$u : \Omega \times \R^+ \to [0,1]$ Konzentration der Tinte.

\ssect{Definition 0.1}{(Erhaltungsprinzip)}
\index{Erhaltungsprinzip}
\begin{enumerate}[(i)]
	\item Physikalisches Prinzip: für eine extensive Zustandsgröße (Masse, Impuls, Energie) gilt: die Änderung dieser Größe in einem beliebigen Volumen $V \subseteq \Omega$ kann nur Transport der Größe über den Rand des Volumens geschehen.
	\item Mathematisches Äquivalent: ist $u(x,t)$ die Dichteverteilung einer extensiven Zustandsgröße,so gilt für ein beliebiges Teilvolumen $V\subseteq \Omega$:
	\begin{align}
	\diff{ }{t} \int\limits_{V} u(x,t) \dint x = -\int\limits_{\partial V} q(x,t)\cdot n(x,t) \dint \sigma(x)
	\end{align}
	Dabei ist $n\in \R^d$ die äußere Normale an den Rand von $V$ und $q$ die Flussdichte der Zustandsgröße.\\
	Annahme: Ruhendes Wasser\\
	$\rightsquigarrow$ mit dem 1. Fickschen Gesetz: " Fluss $\hat{=}$ negativer Gradient der Dichte "\\
	\[
	q(x,t) = - D\cdot \nabla u(x,t),~ D>0,
	\]
	wobei $D$ der Diffusionskoeffizient ist.\\
	$\rightsquigarrow$ mit der Gleichung (0.1):
	\[
	\underbracket{\diff{ }{t} \int\limits_{V} u(x,t)\dint x}_{\int\lim\limits_{V}\partial_t u(x,t)\dint x} =
	\underbracket{\int\lim\limits_{\partial V}D\cdot \nabla u(x,t)\cdot n(x,t)\dint x}_{= \int\lim\limits_{V} \divv(D\cdot \nabla u(x,t))\dint x}
	\]
	$\Rightarrow$ da $V$ beliebig war gilt dies für jeden Punkt:
	\[
	\partial_t u(x,t) = \divv(D\cdot u(x,t)) ~ \forall (x,t)\in \Omega\times \R^+
	\]
	Spezialfälle: $D$ konstant,$d=1$
	\[
	\Rightarrow \partial_t u(x,t) - D\cdot \partial_{xx} u(x,t) = 0
	\]
\end{enumerate}

\ssect{Definition 0.2}{(Anfangs-Randwertproblem für die instationäre Diffusionsgleichung)}
Sei $\Omega \subseteq \R^d$ für $d\in \N$ ein Gebiet (offen, zsmhängend, beschränkt), $T>0$ eine Endzeit und Anfangswerte $u_0\in C^2(\Omega)\cap C^0(\bar{\Omega})$ und Randwerte $g\in C^1([0,T]; C^0(\partial\Omega))$ gegeben.
Dann heißt eine Funktion $u\in C^1([0,T]; C^2(\Omega)\cap C^0(\bar{\Omega})$ \Index{klassische Lösung des Dirichletsproblem für die instationäre Diffusionsgleichung} (homogene Wärmeleitgleichung), falls
\begin{align*}
\partial_t u(x,t) &- \divv(D u(x,t)) = 0,~\forall (x,t)\in \Omega\times (0,T),\\
u(x,t) &= g(x,t),~\forall (x,t)\in \partial \Omega\times (0,T),\\
u(x,t) &= u_0(x), ~\forall x\in \Omega.
\end{align*}

\minisec{Aufgaben der Angewandten Mathematik}
\begin{itemize}
	\item Existenz und Eindeutigkeit von Lösungen
	\item Regularität der Lösung
	\item Beschränktheit der Lösung
	\item Geschlossene Form der Lösung
	\item Numerische Verfahren zur Approximation
	\item Konvergenz gegen die exakte Lösung (\Index{Konvergenzrate})
	\item Visualisierung der Ergebnisse
	\item Validierung des mathematischen Modells (anhand physikalischer Experimente)
\end{itemize}

\ssect{Theorem 0.3}{(Langzeitverhalten/stationäre Diffunsionsgleichung)}
Gilt $g(x,t) = \bar{g}(x)~\forall t\in \R^+$ in Definition 0.2 für eine Funktion $j\in C^0(\partial \Omega)$, so konvergiert die Lösung $u$ für große Zeiten gegen eine Funktion $\bar{u}\in C^2(\Omega) \cap C^0(\bar{\Omega})$, die nicht von der Zeit abhängt.
Die Funktion $\bar{u}$ ist dabei eine \Index{klassische Lösung des Dirichletsproblems für die stationäre Diffusionsgleichung}, d.h. $\bar{u}$ löst
\[
-\divv(D\cdot \nabla \bar{u}(x)) = 0~\forall x\in \Omega;~ \bar{u}(x) = \bar{g}(x) ~\forall x\in \partial \Omega.
\]

\subsubsection{Variationsgleichung}
Physikalisches Prinzip:Energieminimierung\\
Beispiel: Verhalten eines elastischen Körpers\\
$f$: äußere Kraft, $u(x,t)\in \R^d$ Auslenkung/Verschiebevektor, $\sigma$: Spannungstensor (symmetrisch), $\varepsilon(u)=\frac{1}{2}(\nabla u + \nabla u^T)\in \R^{d\times d}$ Verzerrungstensor\\
Die potenzielle Gesamtenergie eine belasteten, elastischen Körpers:
\[
E(u) = \frac{1}{2} \int\limits_\Omega \underbracket{\sigma:\varepsilon(u)}_{\text{Skalarprodukt}}\dint x - \int\lim\limits_{\Omega} f u \dint x,
\]
wobei das Skalarprodukt $A,B\in \R^{n\times n}: A:B = \tr(A^T\cdot B) = \sum_{i=1}^{n}\sum_{j=1}^{n} A_{ij}B_{ij}$.

\minisec{Annahme}
\begin{itemize}
	\item kleine Deformationen
	\item idealisiertes Material
	\item mit dem Hookschen Gesetz gilt: $\sigma(u) = A \cdot \varepsilon(u)$
\end{itemize}

\ssect{Definition 0.4}{(Energieminimierung/Variationsprinzip)}
\begin{enumerate}[(i)]
	\item Physikalisches Prinzip: Ein physikalisches System strebt immer in den Zustand minimaler Energie.
	\item Mathematisches Äquivalent: Sei $u(x,t)$ eine Zustandsvariable und $E(u)$ die Energie des Systems.
	Dann strebt $u$ gegen einen optimalen Zustand $\bar{u}(x)$, der die Energie minimiert.
	D.h., falls $E$ genügend glatt ist, gilt
	\begin{align}
	\diff{ }{\xi}E(\bar{u} + \xi \varphi)|_\xi = 0 \text{ für beliebige Variationen } \varphi.
	\end{align}
	Einsetzen von $E,\sigma$ in (0.2) ergibt:
	\begin{align*}
	0 &= \diff{ }{\xi} \penbrace{\frac{1}{2} \int\lim\limits_{\Omega} A \cdot \varepsilon(\bar{u}+\xi \varphi):\varepsilon(\bar{u}+\xi \varphi)\dint x - \int\limits_{\Omega} f(\bar{u}+\xi \varphi) \dint x}|_{\xi=0}\\
	&= \penbrace{\frac{1}{2} \int\lim\limits_{\Omega} A (\xi \varepsilon(\bar{u}):\varepsilon(\varphi) + \frac{1}{2}\xi\varepsilon(\varphi):\varepsilon(\varphi)\dint x - \int\limits_{\Omega} f\varphi \dint x}|_{\xi=0}
	\end{align*}
	Daraus folgt, dass 
	\[
	\int\lim\limits_{\Omega} - \divv(A\cdot \varepsilon(\bar{u}))\varphi \dint x = \int\lim\limits_{\Omega} A\cdot \varepsilon(\bar{u}):\varepsilon(\varphi)\dint x = \int\lim\limits_{\Omega} f \varphi \dint x ~\forall \text{ ' zulässige ' } \varphi.
	\]
	Mit dem Hauptsatz der Variationsrechnung ergibt dies $-\divv(A\cdot \varepsilon(\bar{u}))=f$, oder ausführlich:
	\[
	-\sum_{i=1}^{n}\partial_{x_i} \sum_{k=1}^{n} A_{jk}\varepsilon(\bar{u})_{ki} = f_j ~\forall 1\le j \le d.
	\]
\end{enumerate}

\subsection{Grundideen Numerischer Methoden}

\subsubsection{Finite Differenzen}
\minisec{Idee}
Approximation von Differenzenoperatoren durch Differenzenquotienten. Sei als Beispiel $d=1$, dann approximiere $u'$ durch:
\begin{itemize}
	\item 'Vorwärtsdifferenzenquotient: $u'(x) \approx \partial^{+h} u(x) := \frac{u(x+h)-u(x)}{h}$
	\item 'Rückwärtsdifferenzenquotient: $u'(x) \approx \partial^{-h} u(x) := \frac{u(x)-u(x-h)}{h}$
	\item 'Zentrierter Differenzenquotient: $u'(x) \approx \partial^{ch} u(x) := \frac{u(x+h)-u(x-h)}{2h}$
\end{itemize}
Obwohl $\partial^{+h} u,\partial^{-h} u,\partial^{ch} u \stackrel{h\to 0}{\to} u'$ (falls $u\in C^1$), führt nicht jede Wahl auf ein konvergentes Verfahren.\\
Beispiel: linearer Transport $a\in \R$, $\partial_t u + a\partial_x u = 0$; mit dem Vorwärtseulerverfahren: $\frac{u(x,t_{n+1})-u(x,t_n)}{\Delta t}$.
Für $a>0$ nutze $\partial^{-h}$, für $a<0$ nutze $\partial^{+h}$ und erhalte ein endlichdimensionales LGS.









\newpage
\printindex
\listoffigures
\end{document}