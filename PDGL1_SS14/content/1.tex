\section{Einleitung}
\label{sec:para1}
\begin{defn}[Partielle Differentialgleichung]
	Sei $\Omega \subset \RR^n$ offen. \marginnote{8. Apr} Eine \Index{partielle Differentialgleichung} (PDGL, PDE) ist eine Gleichung, die eine Funktion $u\colon \Omega \rightarrow \RR$ und ihre partiellen Ableitungen miteinander verknüpft.
	\[ F(D^k u(x), D^{k-1} u(x), \dots, Du(x),u(x),x) = 0, \]
	wobei
	\[F\colon \RR^{n^k} \times \RR^{n^{k-1}} \times \dots \times \RR^n \times \RR \times \Omega \longrightarrow \RR. \]
	$k$ ist die \bet{Ordnung} der partiellen Differentialgleichung. $u$ ist die gesuchte Funktion.
\end{defn}
	
\minisec{Notation}
	\[ u_{x_1} := \frac{\partial u}{\partial x_1}; \quad u_{x_1 x_2} := \frac{\partial^2 u}{\partial x_1 \partial x_2}; \quad \text{usw.} \]
	\[ \nabla u := \begin{pmatrix} \der / \der x_1 \\ \vdots \\ \der / \der x_n \end{pmatrix} u = \begin{pmatrix} u_{x_1} \\ \vdots \\ u_{x_n} \end{pmatrix} \]
	\[ \Delta u := \sum\limits_{i=1}^{n} u_{x_i x_i} = \diver \nabla u \]
	Manchmal ist die erste Variable die Zeit. Dann betrachten wir $\Omega' = (0,\Gamma) \times \Omega$ statt $\Omega$. $t$ bezeichnet dann die Zeitvariable; $(t,x) \in \Omega'$. $\nabla$ und $\Delta$ beziehen sich dann nur auf die Raumvariablen. \\
	Ein Vektor $\alpha = (\alpha_1,\dots,\alpha_n) \in \NN_0^n$ heißt \Index{Multiindex} der Ordnung $|\alpha| = \alpha_1+\dots+\alpha_n$. Damit ist:
	\[ D^\alpha u = \frac{\partial^{|\alpha|}u}{\partial x_1^{\alpha_1} \cdots \partial x_n^{\alpha_n}} \]
	\[ \alpha! = \alpha_1!\alpha_2!\cdots \alpha_n! \]
	\[ x^{\alpha} = x_1^{\alpha_1} x_2^{\alpha_2} \dots x_n^{\alpha_n} \text{ für } x \in \RR^n \]
	
\begin{bsp}
	\begin{itemize}
		\item \Index{Transportgleichung}: $u_t + b(x) \cdot \nabla u = 0$ \\
		$u_t$: z.B. Partikeldichte, $b \in \RR^n$: Flüssigkeitsgeschwindigkeit
		\item \Index{Laplace-Gleichung}: $\Delta u = 0$ \\
		Chemikaliendichte in Lösungsmittel
		\item \Index{Wärmeleitungsgleichung}: $u_t = \Delta u$ \\
		$u$: Temperatur
		\item \Index{Wellengleichung}: $u_{tt} = \Delta u$
	\end{itemize}
\end{bsp}
	
\begin{defn}[PDGL-Problem]\label{def_pdgl_problem}
	Ein \bet{PDGL-Problem} ist eine partielle Differentialgleichung mit Randbedingungen auf einem Teil $\Gamma \subset \partial \Omega$.
\end{defn}
	
\begin{bsp}[PDGL-Probleme]
	\[ \begin{cases}
		\Delta u = 0 & \text{ auf } (0,\pi)^2 \\
		u = 0 & \text{ auf } x_1=0, x_1=\pi, x_2 = 0 \\
		u = \sin^3(x_1) & \text{ auf } x_2 = \pi
		\end{cases} \]
		Eine Lösung ist gegeben durch
		\[ u(x) = \frac{3}{4} \sin x_1 \frac{\sinh x_2}{\sinh 1} - \frac{1}{4} \sin(3x_1) \frac{\sinh 3x_2}{\sinh 3} \]
		
	\[ \begin{cases}
		u_t = \Delta u \text{ auf } (0,\infty) \times (0, \pi) \\
		u(0,x) = \sin^3 (x), u(t,0) = u(t,\pi) = 0 \\
		u \rightarrow 0 \text{ für } t \rightarrow \infty
		\end{cases} \]
		Eine Lösung ist gegeben durch
		\[ u(t,x) = \frac{3}{4} \sin(x) e^{-t} - \frac{1}{4} \sin(3x) e^{-3t} \]
		
	\[ \begin{cases}
		u_{tt} = u_{xx} \text{ auf } (0,\infty) \times (0, \pi) \\
		u(0,x) = \sin^3(x), u_t (0,x) = 0 \\
		u(t,0) = u(t,\pi) = 0
		\end{cases} \]
		Eine Lösung ist gegeben durch
		\[ u(t,x) = \frac{3}{4} \sin (x) \cos (t) - \frac{1}{4} \sin (3x) \cos (3t) \]
\end{bsp}

\begin{defn}[Typen partieller Differentialgleichungen] \label{def_pdgl_typen}
	Eine PDGL heißt \index{partielle Differentialgleichung!linear} \index{partielle Differentialgleichung!semilinear} \index{partielle Differentialgleichung!quasilinear} \index{partielle Differentialgleichung!nichtlinear}
	\begin{itemize}
		\item \bet{linear}, wenn $\sum\limits_{|\alpha| \leq k} a_\alpha (x) D^\alpha u(x) = f(x)$.
		\item \bet{semilinear}, wenn $\sum\limits_{|\alpha| = k} a_\alpha (x) D^\alpha u(x) + a_0(D^{k-1} u(x),\dots,Du(x),u(x),x) = 0$.
		\item \bet{quasilinear}, wenn $\sum\limits_{|\alpha| = k} a_\alpha(D^{k-1} u(x), \dots, Du(x),u(x),x)D^\alpha u(x) + a_0(D^{k-1}u(x),\dots,Du(x),u(x),x) = 0$.
		\item \bet{nichtlinear} in allen anderen Fällen.
	\end{itemize}
\end{defn}
	
\begin{defn}[wohlgestelltes Problem]
\label{def_wohlgestellt}
	Ein (PDGL)-Problem heißt \Index{wohlgestellt} nach Hadamard, wenn
	\begin{enumerate}[(1)]
		\item es eine Lösung besitzt
		\item diese eindeutig ist und
		\item diese stetig von den Daten (Randwerte, rechte Seite) abhängt.
	\end{enumerate}
\end{defn}
	
\mbox{} \\ Wir möchten die Wohlgestelltheit von verschiedenen partiellen Differentialgleichungen untersuchen, ihre Lösungen angeben und charakterisieren. Was verstehen wir unter einer Lösung? Obige Notation setzt eigentlich $k$-fache Differenzierbarkeit von $u$ voraus. Manchmal machen auch weniger reguläre Lösungen Sinn, z.B. schwach differenzierbare Lösungen. Allgemeines Vorgehen:
\begin{enumerate}[(1)]
	\item Zeige Existenz einer (wenig regulären) Lösung.
	\item Schätze die Lösung ab in Termen der Daten
	\item Eindeutigkeit und Regularität herleiten (z.B. $k$-fache Differenzierbarkeit)
\end{enumerate}
\newpage