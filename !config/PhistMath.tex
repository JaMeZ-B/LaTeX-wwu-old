% % % % % % % % % % % % % % % % % % % % % % % % % % % % % % % % %
%		PhistMath.tex											%
%		Self defined math commands								%
%																%
%		Author: Phil Steinhorst									%
% % % % % % % % % % % % % % % % % % % % % % % % % % % % % % % % %

% % % Buchstaben und Zahlen
\newcommand{\CC}{\mathbb{C}}
\newcommand{\FF}{\mathbb{F}}
\newcommand{\HH}{\mathbb{H}}
\newcommand{\KK}{\mathbb{K}}
\newcommand{\NN}{\mathbb{N}}
\newcommand{\OO}{\mathbb{O}}
\newcommand{\QQ}{\mathbb{Q}}
\newcommand{\RR}{\mathbb{R}}
\newcommand{\ZZ}{\mathbb{Z}}
\newcommand{\bigO}{\mathcal{O}}					% Landau-O
\newcommand{\ind}{1\hspace{-0,9ex}1} 			% Indikatorfunktion (Doppeleins)
\newcommand{\qed}{\ifmmode \tag*{$\square$} \else \hfill $\square$ \fi}

% % % Abkürzungen
\newcommand{\ab}[1]{\overline{#1}}					% Abschluss
\newcommand{\bewrueck}{\glqq$\Leftarrow$\grqq:} 	% Beweis Rückrichtung
\newcommand{\bewhin}{\glqq$\Rightarrow$\grqq:}		% Beweis Hinrichtung
\newcommand{\borel}{\mathfrak{B}}					% Borelsche Sigma-Algebra
\newcommand{\setone}{\{1\}}							% Einsmenge
\newcommand{\leb}{\lambda \hspace{-0,95ex}\lambda}	% Lebesgue-Maß (Doppel-Lambda)
\newcommand{\Lp}{\mathcal{L}}						% L^p-Räume
\newcommand{\NT}{\trianglelefteq}					% Normalteiler
\newcommand{\setnull}{\{0\}}						% Nullmenge
\newcommand{\weak}{\rightharpoonup}					% schwache Konvergenz
\newcommand{\weaks}{\overset{*}{\rightharpoonup}}	% schwache *-Konvergenz
\newcommand{\salg}{\mathfrak{A}}					% Sigma-Algebra (Skript-A)
\newcommand{\zyklot}[1]{#1^{(\infty)}}				% zyklotomische Erweiterung


% % % Operatoren
\DeclareMathOperator{\Alt}{Alt} 					% Alternierende n-Linearform
\DeclareMathOperator{\Aut}{Aut} 					% Automorphismen
\DeclareMathOperator{\Bil}{Bil} 					% Bilinearformen
\DeclareMathOperator{\bild}{Bild} 					% Bild
\DeclareMathOperator{\dom}{dom} 					% Domain
\DeclareMathOperator{\diam}{diam}					% Durchmesser
\DeclareMathOperator{\dist}{dist} 					% Distanz
\DeclareMathOperator{\eqs}{\mathrel{\widehat{=}}}	% entspricht
\DeclareMathOperator{\diver}{div} 					% Gradient
\DeclareMathOperator{\EPK}{EPK} 					% Einpunktkompaktifizierung
\DeclareMathOperator{\End}{End} 					% Endomorphismen
\DeclareMathOperator{\esssup}{esssup}				% essentielles Supremum
\DeclareMathOperator{\Gal}{Gal}	 					% Galoisgruppe
\DeclareMathOperator{\ggT}{ggT} 					% ggT
\DeclareMathOperator{\GL}{GL}						% allgemeine lineare Gruppe
\DeclareMathOperator{\grad}{grad} 					% Gradient
\DeclareMathOperator{\Grad}{Grad} 					% Grad
\DeclareMathOperator{\Hess}{Hess} 					% Hesse-Matrix
\DeclareMathOperator{\Hom}{Hom} 					% Homomorphismen
\DeclareMathOperator{\id}{id} 						% identische Abbildung
\DeclareMathOperator{\im}{im} 						% image
\DeclareMathOperator{\Jac}{Jac} 					% Jacobson-Radikal
\DeclareMathOperator{\Kern}{Kern}					% Kern
\DeclareMathOperator{\kgV}{kgV} 					% kgV
\DeclareMathOperator{\Koker}{Koker} 				% Kokern
\DeclareMathOperator{\Cov}{Cov} 					% Kovarianz
\DeclareMathOperator{\Mod}{Mod} 					% Moduln
\DeclareMathOperator{\modu}{mod} 					% Modulo
\DeclareMathOperator{\ord}{ord} 					% Ordnung
\DeclareMathOperator{\der}{\partial}				% Partielle Ableitung
\DeclareMathOperator{\pot}{\mathcal{P}}				% Potenzmenge
\DeclareMathOperator{\prlim}{\varprojlim\limits}	% projektiver Limes
\DeclareMathOperator{\Quot}{Quot}					% Quotientenring
\DeclareMathOperator{\Rang}{Rang} 					% Rang
\DeclareMathOperator{\rot}{rot} 					% Rotation
\DeclareMathOperator{\sgn}{sgn} 					% Signum
\DeclareMathOperator{\Spec}{Spec} 					% Spektrum
\DeclareMathOperator{\SL}{SL} 						% Spezielle lineare Gruppe
\DeclareMathOperator{\SO}{SO} 						% Spezielle orthogonale Gruppe
\DeclareMathOperator{\SU}{SU} 						% Spezielle unitäre Gruppe
\DeclareMathOperator{\Spur}{Spur} 					% Spur
\DeclareMathOperator{\supp}{supp} 					% Träger
\DeclareMathOperator{\Sym}{Sym} 					% Symmetrische Gruppe
\DeclareMathOperator{\tr}{tr} 						% trace

% % % Klammerungen
\DeclarePairedDelimiter{\abs}{\lvert}{\rvert}		% Betrag
\DeclarePairedDelimiter{\ceil}{\lceil}{\rceil}		% aufrunden
\DeclarePairedDelimiter{\floor}{\lfloor}{\rfloor}	% aufrunden
\DeclarePairedDelimiter{\sprod}{\langle}{\right}	% spitze Klammern
\DeclarePairedDelimiter{\enbrace}{(}{)}				% runde Klammern
\DeclarePairedDelimiter{\benbrace}{[}{]}			% eckige Klammern
\DeclarePairedDelimiter{\penbrace}{\{}{\}}			% geschweifte Klammern

% % % Norm
\DeclarePairedDelimiter\doppelstrich{\Vert}{\Vert}
\newcommand{\norm}[2][\relax]{
\ifx#1\relax \ensuremath{\doppelstrich*{#2}}
\else \ensuremath{\doppelstrich*{#2}_{#1}}
\fi}