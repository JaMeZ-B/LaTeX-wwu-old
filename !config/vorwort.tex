\section*{Aktuelle Version verfügbar bei}
\newcommand{\dieBreite}{11cm}
\begin{minipage}{4cm}
	\qrcode[height=3.3cm, version=6]{https://github.com/JaMeZ-B/latex-wwu}
\end{minipage}
\hfill
\begin{minipage}{\dieBreite}
	\includegraphics[height=0.6cm, keepaspectratio]{../!config/Bilder/github_octo.pdf}
	\includegraphics[height=0.6cm, keepaspectratio]{../!config/Bilder/GitHub_Logo.pdf}\\
	\url{https://github.com/JaMeZ-B/latex-wwu} \smallskip\\
	GitHub ist eine Internetplattform, auf der viele OpenSource-Projekte gehostet werden. Diese Plattform nutzen wir zur Zusammenarbeit, also findet man hier neben den PDFs auch die 
	\TeX-Dateien. Außerdem ist über diese Plattform auch direktes Mitarbeiten möglich, siehe nächste Seite.
\end{minipage}\\[1cm]
\begin{minipage}{4cm}
	\qrcode[height=3.3cm, version=6]{https://uni-muenster.sciebo.de/public.php?service=files&t=965ae79080a473eb5b6d927d7d8b0462}
\end{minipage}
\hfill
\begin{minipage}{\dieBreite}
	\raisebox{-2pt}{\includegraphics[height=0.6cm, keepaspectratio]{../!config/Bilder/sciebo_logo.pdf}}
	\resizebox{!}{0.5cm}{\large \textbf{sciebo}} {\large die Campuscloud} \\
	\resizebox{\dieBreite}{!}{\footnotesize\url{https://uni-muenster.sciebo.de/public.php?service=files&t=965ae79080a473eb5b6d927d7d8b0462}}\smallskip\\
	Sciebo ist ein Dropbox-Ersatz der Hochschulen in NRW, der von der Uni Münster in leitender Position auf Basis der OpenSource-Software Owncloud aufgebaut wurde. Wenn man auf den 
	Link klickt, kann man die Freigabe zum eigenen Speicher hinzufügen und hat dann immer automatisch die aktuellste Version.
\end{minipage}\\[1cm]
\begin{minipage}{4cm}
	\qrcode[height=3.3cm, version=6]{btsync://B6WH2DISQ5QVYIRYIEZSF4ZR2IDVKPN3I?n=latex_share}
\end{minipage}
\hfill
\begin{minipage}{\dieBreite}
	\raisebox{-2pt}{\includegraphics[height=0.6cm, keepaspectratio]{../!config/Bilder/bt_sync_logo.pdf}}
	\resizebox{!}{0.5cm}{\large \textbf{Bittorrent} Sync}\\
	\texttt{B6WH2DISQ5QVYIRYIEZSF4ZR2IDVKPN3I} \smallskip\\
	BTSync ist ein peer-to-peer Dateisynchronisations-Tool. Dabei werden die Dateien nur auf den Computern der Teilnehmer an einer Freigabe gespeichert. Ein Mini-Computer ist permanent 
	online, sodass jederzeit die aktuellste Version verfügbar ist. \hrefsym{http://www.getsync.com/intl/de/}{Clients} gibt es für jedes Betriebssystem.
	Zugang ist über das obige \enquote{Secret} bzw. den QR-Code möglich
\end{minipage}\\[1cm]
\hrule \mbox{ }\\[1cm]
\begin{minipage}{4cm}
	\qrcode[height=3.3cm, version=6]{\homepage}
\end{minipage}
\hfill
\begin{minipage}{\dieBreite}
	\resizebox{!}{0.5cm}{\large \textbf{Vorlesungshomepage}}\\
	\resizebox{\dieBreite}{!}{\footnotesize\url{\homepage}}\smallskip\\
	Hier ist ein Link zur offiziellen Vorlesungshomepage.
\end{minipage}
\newpage
\section*{Vorwort --- Mitarbeit am Skript}
Dieses Dokument ist eine Mitschrift aus der Vorlesung \enquote{\fach, \semester}, gelesen von \prof. Der Inhalt entspricht weitestgehend dem Tafelanschrieb. Für die
Korrektheit des Inhalts übernehme ich keinerlei Garantie! Für Bemerkungen und Korrekturen -- und seien es nur Rechtschreibfehler -- bin ich sehr dankbar. 
Korrekturen lassen sich prinzipiell auf drei Wegen einreichen: 
\begin{itemize}
	\item Persönliches Ansprechen in der Uni, Mails an \hrefsymmail{mailto:\mail}{\mail} (gerne auch mit annotieren PDFs) oder Kommentare auf \url{https://github.com/JaMeZ-B/latex-wwu}.
	\item \emph{Direktes} Mitarbeiten am Skript: Den Quellcode poste ich auf GitHub (siehe oben), also stehen vielfältige Möglichkeiten der Zusammenarbeit zur Verfügung:
	Zum Beispiel durch Kommentare am Code über die Website und die Kombination Fork + Pull Request. Wer sich verdient macht oder ein Skript zu einer Vorlesung, die 
	ich nicht besuche, beisteuern will, dem gewähre ich gerne auch Schreibzugriff.
	
	Beachten sollte man dabei, dass dazu ein Account bei \url{github.com} notwendig ist, der allerdings ohne Angabe von persönlichen Daten angelegt werden kann. 
	Wer bei GitHub (bzw. dem zugrunde liegenden Open-Source-Programm \enquote{\texttt{git}}) -- verständlicherweise -- Hilfe beim Einstieg braucht, dem helfe ich gerne 
	weiter. Es gibt aber auch zahlreiche empfehlenswerte Tutorials im Internet.\footnote{zB. \url{https://try.github.io/levels/1/challenges/1}, ist auf Englisch, aber dafür 
	interaktives LearningByDoing}
	\item \emph{Indirektes} Mitarbeiten: \TeX-Dateien per Mail verschicken. 
	
	Dies ist nur dann sinnvoll, wenn man einen ganzen Abschnitt ändern möchte (zB. einen alternativen Beweis geben), da ich die Änderungen dann per Hand einbauen muss! Ich freue mich aber auch über solche Beiträge!
\end{itemize}