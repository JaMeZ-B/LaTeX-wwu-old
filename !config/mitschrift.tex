\RequirePackage[thinlines]{easybmat} %-- muss aufgrund eines Bugs vor etex und tikz geladen werden
\documentclass[a4paper, twoside, headsepline, index=totoc,toc=listof, fontsize=10pt, cleardoublepage=empty, headinclude, DIV=13, BCOR=13mm, titlepage,]{scrartcl}
\usepackage{scrtime} % KOMA, Uhrzeit ermoeglicht
\usepackage{scrpage2} % wie fancyhdr, nur optimiert auf KOMA-Skript, leicht andere Syntax
\usepackage{etoolbox}
\usepackage{letltxmacro}

% Compiler abhaengige IFs
\usepackage{ifxetex,ifluatex}
\newif\ifxetexorluatex
\ifxetex
  \xetexorluatextrue
\else
  \ifluatex
    \xetexorluatextrue
  \else
    \xetexorluatexfalse
  \fi
\fi

%--Farbdefinitionen
%-- muss vor tikz geladen werden
\usepackage[usenames, table, x11names]{xcolor}
\definecolor{dark_gray}{gray}{0.45}
\definecolor{light_gray}{gray}{0.6}

%--Zum Zeichnen
%-- muss vor polyglossia bzw. babel geladen werden
\usepackage{tikz}
\usepackage{tikz-cd}
\usetikzlibrary{external}
\tikzset{>=latex}
\usetikzlibrary{shapes,arrows,intersections}
\usetikzlibrary{calc,3d}
\usetikzlibrary{decorations.pathreplacing,decorations.markings}
\usetikzlibrary{angles}

%-- Konfiguration von tikzexternalize
\tikzexternalize[prefix=tikz/, up to date check=diff]

\ifxetexorluatex
\tikzset{external/system call={xelatex \tikzexternalcheckshellescape %-- verwende LuaLaTeX, wegen dynamischer Speicherallokation
    -halt-on-error -interaction=batchmode -jobname "\image" "\texsource"}}
\else
\tikzset{external/system call={pdflatex \tikzexternalcheckshellescape 
    -halt-on-error -interaction=batchmode -jobname "\image" "\texsource"}}
\fi

%-- tikzexternalize fuer tikzcd deaktivieren, da inkompatibel
\AtBeginEnvironment{tikzcd}{\tikzexternaldisable}
\AtEndEnvironment{tikzcd}{\tikzexternalenable}

%-- um Inkompatibilitaeten von quotes und polyglossia bzw. babel zu vermeiden
\tikzset{
  every picture/.append style={
    execute at begin picture={\shorthandoff{"}},
    execute at end picture={\shorthandon{"}}
  }
}
\usetikzlibrary{quotes}
\usepackage{pgfplots}
\usepgfplotslibrary{colormaps}
\pgfplotsset{compat=1.10}



%-- Mathe
\usepackage{mathtools} % beinhaltet amsmath
\mathtoolsset{showonlyrefs, centercolon}
\newtagform{brackets}[\textbf]{[}{]}
\usetagform{brackets}
\usepackage{wasysym}
\usepackage{amssymb} %zusätzliche Symbole
\usepackage{latexsym} %zusätzliche Symbole
\usepackage{stmaryrd} %für Blitz
\usepackage{nicefrac} %schräge Brüche
\usepackage{cancel} %Befehle zum Durchstreichen
\usepackage{mathdots}
% \usepackage{bm}
\DeclareFontFamily{U}{mathx}{\hyphenchar\font45}
\DeclareFontShape{U}{mathx}{m}{n}{<-> mathx10}{}
\DeclareSymbolFont{mathx}{U}{mathx}{m}{n}
\DeclareMathSymbol{\bigplus}{1}{mathx}{"90}

\DeclareSymbolFont{bbold}{U}{bbold}{m}{n}
\DeclareSymbolFontAlphabet{\mathbbold}{bbold}
\newcommand{\mathds}[1]{\mathbb{#1}} %Um Kompatibilitaet mit frueheren Benutzung von dsfont herzustellen
\newcommand{\ind}{\mathbbold{1}} % charakteristische-Funktion-Eins
\def\mathul#1#2{\color{#1}\underline{{\color{black}#2}}\color{black}} %farbiges Untersteichen im Mathe-Modus
\renewcommand{\le}{\leqslant}
\renewcommand{\ge}{\geqslant}

%-- Alles was mit Schrift und XeteX zu tun hat
\ifxetexorluatex
    % XeLaTeX or LuaTeX
	\ifxetex
		\usepackage{mathspec} %beinhaltet fontspec 
	\else
		\usepackage[no-math]{fontspec}
	\fi
    \usepackage{polyglossia} %babel-ersatz
    \setmainlanguage[spelling=new,babelshorthands=false]{german}
	\setotherlanguage{english}
	\newcommand\glqq{"}
	\newcommand\grqq{"}
	\defaultfontfeatures{Mapping=tex-text, WordSpace={1.4}} %
    \setmainfont[Ligatures=Common, BoldFont={* Bold}, ItalicFont={* Light Italic},SmallCapsFont={Linux Libertine O}, SmallCapsFeatures={Letters=SmallCaps}]{Source Sans Pro}
	\setsansfont[Scale=MatchLowercase,Ligatures=Common, BoldFont={* Medium}]{Ubuntu}
	\ifxetex
		\setallmonofonts[Scale=MatchLowercase, ItalicFont={*}]{Consolas} 
	\else
		\setmonofont[Scale=MatchLowercase, ItalicFont={*}]{Consolas}
	\fi
	\usepackage{xltxtra}
	\usepackage{fontawesome}
	\usepackage{microtype}
\else
    % default: pdfLaTeX
    \usepackage[ngerman]{babel}
    \usepackage[T1]{fontenc}
    \usepackage[utf8]{inputenc}
	\usepackage{textcomp} %verhindert ein paar Fehler bei den Fonts
    \usepackage[babel=true, tracking=true]{microtype}
	\usepackage{lmodern}
	\renewcommand{\familydefault}{\sfdefault} 
\fi


%--Mixed
\usepackage[neverdecrease]{paralist}
\usepackage[german=quotes]{csquotes}
\usepackage{makeidx}
\usepackage{booktabs}
\usepackage{wrapfig}
\usepackage{float}
\usepackage[margin=10pt, font=small, labelfont={sf, bf}, format=plain, indention=1em]{caption}
\captionsetup[wrapfigure]{name=Abb. }
\usepackage{stackrel}
\usepackage{ifthen}
\usepackage{multicol}
\usepackage{qrcode}
\flushbottom

%--Hyphenation
\hyphenation{di-men-si-o-na-le}


%--Unterstreichung
\usepackage[normalem]{ulem}
\setlength{\ULdepth}{1.8pt}

%--Indexverarbeitung
\newcommand{\bet}[1]{\textbf{#1}}
\newcommand{\Index}[1]{\textbf{#1}\index{#1}}
\makeindex
\setindexpreamble{{\noindent \itshape Die \emph{Seitenzahlen} sind mit Hyperlinks zu den entsprechenden Seiten versehen, also anklickbar} \ifxetexorluatex \faHandUp\fi \par \bigskip}
\renewcommand{\indexpagestyle}{scrheadings}


%--Marginnote & todonotes
\usepackage{marginnote}
\renewcommand*{\marginfont}{\color{dark_gray} \itshape \footnotesize }
\usepackage[textsize=small]{todonotes}

%--Todonotes mit tikz/externalize kompatibel machen
\LetLtxMacro{\oldtodo}{\todo}
\renewcommand{\todo}[2][]{\tikzexternaldisable\oldtodo[#1]{#2}\tikzexternalenable}
\LetLtxMacro{\oldmissingfigure}{\missingfigure}
\renewcommand{\missingfigure}[2][]{\tikzexternaldisable\oldmissingfigure[{#1}]{#2}\tikzexternalenable}

%--Konfiguration von Hyperref pdfstartview=FitH, 
\usepackage[hidelinks, pdfpagelabels,  bookmarksopen=true, bookmarksnumbered=true, linkcolor=black, urlcolor=SkyBlue2, plainpages=false,pagebackref, citecolor=black, hypertexnames=true, pdfauthor={Jannes Bantje}, pdfborderstyle={/S/U}, linkbordercolor=SkyBlue2, colorlinks=false]{hyperref}

% \let\orighref\href
\ifxetexorluatex
\newcommand{\hrefsym}[2]{\href{#1}{\texttt{#2\,\faExternalLink}}}
\newcommand{\hrefsymmail}[2]{\href{#1}{\texttt{\faEnvelope\,#2}}}
\renewcommand{\url}[1]{\hrefsym{#1}{\nolinkurl{#1}}}
\fi
\providecommand{\hrefsym}{\href}
\providecommand{\hrefsymmail}{\hrefsym}
% \usepackage{cleveref}
% \crefformat{footnote}{#2\footnotemark[#1]#3}

%--Römische Zahlen
\newcommand{\RM}[1]{\MakeUppercase{\romannumeral #1{}}}

%--Punkte (nach hyperref)
\usepackage{ellipsis}

%%--Abkürzungen etc.
\newcommand{\light}{\text{\Large $\lightning$}}

%-- Definitionen von weiteren Mathe-Befehlen
\DeclareMathOperator{\re}{Re} %Realteil
\DeclareMathOperator{\im}{Im} %Imaginaerteil
\DeclareMathOperator{\id}{id} %identische Abbildung
\DeclareMathOperator{\Sp}{Sp} %Spur
\DeclareMathOperator{\sgn}{sgn} %Signum
\DeclareMathOperator{\alt}{Alt} %Alternierende n-linearFormen
\DeclareMathOperator{\End}{End} %Endomorphismen
\DeclareMathOperator{\Vol}{Vol} %Volumen
\DeclareMathOperator{\dom}{dom} %Domain
\DeclareMathOperator{\Hom}{Hom} %Homomorphismen
\DeclareMathOperator{\bild}{Bild} %Bild
\DeclareMathOperator{\Kern}{Kern }
\DeclareMathOperator{\rg}{rg} %Rang
\DeclareMathOperator{\Rg}{Rg} %Rang
\DeclareMathOperator{\diam}{diam} %Durchmesser
\DeclareMathOperator{\dist}{dist} %Distanz
\DeclareMathOperator{\grad}{grad} %Gradient
\DeclareMathOperator{\dive}{div} %Gradient
\DeclareMathOperator{\rot}{rot} %Rotation
\DeclareMathOperator{\hess}{Hess} %Hesse-Matrix
\DeclareMathOperator{\koker}{Koker} %Kokern
\DeclareMathOperator{\aut}{Aut} %Automorphismen
\DeclareMathOperator{\ord}{ord} %Ordnung
\DeclareMathOperator{\ggT}{ggT} %ggT
\DeclareMathOperator{\kgV}{kgV} %kgV
\DeclareMathOperator{\Gr}{Gr} %Gerade
\DeclareMathOperator{\Kr}{Kr} %Kreis
\DeclareMathOperator{\Char}{char} %Charakteristik
\DeclareMathOperator{\Aut}{Aut} %Automorphismen
% \DeclareMathOperator{\D}{D} %formale Ableitung
\newcommand{\D}{\ensuremath{\mathrm{D}\mkern-1.5mu}}
\newcommand{\mathd}{\ensuremath{\mathrm{d}\mkern-1.0mu}}
\newcommand{\Tmap}{\ensuremath{\mathrm{T}\mkern-1.0mu}}
\newcommand{\opL}{\ensuremath{\mathrm{L}\mkern-0.6mu}}
\DeclareMathOperator{\cov}{cov} %Kovarianz
\DeclareMathOperator{\Gal}{Gal} %Galois
\DeclareMathOperator{\supp}{supp} %Träger
\DeclareMathOperator{\Sym}{Sym} %Symmetrische Gruppe
\DeclareMathOperator{\Zyl}{Zyl} %Zylinder
\DeclareMathOperator{\Mod}{Mod} %Moduln
\DeclareMathOperator{\EPK}{EPK} %Einpunktkompaktifizierung
\DeclareMathOperator{\conj}{conj} %Konjugation
\DeclareMathOperator{\ann}{ann} %Annulator
\DeclareMathOperator{\tor}{tor} %Torsionsmodul
\DeclareMathOperator{\Int}{Int} %Interior/Inneres
\DeclareMathOperator{\Br}{Br} %Brauerergruppe
\DeclareMathOperator{\GL}{GL} %allgemeine lineare Gruppe
\DeclareMathOperator{\SL}{SL} %spezielle lineare Gruppe
\DeclareMathOperator{\SU}{SU} %spezielle unitäre Gruppe
\DeclareMathOperator{\Tr}{Tr} %Spur/Trace
\DeclareMathOperator{\conv}{conv} %Konvexe Teilmenge
\DeclareMathOperator{\Diff}{Diff} %Menge der Diffeomorphismen
\DeclareMathOperator{\Isom}{Isom} %Menge der Isometrien
\DeclareMathOperator{\spec}{spec} %Spektrum
\DeclareMathOperator{\res}{res} %Resolventenmenge
\DeclareMathOperator{\resolv}{R} %Resolventenmenge
% \DeclareMathOperator{\tr}{tr}
\newcommand{\tr}{\mathrm{tr}}
\newcommand{\w}{\mathrm{w}}
\newcommand{\sa}{\mathrm{s.a.}}
\newcommand{\so}{\mathrm{s.o.}}
\newcommand{\weakT}[1]{\ensuremath{\mathcal{T}_{#1}^{\mkern+1.0mu\text{\raisebox{0.4ex}{$\mathrm{w}$}}}}}
\newcommand{\weakTstar}[1]{\ensuremath{\mathcal{T}_{#1}^{\mkern+1.0mu\text{\raisebox{0.4ex}{$\mathrm{w}$}}^*}}}
\newcommand{\finSub}{\subset\mkern-0.7mu \subset}
\newcommand{\rand}[1]{\ensuremath{\partial^{\scriptscriptstyle #1}}}


% -- Kategorien
\DeclareMathOperator{\Ob}{Ob} %Objekte
\DeclareMathOperator{\Mor}{Mor} %Morphismen

% \DeclareMathOperator{\TOP}{TOP} %Topologische Räume
\newcommand{\TOP}{\textsc{Top}}
% \DeclareMathOperator{\HTOP}{HTOP} %Topologische Räume + Homotopieäquivalenz
\newcommand{\HTOP}{\textsc{HTop}}
% \DeclareMathOperator{\VR}{VR} %Vektorräume
\newcommand{\VR}{\textsc{VR}}
% \DeclareMathOperator{\MOD}{MOD} %Moduln
\newcommand{\MOD}{\textsc{Mod}}
% \DeclareMathOperator{\MENGEN}{MENGEN} %Mengen
\newcommand{\MENGEN}{\textsc{Mengen}}
% \DeclareMathOperator{\MAN}{MAN} %Mannigfaltigkeiten
\newcommand{\MAN}{\textsc{Man}}
% \DeclareMathOperator{\GRUPPEN}{GRUPPEN} %Gruppen
\newcommand{\GRUPPEN}{\textsc{Gruppen}}
% \DeclareMathOperator{\ABELGRUPPEN}{ABEL.GRUPPEN} %abelsche Gruppen
\newcommand{\ABELGRUPPEN}{\textsc{Abel.Gruppen}}
% \DeclareMathOperator{\KAT}{KAT} %Kategorien
\newcommand{\KAT}{\textsc{Kat}}
% \DeclareMathOperator{\FUN}{FUN} %Funktoren
\newcommand{\FUN}{\textsc{Fun}}


\newcommand{\op}{\mathrm{op}}
\newcommand{\CW}{\ensuremath{\mathrm{CW}}}
\DeclareMathOperator{\cell}{cell}

\newcommand{\Ker}{\ensuremath{\Kern \,}}
\newcommand{\zirkel}{\rotatebox[origin = c]{-90}{$\varangle$}}
\newcommand{\wirkung}{\!\curvearrowright\!}

\newcommand{\oE}{\OE}

%--Beweisende
\newcommand{\bewende}{\ifmmode \tag*{$\square$} \else \hfill $\square$ \fi}

%--Volumen
\newcommand{\vol}[1]{\ensuremath{\Vol_{#1}}}

%--Skalarprodukt
\DeclarePairedDelimiterX\sprod[2]{\langle}{\rangle}{#1\,\delimsize\vert\,#2}
\DeclarePairedDelimiterX\skal[2]{\langle}{\rangle}{#1\,,\,#2}

%--Betrag, Gaußklammer
\DeclarePairedDelimiter{\abs}{\lvert}{\rvert}
\DeclarePairedDelimiter{\floor}{\lfloor}{\rfloor}
\DeclarePairedDelimiter{\ceil}{\lceil}{\rceil}

%--Norm
\DeclarePairedDelimiter\doppelstrich{\Vert}{\Vert}
\newcommand{\norm}[2][\relax]{
\ifx#1\relax \ensuremath{\doppelstrich*{#2}}
\else \ensuremath{\doppelstrich*{#2}_{#1}}
\fi}


%--Umklammern
\DeclarePairedDelimiter\enbrace{(}{)}
\DeclarePairedDelimiter\benbrace{[}{]}
\DeclarePairedDelimiter\lenbrace{<}{>}
\newcommand{\ssbrace}[1]{{\scriptscriptstyle\enbrace{#1}}}

%--Mengen
\DeclarePairedDelimiterX\mengenA[1]{\lbrace}{\rbrace}{#1}
\DeclarePairedDelimiterX\mengenB[2]{\lbrace}{\rbrace}{#1\, \delimsize\vert \, #2}

\makeatletter
\newcommand{\set}[2][\relax]{
\ifx#1\relax \ensuremath{
\mengenA*{#2}}
\else \ensuremath{%
  \mengenB*{#1}{#2}}
\fi}
\makeatother

\DeclareRobustCommand{\minwidthbox}[2]{%
  \ifmmode
    \expandafter\mathmakebox
  \else
    \expandafter\makebox
  \fi
  [\ifdim#2<\width\width\else#2\fi]{#1}%
}


%--offen und abgeschlossen
\newcommand{\off}{\! \stackrel[\text{offen}]{}{\subset} \!}
\newcommand{\abg}{\! \stackrel[\text{abg.}]{}{\subset} \!}

%--Differential
\newcommand{\diff}[2]{\ensuremath{\frac{{\partial #1}}{{\partial #2}} }}
\newcommand{\diffd}[2]{\ensuremath{\frac{\mathd #1}{\mathd #2} }}

%--Diagonale Linien für Matrizen
\newcommand\tikzmark[1]{%
  \tikzexternaldisable\tikz[overlay,remember picture,baseline] \node [anchor=base] (#1) {};\tikzexternalenable}

\newcommand\MyLine[3][]{%
  \tikzexternaldisable\begin{tikzpicture}[overlay,remember picture]
    \draw[#1] (#2.south) -- (#3.north);
  \end{tikzpicture}\tikzexternalenable}
  
%--Jordankasten
\newcommand{\Jord}{ \tikzexternaldisable \begin{BMAT}{|ccc|}{|ccc|} %
				0 & & \\ %
				1\tikzmark{z} &  \diagdown & \\ %
				& \tikzmark{u}1 \MyLine[thick]{z}{u} &  0 %
			\end{BMAT} \tikzexternalenable}
			
%--Jordankasten mit Lambda
\newcommand{\JordLa}[1][\empty]{%
	\tikzexternaldisable\ifthenelse{\equal{#1}{\empty}}
		{\begin{BMAT}{|ccc|}{|ccc|} %
				\lambda & & \\ %
				1\tikzmark{z} &  \diagdown & \\ %
				& \tikzmark{u}1 \MyLine[thick]{z}{u} &  \lambda  %
			\end{BMAT}}
		{\begin{BMAT}{|ccc|}{|ccc|} %
				\lambda_{#1}  & & \\ %
				1\tikzmark{z} &  \diagdown & \\ %
				& \tikzmark{u}1 \MyLine[thick]{z}{u} &  \lambda_{#1}  %
			\end{BMAT}}\tikzexternalenable}

%--Inhaltsverzeichnis
\usepackage[tocindentauto]{tocstyle}
\usetocstyle{KOMAlike}			
