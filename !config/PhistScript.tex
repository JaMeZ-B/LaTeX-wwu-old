% % % % % % % % % % % % % % % % % % % % % % % % % % % % % % % % %
%		PhistScript.tex											%
%		XeLaTeX-Settings for Scripts							%
%																%
%		Author: Phil Steinhorst									%
% % % % % % % % % % % % % % % % % % % % % % % % % % % % % % % % %

\setlength\parindent{0pt}             % ohne Einrueckung

% % % Konfiguration von scrheadings
\setheadsepline{1pt}[\color{black}]
\pagestyle{scrheadings}
\clearscrheadfoot

% % % Kopf-/Fußzeilenlayout
\providecommand{\shortFach}{\fach}
\ohead{\small \leftmark}
\ofoot[{ \small \thepage}]{ \small \thepage} 
\automark{section}

% Metadaten
\title{\fach}
\author{\verfasser}
\subtitle{\untertitel}
\date{\semester}

% Inhaltsverzeichnis
\setcounter{tocdepth}{1}
\usepackage[tocindentauto]{tocstyle}
\usetocstyle{KOMAlike}	

% Nummerierung
\numberwithin{equation}{section}		% Section in Equation-Nummerierung einbeziehen

% ntheorem package
\newcounter{tcount}						% Zähler für Theoreme
\numberwithin{tcount}{section}			% Section in Theorem-Nummerierung einbeziehen
\usepackage[hyperref]{ntheorem}			% Lade ntheorem mit hyperref-Workaround
\theoremstyle{break}					% Zeilenumbruch nach Theoremkopf
\theorembodyfont{\normalfont}			% nicht kursiv
\theorempreskipamount0.5cm				% Abstand vor Theorem
\theorempostskipamount0.5cm				% Abstand nach Theorem
\newtheorem{thm}[tcount]{Theorem}
\newtheorem{bsp}[tcount]{Beispiel}
\newtheorem{defn}[tcount]{Definition}
\newtheorem{bem}[tcount]{Bemerkung}
\newtheorem{lemma}[tcount]{Lemma}

% Workaround für Tabulatoren im Theorem-Verzeichnis
\makeatletter
\def\thm@@thmline@name#1#2#3#4{%
        \@dottedtocline{-2}{0em}{2.3em}%
                   {\makebox[\widesttheorem][l]{#1 \protect\numberline{#2}}#3}%
                   {#4}}
\@ifpackageloaded{hyperref}{
\def\thm@@thmline@name#1#2#3#4#5{%
    \ifx\\#5\\%
        \@dottedtocline{-2}{0em}{2.3em}%
            {\makebox[\widesttheorem][l]{#1 \protect\numberline{#2}}#3}%
            {#4}
    \else
        \ifHy@linktocpage\relax\relax
            \@dottedtocline{-2}{0em}{2.3em}%
                {\makebox[\widesttheorem][l]{#1 \protect\numberline{#2}}#3}%
                {\hyper@linkstart{link}{#5}{#4}\hyper@linkend}%
        \else
            \@dottedtocline{-2}{0em}{2.3em}%
                {\hyper@linkstart{link}{#5}%
                  {\makebox[\widesttheorem][l]{#1 \protect\numberline{#2}}#3}\hyper@linkend}%
                    {#4}%
        \fi
    \fi}
}
\makeatother
\newlength\widesttheorem
\AtBeginDocument{
  \settowidth{\widesttheorem}{Proposition 10.10\quad}	% Referenzstring für die Breite der ersten Spalte
}

\theoremlisttype{optname}	% nur Theoreme auflisten, die einen Namen haben. 