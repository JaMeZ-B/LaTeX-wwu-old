\section{Elliptische Kurven über verschiedenen Körpern}
\label{sec:para3}
\nextlecture
\subsection{Elliptische Kurven über $\QQ$}
\label{sub:3.1}

\begin{mot}
	Betrachte die elliptische Kurve $E(k)$ jetzt über $k = \QQ$. \marginnote{[16]}
	Wie findet man möglichst viele rationale Punkte (d.h. $P=(x,y) \in \aff^2(\QQ)$) auf der Kurve $E(\QQ)$?
	Die expliziten Formeln zeigen:
	\begin{itemize}
		\item Ist $P$ ein rationaler Punkt auf $E(k)$, so auch $2P, 3P, 4P, \dots$
		\item Sind $P,Q$ zwei rationale Punkte, so auch $P+Q, P+(P+Q) = 2P + Q, 2P + 2Q,$ usw.
	\end{itemize}
	Können so unendlich viele rationale Punkte auf $E(k)$ konstruiert werden?
	Das hängt von der Ordnung des Punktes $P$ in der Gruppe $(E(k),+)$ ab, d.h. von $\ord(P) = \min\{m \in \NN : mP = \oh\}$, falls diese existiert.
	Das ist ziemlich unklar, wie auch Beispiele zeigen:
\end{mot}

\begin{bsp}
\label{bsp_16.2}
	Sei $E(\QQ) \colon y^2 = x^3 + 17$.
	Dann ist $\Delta(E) = -16 \cdot 27 \cdot 17^2 \neq 0$.
	Zwei Punkte sind $P = (-2,3)$ und $Q = (-1,4)$.
	Es ist
	\begin{equation}
	\begin{aligned}
		P + Q &= (4,-9) \\
		2P + Q &= (2,5) \\
		3P + Q &= \enbrace*{\frac{1}{4}, - \frac{33}{8}} \\
		4P + Q &= \enbrace*{\frac{106}{9}, \frac{1097}{27}} \\
		5P + Q &= \enbrace*{-\frac{2228}{961},-\frac{63465}{29791}}\\
		6P + Q &= \enbrace*{\frac{76271}{289},-\frac{21063928}{4913}} \\
		&\vdots
	\end{aligned}
	\end{equation}
	Offenbar werden die Ergebnisse immer komplizierter; unendlich viele rationale Punkte können auf $E(\QQ)$ wohl derart konstruiert werden, d.h. vermutlich hat $P$ keine (endliche) Ordnung.
\end{bsp}

\begin{bsp}
\label{bsp_16.3}
	Sei $E(\QQ)\colon y^2 = x^3+x$.
	Der einzige affine rationale Punkt auf $(0,0)$.
	Dies kann direkt gezeigt werden unter Verwendung, dass die Gleichung $u^4+v^4 = w^2$ nur ganzzahlige Lösungen mit $u=0$ oder $v=0$ hat (was auch schon nicht so schnell zu zeigen ist).
	Es kann dennoch gesagt werden, dass durch $P, 2P = \oh, 3P = P, 4P = \oh, \dots$ alle rationalen Punkte auf $E(\QQ)$ konstruiert werden können.
\end{bsp}

\begin{bsp}
\label{bsp_16.4}
	Sei $E(\QQ)\colon y^2 = x^3 - 4x^2 + 16$.
	Dann ist $\Delta(E) = -16 \cdot (4 \cdot (-4)^3 + 27 \cdot 16^2) \neq 0$.
	Eine kurze Suche liefert die vier rationalen Punkte
	\[
		P_1 = (0,4) \qquad P_2 = (4,4) \qquad P_3 = (0,-4) = -P_1 \qquad P_4 = (4,-4) = -P_2
	\]
	Können hier wie in Beispiel~\ref{bsp_16.2} beliebig viele rationale Punkte konstruiert werden?
	Hier ist die Gerade durch $P_1$ und $P_2$ die Tangente an $E(k)$ in $P_1$, weil $4^2 = x^3 - 4x^2 + 16 \Leftrightarrow 0 = x^2(x-4)$ ist und $x = 0$ doppelte Nullstelle.
	Damit ist $-P_1 = P_1 + P_2 = P_3$, also kann so kein weiterer rationaler Punkt konstruiert werden.
	Auch mit anderen Paaren $P_i$ und $P_j$ der vier Punkte passiert dies.
	Vermutlich gibt es außer den vier angegebenen rationalen Punkten keine weiteren auf $E(\QQ)$.
	Wir haben:
	\begin{equation}
	\begin{aligned}
		P_1 &= (0,4) \\
		2P_1 &= -P_2 = P_4 \\
		3P_1 &= P_1 + P_4 = P_1 - P_2 = P_2 \\
		4P_1 &= P_1 + P_2 = P_3 \\
		5P_1 &= P_3 + P_1 = (P_1 + P_2) + P_1) = 2P_1 + P_2 = -P_2 + P_2 = \oh,
	\end{aligned}
	\end{equation} 
	d.h. $\ord(P_1) = 5$.
	Somit ist $\sprod*{P_1} = \{\oh, P_1, P_2, P_3, P_4\} \simeq \ZZ_5$.
\end{bsp}

Die Beispiele legen folgenden Satz nahe:
\begin{satz}[Satz von Mordell (1922)]
\label{satz_mordell} \label{satz_16.5}
	Sei $E(\QQ)$ eine elliptische Kurve über $\QQ$.
	Dann gibt es eine endliche Liste von Punkten $P_1,\dots,P_s \in E(\QQ)$, sodass alle (rationalen) Punkte auf $E(\QQ)$ von diesen erzeugt werden, d.h. für alle $P \in E(\QQ)$ existieren Zahlen $m_1,\dots,m_s \in \NN_0$ mit
	\[
		P = m_1 P_1 + \dots + m_s P_s
	\]
	Mit anderen Worten: Die Gruppe $(E(\QQ),+)$ ist endlich erzeugt.
	Dabei können die Erzeuger endliche Ordnung haben oder nicht.
	Natürlich sind die Erzeuger nicht unbedingt eindeutig bestimmt.
\end{satz}

\begin{bem}
	\begin{itemize}
		\item In Beispiel~\ref{bsp_16.3} haben wir einen endlichen Erzeuger $P_1 = (0,0), \ord(P_1) = 2$.
		\item In Beispiel~\ref{bsp_16.4} haben wir eventuell einen endlichen Erzeuger $P_1 = (0,4), \ord(P_1)=5$.
		\item In Beispiel~\ref{bsp_16.2} haben wir eventuell einen unendlichen Erzeuger $P_1 = (-2,3)$, welcher eventuell nicht der einzige ist. Die von $P_1$ erzeugte Untergruppe $\ZZ \cdot P_1 := \{m P_1 : m \in \ZZ\} \subseteq E(\QQ)$ ist isomorph zu $\ZZ$ vermöge $m \cdot P_1 \mapsto m$.
	\end{itemize}
\end{bem}

\begin{defn}[Torsionsgruppe]
	Wir können in der Formuliereung von Satz~\ref{satz_16.5} die Unterscheidung zwischen Punkten mit und ohne endliche Ordnung vornehmen.
	Die Teilmenge $T := \{P \in E(\QQ) : \ord(P) \in \NN\}$ aller Punkte von $E(\QQ)$ mit endlicher Ordnung ist offenbar eine Untergruppe, die \Index{Torsionsgruppe} von $E(\QQ)$ heißt.
	Somit hat der Satz von Mordell auch die folgende Formulierung:
\end{defn}

\begin{satz}[Satz von Mordell, Gruppenstruktur]
	Es gibt ein $r = r(E) \in \NN_0$ mit $E(\QQ) \simeq \ZZ^r \times T$. \marginnote{$\ZZ^r$ mit komponentenweiser Addition}
	\index{Satz von Mordell}
\end{satz}

\begin{defn}[Rang]
\label{def_16.9}
	Die Zahl $r(E)$ heißt \Index{Rang} von $E(\QQ)$.
\end{defn}

\begin{bem}
	Die Torsionsgruppe $T$ ist stets endlich, wie aus dem 	\href{https://de.wikipedia.org/wiki/Hauptsatz_\%C3\%BCber_endlich_erzeugte_abelsche_Gruppen}{Struktursatz über endlich erzeugte abelsche Gruppen} gefolgert werden kann.
	Allerdings bleiben Größe von $T$ und Lage der Torsionspunkte $P \in T$ damit unbekannt.
	Weiter kann aber $\#E(\QQ) = \infty \Leftrightarrow r(E) > 0$ gefolgert werden.
\end{bem}

\begin{bsp}
	\begin{itemize}
		\item Für $E(\QQ) \colon y^2 = x^3 - 4$ ist $E(\QQ) \simeq \ZZ^1$, wobei z.B. $P_1 = (2,2)$ Erzeuger ist.
		\item Im Beispiel~\ref{bsp_16.3} und \ref{bsp_16.4} ist $\Rg E(\QQ) = 0.$ (ohne Beweis)
	\end{itemize}
\end{bsp}

\begin{bem}
	Der Rang elliptischer Kurven ist bislang schlecht verstanden.
	Offen, d.h. bislang unbewiesen, ist z.B. die \Index{Rangvermutung}:
	\[
		\limsup_{E(\QQ)} r(E) = \infty
	\]
	D.h. man vermutet, dass es zu jedem $C \in \RR$ eine elliptische Kurve mit $\Rg E(\QQ) > C$ gibt.
	Der aktuelle Weltrekord (2006, von N. Elkies\footnote{\url{https://de.wikipedia.org/wiki/Noam\_Elkies}}) ist eine elliptische Kurve vom Rang $\geq 28$ (da $28$ "unabhängige" Punkte unendlicher Ordnung auf ihre gefunden wurden), die Kurve lautet $y^2 + xy + y = x^3 - x^2 - ax + b$ mit
	\footnotesize
	\begin{equation}
	\begin{aligned}
		a &= 20.067.762.415.575.526.585.033.208.209.338.542.750.930.230.312.178.985.502 \\
		b &= 34.481.611.795.030.556.467.032.985.690.390.720.374.855.944.359.319.180.361.266.008.296.291.939.448.732.243.729
	\end{aligned}
	\end{equation}
	\normalsize
	Die Torsionsgruppe ist deutlich besser verstanden:
\end{bem}

\begin{satz}[Satz von Nagell-Lutz (Nagell 1935, Lutz 1937)]
	(T. Nagell\footnote{\url{https://de.wikipedia.org/wiki/Trygve_Nagell}}, E. Lutz\footnote{\url{https://de.wikipedia.org/wiki/\%C3\%89lizabeth_Lutz}}) \\
	Sei $E(\QQ)$ eine elliptische Kurve mit Gleichung $y^2 = x^3 + ax^2 + bx + c$, $a,b,c \in \ZZ$, und seien $P_1,\dots,P_s$ alle Torsionspunkte, d.h. $T = \{P_1,\dots,P_s\}$.
	Schreibe die $P_i = (x_i,y_i) \in \QQ^2$.
	Dann sind alle $x_i,y_i \in \ZZ$, und für $y_i \neq 0$ gilt $y_i^2 \mid \Delta(E)$. \index{Satz von Nagell-Lutz}
\end{satz}

\begin{satz}[Satz von Mazur (1977)]
	(B. Mazur\footnote{\url{https://de.wikipedia.org/wiki/Barry_Mazur}}) \\
	Sei $E(\QQ)$ eine elliptische Kurve mit Gleichung $y^2 = x^3 + ax^2 + bx + c, a,b,c \in \ZZ$, mit Torsionsuntergruppe $T$.
	Dann ist $T \simeq \ZZ_n$ mit $n \leq 12, n \neq 11$ oder $T \simeq \ZZ_2 \times \ZZ_n$ mit $n \in \{2,4,6,8\}$.
	Andere Torsionsuntergruppen kann es nicht geben, und alle genannten kommen vor. \index{Satz von Mazur}
\end{satz}

Das sind beachtliche, tiefe Sätze.
In Beispiel~\ref{bsp_16.4} ist $T \simeq \ZZ_5$, und man kann sehen, dass der Nagell-Lutz-Satz hier korrekt ist: $4^2 \mid \Delta(E)$. \\
Für $a,b,c \in \ZZ$ kann es höchstens endlich viele Punkte mit ganzzahligen Koordinaten geben:

\begin{satz}[Satz von Siegel (1926)]
	(C. L. Siegel\footnote{\url{https://de.wikipedia.org/wiki/Carl_Ludwig_Siegel}}) \\
	Sei $E(\QQ)\colon y^2 = x^3 + ax^2 + bc + c$ mit $a,b,c \in \ZZ$ eine elliptische Kurve.
	Dann gibt es nur endlich viele Kurvenpunkte $(x,y) \in E(\QQ) \cap \ZZ^2$. \index{Satz von Siegel}
\end{satz}

\begin{bem}
	In Beispiel~\ref{bsp_16.2} haben genau die Punkte $\oh, (-2,\pm -3), (-1,\pm 4), (2,\pm 5), (4,\pm 9), (8, \pm 23), (43,\pm 282), (52, \pm 375), (5324, \pm 378661)$ auf der elliptischen Kurve $E(\QQ)$ ganzzahlige Koordinaten.
\end{bem}

\begin{bem}
	Elliptische Kurve über $\QQ$ sind nicht-singuläre algebraische Kurven vom Geschlecht $1$. Mordell\footnote{\url{https://de.wikipedia.org/wiki/Louis_Mordell}} vermutete, dass jede über $\QQ$ definierte nicht-singuläre algebraische Kurve vom Geschlecht $\geq 2$ höchstens endlich viele Punkte enthält.
	Diese Vermutung wurde 1983 von Faltings\footnote{\url{https://de.wikipedia.org/wiki/Gerd_Faltings}} für beliebige Körper bewiesen, wofür er 1986 auf der ICM in Berkeley mit der Fields-Medaille ausgezeichnet wurde.
\end{bem}

\newpage
\subsection{Elliptische Kurven über $\CC$}
\label{sub:3.2}
\begin{bem}
	Elliptische Kurven über $\CC$ können einerseits über die Weierstraßgleichung dargestellt werden und zum anderen über ihre \Index{Legendre-Normalform} mit einer Gleichung der Form $y^2 = x(x-1)(x-\lambda)$, vgl. Übungsblatt 6, Aufgabe 2(a).
	Wir besprechen hier kurz die dritte Darstellung mittels elliptischer Funktionen.
	Wir möchten hier nur erläutern, warum eine elliptische Kurve über $\CC$ in diesem Sinne ein Torus ist.
	
	Gegeben sei $\tau \in \CC \setminus \RR$. Betrachte das Gitter $\Lambda := \{ a + \tau \cdot b : a,b \in \ZZ\} \subseteq \CC$.	
\end{bem}

\begin{defn}[Elliptische Funktion]
	Eine Funktion $f\colon \CC \setminus \pol \rightarrow \CC$ der Form $f(z) := \frac{f(z)}{g(z)}$, $g,h$ holomorph, $h \neq 0$, heißt \Index{elliptische Funktion}, falls $f(z+w) = f(z)$ für alle $z \in \CC$ und $\omega \in \Lambda$ (sofern $f(z), f(z+\omega)$ definiert ist, wobei $\pol = \{z : h(z) = 0\}$ die Menge der Polstellen von $f$ ist), d.h. wenn $f$ (doppelt-)periodische Funktion zu $\Lambda$ ist.
	Der Körper der elliptischen Funktionen über $\Lambda$ sei $\CC(\Lambda)$. \todo{Zeichnungen einfügen!}
\end{defn}

Eine elliptische Funktion ohne Polstellen (oder ohne Nullstellen) ist konstant (vgl. Satz von Liouville aus der Funktionentheorie).

\begin{defn}[Weierstraß-$\wp$-Funktion, Eisensteinreihe]
	Zu $\Lambda$ definiere
	\begin{equation}
	\begin{aligned}
		\wp \colon \CC \setminus \Lambda &\Longrightarrow \CC \\
		z &\longmapsto \frac{1}{z^2} + \sum_{\omega \in \Lambda \setminus \setnull} \enbrace*{\frac{1}{(z-\omega)^2} - \frac{1}{\omega^2}}
	\end{aligned}
	\end{equation}
	und $G_{2k}(\Lambda) := \sum_{\omega \in \Lambda \setminus \setnull} \frac{1}{\omega^{2k}} \in \CC$. Dann heißt $\wp$ die \Index{Weierstraß-$\wp$-Funktion} und $G_{2k}$ heißt \Index{Eisensteinreihe} vom Gewicht $2k \in \RR_{>2}$.
\end{defn}

\begin{satz}
	Sei $\Lambda$ ein Gitter.
	\begin{enumerate}[(a)]
		\item Die Eisenstein-Reihe $G_{2k}(\Lambda)$ konvergiert absolut für $k > 1$.
		\item Die Reihe der Funktion $\wp$ konvergiert absolut und gleichmäßig auf jeder kompakten Teilmenge von $\CC \setminus \Lambda$.
		Sie definiert eine elliptische Funktion mit zweifachem Pol in jedem Gitterpunkt $\omega \in \Lambda$.
	\end{enumerate}
\end{satz}

Es gilt somit $\wp'(z) = -2 \sum_{\omega \in \Lambda \setminus \setnull} \frac{1}{(z-\omega)^3}$ für $z \in \CC \setminus \Lambda$. 
Die Funktionen $\wp$ und $\wp'$ liefern den "Prototyp" elliptischer Funktionen:
Man kann zeigen, dass jede elliptische Funktion $f$ schreibbar ist als
\[
	f(z) = \frac{P_1(\wp(z))}{Q_1(\wp(z))} + \wp'(z) \cdot \frac{P_2(\wp(z))}{Q_2(\wp(z))}, P_i, Q_i \in \CC[Z].
\]

\begin{satz}
	Es gilt $(\wp'(z))^2 = 4(\wp(z))^3 - g_2 \cdot \wp(z) - g_3$, d.h. $(\wp(z),\wp'(z)) \in E(\CC)$ mit Gleichung $y^2 = 4x^3 - g_2 x - g_3$, wobei $g_2 := 60 G_4, g_3 := 140 G_6$.
	Da $\Delta(E) = g_2^3 - 27g_3^2 \neq 0$, handelt es sich bei $E(\CC)$ um eine elliptische Kurve.
\end{satz}

Wir haben so die Abbildung
\begin{equation}
\begin{aligned}
	\varphi \colon \CC / \Lambda &\longrightarrow \PP^2(\CC) \\
	z+\Lambda &\longmapsto [\wp(z) : \wp'(z) : 1],
\end{aligned}
\end{equation}
wobei $\varphi(0 + \Lambda) = [0:1:0] = \oh$.
Das Bild von $\varphi$ ist genau die genannte elliptische Kurve $E(\CC)$.
Die Abbildung $\varphi \colon \CC/\Lambda \rightarrow E(\CC)$ ist bijektiv und überträgt die Addition "$+$" auf $\CC/\Lambda$, gegeben durch $(x+\Lambda) + (y+\Lambda) := (x+y) + \Lambda$, auf $E(\CC)$, welche sich als unsere bisher studierte Addition $+$ auf $E(\CC)$ erweist.
Der Torus $\CC/\Lambda$ wird so mit der elliptischen Kurve $E(\CC)$ identifiziert.
Umgekehrt ist auch jede elliptische Kurve $(E(\CC),+)$ beschreibbar als Torus $(\CC/\Lambda,+)$.

\nextlecture
\newpage
\subsection{Elliptische Kurven über $\FF_p$ und $\FF_{p^r}$}
\label{sub:3.3}
\begin{bem}
	Elliptische Kurven über endlichen Körpern werden vielfältig eingesetzt, zum einen in der Kryptographie, zum anderen auch in technischen Systemen mit wenig Ressourcen (eingebettete Systeme), z.B. Steuergeräte in Automobilen (elektronische Wegfahrsperren, Tuning-Schutz, Car-To-Car-Kommunikation, etc.).
	Manche Hardware-Implementationen arbeiten über $\FF_{2^r}$ der Charakteristik $2$, bei denen die technische Umsetzung damit günstig ist.
\end{bem}

\subsubsection{Punkte zählen, der Frobenius}
\label{subsub:3.3.1}
\begin{bem}
	Wir studieren zunächst elliptische Kurven über $\FF_p$, wo $p$ prim, mit der "modularen Brille" modulo $p$.
	Das Verhalten dieser Kurven kann ganz anders sein als über $\QQ$:
	Die elliptische Kurve $E(\QQ)\colon y^2 = x^3 + x$ aus Beispiel~\ref{bsp_16.3} etwa enthält den einzigen rationalen Punkt $(0,0)$, über $\FF_p$ hat sie aber viele Punkte:
	Sei $\NN_p := \#E(\FF_p)$ von $y^2 = x^3 + x$, d.h. $N_p$, die Anzahl der Punkte der elliptischen Kurve $E(\FF_p)$, ist die Anzahl der Lösungen von $y^2 = x^3 + x$ modulo $p$.
\end{bem}

\begin{defn}[Defekt]
	Numerische Daten ergeben folgende Tabelle:
	\begin{center}
		\begin{tabular}{c||c|c|c|c|c|c|c|c|c|c|c|c|c|c|c|c}
		$p$ & 2 & 3 & 5 & 7 & 11 & 13 & 17 & 19 & 23 & 29 & 31 & 37 & 41 & 43 & 47 & \dots \\ 
		\hline $N_p-1$ & 2 & 3 & 3 & 7 & 11 & 19 & 15 & 19 & 23 & 19 & 31 & 35 & 31 & 43 & 47 & \dots
		\end{tabular} 
	\end{center}
	Offenbar gilt $p = N_p-1$, falls $p \kon 3 \modu 4$.
	Für $p \kon 1 \modu 4$ sieht der so genannte \Index{Defekt} $a_p := p+1 - N_p$ so aus:
	\begin{center}
		\begin{tabular}{c||c|c|c|c|c|c|c|c|c|c|c}
		$p$ & 5 & 13 & 17 & 29 & 37 & 41 & 53 & 61 & 73 & 89 & \dots \\ 
		\hline $a_p / 2$ & 1 & -3 & 1 & 5 & 1 & 5 & -7 & 5 & -3 & 5 & \dots 
		\end{tabular} 
	\end{center}
	Beobachtung: $p-\enbrace*{\frac{ap}{2}}^2$ ist stets eine Quadratzahl!
\end{defn}

Wir halten fest:
\begin{satz}
\label{satz_17.4}
	Für $E(\FF_p)\colon y^2 = x^3+x$ gilt:
	\begin{enumerate}[(a)]
		\item Ist $p \kon 3 \modu 4$, gilt $N_p = p+1$.
		\item Ist $p \kon 1 \modu 4$, ist $N_p = p+1 \pm 2A$, wobei $p = A^2 + B^2$ mit $2 \nmid A$.
		Dabei gilt "$+$", falls $A \kon 1 \modu 4$ und "$-$", falls $A \kon 3 \modu 4$.
	\end{enumerate}
\end{satz}

Zum Beweis benutzen wir:
\begin{satz}
	Für $p > 2$ und $E(\FF_p) \colon y^2 = x^3 + ax + b$ mit $a,b \in \FF_p$ gilt
	\[
		N_p := \#E(\FF_p) = p+1+\sum_{x \in \FF_p} \leg{x^3+ax+b}{\FF_p},
	\]
	wobei
	\[
		\leg{u}{\FF_p} := \begin{dcases}
			+1, & \text{wenn } u \not\kon 0 \modu p \text{ ein quadratischer Rest } \modu p\text{, d.h. } \exists w \in \ZZ : u \kon w^2 \modu p \\
			-1, & \text{wenn } u \not\kon 0 \modu p \text{ kein quadratischer Rest } \modu p\text{, d.h. } \forall w \in \ZZ : u \not\kon w^2 \modu p \\
			0, & \text{wenn } u \kon 0 \modu p\text{, d.h. } p \mid u
		\end{dcases}
	\]
	das verallgemeinerte \Index{Legendre-Symbol} ist.
\end{satz}

\minisec{Beweis}
	Vgl. Übungsblatt 7, Aufgabe 4(a). \qed
	
\begin{bew}[zu Satz~\ref{satz_17.4}(a)]
	Für $p \kon 3 \modu 4$ ist $\leg{-1}{\FF_p} = (-1)^\frac{p-1}{2} = -1$ nach dem 1. Ergänzungssatz für das Legendre-Symbol,\marginnote{Für die Rechenregeln zum Legendre- Symbol siehe EZT-Skript!} also $\leg{(-x)^3+(-x)}{\FF_p} = -\leg{x^3+x}{\FF_p}$ für das $x \not\kon 0 \modu p$ nach den Rechenregeln für das Legendre-Symbol, sodass
	\[
		0 = \sum_{x \not\kon 0 \modu p} \enbrace*{ \leg{x^3+x}{\FF_p} + \leg{(-x)^3+(-x)}{\FF_p}} = 2 \sum_{x \not\kon 0 \modu p} \leg{x^3 + x}{\FF_p}
	\]
	folgt, denn mit $x$ durchläuft auch $-x$ alle Restklassen $\not\kon 0 \modu p$.
	Also ist $\NN_p = p+1$. \qed
\end{bew}

\begin{bem}
	Für den Defekt $a_p = p+1-N_p$ gilt somit $a_p = - \sum_{x \in \FF_p} \leg{x^3+ax+b}{\FF_p}$.
	Im Absolutbetrag kann $a_p$ nicht allzu groß werden, d.h. $N_p = \#E(\FF_p)$ kann nicht allzu stark von $p$ abweichen:
\end{bem}

\begin{satz}[Satz von Hasse (1933)]
	(H. Hasse\footnote{\url{https://de.wikipedia.org/wiki/Helmut_Hasse}}) \\
	Für den Defekt $a_p$ einer elliptischen Kurve $E(\FF_p)$ gilt $|a_p| \leq 2 \sqrt{p}$.
	Der Satz gilt auch für $E(\FF_{p^r}), r \geq 1$: Es ist $|p^r + 1 - N_{p^r}| \leq 2 \sqrt{p^r}$.
	\index{Satz von Hasse}
\end{satz}

\begin{bem}
	Diese Abschätzung für $|a_p|$ wurde 1920 von E. Artin\footnote{\url{https://de.wikipedia.org/wiki/Emil_Artin}} vermutet.
	Eine Verallgemeinerung zeigte A. Weil\footnote{\url{https://de.wikipedia.org/wiki/Andr\%C3\%A9_Weil}} in den 1940ern, und stark verallgemeinert wurde sie von P. Deligne\footnote{\url{https://de.wikipedia.org/wiki/Pierre_Deligne}} in den 1970ern, wofür er 1978 mit der Fields-Medaille ausgezeichnet wurde.
\end{bem}

\begin{bem}
	Im Beispiel $y^2 = x^3 + x$ mit $p \kon 1 \modu 4$ folgt aus Satz~\ref{satz_17.4}(b) die Hasseschranke, da
	\[
		|a_p| = |p+1-N_p| = |p+1-p-1 \mp 2A| = |\mp 2A| = |\mp 2 \sqrt{p-B^2}| \leq 2 \sqrt{p}.
	\]
\end{bem}

\begin{bem}
	Fragen über die Größen von $N_p$ führen zu offenen Problemen, z.B. die schwache Vermutung von Birch\footnote{\url{https://de.wikipedia.org/wiki/Bryan_Birch}} und Swinnerton-Dyer\footnote{\url{https://de.wikipedia.org/wiki/Peter_Swinnerton-Dyer}} (1963, 1965):
	Für $E(\QQ)$ mit Koeffizienten aus $\ZZ$ sollte
	\[
		\prod_{p \leq x} \frac{N_p}{p} \sim C_E(\log x)^{r(E)}
	\]
	gelten, wobei $C_E$ eine Konstante $> 0$ ist, die nur von $E(\QQ)$ abhängig ist.
	Die Aussage $f(x) \sim g(x)$ für zwei Funktionen $f,g\colon \RR_{>0} \rightarrow \RR_{>0}$ bedeutet, dass $\lim\limits_{x \rightarrow \infty} \frac{f(x)}{g(x)}=1$, d.h. $f$ und $g$ sind asymptotisch gleich.
	Die Zahl $r(E)$ ist der Rang von $E(\QQ)$.
	Numerische Untersuchungen stützen diese Vermutung bislang.
	Sie bedeutet:
	Ist die Anzahl $N_p$ der Punkte auf $E(\FF_p)$ bei Reduktion modulo $p$ signifikant größer als der Erwartungswert, so sollte der Rang $r(E)$ positiv sein.
	Sie stellt damit ein numerisch leicht testbares Kriterium für $r(E) > 0$ dar.
\end{bem}

\begin{bsp}
	Weitere numerische Beobachtungen in anderen Beispielen: \\
	Für $E \colon y^2 = x^3 + 17$ gilt $a_p = 0$ genau für $p \kon 2 \modu 3$, also auch $a_p = 0$ für "die Hälfte" aller Primzahlen.
	Das kommt für "wenige" elliptische Kurven mit Koeffizienten aus $\ZZ$ so heraus.
	Für die Kurve $E \colon y^2 = x^3 - 4x^2 + 16$ etwa haben wir $a_p=0$ nur selten:
	Die einzigen $p < 2000$ mit $a_p = 0$ sind
	\[
		p = 2, 19, 29, 199, 569, 809, 1289, 1439, \dots
	\]
	Welcher der Fälle eintritt, hängt davon ab, ob die elliptische Kurve \Index{komplexe Multiplikation} (CM) hat.
\end{bsp}

\begin{defn}[komplexe Multiplikation (CM)]
	Eine elliptische Kurve $E(k)$ hat \Index{komplexe Multiplikation} (CM), falls sie neben den üblichen Endomorphismen $\Psi_m \colon E(k) \rightarrow E(k), P \mapsto m \cdot P$ mit $m \in \ZZ$ noch weitere hat.
\end{defn}

Elliptische Kurven mit CM haben viele spezielle Eigenschaften, z.B.:
\begin{itemize}
	\item Elliptische Kurven mit CM haben "ebensoviele" $p$ mit $a_p = 0$ wie $p$ mit $a_p \neq 0$.
	\item Elliptische Kurven ohne CM haben nur "wenige" $p$ mit $a_p = 0$.
\end{itemize}
Dennoch konnte N. Elkies 1987 zeigen, dass jede elliptische Kurve $a_p=0$ für unendlich viele $p$ hat.

\begin{bem}
	Beim Übergang von $E(\QQ)$ mit Koeffizienten in $\ZZ$ zu $E(\FF_p)$ zu einer Primzahl $P$ reduzieren wir modulo $p$.
	Nicht immer kommt dabei wieder eine elliptische Kurve heraus, nämlich dann nicht, wenn $\Delta(E(\FF_p)) = 0$ in $\FF_p$ gilt, d.h. $p \mid \Delta(E(\QQ))$ in $\ZZ$.
	Wir sprechen dann von \bet{schlechter Reduktion} (\Index{bad prime}), ansonsten von \bet{guter Reduktion} (\Index{good prime}). \index{gute Reduktion} \index{schlechte Reduktion}
	Die schlechte Reduktion kommt nur für alle Primteiler von $\Delta(E(\QQ)) \in \ZZ$ vor, also nur für endlich viele Primzahlen $p$.
	In diesen Ausnahmefällen verhält sich die kubische Kurve, die durch Reduktion der Gleichung von $E(\QQ)$ modulo $p$ gegeben ist, oft anders; "$+$" gibt es dann nicht.
	Im Beispiel $E(\QQ) \colon y^2 = x^3-4x^2+16$ gilt etwa $N_p \kon 4 \modu 5$ für alle Primzahlen $p$ außer $p = 2$ und $p=11$.
	Tatsächlich sind $p=2$ und $p=11$ hier die Primzahlen mit schlechter Reduktion, da $\Delta(E(\QQ)) 0 -2^{12} \cdot 11$ ist.
\end{bem}

\begin{defn}[Spur des Frobenius]
	Für eine elliptische Kurve $E(\FF_{p^r}), r \geq 1$, wird der Defekt $a_{p^r} = p^r + 1 - N_{p^r}$ auch die \Index{Spur des Frobenius} genannt.
\end{defn}

\begin{defn}[Frobeniusendomorphismus]
	Der \Index{Frobeniusendomorphismus} (kurz: Frobenius\footnote{\url{https://de.wikipedia.org/wiki/Ferdinand_Georg_Frobenius}}) einer elliptischen Kurve $E(\FF_{p^r})$ ist der durch die Abbildung
	\begin{equation}
	\begin{aligned}
		\phi \colon \PP^2(\overline{\FF_{p^r}}) &\longrightarrow \PP^2(\overline{\FF_{p^r}}) \\
		[x:y:z] &\longmapsto \benbrace*{x^{p^r} : y^{p^r} : z^{p^r}}
	\end{aligned}
	\end{equation}
	vermittelte Gruppenhomomorphiusmus $\Phi\colon E(\overline{\FF_{p^r}}) \rightarrow E(\overline{\FF_{p^r}})$.
\end{defn}

\minisec{Beweisskizze}
	\begin{itemize}
		\item $\phi$ ist wirklich eine Abbildung $\PP^2(\overline{\FF_{p^r}})$ in sich.
		\item Ist $E(\FF_{p^r})$ gegeben durch $F(X,Y,Z) = 0$, folgt auch $F(X^{p^r},Y^{p^r},Z^{p^r}) = 0$, weil in $\overline{\FF_{p^r}}$ die Gleichung $(c+d)^{p^r} = c^{p^r} + d^{p^r}$ für $c,d \in \overline{\FF_{p^r}}$ richtig ist.
		Damit ist durch $\Phi$ eine Abbildung von $E(\overline{\FF_{p^r}})$ in sich definiert.
		\item Die Verträglichkeit der Gruppenaddition auf $E(\overline{\FF_{p^r}})$ mit $\Phi$, d.h. die Eigenschaft $\Phi(P_1+P_2) = \Phi(P_1) + \Phi(P_2)$, kann man nachrechnen. \qed
	\end{itemize}
	
\begin{bem}
	Der Frobenius $\Phi$ lässt sich auf allgemeinere Strukturen (genauer: dem Tatemodul) übertragen; dieser lässt eine Matrixdarstellung zu, wobei die Spur dieser Matrix genau $a_{p^r}$ ergibt, daher der Name.
	Dieser Zusammenhang liefert weitere Möglichkeiten, den Defekt $a_{p^r}$ zu bestimmen.
\end{bem}

\begin{anw}[Text in eine elliptische Kurve einbetten]
	Bei der Umsetzung des ElGamal-Verschlüsselungsverfahrens für eine elliptische Kurvengruppe $(G,+) = (E(\FF_{p^r}),+)$, vergleiche Abschnitt~\ref{subsub:1.2.3}, ist es erforderlich, dass sich die Kommunizierenden Alice und Bob darauf einigen, wie man Klartext in eine Folge von Punkten auf der elliptischen Kurve $E(\FF_{p^r})$ übersetzt und wieder zurückerhält.
	Hier ein beispielhaftes Verfahren, wie dies praktisch durchgeführt werden kann:
\end{anw}

\begin{anw}[Schritt 1]
	Man legt ein Alphabet mit $N$ Buchstaben (identifiziert mit $0,1,\dots,N-1$) fest.
	Der Klartext (z.B. ein Wort) habe die Blocklänge $l$.
	Die Zuordnung 
	\[
		w= (a_0 \ a_1 \ \dots \ a_l) \mapsto a_0 N^{l-1} + a_1 N^{l-2} + \dots + a_{l-2} N + a_{l-1} = x_w
	\]
	liefert eine Bijektion zwischen den möglichen Klartextblöcken $w$ und den Zahlen $0 \leq xw < N^l$.
	Eine Zahl $x_w$ soll $x$-Koordinate eines Kurvenpunkts werden.
\end{anw}


\begin{anw}[Schritt 2]
	Für eine gegebene elliptische Kurve $E(\FF_{p^r})$ gibt es aber nicht zu jedem $x \in \FF_{p^r}$ einen Kurvenpunkt $(x_0,y_0) \in E(\FF_{p^r})$.
	Für ein $k \in \NN$ kann man aber die nächste $x$-Koordinate eines Kurvenpunkts $(x_1,y_1) \in E(\FF_{p^r})$ mit $x_0 \leq x_1 \leq x_0+k$ schnell ermitteln; die Wahrscheinlichkeit, dass dies scheitert, d.h. dass ein solches $x_1$ nicht existiert, beträgt schätzungsweise nur etwa $\leg{1}{2}^k$
	(z.B. für $k = 50$ weniger als $10^{-15}$). \\
	Wähle so ein geeignetes $k$ fest und eine elliptische Kurve $E(\FF_{p^r})$ mit $p^r > k \cdot N^l$, d.h. es gibt wohl Kurvenpunkte mit genügend großen $x$-Koordinaten.
\end{anw}

\begin{anw}[Schritt 3]
	Zu $x_w \in \{0,\dots,N^l-1\}$ bestimme $P_w \in E(\FF_{p^r})$ mit $x$-Koordinate $\geq kx_w$, etwa $P_w = (kx_w + j, y)$ mit $j \geq 0$ minimal.
\end{anw}

\begin{bem}
	Wir beobachten: Hat das Verfahren funktioniert, ist dabei $j < k$, sodass durch Berechnung von $x_w = \floor*{\frac{x}{k}}$ für $P_w=(x,y) \in E(\FF_{p^r})$ der Klartext $w$ aus $P_w$ wieder zurückgewonnen werden kann.
\end{bem}

\begin{bsp}
	Für das Alphabet $\{A,B,\dots,Z\} = \{0,1,\dots,25\}$ ist $N = 26$, wähle z.B. $l = 2, k=10$.
	Dann erfüllt $p=6833$ die Bedingung $p > kN^2 = 6760$.
	Ist dazu $E(\FF_p)$ gegeben durch $E(\FF_p)\colon y^2 = x^3 + 5984x + 1180$, kann z.B. der Text "\texttt{KRYPTO}" wie folgt in eine Liste von drei Punkten auf $E(\FF_p)$ umgesetzt werden:
	\begin{center}
		\begin{tabular}{c||c|c|c}
		$w$ & \texttt{KR} & \texttt{YP} & \texttt{TO} \\ 
		\hline $x_w$ & $(10,17)_{26} = 277$ & $(24,15)_{26} = 639$ & $(19,14)_{26} = 508$ \\ 
		\hline $P_w$ & $(2771,353)$ & $(6390,2797)$ & $(5080,238)$
		\end{tabular} 
	\end{center}
\end{bsp}