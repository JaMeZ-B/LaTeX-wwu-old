\documentclass[a4paper, pagesize=pdftex, pdftex, twoside, headsepline, index=totoc,toc=listof, fontsize=10pt, cleardoublepage=empty, headinclude, DIV=13, BCOR=13mm]{scrartcl}

\usepackage[ngerman]{babel}
\usepackage{scrtime} % Bestandteil von KOMA-Skript, ermoeglicht Zugriff auf Uhrzeit des Kompilierens 
\usepackage{scrpage2} % ermöglicht Bearbeiten von Kopf- und Fusszeilen (wie fancyhdr, nur optimiert auf KOMA-Skript, leich andere Syntax)
\usepackage[utf8]{inputenc} % Gibt an in welcher Textcodierung der Code verstandne werden soll
\usepackage{etex} % sehr technisch, ermöglicht LaTeX mehr Speicher zu belegen
\usepackage[T1]{fontenc} % auch sehr technisch; ist wichtig, um die Schriftarten richtig zu behandeln
\usepackage{textcomp} %verhindert ein paar Fehler bei den Fonts
\usepackage{mathtools} % Packet der American Mathematical Society, das viele Mathematik-Umgebungen und -Befehle definiert
\usepackage{amssymb} %zusätzliche Symbole
\usepackage{latexsym} % nochmal zusätzliche Symbole
\usepackage{stmaryrd} % nochmal mehr zusätzliche Symbole, u.a. Blitz für Widerspruchsbeweise ;)
\usepackage{nicefrac} % schräge Brüche, benutzte ich für Quotienvektorräume
\usepackage{paralist} % redefiniert alle Listenbefehle, sodass diese einen optionalen Parameter haben, der die Nummerierung angibt
\usepackage{dsfont} % Schriftart für N,Z,Q,R die ich momentan benutze (mittels \mathds{R} z.B)
\usepackage[pdftex]{graphicx} % Packet, dass das Einbinden von Grafiken aus Dateien ermöglicht
\usepackage{makeidx}% ermöglicht das automatische Anlegen eines Index 
\usepackage{extarrows}
\usepackage{bbold}
\usepackage[hyphens]{url}
\usepackage{algorithmicx}
\usepackage{algpseudocode}


%\usepackage{MnSymbol}
\flushbottom
\usepackage[normalem]{ulem}
\setlength{\ULdepth}{1.8pt}

%--Indexverarbeitung
\newcommand{\bet}[1]{\textbf{#1}} %Betonung von Text
\newcommand{\Index}[1]{\textbf{#1}\index{#1}} % Befehl, der gleichzeitg das Argument hervorhebt und in den Index mitaufnimmt
\makeindex % startet das automatische Sammeln der Index-Einträge
% Ein kleiner Text am Anfang des Index
\setindexpreamble{{\noindent \itshape Die \emph{Seitenzahlen} sind mit Hyperlinks zu den entsprechenden Seiten versehen, also anklickbar!} \par \bigskip}
\renewcommand{\indexpagestyle}{scrheadings} % Seitenstil für den Index festlegen

%--Farbdefinitionen
\usepackage[usenames, table, x11names]{xcolor} %usenames und x11names, aktivieren viele Farben; siehe Dokumentation von xcolor
% Es lassen sich natürlich auch eigene Farben definieren (hier nur Graustufen)
\definecolor{dark_gray}{gray}{0.45}
\definecolor{light_gray}{gray}{0.7}

%--Zum Zeichnen (ich habe es jetzt mal mit aufgenommen, aber es ist eigentlich nochmal ein ganz anderes Thema, sodass ich da jetzt nicht viel zu sagen werde)
\usepackage{tikz} % TikZ steht übrigens für "TikZ ist kein Zeichenprogramm", ein rekursives Akronym ...
\tikzset{>=latex}
\usetikzlibrary{shapes,arrows}
\usetikzlibrary{calc}
\usetikzlibrary{decorations.pathreplacing}
% Hiermit kann man ganz leicht kommutative Diagramme zeichnen (deswegen auch "cd")
\usepackage{tikz-cd}

%--Marginnote, ermöglicht es kleine Notizen an neben den eigentlichen Textkörper zu setzten
\usepackage{marginnote}
\renewcommand*{\marginfont}{\color{Honeydew4} \footnotesize }

%--Schriftarten
\usepackage{lmodern} % neuere Version der Standard-LaTeX-Schriftarten
\renewcommand{\familydefault}{\sfdefault} %Standardschriftart auf die serifenlose Schriftart setzen

%--Hyperref; aktiviert Hyperlinks in der erzeugten PDF-Datei und definiert deren Aussehen
\usepackage[hidelinks, pdfpagelabels,  bookmarksopen=true, bookmarksnumbered=true, linkcolor=black, urlcolor=SkyBlue2, plainpages=false,pagebackref, citecolor=black, hypertexnames=true, pdfauthor={Tobias Wedemeier}, pdfborderstyle={/S/U}, linkbordercolor=SkyBlue2, colorlinks=false]{hyperref}
%--Römische Zahlen
\newcommand{\RM}[1]{\MakeUppercase{\romannumeral #1{}}}



%-- Definitionen von weiteren Mathe-Befehlen, die dann das "richtige" Aussehen haben. Hier sind der Phantasie keine Grenzen gesetzt
\DeclareMathOperator{\id}{id} %identische Abbildung
\DeclareMathOperator{\End}{End} %Endomorphismen
\DeclareMathOperator{\rg}{rg} %Rang
\DeclareMathOperator{\diam}{diam} %Durchmesser
\DeclareMathOperator{\dist}{dist} %Distanz
\DeclareMathOperator{\grad}{grad} %Gradient
\DeclareMathOperator{\rot}{rot} %Rotation
\DeclareMathOperator{\hess}{Hess} %Hesse-Matrix
\DeclareMathOperator{\supp}{supp}
\DeclareMathOperator{\aut}{Aut}
\DeclareMathOperator{\inn}{Inn}
\DeclareMathOperator{\sym}{Sym}
\DeclareMathOperator{\syl}{Syl}
\DeclareMathOperator{\alt}{Alt}
\DeclareMathOperator{\sign}{sign}
\DeclareMathOperator{\Sl}{Sl}
\DeclareMathOperator{\Gl}{Gl}
\DeclareMathOperator{\Quot}{Quot}
\DeclareMathOperator{\ggT}{ggT}
\DeclareMathOperator{\cone}{cone}
\DeclareMathOperator{\Char}{char}
\DeclareMathOperator{\im}{Im}
\DeclareMathOperator{\re}{Re}
\DeclareMathOperator{\Bin}{Bin}
\DeclareMathOperator{\acl}{acl}
\DeclareMathOperator{\cov}{cov}
\DeclareMathOperator{\argmin}{argmin}
\DeclareMathOperator{\argmax}{argmax}
\DeclareMathOperator{\Tol}{TOL}
\DeclareMathOperator{\RK}{RK}
\DeclareMathOperator{\divv}{div}
\DeclareMathOperator{\tr}{tr}
\DeclareMathOperator{\spann}{span}
\DeclareMathOperator{\esssup}{ess sup}
\DeclareMathOperator{\bild}{bild}
\DeclareMathOperator{\cond}{cond}
\DeclareMathOperator{\train}{train}
\DeclareMathOperator{\spur}{spur}
\DeclareMathOperator{\diag}{diag}
%--Skalarprodukt (cooler Befehl, den ich im Internet gefunden habe; benutzt TeX-Befehle)
\makeatletter
\newcommand{\sprod}[2]{\ensuremath{%
  \setbox0=\hbox{\ensuremath{#2}}
  \dimen@\ht0
  \advance\dimen@ by \dp0
  \left\langle \left.#1 \,\rule[-\dp0]{0pt}{\dimen@}\right|#2\right\rangle}}
\makeatother

%--Norm (auch aus dem Internet, wird auch auf der Beispielseite verwandt)
\newcommand{\norm}[1]{
	\ensuremath{\left\Vert#1\right\Vert}
}
\newcommand{\ener}[1]{
	\ensuremath{\left|\left|\left|#1\right|\right|\right|_\mu}
}

%--selbstgeschriebenen Befehle
%--Betrag
\newcommand{\abs}[1]{\ensuremath{\left\vert#1\right\vert}}

%--Umklammern mit passender Größe der Klammern
\newcommand{\enbrace}[1]{\ensuremath{\left( #1\right)}}

%--Mengen
\newcommand{\penbrace}[1]{\ensuremath{\left\{#1\right\}}}
\newcommand{\cenbrace}[1]{\ensuremath{\left[#1\right]}}

%--Differential
\newcommand{\diff}[2]{\ensuremath{\frac{\partial #1}{\partial #2} }}
\newcommand{\difff}[2]{\ensuremath{\frac{\dint #1}{\dint #2} }}

\newcommand{\zz}{\ensuremath{\mathrm{Z\kern-.3em\raise-0.5ex\hbox{$Z$}}}} % zu zeigen ZZ aus dem inet
\setlength{\parindent}{0pt}%absatz nicht einrücken
\newcommand{\lh}[1]{\langle #1 \rangle} %lineare Hülle
\newcommand{\nt}{\trianglelefteqslant} %normalteiler
\newcommand{\pfs}{\mathds{P}-\text{f.s.}} %P fast sicher Konvergenz
\newcommand{\dint}{\mathrm{d}} % d des integrals

\newcommand{\xfrac}[2]{%
	\mbox{\raisebox{-0.4ex}{\ensuremath{\displaystyle #1}\hspace{0.2ex}}%
		{\raisebox{-0.1ex}{\big \backslash}}%
		\raisebox{0.6ex}{\ensuremath{\displaystyle #2}}%
	}%
}
\newcommand{\Pw}{\mathds{P}}
\newcommand{\V}{\mathds{V}}
\newcommand{\E}{\mathds{E}}
\newcommand{\R}{\mathds{R}}
\newcommand{\N}{\mathds{N}}
\newcommand{\Z}{\mathds{Z}}
\newcommand{\Q}{\mathds{Q}}
\newcommand{\G}{\mathcal{G}}
\newcommand{\F}{\mathcal{F}}
\newcommand{\D}{\mathcal{D}}
\newcommand{\C}{\mathds{C}}
\newcommand{\A}{\mathcal{A}}
\newcommand{\mc}[1]{\mathcal{#1}}
\newcommand{\Pfs}[1][\relax]{\ensuremath{
	\ifx#1\relax \Pw-\text{f.s.}
	\else \Pw^{#1}-\text{f.s.}
	\fi}}
\newcommand{\bgl}[1]{\stackrel{#1}{=}}
\newcommand{\dt}[1]{#1 \dots #1}
\newcommand{\ablim}[1]{\limits_{\mathclap{#1}}}
\newcommand{\degree}{\ensuremath{^\circ}}
\newcommand{\adj}{\ensuremath{^\ast}}



\newcommand{\sect}[1]{\section*{#1}\addcontentsline{toc}{section}{#1}}
\newcommand{\ssect}[2]{ \subsubsection*{#1 #2}\addcontentsline{toc}{subsubsection}{#1}}


\usepackage{listings}
\definecolor{light_green}{rgb}{0,0.5,0}
\definecolor{grey}{rgb}{.5,.5,.5}
\lstset{language=Python, commentstyle=\color{light_green}\bfseries,
	keywordstyle=\color{blue}\bfseries,
	stringstyle=\ttfamily\color{orange},
	morekeywords={as},
	deletendkeywords={range,abs,set},
	escapeinside={\%*}{\&*}
}

\lstset{showspaces=false,
	showstringspaces=false, tabsize=4, breaklines=true, rulecolor=\color{black}}
\lstset{numbers=left,basicstyle=\ttfamily\footnotesize, numberstyle=\tiny\color{grey}, numbersep=10pt}

\renewcommand{\lstlistlistingname}{Programmcodeverzeichnis}
\renewcommand{\lstlistingname}{Programmcode}








\newcommand{\vorlesung}{Modellreduktion und partielle Differentialgleichungen}
\newcommand{\Prof}{Dr. Smetana}
\newcommand{\subt}{Mitschrift der Tafelnotizen}

\input{../!config/Tazdr/extra_files/headings.tex}


\numberwithin{equation}{section}


\begin{document}
\maketitle
\thispagestyle{empty}
\cleardoubleoddemptypage

\thispagestyle{empty}
\vspace*{\fill}
\begin{center}
	Hierbei handelt es sich um eine \subt von \textbf{\Prof}, WWU Münster, aus der Vorlesung \textbf{\vorlesung} im Wintersemester 2015/16. 
	Dies ist kein Skript der Vorlesung und keine eigene Arbeit des Autors.\\
	\vspace{2cm}
	Für Fehler in der Mitschrift wird keine Haftung übernommen. 
	Hinweise auf Fehler sind gerne gesehen, hierfür kann man mich in der Uni ansprechen oder alternativ eine e-Mail an: \textit{tobias.wedemeier@gmx.de}\\
	Auch ist eine Mitarbeit über Github möglich.\\
	\vspace{2cm}
	Wenn Teile aus der Vorlesung selber fehlen, können diese gerne an meine e-Mail versandt werden. 
	Ich werde diese dann einarbeiten.\\
\end{center}
\vspace*{\fill}
\cleardoubleoddemptypage

\pagenumbering{Roman}

\tableofcontents
\cleardoubleoddemptypage %sorgt dafür, dass alles folgende erst auf der nächsten freien "rechten" Seite steht
\pagenumbering{arabic}

\section{Einletung und Motivation}

\subsection{Parameterabhängige PDGL}
\label{sub:para_pdgl}
Sei $\Omega\subseteq \R^d$ ein polygonales Gebiet. Zu einem Parametervektor $\mu \in P\subseteq \R^d$ aus einer Menge von 'erlaubten' Parametern ist eine Funktion, z.B. 'Temperatur'
\[
u(\mu): \Omega\to \R
\]
gesucht, so dass $-\nabla\big(\kappa(\mu)\nabla u(\mu)\big) = q(\mu)$ in $\Omega$, wobei $u(\mu) = 0$ auf $\partial\Omega$, mit $\kappa(\mu):\Omega\to \R$ dem 'Wärmeleitkoeffizient' und $q(\mu)$ eine 'Wärmequelle', z.B. $q(\mu)=1$.
%grafik 1
Weiter kann eine Augabe erwünscht sein, z.B.
\[
s(\mu) = \frac{1}{\abs{\Omega_s}} \int\lim\limits_{\Omega_s} u(x,\mu)\dint x,
\]
die mittlere Temperatur auf $\Omega_s$.
%grafik 2
%sub end

\subsection{Definition (schwache Formulierung in Hilberträumen)}
\label{sub:def_hilbert}
Sei $X$ ein reeller Hilbertraum.
Zu $\mu\in P$ ist gesucht ein $u(\mu)\in X$ und eine Ausgabe $s(\mu)\in \R$, so dass
\[
b\big(u(\mu),v;\mu\big) = f(v;\mu), ~ s(\mu) = l\big(u(\mu);\mu\big)~\forall v\in X
\]
für eine Bilinearform $b(\cdot,\cdot;\mu): X\times X \to \R$ und linearen Funktionalen $f(\cdot;\mu),l(\cdot;\mu):X\to \R$.

Die schwache Formulierung für Beispiel 1.1 lautet:
\[
X:= H_0^1(\Omega)= \enbrace{f\in L^2(\Omega)~:+\difff{ }{x_1}f\in L^2(\Omega),~f|_{\partial \Omega = 0}}
\]
Dann kann man die Bilinearform über
\[
b\big(u(\mu),v;\mu\big) := \int\limits_{\Omega} \kappa(\mu)\nabla u(\mu)\nabla v \dint x; f(v;\mu) := \int\lim\limits_{\Omega} q(\mu) v \dint x
\]
ausdrücken und 
\[
s(\mu) = \frac{1}{\abs{\Omega_s}} \int\lim\limits_{\Omega_s} u(x;\mu)\dint x =: l\big(u(\mu);\mu\big)
\]
ABER: Für sehr wenige PDGL's können wir die Lösung analytisch bestimmen.
Daher sind wir an einer numerische Approximation interessiert.
Ein weit verbreitetes Diskretisierungsverfahren ist die Finite Elemente Methode.
Diese Methode basiert auf obiger schwacher Formulierung.
%sub end

\subsection{Definition (hochdimensionales, diskretes Modell)}
\label{sub:def_hochdim}
Sei $X_h\subseteq X$ mit $\dim(X_h) =N_h < \infty$.
Der Index $h$ bezeichnet hier die Gitterweite.
Zu $\mu\in P$ ist gesucht ein $u_h(\mu)\in X_h$ und eine Ausgabe $s_h(\mu)\in \R$, so dass
\begin{align}
b\big(u_h(\mu),v_h;\mu\big) = f(v_h;\mu), ~ s_h(\mu)= l_h\big(v_h(\mu);\mu\big)~\forall v_h\in X_h.
\end{align}
Anwendungen für die Standarddiskretisierungsverfahren sehr teuer oder zu teuer sind:
\minisec{many-query context}
\begin{itemize}
	\item Parameterstudien
	\item Design
	\item Parameteridentifikation / inverse Probleme
	\item Optimierung
	\item Statistische Analyse
\end{itemize}
\minisec{schnelle Simulationsantwort}
\begin{itemize}
	\item Echtzeit-Steuerung technischer Geräte
	\item interaktive Benutzeroberflächen
\end{itemize}
%sub end

\subsection{Parameterabhängige Lösungsmenge}
\label{sub:para_menge}
Sei $\mu :=\{u(\mu)~:~\mu\in P\}\subseteq P$ für $P\in \R^p$ ist die durch $\mu$ parametrisierte Lösungsmenge.
$X$ ist die im Allgemeinen unendlichdimensional.
%grafik
$\Rightarrow$ Motivation für die Suche nach einem 'niedrigdimensionalen' Teilraum $X_N\subseteq X$ zur Approximation von $M$ und einer Approximation $u_N(\mu) \approx u(\mu),~ u_N\in X_N$.
Eine Möglichkeit eine reduzierte Basis zu generieren besteht darin geschickt Parameterwerte $\mu_1\dt{,}\mu_N\in P$ zu wählen und den Raum als $X_N := \spann\{u(\mu_1)\dt{,}u(\mu_N)\}$ zu definieren.
Eine Lösung $u(\mu_i)$ für einen Parameterwert $\mu\in P$ wird auch \bet{Snapshot} genannt.

\subsection{Beispiel}
\label{sub:beispiel_1}
Gesucht ist $u(\cdot;\mu) \in C^2([0,1])$ mit $(1+\mu)u'' = 1$ auf $(0,1)$ und $u(0)=u(1)=1$ für den Parameter $\mu\in P:= [0,1]\subseteq \R$.\\
\bet{Snapshots:}\\
$\mu_1 = 0 \Rightarrow u_1 := u(\cdot;\mu_1) = \frac{1}{2} x^2- \frac{1}{2}x +1$, $\mu_2 = 0 \Rightarrow u_2 := u(\cdot;\mu_2) = \frac{1}{4} x^2- \frac{1}{4}x +1$ und $X_N := \spann\{u_1,u_2\}$.
Dann ist die reduzierte Lösung $u_N(\mu)\in X_N$ gegeben durch
\[
u_N(\mu) = \alpha_1(\mu)u_1 +\alpha_2(\mu) u_2,
\]
mit $\alpha_1 = \frac{2}{\mu+1}-1$ und $\alpha_2 = 2-\frac{2}{\mu-1}$.
Diese erfüllt folgende Fehleraussage und ist somit exakt:
\[
\norm{u_N(\mu)-u(\mu)}_\infty = \sup\limits_{\lambda\in[0,1]} \abs{U_N(x;\mu)-u(x;\mu)} = 0
\]
Da $\alpha_1 + \alpha_2 = 1$ und $0\le \alpha_1,\alpha_2\le 1$ ist $M$ die Menge der Konvexkombinationen von $u_1$ und $u_2$.
%sub end

\subsection{Definition (reduziertes Modell)}
\label{sub:reduziertes_modell}
Sei $X_N\subseteq X$ ein reduzierter Basisraum mit $\dim(X_N)<\infty$.
Zu $\mu\in P$ ist gesucht ein $u_N(\mu)\in X_N$ und eine Ausgabe $s_N(\mu)\in \R$, so dass
\begin{align}
b\big(u_N(\mu),v_N;\mu\big) = f(v_N;\mu),~ s_N(\mu) = l_N\big(u_N(\mu);\mu\big) ~ \forall v_N\in X_N
\end{align}
%sub end

\subsection{Bermerkung (Begrifflichkeit)}
\label{sub:bem_begrifflichkeit}
Zusammengefasst unterscheiden wir zwischen den folgenden drei Modellen:
\begin{enumerate}[1)]
	\item Eine partielle DGL ist ein \bet{analytisches Modell}, welches die analytische Lösung $u(\mu)\in X$ in einem (typischerweise) $\infty$-dimensionalen Funktionenraum charakterisiert ist.
	\item Ein \bet{hochdimensionales, diskretes Modell} ist ein Berechnungsverfahren zur Bestimmung einer Näherung $u_h(\mu)\in X_h$, wobei $X_h$ ein hochdimensionaler Funktionenraum ist.
	Beispiele sind \bet{Finite Elemente} oder \bet{Finite Volumenräume} und typischerweise hat $X_h$ eine Dimension von mindestens $10^5$.
	\item Ein \bet{reduziertes Modell} ist ein Berechnungsverfahren zur Bestimmung einer Näherung $u_N(\mu)\in X_N$ in einem sehr problemangepassten und daher niedrigdimensionalen Raum von typischerweise $\dim X_N < 100$.
	\item \bet{Modellreduktion} beschäftigt sich mit Modellen der Erzeugung von reduzierten Modellen aus hochdimensionalen, diskreten (oder auch analytischen) Modellen und Untersuchungen ihrer Eigenschaften.
\end{enumerate}

\subsection{Organisation der Vorlesung}
\label{sub:org}
\begin{itemize}
	\item[Zentrale Fragen:]
	\item \bet{Reduzierte Basis:} Wie kann ein möglichst kompakter Teilraum konstruiert werden?
	\item \bet{Reduziertes Modell:} Existenz von reduzierten Lösungen $u_N(\mu)$?
	Wie kann eine reduzierte Lösung $u_N(\mu)$ berechnet werden?
	\item \bet{Effizienz:} Wie kann $u_N(\mu)$ schnell berechnet werden?
	\item \bet{Stabilität:} Wie kann die Stabilität des reduzierten Modells für wachsendes $N$ garantiert werden?
	\item\bet{Approximationsgüte:} Warum können wir erwarten, dass eine relativ kleine Anzahl von Basisfunktionen ausreicht?
	\item \bet{Fehlerschätzer:} Kann der Fehler des reduzierten zum vollen Modell beschränkt werden?
	\item \bet{Effektivität:} Kann garantiert werden, dass der Fehlerschätzer den Fehler nicht beliebig überschätzt?
\end{itemize}

\minisec{Vorläufige Gliederung (bis Weihnachten)}
\begin{enumerate}[1)]
	\item Einleitung / Moitavtion
	\item Grundlagen:
	\begin{itemize}
		\item Kurze Einführung in lineare Funktionalanalysis
		\item Kurze Einführung in Finite Elemente
	\end{itemize}
	\item Reduzierte Basis Methoden für lineare, koerzive Probleme
	\begin{itemize}
		\item Reduzierte Basis Verfahren
		\item Offline-/ Online-Zerlegung
		\item Fehlerschätzer
		\item Basisgenerierung
	\end{itemize}
\end{enumerate}
%sub end

%sec end

\section{Grundlagen}

\subsection{Lineare Funktionalanalysis in Hilberträumen}

\subsubsection{Lineare Operatoren}

\ssect{2.1 Definition}{(Hilbertraum)}
Sei $X$ ein reeller Vektorraum mit $(\cdot,\cdot):X\times X\to \R$ ein Skalarprodukt und induzierter Norm $\norm{x}:= \sqrt{(x,x)}$.
falls $X$ vollständig bzgl. $\norm{.}$ , ist $X$ ein (reeller) \bet{Hilbertraum} (HR).

\ssect{2.2 Beispiele}{(Hilbertraum)}
\begin{enumerate}[(1)]
	\item $X:= \R^d$ mit $(x,y) := \sum_{i=1}^{d} x_i y_i$ ist ein HR.
	\item $X:= L^2(\Omega)$ mit $(x,y) := \int_{\Omega} f(x)g(x)\dint x $ ist ein HR.
	\item $X:=C^0([0,1])$ mit $(f,g) := \int_{0}^{1} f(x) g(x) \dint x$ ist kein HR.
\end{enumerate}

\ssect{2.3 Lemma}{ }
Seien $X$ und $Y$ reelle Vektorräume.
Ist die Abbildung $T:X\to Y$ linear und $x_0\in X$, so sind äquivalent:
\begin{enumerate}[(1)]
	\item $T$ ist stetig.
	\item $T$ ist stetig in $x_0$.
	\item $\sup\lim\limits_{\norm{x}_X \le 1} \norm{Tx}_Y < \infty$.
	\item $\exists$ Konstante (mit $\norm{Tx}_Y \le c\norm{x}_X ~\forall x\in X$)
\end{enumerate}

\ssect{2.4 Definition}{(Lineare Operatoren)}
Seinen $X$ und $Y$ reelle Vektorräume.
Wir definieren
\[
L(X;Y) := \enbrace{T:X\to Y~;~ T \text{ ist linear und stetig}}.
\]
Abbildungen in $L(X;Y)$ nennen wir \bet{lineare Operatoren}.
Nach Lemma 2.3 (3) ist für jeden Operator $T\in L(X;Y)$ die \bet{Operatornorm} von $T$ definiert durch
\[
\norm{T}_{L(X;Y)} := \sup\limits_{\norm{x}_X\le 1} \norm{Tx}_Y < \infty,
\]
oder in kurz $\norm{T}$.
Es ist $L(X) := L(X;X)$.

\ssect{2.5 Definition}{(Spezielle lineare Operatoren)}
\begin{enumerate}[(1)]
	\item $X' := L(X;\R)$ ist der \bet{Dualraum} von $X$.
	Die Elemente von $X'$ nennen wir auch \bet{lineare Funktionale}.
	\item Die Menge der kompakten (linearen) Operatoren von $X$ nach $Y$ ist definiert durch
	\[
	K(X;Y) := \enbrace{T\in L(X;Y)~;~ T(\overline{B_1(0)}) \text{ kompakt}}.
	\]
	\item Eine lineare Abbildung $P:X\to X$ heißt (lineare) \bet{Projektion}, falls $P^2=P$.
	\item Für $T\in L(X;Y)$ ist $\ker(T) := \enbrace{x\in X~;~ Tx = 0}$ der \bet{Nullraum} oder \bet{Kern} von $T$.
	Aus der Stetigkeit von $T$ folgt, dass $\ker(T)$ ein abgeschlossener Unterraum ist.
	Der \bet{Bildraum} von $T$ ist $\bild(T) := \enbrace{Tx\in Y~;~x\in X}$.
	\item Ist $T\in L(X;Y)$ bijektiv, so ist $T^{-1}\in L(Y;X)$.
	Dann heißt $T$ (linear, stetiger) \bet{Isomorphismus}.
	\item $T\in L(X;Y)$ heißt \bet{Isometrie}, falls
	\[
	\norm{Tx}_Y = \norm{x}_X ~\forall x\in X
	\]
\end{enumerate}

\ssect{2.6 Beispiel}{ }
Sei $g\in L^2(\Omega)$.
Dann ist nach der Hölderungleichung durch 
\[
T_g f := \int_{\Omega} f(x) g(x)\dint x
\]
ein Funktional $T_g\in  L^2(\Omega)'$ definiert.

\ssect{2.7 Satz}{(Projektionssatz)}
Sei $X$ ein Hilbertraum und $A\subseteq X$ nicht leer, abgeschlossen und konvex.
Dann gibt es genau eine Abbildung $P:X\to A$ mit
\[
\norm{x-Px}_X = \dist(x,A) = \inf\lim\limits_{y\in A} \norm{x-y}_X ~\forall x\in X.
\]
Die Abbildung $P:X\to A$ heißt orthogonale Projektion von $X$ auf $A$.

\bet{Beweis:} [Alt, Satz 2.2, S.96]

\ssect{2.8 Folgerung}{ }
Ist $A\subseteq X$ nicht-leer, abgeschlossen und Unterraum, so ist $P$ linear und $Px\in A$ charakterisiert durch $(x-Px,a)_X = 0~\forall a\in A$.
Falls $\dim(A) = n<\infty$ und $\enbrace{\varphi_i}_{i=1}^h$ Orthonormalbasis von $A$, gilt
\[
Px = \sum_{i=1}^{n} (x, \varphi_i)_X\varphi_i.
\]

\ssect{2.9 Satz}{(Riesz'scher Darstellungssatz)}
Ist $X$ Hilbertraum, so ist $J:X\to X'$ definiert durch
\[
J(v)(w) := (v,w)_X ~\forall v,w\in X 
\]
eine stetige, lineare, bijektive Isometrie.
Insbesondere existiert zu $l\in X'$ ein eindeutiger \bet{Riesz Repräasentant} $V_l := J^{-1}(l)\in X$ mit $l(.) = (v_l,.)_X$.\\

\bet{Beweis:}\\
C-S-Ungleichung: $\abs{J(v)(w)}\le \norm{v}_X\norm{w}_X$.
Dann folgt: $ J(v)\in X'$ mit 
\[
\norm{J(v)}_{X'} = \sup\limits_{w\in X\backslash\{0\}} \frac{\abs{J(v)(w)}}{\norm{w}_X} = \sup\limits_{w\in X\backslash\{0\}} \frac{\abs{(v,w)_X}}{\norm{w}_X} \le \norm{v}_X \Rightarrow J \text{ stetig}.
\]
Da $\abs{J(v)(v)} = \norm{v}_X^2$ folgt:
\[
\sup\limits_{w\in X\backslash\{0\}} \frac{\abs{J(v)(w)}}{\norm{w}_X} \ge \frac{\abs{J(v)(v)}}{\norm{v}_X} = \frac{\norm{v}_x^2}{\norm{v}_X} = \norm{v}_X.
\]
Also ist $J$ eine Isometrie und insbesondere ist $J$ injektiv.\\
Zeige $J$ surjektiv: Sei $l\in X',~l\neq 0$, Kern$(l)$ ist abgeschlossener Teilraum, also existiert $P:X\to \ker(l)$ orthogonale Projektion nach Satz 2.7.
Sei $v_0\in X$mit $l(v_O)=1$. 
Setze $v_1:=v_o-Pv_0\Rightarrow l(v_1) =l(v_0) =1$ und $v_1\neq 0$.
Mit Folgerung 2.8:
\[
\Rightarrow (w,v)_X =0~ \forall x\in\ker(l) \Rightarrow v_1 \perp \ker(l). 
\]
Für $v\in X$ gilt
\[
v\underbracket{v-l(v)}_{\in \ker(l)}\cdot v_1 + l(v)\cdot v_1
\]
und $v-l(v)v_1\in \ker(l)$ wegen
\[
l(v-l(v)v_1) = l(v) -l(v)l(v_1) =0.
\]
Also ist
\begin{align*}
(v_1,v)_X &= (\underbracket{v_1,v-l(v)v_1}_{=0,\text{ da }\ker(l) \perp v_1})_X + (v_1,l(v)v_1)_X\\
&= l(v) \norm{v_1}_X^2\\
&\Rightarrow l(v) = \enbrace{\frac{v_1}{\norm{v_1}_X^2},v}_X = J\enbrace{\frac{v_1}{\norm{v_1}_X^2}}(v).\\
&\Rightarrow l\in \bild(J) \Rightarrow J\text{ bijektiv.}\\
\end{align*}
\hfill $\square$

\ssect{2.10 Folgerung / Beispiel:}{ }
Mit Hilfe des Rieszschen Darstellungssatz können wir damit $L^2(\Omega)$' - den Dualraum von $L^2(\Omega)$ - charakterisieren.
Wie in 2.6 definieren wir für $g\in L^2(\Omega)$ das Funktional
\[
T_g f := \int_{\Omega} f(x)g(x)\dint x.
\]

\ssect{Definition 2.11}{(Bilinearformen)}
Seien $X_1,X_2$ Hilberträume, $b:X_1\times X_2 \to \R$ eine Bilinearform.
\begin{enumerate}[(1)]
	\item Falls 
	\[
	\gamma:= \sup\lim\limits_{u\in X_1\backslash\{0\}} \sup\lim\limits_{v\in X_2\backslash\{0\}} \frac{b(u,v)}{\norm{u}_{X_1}\norm{v}_{X_2}} < \infty
 	\]
 	so ist $b$ stetig mit Stetigkeitskonstante $\gamma$.
 	\item Falls $X=X_1=X_2$, definieren 
 	\[
 	b_s(u,v) = \frac{1}{2} b(u,v)+b(v,u),~b_a = \frac{1}{2} b(u,v)-b(v,u) ~\forall u,v\in X
 	\]
 	den symmetrischen bzw. antisymmetrischen ANteil von $b = b_s +b_a$.
 	\item Falls $X=X_1=X_2$, $b$ stetig und 
 	\[
 	\alpha := \inf\limits_{u\in X\backslash\{0\}} \frac{b(u,u)}{\norm{u}_X^2} > 0
 	\]
 	heißt $b$ Koerziv mit Stetigkeitskonstante $\alpha$.
\end{enumerate}

\ssect{2.12 Bemerkung}{}
\begin{enumerate}[(1)]
	\item $\alpha \in\R$ ist wohldefiniert, denn mit Stetigkeit folgt
	\[
	\frac{b(u,u)}{\norm{u}_X^2} \ge -\gamma \frac{\norm{u}_X\norm{u}_X}{\norm{u}_X^2} = -\gamma.
	\]
	\item $b$ ist koerziv bzgl. $\alpha \Leftrightarrow b_s$ ist koerziv bzgl. $\alpha$.
\end{enumerate}

\ssect{2.13 Satz}{(Operatoren und Bilinearformen)}
Seien $X_1,X_2$ Hilberträume.
\begin{enumerate}[(1)]
	\item Zu $B\in L(X_1,X_2)$ existiert eine eindeutig definierte stetige Bilinearform $b:X_1\times X_2\to \R$ mit 
	\begin{align}
	b(u,v) = (Bu,v)_{X_2} ~\forall u\in X_1,v\in X_2.
	\end{align}
	\item Zu $b:X_1\times X_2 \to \R$ stetige Bilinearform existiert eindeutiges $B\in L(X_1,X_2)$ welches (2.1) erfüllt.
\end{enumerate}

\bet{Beweis:}\\
\begin{enumerate}[(1)]
	\item $b$ definiert durch (2.1) ist bilinear wegen Bilinearität von $(.,.)$ und Linearität von $B$.
	Stetigkeit:
	\[
	b(u,v) = (Bu,v)_{X_2} \stackrel{\text{C.S.}}{\le} \norm{B}\norm{u}_{X_1}\norm{v}_{X_2}
	\]
	daraus folgt $\gamma \le \norm{B} < \infty$.
	\item Sei $u\in X_1$ fest.
	Dann ist $b(u,.):X_2\to\R$ linear und stetig:
	\[
	\sup\limits_{v\in X_2\backslash\{0\}} \frac{b(u,v)}{\norm{v}_{X_2}} \le \sup\limits_{v\in X_2\backslash\{0\}} \frac{\norm{u}_{X_1\norm{v}_{X_2}}}{\norm{v}_{X_2}}\cdot \gamma = \gamma \norm{u}_{X_1} < \infty.
	\]
	Daraus folgt $b(u,.)\in X_2'$ und es existiert nach Satz 2.9 ein eindeutiger Riesz-Repräsentant $v_u\in X_2$ mit $b(u,.) = (v_u,.)$.
	Definiere $B:X_1\to X_2$ durch $Bu := v_u\in X_2$.
	Hiermit (2.1) und Eindeutigkeit klar.
	Linearität damit klar.\\
	Stetigkeit:
	\begin{align*}
	\norm{bu}^2 &= (Bu,Bu) = (v_u,Bu)_{X_2} = b(u,Bu) \le \gamma \norm{u}_{X_1}\norm{Bu}_{X_2}\\
	&\Rightarrow \norm{Bu}_{X_2} \le \gamma \norm{u}_{X_1} \Rightarrow \sup\limits_{u\in X_1\backslash\{0\}} \frac{\norm{Bu}_{X_2}}{\norm{u}_{X_1}} \le \gamma.
	\end{align*}
\end{enumerate}
\hfill $\square$\\

\ssect{2.14 Satz}{von Lax-Milgram}
Sei $x$ HR, $b:X\times X\to\R$ koerzive, stetige Bilinearform mit Koerzivitätskonstante $\alpha$.
Dann existiert ein eindeutiger Operator $B\in L(X)$ mit
\[
b(u,v) = (Bu,v)~\forall u,v\in X.
\]
Ferner gil: $B$ ist bijektiv, $B^{-1}\in L(X)$ mit
\[
\norm{B} \le \gamma \text{ und } \norm{B^{-1}}\le \frac{1}{\alpha}.
\]

\subsubsection{Sobolevräume}

\ssect{2.15 Bemerkung}{(Motivation Sobolevräume)}
Wie in 1.1 motiviert, eignet sich die sogenannte Schwache Formulierung (s. 1.2) einer PDgl besonders gut um Existenz und Eindeutigkeit von Lösungen zu untersuchen.
Die dazu geeigneten Räume sind die \bet{Sobolevräume}.

\ssect{2.16 Definition}{($L_{log}^p(\Omega)$)}
Sei $\Omega\subset \R^d$ ein Gebiet.
Dann ist der Raum $L_{log}^p(\Omega)$ definiter durch 
\[
L_{log}^p(\Omega) := \penbrace{u\in L^p(K) ~|~ \forall K\subset \Omega,~ K\text{ kompakt}}.
\]

\ssect{2.17 Definition}{(schwache Ableitung)}
Sei $\alpha = (\alpha_1\dt{,}\alpha_d)\in \N^d$ ein Multiindex.
Eine Funktion $u\in L_{log}^1(\Omega)$ besitzt eine schwache Ableitung $u_\alpha \in L_{log}^1(\Omega)$, wenn für alle Testfunktionen $\varphi\in C_0^\infty(\Omega)$ gilt
\[
\int_{\Omega} u D^\alpha \varphi = (-1)^{\abs{\alpha}} \cdot \int_{\Omega} u^{(\alpha)} \varphi,
\]
mit $D^\alpha = D_1^{\alpha_1}\cdots D_d^{\alpha_d},~\abs{\alpha} = \alpha_1 \dt{+} \alpha_d$.
Wir schreiben dann auch $u^{(\alpha)} = D^\alpha u$ für die schwache Ableitung.

\ssect{2.18 Lemma}{}
Falls $u\in C^{\abs{\alpha}(\bar{\Omega})}$ und $\abs{\alpha}\ge 1$, gilt: $D^\alpha= u^{(\alpha)}$, d.h. klassische und schwache Ableitung stimmen überein.

\ssect{2.19 Beispiel}{}
Sei $\Omega =(-1,1)$ und $u(x) =\abs{x}$.
Dann ist $u'(x) = -1 (x\le 0), 1 (x> 0)$ die schwache Ableitung von $u$.\\

\bet{Beweis:}\\
Esgilt füt beliebige $\varphi\in C_0^\infty(\Omega)$:
\begin{align*}
Foto
\end{align*}

\ssect{2.20 Beispiel}{}
Im Gegensatz zu $\abs{x}$ ist $v(x)= -1 (x\le 0) 1 (x>0)$ auf $\Omega = (-1,1)$ nicht schwach differenzierbar.

\ssect{2.21 Definition}{(Sobolevräume)}
Seinen $m\in \N_0,~p\in [1,\infty]$ und $u\in L_{log}^p(\Omega)$.
Wir nehmen an, dass alle schwachen partiellen Ableitungen $D^\alpha u$ existieren für $\abs{\alpha} \le m$.
Dann definieren wir die \bet{Sobolevnormen} $\norm{u}_{H^{m,p}(\Omega)}$, durch
\[
\norm{u}_{H^{m,p}(\Omega)} = \enbrace{\sum_{\abs{\alpha}\le m} \norm{D^\alpha u}_{L^p(\Omega)}^p}^{\frac{1}{p}}\text{ falls } 1\le p< \infty
\]
und für $p=\infty$ als 
\[
\norm{u}_{H^{m,p}(\Omega)} := \max\lim\limits_{\abs{\alpha}\le m} \norm{D^\alpha u}_{L^\infty (\Omega)}.
\]
Schließlich definieren wir die \bet{Sobolevräume} $H^{m,p}(\Omega)$ durch
\[
H^{m,p}(\Omega) := \penbrace{u\in L_{log}^p(\Omega) ~|~ \norm{u}_{H^{m,p}(\Omega)}< \infty}.
\]

\ssect{2.22 Bemerkung}{}
Anstelle von $H^{m,p}(\Omega)$ werden die Sobolevräume in der Literatur auch oft mit $W^{m,p}(\Omega)$ bezeichnet.

\ssect{2.23 Beispiel}{}
Seien $\Omega = B_{\frac{1}{2}}(0)\subset \R^2$ und $u(x) = \ln \abs{\ln \abs{x}},~x\in \Omega$.
Dann gilt: $u\in H^{1,2}(\Omega)$, aber $u\notin C^0(\Omega)$.
D.h. Funktionen in $H^{1,p}(\Omega)$ sind in mehreren Raumdimensionen nicht notwendigerweise stetig.

\ssect{2.24 Satz}{(Vollständigkeit von Sobolevräumen)}
Sei $\Omega\subset \R^d$ ein Gebiet.
Damm ist $H^{m,p}(\Omega)$ $1\le p\le \infty,~m\in \N_0$ mit der in 2.21 definierten Norm ein Banachraum,$H^{m,p}(\Omega)$ ist ein Hilbertraum mit dem Skalarprodukt
\[
(u,v)_{H^{m,p}(\Omega)} := \sum_{\abs{\alpha}\le m} (D^\alpha u, D^\alpha v)_{L^2(\Omega)}.
\]
Da wir uns mit Randwertproblemen befassen wollen, ist es notwendig zu klären in welchem Sinne wir bei Sobolevräumen von Randwerten reden können.
Da die Funktionen zunächst nur bis auf Nullmengen definiert sind und der Rand eines Gebietes eine Nullmenge darstellt, auf der man $L^p$-Funktionen beliebig abändern kann.
In der folgenden Definition klären wir zunächst was wir unter Nullrandwerten im schwachen Sinne verstehen wollen.

\ssect{2.25 Definition}{(schwache Nullrandwerte)}
Für $1\le p\le \infty$ und $m\in \N$ definieren wir die Sobolevräume mit Nullrandwerten $H_0^{m,p}(\Omega)$ durch
\[
H_0^{m,p}(\Omega) := \overline{C_0^m(\Omega)}^{\norm{.}_{H^{m,p}(\Omega)}}.
\]

\ssect{2.26 Satz}{}
Für $1\le p<\infty$ ist $H_0^{m,p}(\Omega)$ ein abgeschlossener Teilraum von $H^{m,p}(\Omega)$ und damit ein Banachraum.\\

Dass aus der Definition von $H_0^{m,p}(\Omega)$ tatsächlich folgt, dass solche Funktionen Randwerte besitzen, drückt der folgende Satz aus.

\ssect{2.27 Satz}{(Spursatz)}
Sei $\Omega\subset \R^d$ ein Lipschitz-Gebiet und $1\le p<\infty$.
Dann gibt es einen linearen \bet{Spuroperator} $\tau: H^{1,p}(\Omega) \to L^p(\partial\Omega)$, so dass für $u\in H^{m,p}(\Omega)\cap C^\infty(\bar{\Omega})$ gilt:
\[
\tau u = u|_{\partial\Omega}.
\]
Insbesondere gilt für $u\in H_0^{1,p}(\Omega): \tau u = 0$.\\

\bet{Beweis:}\\
Im Buch von Alt oder von Evans.

\ssect{2.28 Satz}{(2. Soblev'scher Einbettungssatz)}
Sei $1\le p<\infty$, dann gilt:
\[
H^{1,p}((a,b)) \hookrightarrow C^0([a,b]),
\]
d.h. dass (möglicherweise nach Änderung von Funktionswerten auf einer Nullmenge) Funktionen in $H^{1,p}((a,b))$ stetig sind.

Sei nun $\Omega\subset \R^d$ ein Gebiet und $1\le p<\infty$. 
Dann gilt 
\[
H_0^{2,p}(\Omega) \hookrightarrow C^0(\Omega), \text{ falls } 2-\frac{d}{p} > 0.
\]
Ist $\Omega$ ein Lipschitz-Gebiet, so gilt diese Aussage auch für Sobolevräume ohne Nullrandwerte.\\

\bet{Beweis:}\\
Im Buch von Alt.

\ssect{2.29 Satz}{(Poincaré-Friedrichs Ungleichung)}
Sei $\Omega\subset \R^d$ ein Gebiet mit Durchmesser $D:=\diam(\Omega)$ und $1\le p <\infty$.
Dann gibt es eine Konstante $c_p\le 2D$, so dass für alle $v\in H_0^{1,p}(\Omega)$ gilt:
\[
\norm{v}_{L^p(\Omega)}  \le c_p \norm{\nabla v}_{L^p(\Omega)}.
\]

\bet{Beweis:}\\
Siehe Buch von Dziuk.

\subsubsection{Schwache Formulierung elliptischer Randwertprobleme}
Wir betrachten zunächst die stationäre Wärmeleitgleichung.
Sei $\Omega\subset \R^d$ ein Gebiet mit glattem Rand und seinen $q\in C^0(\Omega)$ und $\kappa\in C^1(\Omega)$ mir $\kappa\ge \kappa_1>0$, $\kappa_1\in \R$ Konstante.
Gesucht ist eine Funktion $u\in C^2(\Omega)\cap C^0(\bar{\Omega})$, die sogenannte klassische Lösung, so dass
\begin{align}
\begin{split}
-\nabla (\kappa \nabla u) &= q \text{ in } \Omega,\\
u &= 0 \text{ auf } \partial\Omega. 
\end{split}
\end{align}

Mit Hilfe von schwachen Ableitungen und den Sobolevräumen können wir nun den klassischen Lösungsbegriff verallgemeinern:

\ssect{2.30 Defintion}{(schwache Formulierung der stationären Wärmeleitgleichnung)}
Seien $\Omega\subset \R^d$ ein Lipschitz-Gebiet, $q\in L^2(\Omega)$ und $\kappa \in L^\infty(\Omega)$ mit $0< \kappa_1 \le \kappa$ für eine Konstante $\kappa_1\in \R$ gegeben.
Dann heißt $u\in H_0^{1,p}(\Omega)$ schwache Lösung des Randwertproblems der stationären Wärmeleitgleichung (2.2), falls für alle Testfunktionen $v\in H_0^{1}(\Omega)$ gilt 
\[
\int_{\Omega} \kappa(x) \nabla u(x) \nabla v(x) \dint x = \int_{\Omega} q(x) v(x) \dint x.
\]

\ssect{2.31 Satz}{(Existenz und EIndeutigkeit von Lösungen)}
Unter den Voraussetzungen von Def. 2.30 gibt es genau eine schwache Lösung $u\in H_0^{1}(\Omega)$ des Randwertproblems der stationären Wärmeleitgleichung.\\

\bet{Beweis:}\\
Zunächst wird durch $l(v) := \int_{\Omega} q(x) v(x) \dint x$ ein lineares Funktional in $(H_0^{1}(\Omega))'$ definiert, denn
\[
\norm{l(v)}_{(H_0^1(\Omega))'} = \sup\limits_{v\in (H_0^1(\Omega))\backslash \{0\}} \frac{(q,v)_{L^2(\Omega)}}{\norm{v}_{H^1(\Omega)}} \le \norm{q}_{L^2(\Omega)}  \infty.
\]
Ferner wird wegen der Poincaré-Friedrichs Ungleichung durch
\[
(w,v)_{H_0^1(\Omega)} := \int_{\Omega} \nabla w(x) \nabla v(x)\dint x
\]
ein Skalarprodukt auf dem Hilbertraum $H_0^1(\Omega)$ definiert.
Daher existiert nach dem Riesz'schen Darstellungsatz 2.9 ein eindeutiger Riesz-Repräsentant $w_l$ mit $l(.) = (w_l,.)_{H_0^1(\Omega)}$.

Um den Beweis zu schließen, müssen wir noch nachweisen, dassdie Bilinearform $b:H_0^1(\Omega)\times H_0^1(\Omega) \to \R$ definiert durch
\[
b(w,v) := \int_{\Omega} \kappa(x) \nabla w(x) \nabla v(x) \dint x
\]
die Voraussetzungen des Satzes von Lax-Milgram erfüllt.
Wir müssen also zeigen, dass die Bilinearform $b$ stetig und koerziv ist.\\
\bet{Stetigkeit:}
\[
b(w,v) \le \norm{\kappa}_{L^\infty(\Omega)} \norm{w}_{H_0^1(\Omega)} \norm{v}_{H_0^1(\Omega)}.
\]
\bet{Koerzivität:}
\[
b(w,v) = \ge \kappa_1 \norm{w}_{H_0^1(\Omega)}^2.
\]
Damit existiert ein bijektiver Operator $B\in L(H_0^1(\Omega))$ mit $b(u,v)= (Bu,v)_{H_0^1(\Omega)}$ und wir definieren die eindeutige Lösung $u\in H_0^^(\Omega)$ des Randwertproblems als $u := B^{-1}w_l$, wobei $w_l$ der eindeutige Riesz-Repräsentant mit $l(.)= (w_l,.)_{H_0^1(\Omega)}$ war.
\hfill $\square$

\ssect{2.32 Bemerkung}{}
Mit der gleichen Beweistechnik lassen sich auch allgemeinere PDgl'en behandeln, wie zum Beispiel das Randwertproblem in Divergenzform
\begin{align*}
-\nabla (A(x)\nabla u) +b(x)\nabla u + c(x) u &= q \text{ in } \Omega,\\
u &= 0 \text{ auf } \partial\Omega.
\end{align*}

$A(x) \in C^1(\Omega,\R^{d\times d}),~b(x)\in C^0(\Omega,\R^d),~c(x)\in C^1(\Omega)$, wobei die Koeffizienten gewisse Anforderungen erfüllen müssen damit die Koerzivität der entsprechenden Bilinearform nachgewiesen werden können.

\ssect{2.33 Bemerkung}{(Reduktion auf Nullrandwerte)}
Zur Betrachtung von allgemeinen Dirichletrandwerten, kann man wie folgt vorgehen.
Seien $g_D\in H^1(\Omega)$ und $q\in L^2(\Omega),~\kappa\in L^\infty(\Omega),~\kappa\ge \kappa_1 >0$. 
Dann ist $u\in H^1(\Omega)$ schwache Lösung von $-\nabla(\kappa(x)\nabla u(x)) = q$ in $\Omega$ und $u=g$ auf $\partial\Omega$.
Wenn gilt $\tilde{u}:= u-g_D \in H_0^1(\Omega)$ und für alle $v\in H_0^1(\Omega)$ gilt (2.3).
Dabei ist zu bemerken, dass mit der Definition von $\tilde{u}$ (2.3) äquivalent ist zu $\int_{\Omega} \kappa(x)\nabla \tilde{u} \nabla v(x)\dint x = \int_\Omega q(x) v(x)\dint x - \int_\Omega \nabla g_D(x) \nabla v(x) \dint x$.
Die Existenz und Eindeutigkeit einer Lösung folgt dann daraus, dass durch $l(v) := \int_\Omega q(x) v(x)\dint x - \int_\Omega \nabla g_D(x) \nabla v(x) \dint x$ ein lin. Funktional in $(H_0^1(\Omega))'$ definiert wird.

\ssect{2.34 Definition}{(schwache Formulierung)}
Seien $X$ reeller Hilbertraum, $b:X\times X\to \R$ eine stetige und koerzive Bilinearform mit Stetigkeitskonstante $\gamma$ und Koerzivitätskonstante $\alpha$ und $l\in X'$.
Dann bezeichnen wir mit $u\in X$ die eindeutige Lösung des Problems
\begin{align}
b(u,v) = l(v) ~\forall v\in X.
\end{align}

\subsubsection{Regularität}
Zur Motivation betrachte in einer Raumdimension die Dgl. $u''(x) = q(x)$ mit einer stetigen Funktion $q(x)$.
Dann folgt mit dem Hauptsatz der Differential- und Integralrechnung, dass bereits $u\in C^2$ gelten muss.

\ssect{2.35 Satz}{($H^2$-Regularität)}
Sei $\Omega$ ein Gebiet mit glattem Rand (es gelte $\partial\Omega$ ist in $C^2$) oder ein konvexes Lipschitz-Gebiet.
Ferner seien $q\in L^2(\Omega)$ und $\kappa\in C^1(\bar{\Omega})$.
Dann gilt für die eindeutige schwache Lösung $u\in H_0^1(\Omega)$ der stationären Wärmeleitungsgleichung (2.2) dass $u\in H^2(\Omega)$ und dass eine Konstante $c>0$ existiert, so dass die folgende Abschätzung gilt:
\[
\norm{u}_{H^2(\Omega)} \le c\norm{q}_{L^2(\Omega)}
\]

\bet{Beweis:}\\
Für glatten Rand: Buch von Evans.

\ssect{2.36 Bemerkung}{}
Betrachtet man nicht konvexe Lipschitz-Gebiete, so kann man im Allgemeinen keine Lösung $u\in H^2(\Omega)$ erwarten.

\subsection{Ritz-Galerkin Verfahren und absrtrakte Fehlerabschätzungen}
In diesem Abschnitt wollen wir uns mit der Approximation der Lösung der schwachen Formulierung von (2.4) befassen.

\ssect{2.37 Definition}{(Ritz-Galerkin Verfahren)}
Seien $X,b$ wie in Definition 2.34 und $X_m\subset X$ mit $\dim (X_m) = m$ ein Unterraum.
Dann ist die \bet{Ritz-Galerkin Approximation´} $u_m\in X_m$ definiert durch
\[
b(u_m,v_m)=l(v_m)~\forall v_m\in X_m.
\]

\ssect{2.38 Bemerkung}{}
Die Existenz und Eindeutigkeit von $u_m$ folgt unmittelbar aus dem Satz von Lax-Milgram 2.14, da der Unterraum $X_m$ wieder ein Hilbertraum mit dem aus $X$ geerbten Skalarprodukt ist.

\ssect{2.39 Satz}{(Abstrakte Fehlerabschätzung/Lemma von Céa)}
Seien $X,X_m,b,u$ und $u_m$ wie in den Definitionen 2.34 und 2.37 definiert. 
Dann gilt die abstrakte Fehlerabschätzung
\[
\norm{u-u_m}_X \le \frac{\gamma}{\alpha} \inf\limits_{v_m\in X_m} \norm{u-v_m}_X.
\]

Außerdem gilt die Galerkin-Orthogonalität
\[
b(u-u_m,v_m)=0~\forall v_m\in X_m.
\]

\bet{Beweis:}\\
Wir zeigen zunächst die Galerkin-Orthogonalität:
Dazu sei $v_m\in X_m$ und es folgt mit $X_m\subset X$:
\[
b(u-u_m,v_m) = b(u,v_m)-b(u_m,v_m) = l(v_m)-l(v_m)= 0.
\]
Mit der Stetigkeit und Koerzivität von $b$ folgt weiter
\begin{align*}
\alpha \norm{u-u_m}_X^2 &\le b(u-u_m,u-u_m) = b(u-u_m,u-v_m)\\
&\le \gamma\norm{u-u_m}_X\norm{u-v_m}_X\\
&\Rightarrow \norm{u-u_m}_X \le \frac{\gamma}{\alpha} \norm{u-v_m}_X
\end{align*}
Gehe auf beiden Seiten der Ungleichung zum Infimum über, dann folgt die Behauptung.
\hfill $\square$

\ssect{2.40 Bemerkung}{}
Die abstrakte Fehlerabschätzung zeigt, dass der Fehler zwischen Ritz-Galerkin Approximationen und exakter Lösung abgeschätzt werden kann durch die Bestapproximation in dem Teilraum $X_m$.
Die weitere numerische Analyse beruht somit allein auf der Approximationstheorie.
Insbesondere bestimmt im wesentlichen der Teilraum $X_m$ die Approximationsgüte.

\ssect{2.41 Beispiel}{(mögliche Wahl von Teilräumen)}
Betrachten wir konkret die Stationäre Wärmeleitungsgleichung (2.2) oder allgemeiner ein elliptisches Problem in Divergenzform mit $X=H^1_0(\Omega)$ auf einem Gebiet $\Omega\subset \R^d$, so sind neben den Finiten Elemente Verfahren, die wir im nächsten Abschnitt betrachten wollen,vor allem folgende Wahlen von Teilräumen gebräuchlich:
\begin{itemize}
	\item Polynomräume $X_M := \Pw^{k(m)}(\Omega)\cap \penbrace{v_m\in C^0(\bar{\Omega})~|+v_m=0\text{ auf }\partial\Omega}$, wobei $\Pw^{k(m)}(\Omega)$ der Raum der Polynome mit Grad $\le k(m)$ über $\Omega$ ist.
	Die zugehörigen Verfahren nennt man \bet{Spektralverfahren}.
	\item $X_m := \spann \penbrace{u_i\in X~|~Lu_i=\lambda u_i,~i=1\dt{,}m}$ wobei $u_i$ die $i$-te Eigenfunktion des zugrundeliegenden Differentialoperators $L$ ist.
	\item $X_m := \spann \penbrace{u_i\in X~|~ \Delta u_i =\lambda u_i,~i=1\dt{,}m}$, wobei $u_i$ die $i$-te Eigenfunktion des Laplaceoperators $\Delta$ ist.
\end{itemize}

\ssect{2.42 Folgerung}{(Matrix-Vektor von Ritz-Galerkin Verfahren)}
Seien $X,X_m,b,u$ un d$u_m$ wie in den Definitionen 2.34 und 2.37 definiert und sei zudem $X_m$ endlichdimensional, mit Dimension $m := \dim X_m$.
Ist dann $\Phi := \penbrace{\varphi_1\dt{,}\varphi_m}$ eine Basis von $X_m$ so folgt mit der Darstellung $u_m = \sum_{i=1}^{m}U_i\varphi_i$ aus der Definition von $u_m$
\[
b\enbrace{\sum_{i=1}^{m}U_i\varphi_i,\varphi_j} = l(\varphi_j),~j=1\dt{,}m.
\]
Durch Ausnutzen der Linearität von $b$ im 1. Argument folgt weiter:
\[
\sum_{i=1}^{m} b(\varphi_i,\varphi_j)u_i = l(\varphi_j),~j=1\dt{,}m.
\]
Definieren wir also die Matrix $\bet{S}\in \R^{m\times m}$ durch $\bet{S}_{ji} := b(\varphi_i,\varphi_j),~i,j=1\dt{,}m$ und die Vektoren $\bet{u},\bet{l}\in \R^m$ durch $\bet{u}_i:= U_i,~\bet{l}_i := l(\varphi_i),~i=1\dt{,}m$, so ist $u_m$ genau dann Lösung des Ritz-Galerkin Verfahren, wenn $u$ das folgende lineare Gleichungssystem löst: $\bet{Su}=\bet{l}$.

\subsection{Finite Elemente Verfahren}
Finite Elemente Verfahren sind Spezialfälle von Ritz-Galerkin Verfahren für eine bestimmte Klasse von Teilräumen $X_h\subset X$, wobei $X_h$ der \bet{Finite Elemente Raum} ist.
Die Konstruktion von $x_h$ im Falle von Finite Elemente (FE) Verfahren beruht auf einer Zerlegung des Gebietes $\Omega$ in nicht überlappende Teilgebiete, die selbst wiederum einfache geometrische Objekte sind.
Die einfachste Klasse von Finiten Elementen sind Lagrange Elemente, auf welche wir uns in dieser Vorlesung beschränken werden.
Ferner betrachten wir nur Teilräume $X_h$ welche auf einer simplizialen Zerlegung des Gebietes $\Omega$ beruhen.
In zwei Raumdimensionen besteht das Rechengitter aus Dreiecken, in drei Raumdimensionen aus Tetraedern.
Eingeschränkt auf einen Simplex wird eine Funktion aus $X_h$, dann ein Polynom mit Grad $\le k,~k\in \N$ sein.
Für andere FE siehe z.B. das Buch von Brenner und Scott.

\ssect{2.43 Definition}{(Simplex)}
Seinen $s\in \penbrace{1\dt{,}d}$ und $a_0\dt{,}a_s\in \R^d$ Punkte, so dass $(a_j-a_0)_{j=1\dt{,}s}$ linear unabhängig sind.
Dann heißt 
\[
T:= \penbrace{x\in \R^d~|~ x=\sum_{i=0}^{s}\lambda_ia_i,~0\le \lambda_i,~\sum_{i=0}^{s}\lambda_i=1}
\]
nicht-degeneriertes $s$-dimensionaler Simplex im $\R^d$.
Die Punkte $a_0\dt{,}a_s$ heißen Ecken des Simplex.
Ist $r\in \{0\dt{,}s\}$ und $\tilde{a}_0\dt{,}\tilde{a}_r\in \{a_0\dt{,}a_s\}$, so heißt
\[
\tilde{T}:= \penbrace{x\in \R^d~|~ x=\sum_{i=0}^{s}\lambda_i\tilde{a}_i,~0\le \lambda_i,~\sum_{i=0}^{s}\lambda_i=1}
\]
$r$-dimensionales Seitensimplex von $T$.
Die nulldimensionalen Seitensimplexe heißen Ecken, die eindimensionalen Kanten.
Wir bezeichnen mit $T_0$ den Simplex zu den Punkten $a_0=e_0=(0\dt{,}0),a_i = e_i,~i=1\dt{,}d$, $T_0$ heißt $d$-dimensionaler Einheitssimplex.
Der \bet{Durchmesser} von $T$ ist gegeben durch $h(T):= \diam (T) = \max_{i,j=1\dt{,}s}\abs{a_i-a_j}$. 
Mit 
\[
\rho(T):= 2\sup\{R~|~B_R(x_0)\subset T\}
\]
bezeichnen wir den \bet{Inkugeldurchmesser} von $T$ und mit 
\[
\delta(T):= \frac{h(T)}{\rho(T)}
\]
den Quotienten aus $h$ und $\rho$.

\ssect{2.44 Definition}{(Baryzentrische Koordinaten)}
Die baryzentrsichen Koordinaten $\lambda_0\dt{,}\lambda_s\in [0,1]$ eines Punktes $x\in T$ des $s$-dim. Simplex $T$ sind die Lösung des linearen Gleichungssystems
\[
x= \sum_{i=0}^{s}\lambda_i a_i,~ \sum_{i=0}^{s} \lambda_i =1.
\]
Der Schwerpunkt $x_s$ von $T$ ist definiert durch $x_s := \frac{1}{s+1}\sum_{i=0}^{s}a_i$ und hat die baryzentrischen Koordinaten $\lambda_i :=\frac{1}{s+1}$.
Für die Eckpunkte $a_k$ von $T$ sind die baryzentrischen Koordinaten gegeben durch $\lambda_k=1,~\lambda_i =0,~i\neq k$.
Die baryzentrischen Koordinaten sind eindeutig bestimmt.

\ssect{2.45 Lemma}{(Referenzabbildung)} 
Jedes $s$-dimensionale Simplex $T$ im $\R^s$ ist affin äquivalent zum Einheitssimplex $T_0$ der gleichen Dimension.
Die eindeutige affine Abbildung $F:T_0\to T, ~ F(x) =Ax+b,~A\in \R^{s\times s},~b\in \R^s, ~\det A \neq 0$ mit $F(e_j)= a_j,~j=0\dt{,}
s$ heißt \bet{Referenzabbildung}.
$F$ ist invertierbar und es gelten die Abschätzungen
\[
\norm{\nabla F} = \norm{A}\le \frac{h(T)}{\rho(T_0)},~\norm{\nabla(F^{-1})}= \norm{A^{-1}}\le \frac{h(T_0)}{\rho(T)}
\]
sowie
\[
c\cdot \rho(T)^s \le \abs{\det (\nabla F)}= \abs{\det A} =\frac{\abs{T}}{\abs{T_0}}\le C\cdot h(T)^s,~c,C>0.
\]

\bet{Beweis:}\\
Im Buch von Dziuk.

\ssect{2.46 Definition}{(Zulässige Triangulierung)}
Sei $\Omega\subset\R^d$ ein Gebiet und 
\[
\mathbb{T}_h := \penbrace{T_j~|~j=1\dt{,}m,~T_j\text{ ist $d$-dim. Simplex im }\R^d}.
\]
$\mathbb{T}_h$ heißt zulässige Triangulierung der Feinheit $h$ und Gute $\rho$ von $\Omega$, falls gilt:
\[
\bar{\Omega} = \bigcup\limits_{j=1}^{m} T_j,~\partial\Omega = \bigcup\limits_{j=1}^{m} \tilde{T}_j,
\]
wobei $\tilde{T}_j$ Flächen der Simplexe $T_j$ sind.
Für je zwei $T_1,T_2\in \mathbb{T}_h$ mit $S:=T_1\cap T_2$ gilt $S=\emptyset$ oder $S$ ist $(d-k)$-dim. Seitensimplex von $T_1$ und $T_2$ für ein $k\in \{1\dt{,}d\}$.
Mit $h:=\max_{j=1\dt{,}m}h(T_j)$ und $\rho := \min_{j=1\dt{,}m}\rho(T_j)$.\\

Zur Definition von Finite Elemente Räumen basierend auf einer Triangulierung $\mathbb{T}_h$ müssen wir nun lediglich lokale Funktionenräume auf dem Simplexen $T\in\mathbb{T}_h$ angeben und festlegen wie solche lokalen Funktionen global zusammengesetzt werden.
Ein Tripel bestehend aus einem geometrischen Objekt $T$, einer lokalen Basis $\Phi$ und lokalen Freiheitsgraden $\delta$, wollen wir um folgenden \bet{Element} nennen.

\ssect{2.47 Definition}{(lineares simpliziales Lagrange Element)}
Sei $T\subset\R^d$ ein $d$-dim. Simplex.
Sei $\delta := \penbrace{a_k~|~k=0\dt{,}d}$ die Menge der Ecken von $T$.
Dann ist durch Angabe von Werten in den Punkten $a_k\in \delta$ eindeutig eine lineare Funktion $p\in \Pw^1(T)$ definiert.
Durch $\Phi := \penbrace{\varphi_i~|~\varphi_i(a_k)=\delta_{ik}~i,k=1\dt{,}d}$ ist eine modale Basis von $\Pw^1(T)$ gegeben.
Wir nennen das Tripel $(T,\Phi,\delta)$ \bet{lineres simpliziales Lagrange Element}.
Die Basisfunktionen $\varphi_i\in\Phi$ werden \bet{Formfunktionen} oder im Englischen \bet{Shapefunctions} genannt und $\delta$ ist die Menge der \bet{modalen Variablen}.
Zur Wohldefiniertheit kann man im Buch von Dziuk nachschauen.

\ssect{2.48 Beispiel}{(lineares Lagrange Element für $d=2$)}
Wir betrachten das Einheitsdreieck $T_0$ mit Eckpunkten $a_0^0=(0,0),a_1^0=(1,0),a_2^0=(0,1)$.
Die Formfunktionen sind dann gegeben durch
\[
\varphi_0^0(x,y)= 1-x-y,~\varphi_1^0(x,y) = x,~\varphi_2^0(x,y) = y.
\]
Sind $p(a_0^0),p(a_1^0),p(a_2^0)$ Funktionswerte einer linearen Funktion $p\in\Pw^1(T_0)$, so ist $p$ gegeben durch
\[
p(x,y) = \sum_{i=0}^{2} p(a_i^0)\varphi_i^0(x,y).
\]
Für ein beliebiges Dreieck $T\subset\R^2$ erhält man das Lagrange Element mit Hilfe der Referenzabbildung $F:T_0\to T$ aus Lemma 2.45.

\ssect{2.49 Bemerkung}{}
Das Beispiel 2.48 zeigt, dass es ausreicht ein Finites Element auf einer Referenzgeometrie zu definieren.
Durch die Referenzabbildung erhält man dann die entsprechende Klasse von Elementen auf beliebigen Geometrien im Raum.\\

Ein Finites Element legt lediglich ein lokalen Funktionenraum auf einen Simplex $T$ fest, um zu einem Unterraum von $H^1_0(\Omega)$ zu gelangen, müssen wir zusätzlich festlegen auf welche Weise die lokalen Funktionen global zusammengesetzt werden.

\ssect{2.50 Definition}{(linearer Finite Elemente Raum $S_h^1$)}
\begin{enumerate}[(1)]
\item Sei $\Omega\subset\R^d$ und $\mathbb{T}_h$ eine zulässige Triangulierung von $\Omega$.
Wir definieren den Raum der \bet{linearen Finite Elemente} auf simplizialen Gittern $S_h^1$ durch
\[
S_h^1 := \penbrace{v_h\in C^0(\Omega)~|~v_h|_T\in \Pw^1(\Omega),~T\in\mathbb{T}_h}.
\]
\item Sind $\bar{a}_j,~j=1\dt{,}N_h$ die Ecken der Triangulierung $\mathbb{T}_h$, so idt eine Funktion $v_h\in S_h^1$ durch die Vorgabe von Funktionswerten in den Ecken $v_h(\bar{a}_j)$ eindeutig definiert.
Insbesondere gilt $\dim(S_h^1)= \mathcal{N}_h$.
Eine Basis von $S_h^1$ ist durch die Funktionen
\[
\bar{\varphi}_i\in S_h^1,~\bar{\varphi}_i(\bar{a}_j)=\delta_{ij},~i,j=1\dt{,}\mathcal{N}
\]
gegeben.
Diese Basis heißt \bet{Knotenbasis} oder \bet{modale Basis}.
\item Ist $(T_0,\Phi,\delta)$ das lineare Lagrange Element auf dem Einheitssimplex $T_0$ und $v_h\in S_h^1$ gegeben durch
\[
v_h(x):=\sum_{i=1}^{\mathcal{N}_h}v_n(\bar{a}_i)\bar{\varphi}_i(x),
\]
so gilt für beliebige Simplexe $T\in \mathbb{T}_h$ mit Ecken $a_0\dt{,}a_d$
\[
v_h|_T(x) = \sum_{i=1}^{d}v_h(a_i)\varphi_i^0(T^{-1}(x)),
\]
wobei $F:T_0\to T$ die referenzabbildung und $\varphi_i^0\in \Phi$ die Formfunktionen von $T_0$ sind.
\end{enumerate}

\ssect{2.51 Definition}{(lineares Finite Elemente Verfahren)}
Sei $\Omega\subset\R^d$ und $\mathbb{T}_h$ zuläassige Triangulierungvon $\Omega$.
Seien $X:=H_0^1(\Omega)$ und $X_h:= S_{h,0}^1:=S_h^1\cap \penbrace{v\in C^1(\Omega)~|~v=0 \text{ auf }\partial\Omega}$.
Weiter seien eine stetige und koerzive Bilinearform $b:X\times X\to \R$ und ein $l\in X'$ gegeben.
Dann ist $X_h\subset X$ ein teilraum und $u_h\in X_h$ heißt Lösung des \bet{linearen Finite Elemente Verfarhens} für das Problem aus 2.34, falls gilt:
\[
b(u_h,v_h)= l(v_h) ~\forall v_h\in X_h.
\]

\ssect{2.52 Satz}{(A priori Fehlerabschätzung)}
Sei $\Omega\subset\R^d,~d\le 3$ ein Lipschitz-Gebiet und $\mathbb{T}_h$ eine zulässige Triangulierung von $\Omega$ mit $\sigma(T)\le \sigma<\infty,~\sigma\in \R,~\forall T\in \mathbb{T}_h$.
Seien $X,X_h,b,u_h$ wie in Definition 2.51 und $u\in X$ wie in Definition 2.34.
Liegt nun $u\in H_2(\Omega)$ so gibt es eine Konstante $c>0,c\in \R$, die nur von $d,\sigma,\Omega$ abhängt, so dass gilt:
\begin{align}
\norm{u-u_h}_{H^1(\Omega)}\le ch\abs{u}_{H^2(\Omega)},
\end{align}
wobei $\abs{u}_{H^2(\Omega)}:= \norm{D^2u}_{L^2(\Omega)}$.\\

\bet{Beweis:}\\
Für Poissonproblem siehe Buch von Dziuk.

\ssect{2.53 Bemerkung}{(a priori $\leftrightarrow$ a posteriori Fehlerabschätzung)}
Satz 2.52 macht eine Aussage über die Konvergenz des linearen FE Verfahrens.
Da auf der rechten Seite der Ungleichung (2.4) aber der Term $\abs{u}_{H^2(\Omega)}$ auftaucht, ist (2.4) nicht geeignet um den tatsächlichen Wert des Approximationsfehlers abzuschätzen.
Zu diesem Zweck leitet man A posteriori Fehlerabschätzungen her, bei denen der Fehler ausschließlich durch berechenbare Größen abgeschätzt wird.
Für eine Übersicht über A posteriori Fehlerabschätzungen für FE Verfahren verweisen wir auf das Buch von Verfürth.

\ssect{2.54 Bemerkung}{}
Betrachten wir Definition 2.51 des linearen FE Verfahrens, so stellen wir zunächst fest, dass wir in der schwachen Formulierung exakte Integrale bestimmen müssen, was im Allgemeinen nicht realisierbar ist.
in der Praxis verwendet man Quadraturformeln.
Ferner schränkt uns Definition 2.51 auf polygonal berandete Gebiete ein.
Um auch Gebiete mit glattem Rand behandeln zu können , kann man z.B. eine Gebietsapproximation durchführen bei dem alle Ecken auf dem Rand des polygonal berandeten approximierenden Gebiets $\partial\Omega_h$ auch auf $\partial\Omega$ liegen.
Hier ist dann $X_h$ nicht Teilraum von $X$ und das Lemma von Céa nicht anwendbar.
Allerdings kann man in beiden Fällen unter gewissen Voraussetzungen zeigen, dass die zusätzlichen Approximationsfehler die Konvergenzordnung des FE Verfahrens nicht beeinflussen.
Für weitere Details siehe z.B. das Buch von Dziuk oder Brenner und Scott.

\section{Reduzierte Basis Methoden für lineare, koerzive Probleme}

\subsection{Parameterabhängigkeit}

\ssect{3.1 Definition}{(parametrische Formen)}
Sei $\mathcal{P}\subset \R^d$ eine beschränkte Parametermenge.
Dann nennen wir 
\begin{enumerate}[(1)]
	\item $f:X\times \mathcal{P}\to \R$ eine parametrische stetige Linearform oder ein parametrisches stetiges lineares Funktional, falls $\forall \mu\in\mathcal{P}: f(.,\mu)\in X$.
	\item Wir nennen $b:X_1\times X_2\times \mathcal{P}\to \R$ eine parametrische stetige koerzive Bilinearform, falls $\forall \mu\in \mathcal{P}: b(.,.,\mu):X_1\times X_2\to \R$ bilinear stetig und koerziv ist. Wir bezeichnen die Stetigkeitskonstante mit $\gamma(\mu)$ und die Koerzivitätskonstante mit $\alpha(\mu)$.
\end{enumerate}

\ssect{3.2 Bemerkung}{}
Eine parametrische stetige Bi-/Linearform ist nicht unbedingt stetig bzgl. $\mu$.
Betrachte dazu das Beispiel $X=\R,\mathcal{P}=[0,1],f:X\times \mathcal{P}\to\R$ mit
\[
f(x,\mu):=\left\{\begin{array}{l}x,\text{ falls }\mu<\frac{1}{2}\\ \frac{1}{2}x,\text{ sonst.}  \end{array}\right.
\]

\ssect{3.3 Defintion}{(Parametrische Beschränktheit, Stetigkeit)}
\begin{enumerate}[(1)]
	\item Wir nennen eine parametrische stetige Linearform $f$ beschränkt bzw. Bilinearform$b$ \bet{gleichmäßig beschränkt} bzgl. $\mu$, falls $\gamma_0,\gamma_1\in\R^+$ existieren so dass
	\[
	\sup\limits_{\mu\in\mathcal{P}}\norm{f(.,\mu)}_X\le \gamma_0 \text{ bzw. } \sup\limits_{\mu\in\mathcal{P}}\gamma(\mu)\le \gamma_1.
	\]
	\item Wir nennen $b$ \bet{glm. koerziv} bzgl. $\mu$, falls ein $\alpha_0>0$ existiert so dass
	\[
	\inf\limits_{\mu \in \mathcal{P}} \alpha(\mu)\ge \alpha_0>0.
	\]
	\item Wir nennen $f$ bzw. $b$ Lipschitz-stetig bzgl. $\mu$, falls ein $L_f\in \R^+$ bzw. ein $L_b\in \R^+$ existiert, so dass für alle $\mu_1,\mu_2\in\mathcal{P}$ gilt
	\[
	\abs{f(u,\mu_1)-f(u,\mu_2)}\le L_f\norm{u}_X\norm{\mu_1-\mu_2} ~\forall  u\in X,
	\]
	bzw.
	\[
	\abs{b(u,v,\mu_1)-b(u,v,\mu_2)}\le L_b\norm{u}_{X_1}\norm{v}_{X_2}\norm{\mu_1-\mu_2}~\forall u\in X_1,v\in X_2.
	\]
\end{enumerate}

\ssect{3.4 Lemma}{(Energienorm)}
Sei $X$ HR, $b:X\times X\times \mathcal{P}\to \R$ parametrische , koerzive, stetige Bilinearform.
Dann ist für $\mu\in\mathcal{P}$ durch
\[
(((u,v)))_\mu := b_s(u,v;\mu)
\]
ein Skalarprodukt auf $X$ und durch
\[
|||u|||_\mu := \sqrt{(((u,u)))_\mu}
\]
die \bet{Energienorm} definiert.
Diese ist äquivalent zur $X$-Norm und es gilt
\[
\sqrt{\alpha(mu)}\norm{u}_X \le |||u|||_\mu \le \sqrt{\gamma(\mu)}\norm{u}_X ~\forall u\in X.
\]

\bet{Beweis:}\\
Skalarprodukt klar wegen Bilinearität, Stetigkeit und Koerzivität.
Normäquivalenz folgt aus Stetigkeit und Koerzivtät von $b_s$:
\[
\alpha(\mu)\norm{v}_X^2\le b_s(v,v;\mu) \le \gamma(\mu) \norm{v}_X^2.
\]

\ssect{3.5 Definition}{(Parametrische schwache Formulierung; Parametrisches Variationsproblem ($P(\mu)$)}
Sei $X$ HR, $\mathcal{P}\subset \R^p$ beschränkt, $b: x\times X\times \mathcal{P}\to \R$ parametrische, stetige, koerzive Bilinearform, $f,l:X\times \mathcal{P}\to \R$ parametrische stetige Linearform.
Zu $\mu \in\mathcal{P}$ bezeichnet $u(\mu)\in X$ die eindeutige Lösung des parametrischen Variationsproblems
\begin{align}
b(u(\mu),v;\mu) = f(v,\mu) ~\forall v\in X,
\end{align}
mit Ausgabe $s(\mu)=l(u(\mu),\mu)$.

\ssect{3.6 Bemerkung}{}
Existenz und Eindeutigkeit der Lösung $u(\mu)$ folgen mit dem Satz von Lax Milgram.

\ssect{3.7 Definition}{(schwache Formulierung der parametrischen, stationären Wärmeleitungsgleichung)}
Seien $\Omega\subset\R^d$ Lipschitz-Gebiet, $\mathcal{P}\subset \R^p$ beschränkt, $q(\mu)\in L^2(\Omega)$ und $\kappa(\mu)\in L^\infty(\Omega)$ mit $0<\kappa_1\le \kappa(\mu)$ für alle $\mu\in\mathcal{P}$ und Konstante $\kappa_1\in \R^+$.
Dann heißt $u(\mu)\in H_0^{1}(\Omega)$ schwache Formulierung des RWP der parametrischen, stationären WLG aus 1.1, falls gilt
\[
\int_{\Omega} \kappa(x;\mu)\nabla u(x;\mu) \nabla v(x)\dint x = \int_{\Omega} q(x,\mu)v(x)\dint x~\forall v\in H_0^1(\Omega).
\]

\ssect{3.8 Folgerung}{(Existenz und Eindeutigkeit von Lösungen)}
Unter den Voraussetzungen von Definition 3.7 gibt es für jedes $\mu\in\mathcal{P}$ genau eine schwache Lösung $u(\mu)\in H_0^1(\Omega)$ des RWP der parametrischen, stationären WLG aus 1.1.\\

\bet{Beweis:}\\
Analog zum Beweis von Satz 2.31.

\ssect{3.9 Definition}{((lineares) FE Verfahren für parametrsiche Variationsprobleme ($P_h(\mu)$)}
Sei $X$ HR, $\mathcal{P}\in \R^p$ beschränkt, $b:X\times X\times \mathcal{P}\to \R$ parametrische, stetige, koerzive Bilinearform, $f,l:X\times \mathcal{P}\to \R$ parametrische, stetige Linearform.
Sei ferner $\mathbb{T}_h$ eine zulässige Triangulierung des Rechengebietes $\Omega\subset \R^d$ und $X_h\subset X$ eine zugehöriger (linearer) Finite Elemente Raum, wobei $X_h$ Unterraum von $X$.
Zu $\mu\in\mathcal{P}$ heißt $u_h(\mu)\in X_h$ Lösung des (linearen) FE Verfahrens für das parametrische Variationsproblem, falls gilt
\[
b(u_h(\mu),v_h;\mu) = f(v_h;\mu) ~\forall v_h\in X_h,~ s_h(\mu)= l(u_h(\mu);\mu).
\]

\ssect{3.10 Bemerkung}{}
Das Verfahren aus Definition 3.9 ist nach Bemerkung 1.6 ein "hochdimensionales, diskretes" Modell.

\subsection{Reduzierte Basisverfahren}

\ssect{3.11 Definition}{(Reduzierte Basis, Reduzierte Basis Räume)}
Sei $S_N := \penbrace{\mu^1\dt{,}\mu^N}\subset \mathcal{P}$ eine Menge von Parametern mit (oBdA) linear unabhängigen Lösungen $\penbrace{u(\mu^i)}_{i=1}^N$ von $(P_h(\mu))$.
Dann ist $X_N := \spann\penbrace{u(\mu^i)}_{i=1}^N$ ein $N$-dimensionaler \bet{Lagrange Reduzierte Basis-Raum}.
Eine Basis $\Phi_N :=\penbrace{\phi_1\dt{,}\phi_N}\subset X_h$ eines Reduzierte Basis-Raumes ist eine Reduzierte Basis (RB).

\ssect{3.12 Bemerkung}{}
Es existieren weitere Arten von RB-Räumen.
Im weiteren Verlauf der Vorlesung werden wir z.B. noch \bet{POD-Räume} kennenlernen.
Auch die POD-Räume werden aus sogenannten Snapshots, d.h. Lösungen $u_h(\mu^i),~1\le i\le k$ mit $k\gg N$, erzeugt.

\ssect{3.13 Definition}{(RB-Modell $(P_N(\mu))$, symmetrischer Fall)}
Sei ein Problem $P(\mu)$ und ein diskretes Modell $(P_h(\mu))$ gegeben und zusätzlich gelte $b$ symmetrisch und $f=l$ ("compliant").
Sei $X_h\subset X$ ein RB-Raum.
Zu $\mu \in \mathcal{P}$ ist die RB-Lösung $u_N(\mu)\in X_N$ und die RB-Ausgabe $s_N(\mu)\in \R$ gesucht, so dass
\[
b(u_N(\mu),v;\mu) = f(v;\mu) ~\forall v\in X_N
\]
und 
\[
s_N(\mu) = l(u_N(\mu);\mu).
\]

\ssect{3.14 Bemerkung}{}
Falls $b$ nicht symmetrisch oder $f\neq l$ ist obiges immer noch sinnvoll, aber es bestehen bessere Möglichkeiten $s_N(\mu)$ mittels eines dualen Problems zu bestimmen.

\ssect{3.15 Bemerkung}{}
Da $X_N\subset X_h\subset X$, $X_N$ Teilraum von $X_h$, ist das RB-Modell ein Ritz-Galerkin Verfahren.

\ssect{3.16 Folgerung}{(Existenz, Eindeutigkeit, Stabilität, Wohlgestelltheit)}
ZU $\mu\in\mathcal{P}$ existiert eine eindeutige RB-Lösung $u_N(\mu)\in X_N$ und RB-Ausgabe $S_N(\mu)$ von $(P_N(\mu))$.
Diese sind beschränkt durch $\norm{u_N(\mu)}_X \le \frac{1}{\alpha(\mu)} \norm{f(.;\mu)}_{X}$ und $\abs{s_N(\mu)}\le \frac{1}{\alpha(\mu)}\norm{f(.;\mu)}_{X'}\norm{l(.;\mu)}_{X'}.$\\

\bet{Beweis:}\\
Existenz und Eindeutigkeit von $u_N(\mu)$ folgt mit dem Satz von Lax-Milgram, wobei
\[
\alpha_N(\mu) := \inf\limits_{u\in X_N} \frac{b(u,u:\mu)}{\norm{u}_X^2} \ge \inf\limits_{u\in X} \frac{b(u,u:\mu)}{\norm{u}_X^2} = \alpha(\mu) > 0.
\]
Dann ist auch $s_N(\mu) = l(u_N(\mu);\mu)$ eindeutig und die Stabilität folgt mit
\[
\norm{u_N(\mu)}_X = \norm{B^{-1}(\mu) v_f(\mu)}_X \le \norm{B^{-1}(\mu)}_X \norm{v_f(\mu)}_X \le \frac{1}{\alpha(\mu)}\norm{f(.;\mu)}_X.
\]
Hierbei ist $B(\mu)$ der eindeutige invertierbare Operator aus dem Satz von Lax-Milgram und $v_f(\mu)$ der Riesz-Repräsentant von $f(.;\mu)\in X_N'$.
\[
\abs{s_N(\mu)} = \abs{l(u_N(\mu);\mu)} \le \norm{l(.;\mu)}_{X'}\norm{u_N(\mu)}_X \le \frac{1}{\alpha(\mu)} \norm{f(.;\mu)}_{X'}\norm{l(.;\mu)}_{X'}
\]
\hfill $\square$

\ssect{3.17 Folgerung}{(Galerkin-Projektion, Galerkin-Orthogonalität)}
Zu $\mu\in\mathcal{P},X_h,X_N$ HR mit Energieskalarprodukt $(((.,.)))_\mu,P_\mu:X_h\to X_N$ die orthogonale Projektion aus Satz 2.7, $u_h(\mu),u_N(\mu)$ Lösungen von $(P_h(\mu))$ bzw.$(P_N(\mu))$ und der Fehler $e_N(\mu) = u_h(\mu)-u_N(\mu)$.
Dann gilt
\begin{enumerate}[(1)]
	\item $u_N(\mu) = P_\mu(u_h(\mu))$ "Galerkin-Projektion"
	\item $(((e_N(\mu),v_N)))_\mu = 0 ~\forall v_N\in X_N$ "Galerkin-Orthogonalität"
\end{enumerate}

\bet{Beweis:}\\
Lemma 3.4 impliziert $(X_h, (((.,.)))_\mu)$ HR und $X_N = \spann\{\Phi_i \}_{i=1}^N$ endlichdimensional, also abgeschlossen ist. 
Daher ist $P_\mu$ nach Satz 2.7 wohldefiniert.\\
Mit Folgerung 2.8 folgt
\begin{align*}
&(((P_\mu(u_h(\mu)-u_h(\mu),\Phi_i)))_\mu = 0,~i=1\dt{,}N\\
\Leftrightarrow &b(P_\mu(u_h(\mu))-u_h(\mu),\Phi_i; \mu) = 0,~i=1\dt{,}N\\
\Leftrightarrow &b(P_\mu(u_h(\mu)),\Phi_i; \mu) = b(u_h(\mu),\Phi_i;\mu),~i=1\dt{,}N\\
\Leftrightarrow &b(P_\mu(u_h(\mu)),\Phi_i; \mu) = f(\Phi_i;\mu),~i=1\dt{,}N\\
\end{align*}
Da $u_N(\mu)$ eindeutig folgt $P_\mu(u_n(\mu)) = u_N(\mu)$ daraus folgt (1). (2) folgt entweder aus 2.8 oder Satz 2.39.

\ssect{3.18 Folgerung}{}
Sei $\mu\in\mathcal{P},u_h(\mu),u_N(\mu)$ Lösungen von $(P_h(\mu))$ bzw. $(P_N(\mu))$.
Falls $u_h(\mu)\in X_N \Rightarrow u_N(\mu) = u_h(\mu)$.\\

\bet{Beweis:}\\
Da $u_h(\mu),u_N(\mu)\in X_N \Rightarrow e_N(\mu) := u_h(\mu) - u_N(\mu)\in X_h$ und $(((e_N(\mu),v_N)))_\mu = 0 ~\forall v_N\in X_N$ nach Folgerung 3.17 (2).
Damit gilt $(((e_N(\mu),l_N(\mu))))_\mu = 0 \Rightarrow e_N(\mu) = 0 \Rightarrow u_N(\mu) = u_h(\mu)$.
\hfill $\square$

\ssect{3.19 Satz}{(abstrakte Fehlerabschätzung; Relation zur Bestapproximation)}
Sei $\mu \in \mathcal{P}$ und $ u_h(\mu),s_h(\mu)$ bzw. $u_N(\mu),s_N(\mu)$ Lösungen von $(P_h(\mu))$ bzw. $(P_N(\mu))$.
Dann gilt:
\begin{enumerate}[(1)]
	\item Der Fehler der ($\mu$-abhängigen Energienorm) erfüllt
	\[
	|||u_h(\mu)-u_N(\mu)|||_\mu = \inf\limits_{v\in X_N} |||u_(\mu)-v|||_\mu.
	\]
	\item Der Fehler in der ($\mu$-unabhängigen) $X$-Norm erfüllt
	\[
	\norm{u_h(\mu)-u_N(\mu)}_X \le \sqrt{\frac{\gamma(\mu)}{\alpha(\mu)}} \inf\limits_{v\in X_N} \norm{u(\mu)-v}_X.
	\]
	mit $\gamma(\mu),\alpha(\mu)$ Stetigkeits- bzw. Koerzivitätskonstante.
	\item Für den Ausgabefehler gilt (wegen $f=l$)
	\[
	0\le s_h(\mu)-s_N(\mu) = |||u_h(\mu)-u_N(\mu)|||_\mu^2 = \inf\limits_{v\in X_N} |||u_h(\mu)-v|||\mu^2 \le \gamma(\mu) \inf\limits_{v\in X_N} \norm{u_h(\mu)-v}_X^2.
	\]
\end{enumerate}

\bet{Beweis:}\\
\begin{enumerate}[(1)]
	\item Nach Folgerung 3.17 ist $u_N(\mu)$ orthogonale Projektion, also Bestapproximation
	\[
	|||u_h(\mu)-u_N(\mu)|||_\mu \bgl{3.17(1)} |||u_h(\mu)-P_\mu(u_N(\mu))|||_\mu \bgl{2.7} = \inf\limits_{v\in X_h} |||u_h(\mu)-v|||_\mu.
	\]
	\item Mit der Normäquivalenz 3.4 folgt 
	\[
	\sqrt{\alpha_h(\mu)}\norm{u_h(\mu)-u_N(\mu)}_X \stackrel{3.4}{\le} |||u_h(\mu)-u_N(\mu)|||_\mu \bgl{(1)} \inf\limits_{v\in X_N} |||u_h(\mu)-v|||_\mu \stackrel{3.4}{\le} \sqrt{\gamma_h(\mu)} \inf\limits_{v\in X_h} \norm{u_h(\mu)-v}_X,
	\]
	wobei $\alpha_h(\mu) := \inf_{v\in X_h} \frac{b(v,v;\mu)}{\norm{v}_X^2}$ und $\gamma_h(\mu):= \sup_{u,v\in X_h} \frac{b(u,v;\mu)}{\norm{u}_X\norm{v}_X}$.
	Wie in Beweis von Folgerung 3.16 folgt $\alpha_h(\mu)\ge \alpha(\mu)$ und $\gamma_h(\mu)\le \gamma(\mu)~\forall \mu\in \mathcal{P}$ und damit die Behauptung.
	\item
	\[
	s_h(\mu)-s_N(\mu) \bgl{Def} l(u_h(\mu);\mu)-l(u_N(\mu);\mu) \bgl{l=f} f(u_h(\mu))-f(u_N(\mu)) \bgl{(P_h(\mu)} b(u_h(\mu),u_h(\mu)-u_N(\mu);\mu) = b(u_h(\mu),u_h(\mu)-u_N(\mu);\mu)- b(u_h(\mu)-u_N(\mu),u_N(\mu);\mu) = b(u_h(\mu)-u_N(\mu),u_h(\mu)-u_N(\mu);\mu)
	\]
	Damit folgt
	\[
	s_h(\mu)-s_N(\mu) = |||u_h(\mu)-u_N(\mu)|||_\mu^2 \bgl{(1)} \inf\limits_{v\in X_h} |||u_h(\mu)-v|||_\mu^2 \stackrel{3.4}{\le} \gamma(\mu)\inf\limits_{v\in X_N} \norm{u_h(\mu)-v}_X^2
	\]
	Insbesondere gilt auch $s_h(\mu)-s_N(\mu) = |||u_h(\mu)-u_N(\mu)|||_\mu^2 \ge 0$.
\end{enumerate}
\hfill $\square$

\ssect{3.20 Bemerkung}{}
\begin{enumerate}[(1)]
	\item $s_N(\mu)$ ist also untere Schranke für $s_h(\mu)$.
	\item Der Ausgabefehler ist im Allgemeinen sehr klein, da das Quadrat des RB-Fehlers eingeht.
	\item Mit dem Lemmavon Céa (Satz 2.39) erhalten wir für nicht notwendigerweise symmetrische Bilinearformen $\norm{u_h(\mu)-u_N(\mu)}_X \le \frac{\gamma(\mu)}{\alpha(\mu)} \inf_{v\in X_N} \norm{u_h(\mu)-v}_X$.
	Damit ist Satz 3.19 eine Verschärfung für symmetrische Bilinearformen.
\end{enumerate}

\ssect{3.21 Korollar}{(Monotoner Fehlerabfall in der Energienorm)}
Sei $(X_N)_{N=1}^{N_{\max}}$ Folge von RB-Räumen mit $X_N\subseteq X_{N'}$ für $N\le N'\le N_{\max}$ ("Hierarchische Räume") und $e_N(\mu)= u_h(\mu)-u_N(\mu)$ für $\mu\in \mathcal{P}$.
Dann ist die Folge $(|||e_N(\mu)|||)_{N=1}^{N_{\max}}$ monoton fallend.\\

\bet{Beweis:}\\
\[
|||e_N(\mu)|||\mu = \inf\limits_{v\in X_N} |||u_h(\mu)-v|||_\mu \ge  \inf\limits_{v\in X_{N'}} |||u_h(\mu)-v|||_\mu = |||e_{N'}(\mu)|||_\mu.
\]
\hfill $\square$

\ssect{3.22 Bemerkung}{}
\begin{enumerate}[(1)]
	\item "Worst Case" ist eine Stagnation des Fehlers (unrealistisch, da jeder neue Basisvektor orthogonal zu $e_N(\mu)$ sein müsste).
	In der Praxis ist bei geschickter Basiswahl exponentielle Konvergenz zu beobachten.
	\item Monotonie gilt nicht notwendigerweise für andere Normen trotz Normenäquivalenz:
	\[
	c|||e_N(\mu)|||_\mu \le \norm{e_N(\mu)} \le C |||e_N(\mu)|||_\mu
	\]
	mit $c,C$ Konstanten unabhängig von $N$.
	Fehlernorm $\norm{e_N(\mu)}$ kann gelegentlich anwachsen, bleibt aber in einem "Korridor" um $|||e_N(\mu)|||_\mu$.\\
	"Beweis":
	\[
	\norm{e_{N'}(\mu)} \le C|||e_{N'}(\mu)|||_\mu \le C|||e_N(\mu)|||_\mu \le \frac{C}{c} \norm{e_N(\mu)}
	\]
\end{enumerate}

\ssect{3.23 Folgerung}{(Fehlerabschätzung für den Fehler zwischen exakter Lösung und RB-Lösung)}
Sei $\mu\in\mathcal{P}$ und $u_(\mu),s(\mu),u_h(\mu),s_h(\mu)$ bzw. $u_N(\mu),s_N(\mu)$ Lösung von $(P(\mu)),(P_h(\mu))$ bzw. $(P_N(\mu))$, wobei $X_h$ linearer Finite Elemente Raum.
Zusätzlich gelte $b$ symmetrisch und $f=l$ ("compliant").
Liegt nun $u\in H^2(\Omega)$, so gibt es eine Konstante $c>0$ die nur von $d,\sigma$ und $\Omega$ abhängt, so dass gilt:
\begin{enumerate}
	\item Der Fehler in der ($\mu$-unabhängigen) $X$-Norm erfüllt
	\[
	\norm{u(\mu)-u_h(\mu)}_X \le \sqrt{\frac{\gamma(\mu)}{\alpha(\mu)}} \enbrace{ch\abs{u(\mu)}_{H^2(\Omega)} + \inf\limits_{v\in X_N} \norm{u_h(\mu)-v}_X}
	\]
	\item Für den Ausgabefehler gilt:
	\[
	0 \le s(\mu)-s_N(\mu) \le \gamma(\mu) \enbrace{c^2h^2\abs{u(\mu)}_{H^2(\Omega)} + \inf\limits_{v\in X_N} \norm{u_h(\mu)-v}_X^2}
	\]
	Beweis: Analog zum Beweis von Satz 3.19 unter Verwendung von Satz 2.52.
\end{enumerate}

\ssect{3.24 Bemerkung}{}
Den Fehleranteil $\norm{u(\mu)-u_h(\mu)}_X$ nennt man \bet{Diskretisierungsfehler} und den Anteil $\norm{u_h(\mu)-u_N(\mu)}$ nennt man \bet{Modellfehler}.
Um eine gute Approximation der exakten Lösung $u(\mu)$ und der Ausgabe $s(\mu)$ zu erhalten, müssen beide Fehleranteile klein sein.

\ssect{3.25 Satz}{(Lipschitzstetigkeit)}
Falls $b$ und $f$ gleichmäßig beschränkt und Lipschitz-stetig bzgl. $\mu$ und $b$ gleichmäßig koerziv bzgl. $\mu$, so sind auch die Lösungen $u_N(\mu)$ und $s_N(\mu)$ von $(P_N(\mu))$ Lipschitz-stetig bzgl. $\mu$.

\ssect{3.26 Satz}{(Gleichungssytem und numersische Stabilität)}
Sei $\Phi_N = \penbrace{\phi_1\dt{,}\phi_N}$ eine reduzierte Basis von $X_N$.
Für $\mu\in\mathcal{P}$ definieren wir $\mathbb{B}_N(\mu) \in\R^{N\times N}$ und $\mathbb{F}_N(\mu)\in \R^N$ durch 
\[
(\mathbb{B}_N(\mu))_{nm} := b(\phi_m,\phi_n;\mu),~ (\mathbb{F}_N(\mu)_n := f(\phi_n;\mu)
\]
und
\[
\mathbb{U}_N(\mu) = (U_1^N(\mu)\dt{,}U_N^N(\mu))\in \R^N
\]
als Lösung von 
\begin{align}
\mathbb{B}_N(\mu)\mathbb{U}_N(\mu) = \mathbb{F}_N(\mu).
\end{align}
\begin{enumerate}[(1)]
	\item Dann ist $u_N(\mu)= \sum_{n=1}^{N} U_n^N(\mu)\phi_n$ und $s_N(\mu) = \mathbb{F}_N^T(\mu)\mathbb{U}_N(\mu)$ Lösung von $(P_N(\mu))$.
	\item Falls $\Phi_N$ orthogonal, so ist die Kondition von (3.2) unabhängig von beschränkt durch
	\[
	\cond_2(\mathbb{B}_N(\mu)) = \norm{\mathbb{B}_N(\mu)}_2 \norm{\mathbb{B}_N^{-1}(\mu)}_2\le \frac{\gamma(\mu)}{\alpha(\mu)}.
	\]
\end{enumerate}

\bet{Beweis:}\\
(1) klar.\\
(2): Wegen Symmetrie con $\mathbb{B}_N(\mu)$ ist 
\begin{align}
\cond_2(\mathbb{B}_N(\mu)) = \frac{\abs{\lambda_{\max}}}{\abs{\lambda_{\min}}}
\end{align}
mit betragsmäßig größten/kleinstem Eigenwert $\lambda_{\max},\lambda_{\min}$ von $\mathbb{B}_N(\mu)$.
Sei $\mathbb{U}_{\max} = (U_n)_{n=1}^N\in \R^N$ Eigenvektor zu $\lambda_{\max}$ und $u_{\max} := \sum_{n=1}^{N} U_n\phi_n$.
Dann gilt
\begin{align*}
\lambda_{\max} \norm{\mathbb{U}_{\max}}_2^2 &= \lambda_{\max} \mathbb{U}_{\max}^T\cdot\mathbb{U}_{\max} = \mathbb{U}_{\max}^T \mathbb{B}_N(\mu) \mathbb{U}_{\max}\\
&= \sum_{n,m=1}^N U_nU_m b(\phi_n,\phi_m;\mu) = b(\sum_{n=1}^{N} U_n\phi_n,\sum_{m=1}^{N} U_m\phi_m;\mu) \\
&= b(u_{\max},u_{\max};\mu)
\end{align*}
Aus der Orthogonalität folgt
\[
\norm{u_{\max}}_X^2 = 
\]

\ssect{3.27 Bemerkung}{}
Im Gegensatz zu großen aber dünn besetzten Matrizen bei FEM ist (3.2) klein aber voll besetzt, weil $\phi_i$ im Allgemeinen keinen disjunkten Träger haben.

\ssect{3.28 Folgerung}{}
Sei $\bar{\varphi}_i,~i=1\dt{,}\mathcal{N}_h$ die Knotenbasis von $X_h$ wie in Def. 2.50 definiert.
Für $\mu\in \mathcal{P}$ definieren wir $\mathbb{B}_h(\mu)\in \R^{\mathcal{N}_h\times \mathcal{N}_h}$ und $\mathbb{F}_h(\mu)\in \R^{\mathcal{N}_h}$ durch
\begin{align}
(\mathbb{B}_h(\mu))_{ij} := b(\bar{\varphi}_j,\bar{\varphi}_j;\mu),~ (\mathbb{F}_h(\mu))_i := f(\bar{\varphi}_i;\mu),~1\le i,j\le \mathcal{N}_h.
\end{align}
Indem wir dnun die RB-Basisfunktionen in der Knotenbasis darstellen
\begin{align}
\Phi_n = \sum_{i=1}^{\mathcal{N}_h} \phi_h^i\bar{\varphi}_i,~n=1\dt{,}N
\end{align}
können wir eine Transformationsmatrix $\mathbb{V}\in \R^{\mathcal{N}_h\times N}$ definieren deren Spalten die Koeffizienten der RB-Basisfunktionen in (3.5) enthalten:
\begin{align}
\mathbb{V}_{in} := \phi_n^i,~1\le i\le \mathcal{N}_h,~1\le n\le N.
\end{align}
Dann gilt für $\mathbb{B}_N(\mu)$ und $\mathbb{F}_N(\mu)$ aus Satz 3.25:
\[
\mathbb{B}_N(\mu)= \mathbb{V}^T\mathbb{B}_h(\mu)\mathbb{V} \text{ und } \mathbb{F}_N(\mu) = \mathbb{V}^T\mathbb{F}_h(\mu).
\]

\bet{Beweis:}\\
\begin{align*}
(\mathbb{V}^T\mathbb{B}_h(\mu)\mathbb{V})_{mn} &= \sum_{r,s=1}^{\mathcal{N}_h} (\mathbb{V}^T)_{mr} (\mathbb{B}_h(\mu))_rs (\mathbb{V}_{sn})\\
&= \sum_{r,s=1}^{\mathcal{N}_h} \phi_m^r b(\bar{\varphi}_s,\bar{\varphi}_r;\mu) \phi_n^s\\
&= b\enbrace{\sum_{s=1}^{\mathcal{N}_h}\phi_n^s \bar{\varphi}_s,\sum_{r=1}^{\mathcal{N}_h}\phi_m^r \bar{\varphi}_r;\mu}\\
&= b(\phi_n,\phi_m;\mu) = (\mathbb{B}_N(\mu))_{mn}.
\end{align*}
\hfill $\square$

\subsection{Offline/Online Zerlegung des RB-Modells}

\ssect{3.29 Bemerkung}{(Komplexitätsbetrachtungen)}
Da $\mathbb{B}_h(\mu)$ aus (3.4) dünn besetzt ist, erfordert die Berechnung von $u_h(\mu)$ $ \mathcal{O}(\mathcal{N}_h^2)$ Rechenschritte.
Da $\mathbb{B}_N(\mu)$ vollbesetzt ist das lineare Gleichungssystem (LGS) (3.2) in $\mathcal{O}(N^3)$ Rechenschritten lösbar.
Daher ist nur für $N\ll \mathcal{N}_h$ da RB-Modell ein Gewinn.
Genauere Betrachtung der Berechnung einer reduzierten Lösung $u_N(\mu)$:
\begin{enumerate}[(1)]
	\item $N$ Snapshots, also $N$ Lösungen $u_h(\mu)$ von $(P_h(\mu))$ berechnen: $\mathcal{O}(N\mathcal{N}_h^2)$. ('Offline')
	\item $N^2$ Auswertungen von $b(\phi_m,\phi_n;\mu)$: $\mathcal{O}(N^2\mathcal{N}_h)$.
	\item $N$ Auswertungen von $f(\phi_n;\mu)$: $\mathcal{O}(N\mathcal{N}_h)$.
	\item Die Lösung des LGS (3.2): $\mathcal{O}(N^3)$. ('Online')
\end{enumerate}
Damit lohnt sich das RB-Modell für ein einzelnes $\mu\in\mathcal{P}$ oder wenige $\mu\in\mathcal{P}$ \uline{nicht}.
Wenn wir $(P_N(\mu))$ aber für viele verschiedene Parameter $\mu\in\mathcal{P}$ lösen müssen, wie zum Beispiel in einem \textit{many-query} Kontext lohnt sich das RB-Modell, wenn man eine sogenannte \bet{Offline/Online Zerlegung} durchführt.

In der einmalig durchgeführten \bet{Offline-Phase} werden \uline{$\mu$-unabhängige, hochdimensionale} Größen in $\mathcal{O}(\mathcal{N}_h^m),m\in \N$ Rechenschritten, typischerweise teuer vorberechnet.
In der \uline{vielfach} durchgeführten \bet{Online-Phase} werden die Offline-Daten kombiniert um \uline{$\mu$-abhängige} Größen wie das reduzierte LGS (3.2) zu assemblieren.
Die RB-Lösung $u_N(\mu)$ und $s_N(\mu)$ können dann schnell berechnet werden, wobei die Anzahl der Rechenschritte idealerweise $\mathcal{O}(N^k),k\in \N$ ist, d.h. unabhängig von $\mathcal{N}_h$.

Vor diesem Hintergrund können wir Schritt 1 klar der Offline-Phase und Schritt 4 der Online-Phase zu ordnen.
Schritt 2 und 3 lassen sich direkt keiner der beiden Phasen klar zu ordnen, da sie sowohl teure als auch Parameter-abhängige Operationen benötigen.
Um eine klare Trennung auch von Schritt 2 und 3 zu erreichen benötigen wir eine spezielle Struktur der Bilinearform $b(.,.;\mu)$ und Linearform $f(.;\mu)$.
 
\ssect{3.30 Definition}{}
Seinen $X,X_1,X_2$ HR, $\mathcal{P}$ beschränkte Parametermenge.
\begin{enumerate}[(1)]
	\item Eine Funktion $v:\mathcal{P}\to X$ nennen wir \bet{affin parametrisch}, falls Funktionen $v^q\in X$ und Koeffizientenfunktionen $\theta_v^q: \mathcal{P}\to \R$ für $q=1\dt{,}Q_v$ existieren, so dass
	\[
	v(x;\mu) := \sum_{q=1}^{Q_v} \theta_v^q(\mu)v^q(x).
	\]
	\item Eine parametrische stetige Linearform $f:X\times \mathcal{P}\to \R$ bzw. stetige Bilinearform $b:X_1\times X_2\times \mathcal{P}\to \R$ ist \bet{affin parametrisch}, falls $f^q\in X'$ und $\theta_f^q:\mathcal{P}\to \R$ für $q=1\dt{,} Q_f$ bzw. $b^q:X_1\times X_2 \to \R$ und $\theta_B^q: \mathcal{P}\to \R$ für $q=1\dt{,}Q_b$ existieren, so dass
	\begin{align*}
	f(v;\mu) &= \sum_{q=1}^{Q_f} \theta_f^q(\mu) f^q(v)~\forall v\in X \text{ bzw.}\\
	b(u,v;\mu) &= \sum_{q=1}^{Q_b} \theta_b^q(\mu) b^q(u,v)~\forall u\in X_1,v\in X_2.
	\end{align*}
\end{enumerate}

\ssect{3.31 Folgerung}{(Offline/Online-Zerlegung von $(P_N(\mu))$)}
Sei $(P_N(\mu))$ gegeben und $b,f$ affin parametrisch. Dann erlaubt $(P_N(\mu))$ die folgende Offline/Online-Zerlegung:
\begin{itemize}
	\item[Offline-Phase:]
	Nach Berechnung einer reduzierten Basis $\Phi_N := \penbrace{\phi_1\dt{,}\phi_N}$ assemblieren wir die parameterunabhägige Matrizen und Vektoren $\mathbb{B}_N^q\in \R^{N\times N}$ und $\mathbb{F}_N^q\in \R^N$, definiert durch
	\[
	(\mathbb{B}_N^q)_{nm} := b^q(\phi_m,\phi_n),~1\le m,n \le N,~1\le q\le Q_b,
	\]
	\[
	(\mathbb{F}_N^q)_n := f^q(\phi_n),~1\le n\le N,~1\le q\le Q_f.
	\]
	\item[Online-Phase]
	Für einen gegebenen Parametervektor $\mu\in\mathcal{P}$ werten wir die parameterabhängigen Koeffizientenfunktionen $\theta_b^q(\mu),\theta_f^q(\mu)$ für $1\le q\le Q_b,Q_f$ aus und assemblieren die Matrix und den Vektor
	\[
	\mathbb{B}_N(\mu) := \sum_{q=1}^{Q_b} \theta_b^q(\mu) \mathbb{B}_N^q \text{ bzw. } \mathbb{F}_N(\mu) := \sum_{q=1}^{Q_f} \theta_f^q(\mu) \mathbb{F}_N^q,
	\]
	welche mit der Matrix und dem Vektor aus dem LGS (3.2) aus Satz 3.26 übereinstimmen.
	Dieses LGS kann dann nach $u_N(\mu)$ und $s_N(\mu)$ gelöst werden.
\end{itemize}

\ssect{3.32 Bemerkung}{}
Die Matrizen $\mathbb{B}_N^q\in \R^{N\times N}$ und die Vektoren $\mathbb{F}_N^q\in \R^N$ können mit dem in Folgerung 3.28 beschriebenen Verfahren einfach aus den entsprechenden FE-Matrizen und den FE-Vektoren mit der in (3.6) definierten Transformationsmatrix assembliert werden.

\ssect{3.33 Bemerkung}{(Rechenaufwand/Laufzeit)}
\begin{enumerate}[(1)]
	\item Für die Berechnung der Snapshots und der anschließenden Assemblierung von $\mathbb{B}_N^q,\mathbb{F}_N^q$ benötigen wir $\mathcal{O}(N\mathcal{N}_h^2 + N^2\mathcal{N}_hQ_b + N\mathcal{N}_hQ_f)$ Rechenschritte.
	In der Online-Phase kann dann die Assemblierung und das Lösen von LGS (3.2) in $\mathcal{O}(N^2Q_b + NQ_f + N^3)$ Rechenschritten erfolgen.
	Insbesondere hängt die Komplexität der Online-Phase nicht von $\mathcal{N}_h$ ab.
	\item Die Offline/Online Zerlegung lässt sich auch in einem Laufzeitdiagramm veranschaulichen. $t_{hoch},t_{offline},t_{online}$ bezeichnen die Laufzeit dür eine Lösung des hochdimensionalen, diskreten Problems $(P_h(\mu))$, die Offline- und die Online-Phase von $(P_N(\mu))$.
	Wir nehmen an, dass diese Zeiten für unterschiedliche Parameter jeweils dieselben sind und erhalten dadurch einen linearen Zusammenhang zwischen der gesamten benötigten Laufzeit und der Anzahl $k$ von Berechnungen der Lösungen $u_h(\mu),u_N(\mu)$.
	Die Gesamtlaufzeit für $k$ hochdimensionale Lösungen ist $t_h(k) = k\cdot t_{hoch}$, während das reduzierte Modell eine Laufzeit von $t_N(k) = t_{offline} + k\cdot t_{online}$ benötigt.
	Wie bereits in 3.29 erwähnt lohnt sich ein RB-Modell bei mehr als $k* := \frac{t_{offline}}{t_{hoch}-t_{online}}$ benötigten Approximationen von $u(\mu)$.
\end{enumerate}

\subsection{A posteriori Fehlerschätzer}

\subsubsection{A posteriori Fehlerschranken und Effektivität}

\ssect{3.34 Lemma}{(Fehler-Residuum Beziehung)}
Für $\mu \in \mathcal{P}$ definieren wir mittels der RB-Lösung $u_N(\mu)$ das Residuum $r(.;\mu)\in X_h'$ durch
\begin{align}
r(v;\mu) := f(v;\mu)- b(u_N(\mu),v;\mu) \forall v\in X_h
\end{align}
und den zugehörigen Riesz-Repräsentanten $R(\mu)\in X_h$ als Lösung von
\begin{align}
(R(\mu),v)_X = r(v;\mu) ~\forall v\in X_h.
\end{align}
Dann erfüllt der Fehler $e_N(\mu) := u_h(\mu)-u_N(\mu)$
\begin{align}
b(e_N(\mu),v;\mu) = r(v;\mu) ~\forall v\in X_h.
\end{align}

\bet{Beweis:}\\
\begin{align*}
b(e_N(\mu),v;\mu) &= b(u_h(\mu)-u_N(\mu),v;\mu)\\
&= b(u_h(\mu),v;\mu) - b(u_N(\mu),v;\mu)\\
&= f(v;\mu) - b(u_N(\mu),v;\mu) = r(v;\mu)
\end{align*}
\hfill $\square$

\ssect{3.35 Satz}{(A posteriori Fehlerschätzer)}
Für $\mu\in\mathcal{P}$ seinen $u_h(\mu),s_h(\mu)$ Lösungen von $(P_h(\mu))$ und $u_N(\mu),s_N(\mu)$ Lösungen von $(P_N(\mu))$.
Ferner sei $\alpha_{LB}(\mu) > 0$ eine berechenbare untere Schranke für die Koerzivitätskonstante $\alpha_h(\mu)$ von $b(.,.;\mu)$ und $R(\mu)$ der Riesz-Repräsentant des Residuums aus Lemma 3.34.
Dann erfüllen die A posteriori Fehlerschätzer , definiert durch 
\begin{align}
\Delta_N^{en}(\mu) := \frac{\norm{R(\mu)}_X}{\sqrt{\alpha_{LB}(\mu)}} \text{ und } \Delta_N^s(\mu) := \frac{\norm{R(\mu)}_X^2}{\alpha_{LB}(\mu)},
\end{align}
die folgenden Ungleichungen
\begin{align}
|||u_h(\mu)-u_N(\mu)|||_\mu = |||e_N(\mu)|||_\mu \le \Delta_N^{en}(\mu)\\
s_h(\mu)-s_N(\mu) \le \Delta_N^s(\mu).
\end{align}

\bet{Beweis:}\\
Testen von Gleichung (3.9) mit $e_N(\mu)$ ergibt:
\begin{align*}
|||e_N(\mu)|||_\mu &\bgl{Def} b(e_N(\mu),e_N(\mu);\mu) \bgl{(3.9)} r(e_N(\mu);\mu)\\
&\bgl{(3.8)} (R(\mu),e_N(\mu))_X \stackrel{C.S}{\le} \norm{R(\mu)}_X\norm{e_N(\mu)}_X\\
&\bgl{3.4} \frac{1}{\sqrt{\alpha_h(\mu)}} \norm{R(\mu)}_X |||e_N(\mu)|||_\mu \le \frac{1}{\sqrt{\alpha_{LB}(\mu)}} \norm{R(\mu)}_X |||e_N(\mu)|||_\mu\\
&\Rightarrow |||e_N(\mu)|||_\mu \le \Delta_N^{en}(\mu) \Rightarrow (3.11)
\end{align*}
Aus Satz 3.19 folgt
\[
s_h(\mu) -s_N(\mu) = |||e_n(\mu)|||_\mu^2 \le (\Delta_N^{en}(\mu))^2 = \Delta_N^s(\mu).
\]
\hfill $\square$

\ssect{3.36 Folgerung}{}
Durch $\hat{s}_N(\mu) := s_N(\mu) + \Delta_N^s(\mu)$ ist eine obere Schranke für $s_h(\mu)$ gegeben, das heißt es gilt
\[
s_h(mu) \le \hat{s}_N(\mu).
\]

\bet{Beweis:}\\
Folgt direkt aus (3.12).

\ssect{3.37 Bemerkung}{}
Das Beschränken des Fehlers durch das Residuum ist eine Standardtechnik zum Herleiten von A posteriori Fehlerschätzern für FEM.
Da in diesem Fall dein Schätzer für den Fehler $|||u(\mu)-u_h(\mu)|||_\mu$ gesucht wird, ist $X$ unendlich-dimensional und die Norm $\norm{r(.;\mu)}_{X'}$ kann nicht berechnet werden.
Im Fall von RB Methoden ist $\norm{r(.;\mu)}_{X_h'}$ mit Hilfe des Riesz-Repräsentanten berechenbar.

\ssect{3.38 Bemerkung}{}
Da $\Delta_N^{en}(\mu)$ und $\Delta_N^s(\mu)$ unter den Voraussetzungen von Satz 3.35 obere Schranken für die Fehler sind, werden sie auch als \uline{rigorose} Fehlerschranken bezeichnet.
Bei A posteriori Fehlerschätzern für FEM treten häufig Konstanten in den Abschätzungen auf, welche nicht entsprechend nach oben/unten durch berechenbare Konstanten beschränkt werden können, so dass in diesen Fällen die A posteriori Fehlerschätzer den Fehler auch unterschätzen können;
sie also keine rigorose Fehlerschranken zu sein brauchen.
Mit Hilfe des Fehlerschätzers können wir die Dimension des RB-Raumes so bestimmen, dass der Approximationsfehler kleiner als eine vorgegebene Toleranz ist.
Um ein möglichest effizientes Verfahren zu erhalten ist es daher wünschenswert, dass der Quotient $\frac{\Delta_N^{en}(\mu}{|||e_N(\mu)|||_\mu}$ möglichst nahe an 1 ist.
Er ist $\ge 1$ wegen Satz 3.35.
Diesen Quotienten werden wir im Folgenden weiter untersuchen.

\ssect{3.39 Satz}{(Effektivitäten der Fehlerschätzer)}
Wir definieren die Effektivitäten $\eta_N^{en}(\mu)$ und $\eta_N^s(\mu)$ der Fehlerschätzer $\Delta_N^{en}(\mu)$ und $\Delta_N^s(\mu)$, definiert in (3.10), durch
\begin{align}
\eta_N^{en}(\mu) := \frac{\Delta_N^{en}(\mu)}{|||u_h(\mu)-u_N(\mu)|||_\mu} \text{ und } \eta_N^s(\mu) :=\frac{\Delta_N^s(\mu)}{s_h(\mu)-s_N(\mu)}.
\end{align}
Unter den Voraussetzungen von Satz 3.35 gilt dann
\begin{align}
&\eta_N^{en}(\mu) \le \sqrt{\frac{\gamma(\mu)}{\alpha_{LB}(\mu)}} \text{ und }\\
&\eta_N^s(\mu)\le \frac{\gamma(\mu)}{\alpha_{LB}(\mu)}.
\end{align}

\bet{Beweis:}\\
Zunächst folgt aus der Definition de Riesz-Repräsentanten und des Residuums in Lemma 3.34:
\begin{align*}
\norm{R(\mu)}_X^2 = (R(\mu),R(\mu))_X \bgl{(3.8)} r(R(\mu);\mu)\\
&\bgl{(3.9)} b(e_N(\mu),R(\mu);\mu \stackrel{C.S}{\le} |||e_N(\mu)|||_\mu |||R(\mu)|||_\mu\\
&\stackrel{3.4}{\le} |||e_N(\mu)|||_\mu \sqrt{\gamma(\mu)}\norm{R(\mu)}_X.
\end{align*}
\begin{align}
\Rightarrow \norm{R(\mu)}_X \le |||e_N(\mu)|||_\mu \sqrt{\gamma(\mu)}
\end{align}
\begin{align*}
\eta_N^{en}(\mu) &= \frac{\Delta_N^{en}(\mu)}{|||e_N(\mu)|||_\mu} \bgl{Def} \frac{\norm{R(\mu)}_X}{\sqrt{\alpha_{LB}(\mu)}|||e_N(\mu)|||_\mu}\\
&\stackrel{(3.16)}{\le} \sqrt{\frac{\gamma(\mu)}{\alpha_{LB}(\mu)}} \frac{|||e_N(\mu)|||_\mu}{|||e_N(\mu)|||_\mu}\\
&\Rightarrow (3.14)
\end{align*}
Aus Satz 3.19 folgt dann:
\begin{align*}
\eta_N^s(\mu)\bgl{Def} \frac{\Delta_N^s(\mu)}{s_h(\mu)s_N(\mu)} \bgl{Def/3.19} \frac{(\Delta_N^{en}(\mu))^2}{|||e_N(\mu)|||_\mu^2} \stackrel{(3.14)}{\le} \frac{\gamma(\mu)}{\alpha_{LB}(\mu)} \Rightarrow (3.15).
\end{align*}

\ssect{3.40 Folgerung}{}
Falls $u_h(\mu) = u_N(\mu)$ dann gilt automatisch $\Delta_N^{en}(\mu) = \Delta_N^s(\mu) = 0$.\\

\bet{Beweis:}\\
Folgt direkt aus Satz 3.39, kann aber auch unabhängig davon wie folgt eingesehen werden:
Da $0=b(0,v;\mu) = b(e_N(\mu),v;\mu) \bgl{(3.9)} r(v;\mu) \bgl{(3.8)} (R(\mu),v)_X$ für alle $v\in X$ gilt, folgt $\norm{R(\mu)}_X=0$ und damit $\Delta_N^{en}(\mu) = \Delta_N^s(\mu) = 0$.

\ssect{3.41 Bemerkung}{}
Folgerung 3.40 ist insbesondere dann relevant, wenn für einen Fehlerschätzer Schranken für die Effektivität (noch) nicht verfügbar sind.

\ssect{3.42 Folgerung}{(Fehlerschätzer für die $X$-Norm)}
Unter den Voraussetzungen von Satz 3.35 gilt für den Fehlerschätzer $\Delta_N(\mu) := \frac{\norm{R(\mu)}_X}{\alpha_{LB}(\mu)}$, dass 
\[
\norm{u_h(\mu)-u_N(\mu)}_X \le \Delta_N(\mu).
\] 
Ferner gilt für die Effektivität des Fehlerschätzers $\eta_N(\mu) := \frac{\Delta_N(\mu)}{\norm{u_h(\mu)-u_N(\mu)}_X}$ die folgende Schranke
\[
\eta_N(\mu) \le \frac{\gamma(\mu)}{\alpha_{LB}(\mu)}.
\]

\bet{Beweis:}\\
Analog zu den Beweisen von Satz 3.35/3.39 unter Verwendung von Lemma 3.4.
\hfill $\square$\\

Zusätzlich zu absoluten Fehlerschätzern wollen wir schließlich noch relative Fehlerschätzer herleiten und die zugehörigen Effektivitäten untersuchen.

\ssect{3.34 Satz}{(Relative Fehlerschätzer)}
Wir definieren die relativen Fehlerschätzer 
\[
\Delta_N^{en,rel}(\mu) := 2\frac{\norm{R(\mu)}_X}{\sqrt{\alpha_{LB}(\mu)}}\cdot \frac{1}{|||u_N(\mu)|||_\mu}
\]
für den relativen Fehler in der Energienorm,
\[
\Delta_N^{rel}(\mu) := 2\frac{\norm{R(\mu)}_X}{\alpha_{LB}(\mu)}\cdot \frac{1}{||u_N(\mu)||_X}
\]
für den relativen Fehler in der $X$-Norm und 
\[
\Delta_N^{s,rel}(\mu) := \frac{\norm{R(\mu)}_X^2}{\alpha_{LB}(\mu)s_N(\mu)}
\]
für den relativen Ausgabefehler.
Dann gilt unter den Voraussetzungen von Satz 3.35 und falls $\Delta_N^{en,rel}(\mu)\le 1, \Delta_N^{rel}(\mu)\le 1$
\begin{align}
\frac{|||u_h(\mu)-u_N(\mu)|||_\mu}{|||u_h(\mu)|||_\mu} &\le \Delta_N^{en,rel}(\mu),\\
\frac{||u_h(\mu)-u_N(\mu)||_X}{||u_h(\mu)||_X} &\le \Delta_N^{rel}(\mu),\\
\frac{s_h(\mu)-s_N(\mu)}{s_h(\mu)} &\le \Delta_N^{s,rel}(\mu).
\end{align}

\bet{Beweis:}\\
Falls $\Delta_N^{en,rel}(\mu)\le 1$, so gilt
\begin{align}
\abs{\frac{|||u_h(\mu)|||_\mu-|||u_N(\mu)|||_\mu}{|||u_N(\mu)|||_\mu}} \le \frac{|||u_h(\mu)-u_N(\mu)|||_\mu}{|||u_N(\mu)|||_\mu} \stackrel{(3.11)}{\le} \frac{\norm{R(\mu)}_X}{\alpha_{LB}(\mu)|||u_N(\mu)|||_\mu} = \frac{\Delta_N^{en,rel}(\mu)}{2} \le \frac{1}{2}.
\end{align}
Aus (3.20) folgt:
\[
|||u_N(\mu)|||_\mu - |||u_h(\mu)|||_\mu \le \frac{1}{2} |||u_N(\mu)|||_\mu
\]
\begin{align}
\Rightarrow \frac{1}{2} |||u_N(\mu)|||_\mu \le |||u_h(\mu)|||\mu.
\end{align}
Damit gilt:
\begin{align*}
\frac{|||e_N(\mu)|||_\mu}{|||u_h(\mu)|||_\mu} \stackrel{(3.11)}{\le} \frac{\norm{R(\mu)}_X}{\sqrt{\alpha_{LB}(\mu)}|||u_h(\mu)|||_\mu} \stackrel{(3.21)}{\le} \frac{\norm{R(\mu)}_X}{\sqrt{\alpha_{LB}(\mu)}|||u_N(\mu)|||_\mu}\cdot 2 = \Delta_N^{en,rel}(\mu).
\end{align*}
Daraus folgt (3.17).
Und (3.18) folgt analog zu (3.17).
Schließlich gilt 
\begin{align*}
\frac{s_h(\mu)-s_N(\mu)}{s_h(\mu)} \stackrel{(3.12)}{\le} \frac{\Delta_N^{s}(\mu)}{s_h(\mu)} \stackrel{3.19}{\le} \frac{\Delta_N^s(\mu)}{s_N(\mu)} = \Delta_N^{s,rel}(\mu).
\end{align*}
\hfill $\square$

\ssect{3.44 Satz}{(Effektivitäten der relativen Fehlerschätzer)}
Wir definieren die Effektivitäten der relativen Fehlerschätzer $\eta_N^{en,rel}(\mu),\eta_N^{rel}(\mu),\eta_N^{s,rel}(\mu)$ wie folgt:
\begin{align*}
\eta_N^{en,rel}(\mu) := \frac{\Delta_N^{en,rel}(\mu)}{|||e_N(\mu)|||_\mu / |||u_h(\mu)|||_\mu},~ \eta_N^{rel}(\mu) := \frac{\Delta_N^{rel}(\mu)}{||e_N(\mu)||_X / ||u_h(\mu)||_X},~ \eta_N^{s,rel}(\mu) := \frac{\Delta_N^{s,rel}(\mu)}{(s_h(\mu)-s_N(\mu)/ s_h(\mu)}.
\end{align*}
Dann gilt unter den Voraussetzungen von Satz 3.35, falls $\Delta_N^{en,rel}(\mu)\le 1,\Delta_N^{rel}(\mu)\le 1,\Delta_N^{s,rel}(\mu)\le 1$:
\begin{align*}
\Delta_N^{en,rel}(\mu) \le 3\sqrt{\frac{\gamma(\mu)}{\alpha_{LB}(\mu)}} \tag{3.21b},
\end{align*}
\begin{align}
\Delta_N^{rel}(\mu) &\le 3\frac{\gamma(\mu)}{\alpha_{LB}(\mu)}\\
\eta_N^{s,rel}(\mu) &\le 2\frac{\gamma(\mu)}{\alpha_{LB}(\mu)}.
\end{align}

\bet{Beweis:}\\
Wie in Beweis von Satz 3.43 impliziert $\Delta_N^{en,rel}(\mu)\le 1$ dass
\[
\abs{\frac{|||u_h(\mu)|||_\mu-|||u_N(\mu)|||_\mu}{|||u_N(\mu)|||_\mu}}  \le \frac{1}{2}.
\]
Damit gilt $|||u_h(\mu)|||_\mu-|||u_N(\mu)|||_\mu \le \frac{1}{2} |||u_N(\mu)|||_\mu$ und damit
\begin{align}
|||u_h(\mu)|||_\mu \le \frac{3}{2} |||u_N(\mu)|||_\mu.
\end{align}
Damit folgt:
\begin{align*}
\eta_N^{en,rel}(\mu) &\bgl{Def} \frac{2\norm{R(\mu)}_X}{\sqrt{\alpha_{LB}(\mu)}|||u_h(\mu)|||_\mu} \frac{|||u_h(\mu)|||_\mu}{|||e_N(\mu)|||_\mu}\\
&\stackrel{(3.16)}{\le} 2\frac{\sqrt{\gamma(\mu)}\norm{e_N(\mu)}_X}{\sqrt{\alpha_{LB}(\mu)}|||u_h(\mu)|||_\mu} \frac{|||u_h(\mu)|||_\mu}{|||e_N(\mu)|||_\mu} \stackrel{(3.24)}{\le} 3 \sqrt{\frac{\gamma(\mu)}{\alpha_{LB}(\mu)}} \Rightarrow (3.21b).
\end{align*}
(3.22) folgt analog zu (3.21b).
Schließlich gilt
\begin{align*}
\eta_N^{s,rel}(\mu) = \frac{\Delta_N^s(\mu) / s_N(\mu)}{(s_h(\mu)-s_N(\mu)) / s_h(\mu)} = \eta_N^s(\mu) \frac{s_h(\mu)}{s_N(\mu)} \stackrel{(3.15)}{\le} \frac{\gamma(\mu)}{\alpha_{LB}(\mu)} \frac{s_h(\mu)}{s_N(\mu)}.
\end{align*}
Der letzte Faktor ist beschränkt, da
\[
\frac{s_h(\mu)}{s_N(\mu)} = 1 + \frac{s_h(\mu)-s_N(\mu)}{s_N(\mu)} \stackrel{(3.19)}{\le} 1 + \Delta_N^{s,rel}(\mu) \le 2 \Rightarrow (3.23).
\]
\hfill $\square$

\ssect{3.45 Bemerkung}{}
Durch "Tauschen" des Skalarprodukts/der Norm auf $X_h$ können die Fehlerschätzer und Effektivitäten verbessert werden, ohne die Ausgabe des reduzierten Modells oder die reduzierte Lösung zu verändern.
Wähle $\bar{\mu}\in\mathcal{P}$ fest und betrachte auf $X_h$ das Skalarprodukt $(((.,.)))_{\bar{\mu}}$ mit induzierter Norm $|||.|||_{\bar{\mu}}$.
Wegen Lemma 3.4 Normäquivalenz folgt aus der Stetigkeit und Koerzivität der Bilinearform $b$ auf $X_h$ bzgl. der $X$-Norm die Stetigkeit und Koerzivität auf $X_h$ bzgl. der $|||.|||_{\bar{\mu}}$-Norm und umgekehrt.
Gleiches gilt für die Stetigkeit von $l$ und $f$.
Dann gilt $\alpha_h(\bar{\mu}) := \inf_{v\in X_h} \frac{b(v,v;\bar{\mu})}{|||v|||_{\bar{\mu}}^2} = 1$ und $\gamma_h(\bar{\mu}) := \sup_{u,v\in X_h} \frac{b(u,v;\bar{\mu})}{|||u|||_{\bar{\mu}}|||v|||_{\bar{\mu}}}  = \sup_{u,v\in X_h} \frac{(((u,v)))_{\bar{\mu}}}{|||u|||_{\bar{\mu}}|||v|||_{\bar{\mu}}} \le 1$.
Für einen Fehlerschätzer welcher auf der $|||.|||_{\bar{\mu}}$-Norm des Riesz-Repräsentanten $R(\mu)$ basiert gilt also $\eta(\bar{\mu})=1$.
Er ist in diesem Sinne optimal.
Nimmt man an, dass $\alpha_h(\mu)$ und $\gamma_h(\mu)$ stetig von Parameter $\mu$ abhängen, kann man erwarten auch in einer Umgebung von $\bar{\mu}$ sehr effektive Fehlerschätzer zu erhalten.

\subsubsection{Offline/Online-Zerlegung des Fehlerschätzers}
Damit wir in der Online-Phase verifizeren können, dass der Approximationsfehler unter einer vorgegeben Toleranz liegt, ist es wichtig, dass wir auch den Fehlerschätzer Offline/Online zerlegen können.
Die Erkenntnis, dass sich die affine Parameterabhängigkeit der Bilinearform $b$ und der Linearform $f$ auf das Residuum und die Norm des Riesz-Repräsentanten überträgt ist hierbei von zentraler Bedeutung.
Mit Lemma 3.34 und der afiinen Parameterabhängigkeit von $b$ und $f$ folgt:
\begin{align*}
(R(\mu),v) &= r(v;\mu) = f(v;\mu) - b(u_N(\mu),v;\mu)\\
&= \sum_{q=1}^{Q_f} \theta_f^q(\mu)f^q(v)- \sum_{q=1}^{Q_b}\sum_{n=1}^{N} \theta_b^q(\mu)U_n^N(\mu)b^q(\phi_n,v).
\end{align*}
Nach dem Riesz'schen Darstellungssatz 2.9 existieren $R_f^q\in X-h$ mit 
\begin{align}
(R_f^q,v)_X = f^q(v) ~\forall v\in X_h,~1\le q\le Q_f
\end{align}
und $R_b^{q,n}\in X_h$ mit
\begin{align}
(R_b^{q,n},v)_X = b^q(\phi_n,v) ~\forall v\in X_h,~1\le q\le Q_b,~1\le n\le N.
\end{align}
Daher können wir weiter umformen:
\begin{align*}
(R(\mu),v)_X &= \sum_{q=1}^{Q_f} \theta_f^q(\mu)(R_f^q,v)_X - \sum_{q=1}^{Q_b}\sum_{n=1}^{N} \theta_b^q(\mu)U_n^N(\mu)(R_b^{q,n},v)_X\\
&= \enbrace{\sum_{q=1}^{Q_f} \theta_f^q(\mu)R_f^q - \sum_{q=1}^{Q_b}\sum_{n=1}^{N} \theta_b^q(\mu)U_n^N(\mu)R_b^{q,n},v}_X ~\forall v\in X_h\\
&\Rightarrow R(\mu) =  \sum_{q=1}^{Q_f} \theta_f^q(\mu)R_f^q - \sum_{q=1}^{Q_b}\sum_{n=1}^{N} \theta_b^q(\mu)U_n^N(\mu) R_b^{q,n}.
\end{align*}
Wir fassen dieses Resultat im folgenden Lemma zusammen.

\ssect{3.46 Lemma}{(Affine Parameterabhängigkeit von $R(\mu)$}
Seien $b,f$ affin parametrisch und $R_f^q,R_b^{q,n}\in X_h$ definiert wie in (3.25)/(3.26) .
Sei $Q_R:= Q_f+N\cdot Q_b$ und $R_R^q$ für $1\le q\le Q_R$ eine Aufzählung von $R_f^q,R_b^{q,n}$:
\begin{align*}
(R_R^1\dt{,}R_R^{Q_R}) := (R_f^1\dt{,}R_f^{Q_f},R_b^{1,1}\dt{,}R_b^{Q_b,1},R_b^{1,2}\dt{,}R_b^{Q_b,2}\dt{,}R_b^{1,N}\dt{,}R_b^{N,Q_b}).
\end{align*}
Für $\mu\in\mathcal{P}$ sei $u_N(\mu) = \sum_{n=1}^{N} U_n^N(\mu) \phi_n$ die Lösung von $(P_N(\mu))$.
Dann definieren wir $\theta_R^q: \mathcal{P}\to \R,~1\le q\le Q_R$ durch
\begin{align*}
(\theta_R^1(\mu)\dt{,}\theta_R^{Q_R}(\mu)) := (\theta_f^1(\mu)\dt{,}\theta_f^{Q_f}(\mu), -\theta_b^1(\mu)U_1^N(\mu)\dt{,} -Q_b^{Q_b}(\mu)\dt{,} -\theta_b^{Q_b}(\mu)U_N^N(\mu)).
\end{align*}
Dann ist der Riesz-Repräsentant $R(\mu)\in X_h$ des Residuums affin parametrisch:
\begin{align*}
R(\mu) = \sum_{q=1}^{Q_R} \theta_R^q(\mu)R_R^q.
\end{align*}

























\cleardoubleoddemptypage
\pagenumbering{Alph}
\setcounter{page}{1}


\printindex
\listoffigures
\end{document}