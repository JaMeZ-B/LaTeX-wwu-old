\documentclass[a4paper, pagesize=pdftex, pdftex, twoside, headsepline, index=totoc,toc=listof, fontsize=10pt, cleardoublepage=empty, headinclude, DIV=13, BCOR=13mm]{scrartcl}

\usepackage[ngerman]{babel}
\usepackage{scrtime} % Bestandteil von KOMA-Skript, ermoeglicht Zugriff auf Uhrzeit des Kompilierens 
\usepackage{scrpage2} % ermöglicht Bearbeiten von Kopf- und Fusszeilen (wie fancyhdr, nur optimiert auf KOMA-Skript, leich andere Syntax)
\usepackage[utf8]{inputenc} % Gibt an in welcher Textcodierung der Code verstandne werden soll
\usepackage{etex} % sehr technisch, ermöglicht LaTeX mehr Speicher zu belegen
\usepackage[T1]{fontenc} % auch sehr technisch; ist wichtig, um die Schriftarten richtig zu behandeln
\usepackage{textcomp} %verhindert ein paar Fehler bei den Fonts
\usepackage{mathtools} % Packet der American Mathematical Society, das viele Mathematik-Umgebungen und -Befehle definiert
\usepackage{amssymb} %zusätzliche Symbole
\usepackage{latexsym} % nochmal zusätzliche Symbole
\usepackage{stmaryrd} % nochmal mehr zusätzliche Symbole, u.a. Blitz für Widerspruchsbeweise ;)
\usepackage{nicefrac} % schräge Brüche, benutzte ich für Quotienvektorräume
\usepackage{paralist} % redefiniert alle Listenbefehle, sodass diese einen optionalen Parameter haben, der die Nummerierung angibt
\usepackage{dsfont} % Schriftart für N,Z,Q,R die ich momentan benutze (mittels \mathds{R} z.B)
\usepackage[pdftex]{graphicx} % Packet, dass das Einbinden von Grafiken aus Dateien ermöglicht
\usepackage{makeidx}% ermöglicht das automatische Anlegen eines Index 
\usepackage{extarrows}
\usepackage{bbold}
\usepackage[hyphens]{url}
\usepackage{algorithmicx}
\usepackage{algpseudocode}


%\usepackage{MnSymbol}
\flushbottom
\usepackage[normalem]{ulem}
\setlength{\ULdepth}{1.8pt}

%--Indexverarbeitung
\newcommand{\bet}[1]{\textbf{#1}} %Betonung von Text
\newcommand{\Index}[1]{\textbf{#1}\index{#1}} % Befehl, der gleichzeitg das Argument hervorhebt und in den Index mitaufnimmt
\makeindex % startet das automatische Sammeln der Index-Einträge
% Ein kleiner Text am Anfang des Index
\setindexpreamble{{\noindent \itshape Die \emph{Seitenzahlen} sind mit Hyperlinks zu den entsprechenden Seiten versehen, also anklickbar!} \par \bigskip}
\renewcommand{\indexpagestyle}{scrheadings} % Seitenstil für den Index festlegen

%--Farbdefinitionen
\usepackage[usenames, table, x11names]{xcolor} %usenames und x11names, aktivieren viele Farben; siehe Dokumentation von xcolor
% Es lassen sich natürlich auch eigene Farben definieren (hier nur Graustufen)
\definecolor{dark_gray}{gray}{0.45}
\definecolor{light_gray}{gray}{0.7}

%--Zum Zeichnen (ich habe es jetzt mal mit aufgenommen, aber es ist eigentlich nochmal ein ganz anderes Thema, sodass ich da jetzt nicht viel zu sagen werde)
\usepackage{tikz} % TikZ steht übrigens für "TikZ ist kein Zeichenprogramm", ein rekursives Akronym ...
\tikzset{>=latex}
\usetikzlibrary{shapes,arrows}
\usetikzlibrary{calc}
\usetikzlibrary{decorations.pathreplacing}
% Hiermit kann man ganz leicht kommutative Diagramme zeichnen (deswegen auch "cd")
\usepackage{tikz-cd}

%--Marginnote, ermöglicht es kleine Notizen an neben den eigentlichen Textkörper zu setzten
\usepackage{marginnote}
\renewcommand*{\marginfont}{\color{Honeydew4} \footnotesize }

%--Schriftarten
\usepackage{lmodern} % neuere Version der Standard-LaTeX-Schriftarten
\renewcommand{\familydefault}{\sfdefault} %Standardschriftart auf die serifenlose Schriftart setzen

%--Hyperref; aktiviert Hyperlinks in der erzeugten PDF-Datei und definiert deren Aussehen
\usepackage[hidelinks, pdfpagelabels,  bookmarksopen=true, bookmarksnumbered=true, linkcolor=black, urlcolor=SkyBlue2, plainpages=false,pagebackref, citecolor=black, hypertexnames=true, pdfauthor={Tobias Wedemeier}, pdfborderstyle={/S/U}, linkbordercolor=SkyBlue2, colorlinks=false]{hyperref}
%--Römische Zahlen
\newcommand{\RM}[1]{\MakeUppercase{\romannumeral #1{}}}



%-- Definitionen von weiteren Mathe-Befehlen, die dann das "richtige" Aussehen haben. Hier sind der Phantasie keine Grenzen gesetzt
\DeclareMathOperator{\id}{id} %identische Abbildung
\DeclareMathOperator{\End}{End} %Endomorphismen
\DeclareMathOperator{\rg}{rg} %Rang
\DeclareMathOperator{\diam}{diam} %Durchmesser
\DeclareMathOperator{\dist}{dist} %Distanz
\DeclareMathOperator{\grad}{grad} %Gradient
\DeclareMathOperator{\rot}{rot} %Rotation
\DeclareMathOperator{\hess}{Hess} %Hesse-Matrix
\DeclareMathOperator{\supp}{supp}
\DeclareMathOperator{\aut}{Aut}
\DeclareMathOperator{\inn}{Inn}
\DeclareMathOperator{\sym}{Sym}
\DeclareMathOperator{\syl}{Syl}
\DeclareMathOperator{\alt}{Alt}
\DeclareMathOperator{\sign}{sign}
\DeclareMathOperator{\Sl}{Sl}
\DeclareMathOperator{\Gl}{Gl}
\DeclareMathOperator{\Quot}{Quot}
\DeclareMathOperator{\ggT}{ggT}
\DeclareMathOperator{\cone}{cone}
\DeclareMathOperator{\Char}{char}
\DeclareMathOperator{\im}{Im}
\DeclareMathOperator{\re}{Re}
\DeclareMathOperator{\Bin}{Bin}
\DeclareMathOperator{\acl}{acl}
\DeclareMathOperator{\cov}{cov}
\DeclareMathOperator{\argmin}{argmin}
\DeclareMathOperator{\argmax}{argmax}
\DeclareMathOperator{\Tol}{TOL}
\DeclareMathOperator{\RK}{RK}
\DeclareMathOperator{\divv}{div}
\DeclareMathOperator{\tr}{tr}
\DeclareMathOperator{\spann}{span}
\DeclareMathOperator{\esssup}{ess sup}
\DeclareMathOperator{\bild}{bild}
\DeclareMathOperator{\cond}{cond}
\DeclareMathOperator{\train}{train}
\DeclareMathOperator{\spur}{spur}
\DeclareMathOperator{\diag}{diag}
%--Skalarprodukt (cooler Befehl, den ich im Internet gefunden habe; benutzt TeX-Befehle)
\makeatletter
\newcommand{\sprod}[2]{\ensuremath{%
  \setbox0=\hbox{\ensuremath{#2}}
  \dimen@\ht0
  \advance\dimen@ by \dp0
  \left\langle \left.#1 \,\rule[-\dp0]{0pt}{\dimen@}\right|#2\right\rangle}}
\makeatother

%--Norm (auch aus dem Internet, wird auch auf der Beispielseite verwandt)
\newcommand{\norm}[1]{
	\ensuremath{\left\Vert#1\right\Vert}
}
\newcommand{\ener}[1]{
	\ensuremath{\left|\left|\left|#1\right|\right|\right|_\mu}
}

%--selbstgeschriebenen Befehle
%--Betrag
\newcommand{\abs}[1]{\ensuremath{\left\vert#1\right\vert}}

%--Umklammern mit passender Größe der Klammern
\newcommand{\enbrace}[1]{\ensuremath{\left( #1\right)}}

%--Mengen
\newcommand{\penbrace}[1]{\ensuremath{\left\{#1\right\}}}
\newcommand{\cenbrace}[1]{\ensuremath{\left[#1\right]}}

%--Differential
\newcommand{\diff}[2]{\ensuremath{\frac{\partial #1}{\partial #2} }}
\newcommand{\difff}[2]{\ensuremath{\frac{\dint #1}{\dint #2} }}

\newcommand{\zz}{\ensuremath{\mathrm{Z\kern-.3em\raise-0.5ex\hbox{$Z$}}}} % zu zeigen ZZ aus dem inet
\setlength{\parindent}{0pt}%absatz nicht einrücken
\newcommand{\lh}[1]{\langle #1 \rangle} %lineare Hülle
\newcommand{\nt}{\trianglelefteqslant} %normalteiler
\newcommand{\pfs}{\mathds{P}-\text{f.s.}} %P fast sicher Konvergenz
\newcommand{\dint}{\mathrm{d}} % d des integrals

\newcommand{\xfrac}[2]{%
	\mbox{\raisebox{-0.4ex}{\ensuremath{\displaystyle #1}\hspace{0.2ex}}%
		{\raisebox{-0.1ex}{\big \backslash}}%
		\raisebox{0.6ex}{\ensuremath{\displaystyle #2}}%
	}%
}
\newcommand{\Pw}{\mathds{P}}
\newcommand{\V}{\mathds{V}}
\newcommand{\E}{\mathds{E}}
\newcommand{\R}{\mathds{R}}
\newcommand{\N}{\mathds{N}}
\newcommand{\Z}{\mathds{Z}}
\newcommand{\Q}{\mathds{Q}}
\newcommand{\G}{\mathcal{G}}
\newcommand{\F}{\mathcal{F}}
\newcommand{\D}{\mathcal{D}}
\newcommand{\C}{\mathds{C}}
\newcommand{\A}{\mathcal{A}}
\newcommand{\mc}[1]{\mathcal{#1}}
\newcommand{\Pfs}[1][\relax]{\ensuremath{
	\ifx#1\relax \Pw-\text{f.s.}
	\else \Pw^{#1}-\text{f.s.}
	\fi}}
\newcommand{\bgl}[1]{\stackrel{#1}{=}}
\newcommand{\dt}[1]{#1 \dots #1}
\newcommand{\ablim}[1]{\limits_{\mathclap{#1}}}
\newcommand{\degree}{\ensuremath{^\circ}}
\newcommand{\adj}{\ensuremath{^\ast}}



\newcommand{\sect}[1]{\section*{#1}\addcontentsline{toc}{section}{#1}}
\newcommand{\ssect}[2]{ \subsubsection*{#1 #2}\addcontentsline{toc}{subsubsection}{#1}}


\usepackage{listings}
\definecolor{light_green}{rgb}{0,0.5,0}
\definecolor{grey}{rgb}{.5,.5,.5}
\lstset{language=Python, commentstyle=\color{light_green}\bfseries,
	keywordstyle=\color{blue}\bfseries,
	stringstyle=\ttfamily\color{orange},
	morekeywords={as},
	deletendkeywords={range,abs,set},
	escapeinside={\%*}{\&*}
}

\lstset{showspaces=false,
	showstringspaces=false, tabsize=4, breaklines=true, rulecolor=\color{black}}
\lstset{numbers=left,basicstyle=\ttfamily\footnotesize, numberstyle=\tiny\color{grey}, numbersep=10pt}

\renewcommand{\lstlistlistingname}{Programmcodeverzeichnis}
\renewcommand{\lstlistingname}{Programmcode}








\newcommand{\vorlesung}{Modellreduktion und partielle Differentialgleichungen}
\newcommand{\Prof}{Dr. Smetana}
\newcommand{\subt}{Mitschrift der Tafelnotizen}

\input{../!config/Tazdr/extra_files/headings.tex}


\numberwithin{equation}{section}


\begin{document}
\maketitle
\thispagestyle{empty}
\cleardoubleoddemptypage

\thispagestyle{empty}
\vspace*{\fill}
\begin{center}
	Hierbei handelt es sich um eine \subt von \textbf{\Prof}, WWU Münster, aus der Vorlesung \textbf{\vorlesung} im Wintersemester 2015/16. 
	Dies ist kein Skript der Vorlesung und keine eigene Arbeit des Autors.\\
	\vspace{2cm}
	Für Fehler in der Mitschrift wird keine Haftung übernommen. 
	Hinweise auf Fehler sind gerne gesehen, hierfür kann man mich in der Uni ansprechen oder alternativ eine e-Mail an: \textit{tobias.wedemeier@gmx.de}\\
	Auch ist eine Mitarbeit über Github möglich.\\
	\vspace{2cm}
	Wenn Teile aus der Vorlesung selber fehlen, können diese gerne an meine e-Mail versandt werden. 
	Ich werde diese dann einarbeiten.\\
\end{center}
\vspace*{\fill}
\cleardoubleoddemptypage

\pagenumbering{Roman}

\tableofcontents
\cleardoubleoddemptypage %sorgt dafür, dass alles folgende erst auf der nächsten freien "rechten" Seite steht


\section{Einletung und Motivation}

\subsection{Parameterabhängige PDGL}
\label{sub:para_pdgl}
Sei $\Omega\subseteq \R^d$ ein polygonales Gebiet. Zu einem Parametervektor $\mu \in P\subseteq \R^d$ aus einer Menge von 'erlaubten' Parametern ist eine Funktion, z.B. 'Temperatur'
\[
u(\mu): \Omega\to \R
\]
gesucht, so dass $-\nabla\big(\kappa(\mu)\nabla u(\mu)\big) = q(\mu)$ in $\Omega$, wobei $u(\mu) = 0$ auf $\partial\Omega$, mit $\kappa(\mu):\Omega\to \R$ dem 'Wärmeleitkoeffizient' und $q(\mu)$ eine 'Wärmequelle', z.B. $q(\mu)=1$.
%grafik 1
Weiter kann eine Augabe erwünscht sein, z.B.
\[
s(\mu) = \frac{1}{\abs{\Omega_s}} \int\lim\limits_{\Omega_s} u(x,\mu)\dint x,
\]
die mittlere Temperatur auf $\Omega_s$.
%grafik 2
%sub end

\subsection{Definition (schwache Formulierung in Hilberträumen)}
\label{sub:def_hilbert}
Sei $X$ ein reeller Hilbertraum.
Zu $\mu\in P$ ist gesucht ein $u(\mu)\in X$ und eine Ausgabe $s(\mu)\in \R$, so dass
\[
b\big(u(\mu),v;\mu\big) = f(v;\mu), ~ s(\mu) = l\big(u(\mu);\mu\big)~\forall v\in X
\]
für eine Bilinearform $b(\cdot,\cdot;\mu): X\times X \to \R$ und linearen Funktionalen $f(\cdot;\mu),l(\cdot;\mu):X\to \R$.

Die schwache Formulierung für Beispiel 1.1 lautet:
\[
X:= H_0^1(\Omega)= \enbrace{f\in L^2(\Omega)~:+\difff{ }{x_1}f\in L^2(\Omega),~f|_{\partial \Omega = 0}}
\]
Dann kann man die Bilinearform über
\[
b\big(u(\mu),v;\mu\big) := \int\limits_{\Omega} \kappa(\mu)\nabla u(\mu)\nabla v \dint x; f(v;\mu) := \int\lim\limits_{\Omega} q(\mu) v \dint x
\]
ausdrücken und 
\[
s(\mu) = \frac{1}{\abs{\Omega_s}} \int\lim\limits_{\Omega_s} u(x;\mu)\dint x =: l\big(u(\mu);\mu\big)
\]
ABER: Für sehr wenige PDGL's können wir die Lösung analytisch bestimmen.
Daher sind wir an einer numerische Approximation interessiert.
Ein weit verbreitetes Diskretisierungsverfahren ist die Finite Elemente Methode.
Diese Methode basiert auf obiger schwacher Formulierung.
%sub end

\subsection{Definition (hochdimensionales, diskretes Modell)}
\label{sub:def_hochdim}
Sei $X_h\subseteq X$ mit $\dim(X_h) =N_h < \infty$.
Der Index $h$ bezeichnet hier die Gitterweite.
Zu $\mu\in P$ ist gesucht ein $u_h(\mu)\in X_h$ und eine Ausgabe $s_h(\mu)\in \R$, so dass
\begin{align}
b\big(u_h(\mu),v_h;\mu\big) = f(v_h;\mu), ~ s_h(\mu)= l_h\big(v_h(\mu);\mu\big)~\forall v_h\in X_h.
\end{align}
Anwendungen für die Standarddiskretisierungsverfahren sehr teuer oder zu teuer sind:
\minisec{many-query context}
\begin{itemize}
	\item Parameterstudien
	\item Design
	\item Parameteridentifikation / inverse Probleme
	\item Optimierung
	\item Statistische Analyse
\end{itemize}
\minisec{schnelle Simulationsantwort}
\begin{itemize}
	\item Echtzeit-Steuerung technischer Geräte
	\item interaktive Benutzeroberflächen
\end{itemize}
%sub end

\subsection{Parameterabhängige Lösungsmenge}
\label{sub:para_menge}
Sei $\mu :=\{u(\mu)~:~\mu\in P\}\subseteq P$ für $P\in \R^p$ ist die durch $\mu$ parametrisierte Lösungsmenge.
$X$ ist die im Allgemeinen unendlichdimensional.
%grafik
$\Rightarrow$ Motivation für die Suche nach einem 'niedrigdimensionalen' Teilraum $X_N\subseteq X$ zur Approximation von $M$ und einer Approximation $u_N(\mu) \approx u(\mu),~ u_N\in X_N$.
Eine Möglichkeit eine reduzierte Basis zu generieren besteht darin geschickt Parameterwerte $\mu_1\dt{,}\mu_N\in P$ zu wählen und den Raum als $X_N := \spann\{u(\mu_1)\dt{,}u(\mu_N)\}$ zu definieren.
Eine Lösung $u(\mu_i)$ für einen Parameterwert $\mu\in P$ wird auch \bet{Snapshot} genannt.

\subsection{Beispiel}
\label{sub:beispiel_1}
Gesucht ist $u(\cdot;\mu) \in C^2([0,1])$ mit $(1+\mu)u'' = 1$ auf $(0,1)$ und $u(0)=u(1)=1$ für den Parameter $\mu\in P:= [0,1]\subseteq \R$.\\
\bet{Snapshots:}\\
$\mu_1 = 0 \Rightarrow u_1 := u(\cdot;\mu_1) = \frac{1}{2} x^2- \frac{1}{2}x +1$, $\mu_2 = 0 \Rightarrow u_2 := u(\cdot;\mu_2) = \frac{1}{4} x^2- \frac{1}{4}x +1$ und $X_N := \spann\{u_1,u_2\}$.
Dann ist die reduzierte Lösung $u_N(\mu)\in X_N$ gegeben durch
\[
u_N(\mu) = \alpha_1(\mu)u_1 +\alpha_2(\mu) u_2,
\]
mit $\alpha_1 = \frac{2}{\mu+1}-1$ und $\alpha_2 = 2-\frac{2}{\mu-1}$.
Diese erfüllt folgende Fehleraussage und ist somit exakt:
\[
\norm{u_N(\mu)-u(\mu)}_\infty = \sup\limits_{\lambda\in[0,1]} \abs{U_N(x;\mu)-u(x;\mu)} = 0
\]
Da $\alpha_1 + \alpha_2 = 1$ und $0\le \alpha_1,\alpha_2\le 1$ ist $M$ die Menge der Konvexkombinationen von $u_1$ und $u_2$.
%sub end

\subsection{Definition (reduziertes Modell)}
\label{sub:reduziertes_modell}
Sei $X_N\subseteq X$ ein reduzierter Basisraum mit $\dim(X_N)<\infty$.
Zu $\mu\in P$ ist gesucht ein $u_N(\mu)\in X_N$ und eine Ausgabe $s_N(\mu)\in \R$, so dass
\begin{align}
b\big(u_N(\mu),v_N;\mu\big) = f(v_N;\mu),~ s_N(\mu) = l_N\big(u_N(\mu);\mu\big) ~ \forall v_N\in X_N
\end{align}
%sub end

\subsection{Bermerkung (Begrifflichkeit)}
\label{sub:bem_begrifflichkeit}
Zusammengefasst unterscheiden wir zwischen den folgenden drei Modellen:
\begin{enumerate}[1)]
	\item Eine partielle DGL ist ein \bet{analytisches Modell}, welches die analytische Lösung $u(\mu)\in X$ in einem (typischerweise) $\infty$-dimensionalen Funktionenraum charakterisiert ist.
	\item Ein \bet{hochdimensionales, diskretes Modell} ist ein Berechnungsverfahren zur Bestimmung einer Näherung $u_h(\mu)\in X_h$, wobei $X_h$ ein hochdimensionaler Funktionenraum ist.
	Beispiele sind \bet{Finite Elemente} oder \bet{Finite Volumenräume} und typischerweise hat $X_h$ eine Dimension von mindestens $10^5$.
	\item Ein \bet{reduziertes Modell} ist ein Berechnungsverfahren zur Bestimmung einer Näherung $u_N(\mu)\in X_N$ in einem sehr problemangepassten und daher niedrigdimensionalen Raum von typischerweise $\dim X_N < 100$.
	\item \bet{Modellreduktion} beschäftigt sich mit Modellen der Erzeugung von reduzierten Modellen aus hochdimensionalen, diskreten (oder auch analytischen) Modellen und Untersuchungen ihrer Eigenschaften.
\end{enumerate}

\subsection{Organisation der Vorlesung}
\label{sub:org}
\begin{itemize}
	\item[Zentrale Fragen:]
	\item \bet{Reduzierte Basis:} Wie kann ein möglichst kompakter Teilraum konstruiert werden?
	\item \bet{Reduziertes Modell:} Existenz von reduzierten Lösungen $u_N(\mu)$?
	Wie kann eine reduzierte Lösung $u_N(\mu)$ berechnet werden?
	\item \bet{Effizienz:} Wie kann $u_N(\mu)$ schnell berechnet werden?
	\item \bet{Stabilität:} Wie kann die Stabilität des reduzierten Modells für wachsendes $N$ garantiert werden?
	\item\bet{Approximationsgüte:} Warum können wir erwarten, dass eine relativ kleine Anzahl von Basisfunktionen ausreicht?
	\item \bet{Fehlerschätzer:} Kann der Fehler des reduzierten zum vollen Modell beschränkt werden?
	\item \bet{Effektivität:} Kann garantiert werden, dass der Fehlerschätzer den Fehler nicht beliebig überschätzt?
\end{itemize}

\minisec{Vorläufige Gliederung (bis Weihnachten)}
\begin{enumerate}[1)]
	\item Einleitung / Moitavtion
	\item Grundlagen:
	\begin{itemize}
		\item Kurze Einführung in lineare Funktionalanalysis
		\item Kurze Einführung in Finite Elemente
	\end{itemize}
	\item Reduzierte Basis Methoden für lineare, koerzive Probleme
	\begin{itemize}
		\item Reduzierte Basis Verfahren
		\item Offline-/ Online-Zerlegung
		\item Fehlerschätzer
		\item Basisgenerierung
	\end{itemize}
\end{enumerate}
%sub end

%sec end

\section{Grundlagen}

\subsection{Lineare Funktionalanalysis in Hilberträumen}

\subsubsection{Lineare Operatoren}

\ssect{2.1 Definition}{(Hilbertraum)}
Sei $X$ ein reeller Vektorraum mit $(\cdot,\cdot):X\times X\to \R$ ein Skalarprodukt und induzierter Norm $\norm{x}:= \sqrt{(x,x)}$.
falls $X$ vollständig bzgl. $\norm{.}$ , ist $X$ ein (reeller) \bet{Hilbertraum} (HR).

\ssect{2.2 Beispiele}{(Hilbertraum)}
\begin{enumerate}[(1)]
	\item $X:= \R^d$ mit $(x,y) := \sum_{i=1}^{d} x_i y_i$ ist ein HR.
	\item $X:= L^2(\Omega)$ mit $(x,y) := \int_{\Omega} f(x)g(x)\dint x $ ist ein HR.
	\item $X:=C^0([0,1])$ mit $(f,g) := \int_{0}^{1} f(x) g(x) \dint x$ ist kein HR.
\end{enumerate}

\ssect{2.3 Lemma}{ }
Seien $X$ und $Y$ reelle vektorräume.
Ist die Abbildung $T:X\to Y$ linear und $x_0\in X$, so sind äquivalent:
\begin{enumerate}[(1)]
	\item $T$ ist stetig.
	\item $T$ ist stetig in $x_0$.
	\item $\sup\lim\limits_{\norm{x}_X \le 1} \norm{Tx}_Y < \infty$.
	\item $\exists$ Konstante (mit $\norm{Tx}_Y \le c\norm{x}_X ~\forall x\in X$)
\end{enumerate}

\ssect{2.4 Definition}{(Lineare Operatoren)}
Seinen $X$ und $Y$ reelle Vektorräume.
Wir definieren
\[
L(X;Y) := \enbrace{T:X\to Y~;~ T \text{ ist linear und stetig}}.
\]
Abbildungen in $L(X;Y)$ nennen wir \bet{lineare Operatoren}.
Nach Lemma 2.3 (3) ist für jeden Operator $T\in L(X;Y)$ die \bet{Operatornorm} von $T$ definiert durch
\[
\norm{T}_{L(X;Y)} := \sup\limits_{\norm{x}_X\le 1} \norm{Tx}_Y < \infty,
\]
oder in kurz $\norm{T}$.
Es ist $L(X) := L(X;X)$.

\ssect{2.5 Definition}{(Spezielle lineare Operatoren)}
\begin{enumerate}[(1)]
	\item $X' := L(X;\R)$ ist der \bet{Dualraum} von $X$.
	Die Elemente von $X'$ nennen wir auch \bet{lineare Funktionale}.
	\item Die Menge der kompakten (linearen) Operatoren von $X$ nach $Y$ ist definiert durch
	\[
	K(X;Y) := \enbrace{T\in L(X;Y)~;~ T(\overline{B_1(0)}) \text{ kompakt}}.
	\]
	\item Eine lineare Abbildung $P:X\to X$ heißt (lineare) \bet{Projektion}, falls $P^2=P$.
	\item Für $T\in L(X;Y)$ ist $\ker(T) := \enbrace{x\in X~;~ Tx = 0}$ der \bet{Nullraum} oder \bet{Kern} von $T$.
	Aus der Stetigkeit von $T$ folgt, dass $\ker(T)$ ein abgeschlossener Unterraum ist.
	Der \bet{Bildraum} von $T$ ist $\bild(T) := \enbrace{Tx\in Y~;~x\in X}$.
	\item Ist $T\in L(X;Y)$ bijektiv, so ist $T^{-1}\in L(Y;X)$.
	Dann heißt $T$ (linear, stetiger) \bet{Isomorphismus}.
	\item $T\in L(X;Y)$ heißt \bet{Isometrie}, falls
	\[
	\norm{Tx}_Y = \norm{x}_X ~\forall x\in X
	\]
\end{enumerate}

\ssect{2.6 Beispiel}{ }
Sei $g\in L^2(\Omega)$.
Dann ist nach der Hölderungleichung durch 
\[
T_g f := \int_{\Omega} f(x) g(x)\dint x
\]
ein Funktional $T_g\in  L^2(\Omega)'$ definiert.

\ssect{2.7 Satz}{(Projektionssatz)}
Sei $X$ ein Hilbertraum und $A\subseteq X$ nicht leer, abgeschlossen und konvex.
Dann gibt es genau eine Abbildung $P:X\to A$ mit
\[
\norm{x-Px}_X = \dist(x,A) = \inf\lim\limits_{y\in A} \norm{x-y}_X ~\forall x\in X.
\]
Die Abbildung $P:X\to A$ heißt orthogonale Projektion von $X$ auf $A$.

\bet{Beweis:} [Alt, Satz 2.2, S.96]

\ssect{2.8 Folgerung}{ }
Ist $A\subseteq X$ nicht-leer, abgeschlossen und Unterraum, so ist $P$ linear und $Px\in A$ charakterisiert durch $(x-Px,a)_X = 0~\forall a\in A$.
Falls $\dim(A) = n<\infty$ und $\enbrace{\varphi_i}_{i=1}^h$ Orthonormalbasis von $A$, gilt
\[
Px = \sum_{i=1}^{n} (x, \varphi_i)_X\varphi_i.
\]

\ssect{2.9 Satz}{(Riesz'scher darstellungssatz)}
Ist $X$ Hilbertraum, so ist $J:X\to X'$ definiert durch
\[
J(v)(w) := (v,w)_X ~\forall v,w\in X 
\]
eine stetige, lineare, bijektive Isometrie.
Insbesondere existiert zu $l\in X'$ ein eindeutiger \bet{Riesz Repräasentant} $V_l := J^{-1}(l)\in X$ mit $l(.) = (v_l,.)_X$.

\bet{Beweis:}\\
C-S-Ungleichung: $\abs{J(v)(w)}\le \norm{v}_X\norm{w}_X$.
Dann folgt: $ J(v)\in X'$ mit 
\[
\norm{J(v)}_{X'} = \sup\limits_{w\in X\backslash\{0\}} \frac{\abs{J(v)(w)}}{\norm{w}_X} = \sup\limits_{w\in X\backslash\{0\}} \frac{\abs{(v,w)_X}}{\norm{w}_X} \le \norm{v}_X \Rightarrow J \text{ stetig}.
\]
Da $\abs{J(v)(v)} = \norm{v}_X^2$ folgt:
\[
\sup\limits_{w\in X\backslash\{0\}} \frac{\abs{J(v)(w)}}{\norm{w}_X} \ge \frac{\abs{J(v)(v)}}{\norm{v}_X} = \frac{\norm{v}_x^2}{\norm{v}_X} = \norm{v}_X.
\]
Also ist $J$ eine Isometrie und insbesondere ist $J$ injektiv.





















\cleardoubleoddemptypage
\pagenumbering{Alph}
\setcounter{page}{1}


\printindex
\listoffigures
\end{document}