\documentclass[a4paper, pagesize=pdftex, pdftex, twoside, headsepline, index=totoc,toc=listof, fontsize=10pt, cleardoublepage=empty, headinclude, DIV=13, BCOR=13mm]{scrartcl}

\usepackage[ngerman]{babel}
\usepackage{scrtime} % Bestandteil von KOMA-Skript, ermoeglicht Zugriff auf Uhrzeit des Kompilierens 
\usepackage{scrpage2} % ermöglicht Bearbeiten von Kopf- und Fusszeilen (wie fancyhdr, nur optimiert auf KOMA-Skript, leich andere Syntax)
\usepackage[utf8]{inputenc} % Gibt an in welcher Textcodierung der Code verstandne werden soll
\usepackage{etex} % sehr technisch, ermöglicht LaTeX mehr Speicher zu belegen
\usepackage[T1]{fontenc} % auch sehr technisch; ist wichtig, um die Schriftarten richtig zu behandeln
\usepackage{textcomp} %verhindert ein paar Fehler bei den Fonts
\usepackage{amsmath} % Packet der American Mathematical Society, das viele Mathematik-Umgebungen und -Befehle definiert
\usepackage{amssymb} %zusätzliche Symbole
\usepackage{latexsym} % nochmal zusätzliche Symbole
\usepackage{stmaryrd} % nochmal mehr zusätzliche Symbole, u.a. Blitz für Widerspruchsbeweise ;)
\usepackage{nicefrac} % schräge Brüche, benutzte ich für Quotienvektorräume
\usepackage{paralist} % redefiniert alle Listenbefehle, sodass diese einen optionalen Parameter haben, der die Nummerierung angibt
\usepackage{dsfont} % Schriftart für N,Z,Q,R die ich momentan benutze (mittels \mathds{R} z.B)
\usepackage[pdftex]{graphicx} % Packet, dass das Einbinden von Grafiken aus Dateien ermöglicht
\usepackage{makeidx}% ermöglicht das automatische Anlegen eines Index 
\usepackage{extarrows}
\usepackage{bbold}
\usepackage{mathtools}
%\usepackage{MnSymbol}

\flushbottom
\usepackage[normalem]{ulem}
\setlength{\ULdepth}{1.8pt}

%--Indexverarbeitung
\newcommand{\bet}[1]{\uline{\textbf{#1}}} %Betonung von Text
\newcommand{\Index}[1]{\uline{\textbf{#1}}\index{#1}} % Befehl, der gleichzeitg das Argument hervorhebt und in den Index mitaufnimmt
\makeindex % startet das automatische Sammeln der Index-Einträge
% Ein kleiner Text am Anfang des Index
\setindexpreamble{{\noindent \itshape Die \emph{Seitenzahlen} sind mit Hyperlinks zu den entsprechenden Seiten versehen, also anklickbar!} \par \bigskip}
\renewcommand{\indexpagestyle}{scrheadings} % Seitenstil für den Index festlegen

%--Farbdefinitionen
\usepackage[usenames, table, x11names]{xcolor} %usenames und x11names, aktivieren viele Farben; siehe Dokumentation von xcolor
% Es lassen sich natürlich auch eigene Farben definieren (hier nur Graustufen)
\definecolor{dark_gray}{gray}{0.45}
\definecolor{light_gray}{gray}{0.7}

%--Zum Zeichnen (ich habe es jetzt mal mit aufgenommen, aber es ist eigentlich nochmal ein ganz anderes Thema, sodass ich da jetzt nicht viel zu sagen werde)
\usepackage{tikz} % TikZ steht übrigens für "TikZ ist kein Zeichenprogramm", ein rekursives Akronym ...
\tikzset{>=latex}
\usetikzlibrary{shapes,arrows}
\usetikzlibrary{calc}
\usetikzlibrary{decorations.pathreplacing}
% Hiermit kann man ganz leicht kommutative Diagramme zeichnen (deswegen auch "cd")
\usepackage{tikz-cd}

%--Marginnote, ermöglicht es kleine Notizen an neben den eigentlichen Textkörper zu setzten
\usepackage{marginnote}
\renewcommand*{\marginfont}{\color{Honeydew4} \footnotesize }

%--Schriftarten
\usepackage{lmodern} % neuere Version der Standard-LaTeX-Schriftarten
\renewcommand{\familydefault}{\sfdefault} %Standardschriftart auf die serifenlose Schriftart setzen

%--Hyperref; aktiviert Hyperlinks in der erzeugten PDF-Datei und definiert deren Aussehen
\usepackage[colorlinks, pdfpagelabels, pdfstartview=FitH, bookmarksopen=true, bookmarksnumbered=true,linkcolor=black,urlcolor=SkyBlue2, plainpages=false, hypertexnames=false, citecolor=black, hypertexnames=true]{hyperref}

%--Römische Zahlen
\newcommand{\RM}[1]{\MakeUppercase{\romannumeral #1{}}}



%-- Definitionen von weiteren Mathe-Befehlen, die dann das "richtige" Aussehen haben. Hier sind der Phantasie keine Grenzen gesetzt
\DeclareMathOperator{\id}{id} %identische Abbildung
\DeclareMathOperator{\End}{End} %Endomorphismen
\DeclareMathOperator{\rg}{rg} %Rang
\DeclareMathOperator{\diam}{diam} %Durchmesser
\DeclareMathOperator{\dist}{dist} %Distanz
\DeclareMathOperator{\grad}{grad} %Gradient
\DeclareMathOperator{\rot}{rot} %Rotation
\DeclareMathOperator{\hess}{Hess} %Hesse-Matrix
\DeclareMathOperator{\supp}{supp}
\DeclareMathOperator{\aut}{Aut}
\DeclareMathOperator{\inn}{Inn}
\DeclareMathOperator{\sym}{Sym}
\DeclareMathOperator{\syl}{Syl}
\DeclareMathOperator{\alt}{Alt}
\DeclareMathOperator{\sign}{sign}
\DeclareMathOperator{\Sl}{Sl}
%--Skalarprodukt (cooler Befehl, den ich im Internet gefunden habe; benutzt TeX-Befehle)
\makeatletter
\newcommand{\sprod}[2]{\ensuremath{%
  \setbox0=\hbox{\ensuremath{#2}}
  \dimen@\ht0
  \advance\dimen@ by \dp0
  \left\langle \left.#1 \,\rule[-\dp0]{0pt}{\dimen@}\right|#2\right\rangle}}
\makeatother

%--Norm (auch aus dem Internet, wird auch auf der Beispielseite verwandt)
\newcommand{\norm}[2][\relax]{
\ifx#1\relax \ensuremath{\left\Vert#2\right\Vert}
\else \ensuremath{\left\Vert#2\right\Vert_{#1}}
\fi}


%--selbstgeschriebenen Befehle
%--Betrag
\newcommand{\abs}[1]{\ensuremath{\left\vert#1\right\vert}}

%--Umklammern mit passender Größe der Klammern
\newcommand{\enbrace}[1]{\ensuremath{\left( #1\right)}}

%--Mengen
\newcommand{\penbrace}[1]{\ensuremath{\left\{#1\right\}}}

%--Differential
\newcommand{\diff}[2]{\ensuremath{\frac{\partial #1}{\partial #2} }}

\newcommand{\zz}{\ensuremath{\mathrm{Z\kern-.3em\raise-0.5ex\hbox{$Z$}}}} % zu zeigen ZZ aus dem inet
\setlength{\parindent}{0pt}%absatz nicht einrücken
\newcommand{\lh}[1]{\langle #1 \rangle} %lineare Hülle
\newcommand{\nt}{\trianglelefteqslant} %normalteiler
\newcommand{\pfs}{\mathds{P}-\text{f.s.}} %P-f.s. konvergenz
\newcommand{\dint}{\mathrm{d}} % d des integrals

\newcommand{\xfrac}[2]{%
	\mbox{\raisebox{-0.4ex}{\ensuremath{\displaystyle #1}\hspace{0.2ex}}%
		{\raisebox{-0.1ex}{\big \backslash}}%
		\raisebox{0.6ex}{\ensuremath{\displaystyle #2}}%
	}%
}
\newcommand{\Pw}{\mathds{P}}
\newcommand{\E}{\mathds{E}}
\newcommand{\R}{\mathds{R}}
\newcommand{\N}{\mathds{N}}
\newcommand{\Z}{\mathds{Z}}
\newcommand{\Q}{\mathds{Q}}
\newcommand{\G}{\mathcal{G}}
\newcommand{\F}{\mathcal{F}}
\newcommand{\D}{\mathcal{D}}
\newcommand{\C}{\mathds{C}}

\newcommand{\sect}[1]{\section*{#1}\addcontentsline{toc}{section}{#1}}
\newcommand{\ssect}[1]{\subsection*{#1}\addcontentsline{toc}{subsection}{#1}}