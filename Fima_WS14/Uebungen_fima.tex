\documentclass[a4paper, pagesize=pdftex, pdftex, twoside, headsepline, index=totoc,toc=listof, fontsize=10pt, cleardoublepage=empty, headinclude, DIV=13, BCOR=13mm]{scrartcl}

\usepackage[ngerman]{babel}
\usepackage{scrtime} % Bestandteil von KOMA-Skript, ermoeglicht Zugriff auf Uhrzeit des Kompilierens 
\usepackage{scrpage2} % ermöglicht Bearbeiten von Kopf- und Fusszeilen (wie fancyhdr, nur optimiert auf KOMA-Skript, leich andere Syntax)
\usepackage[utf8]{inputenc} % Gibt an in welcher Textcodierung der Code verstandne werden soll
\usepackage{etex} % sehr technisch, ermöglicht LaTeX mehr Speicher zu belegen
\usepackage[T1]{fontenc} % auch sehr technisch; ist wichtig, um die Schriftarten richtig zu behandeln
\usepackage{textcomp} %verhindert ein paar Fehler bei den Fonts
\usepackage{amsmath} % Packet der American Mathematical Society, das viele Mathematik-Umgebungen und -Befehle definiert
\usepackage{amssymb} %zusätzliche Symbole
\usepackage{latexsym} % nochmal zusätzliche Symbole
\usepackage{stmaryrd} % nochmal mehr zusätzliche Symbole, u.a. Blitz für Widerspruchsbeweise ;)
\usepackage{nicefrac} % schräge Brüche, benutzte ich für Quotienvektorräume
\usepackage{paralist} % redefiniert alle Listenbefehle, sodass diese einen optionalen Parameter haben, der die Nummerierung angibt
\usepackage{dsfont} % Schriftart für N,Z,Q,R die ich momentan benutze (mittels \mathds{R} z.B)
\usepackage[pdftex]{graphicx} % Packet, dass das Einbinden von Grafiken aus Dateien ermöglicht
\usepackage{makeidx}% ermöglicht das automatische Anlegen eines Index 
\usepackage{extarrows}
\usepackage{bbold}
\usepackage{mathtools}
%\usepackage{MnSymbol}

\flushbottom
\usepackage[normalem]{ulem}
\setlength{\ULdepth}{1.8pt}

%--Indexverarbeitung
\newcommand{\bet}[1]{\uline{\textbf{#1}}} %Betonung von Text
\newcommand{\Index}[1]{\uline{\textbf{#1}}\index{#1}} % Befehl, der gleichzeitg das Argument hervorhebt und in den Index mitaufnimmt
\makeindex % startet das automatische Sammeln der Index-Einträge
% Ein kleiner Text am Anfang des Index
\setindexpreamble{{\noindent \itshape Die \emph{Seitenzahlen} sind mit Hyperlinks zu den entsprechenden Seiten versehen, also anklickbar!} \par \bigskip}
\renewcommand{\indexpagestyle}{scrheadings} % Seitenstil für den Index festlegen

%--Farbdefinitionen
\usepackage[usenames, table, x11names]{xcolor} %usenames und x11names, aktivieren viele Farben; siehe Dokumentation von xcolor
% Es lassen sich natürlich auch eigene Farben definieren (hier nur Graustufen)
\definecolor{dark_gray}{gray}{0.45}
\definecolor{light_gray}{gray}{0.7}

%--Zum Zeichnen (ich habe es jetzt mal mit aufgenommen, aber es ist eigentlich nochmal ein ganz anderes Thema, sodass ich da jetzt nicht viel zu sagen werde)
\usepackage{tikz} % TikZ steht übrigens für "TikZ ist kein Zeichenprogramm", ein rekursives Akronym ...
\tikzset{>=latex}
\usetikzlibrary{shapes,arrows}
\usetikzlibrary{calc}
\usetikzlibrary{decorations.pathreplacing}
% Hiermit kann man ganz leicht kommutative Diagramme zeichnen (deswegen auch "cd")
\usepackage{tikz-cd}

%--Marginnote, ermöglicht es kleine Notizen an neben den eigentlichen Textkörper zu setzten
\usepackage{marginnote}
\renewcommand*{\marginfont}{\color{Honeydew4} \footnotesize }

%--Schriftarten
\usepackage{lmodern} % neuere Version der Standard-LaTeX-Schriftarten
\renewcommand{\familydefault}{\sfdefault} %Standardschriftart auf die serifenlose Schriftart setzen

%--Hyperref; aktiviert Hyperlinks in der erzeugten PDF-Datei und definiert deren Aussehen
\usepackage[colorlinks, pdfpagelabels, pdfstartview=FitH, bookmarksopen=true, bookmarksnumbered=true,linkcolor=black,urlcolor=SkyBlue2, plainpages=false, hypertexnames=false, citecolor=black, hypertexnames=true]{hyperref}

%--Römische Zahlen
\newcommand{\RM}[1]{\MakeUppercase{\romannumeral #1{}}}



%-- Definitionen von weiteren Mathe-Befehlen, die dann das "richtige" Aussehen haben. Hier sind der Phantasie keine Grenzen gesetzt
\DeclareMathOperator{\id}{id} %identische Abbildung
\DeclareMathOperator{\End}{End} %Endomorphismen
\DeclareMathOperator{\rg}{rg} %Rang
\DeclareMathOperator{\diam}{diam} %Durchmesser
\DeclareMathOperator{\dist}{dist} %Distanz
\DeclareMathOperator{\grad}{grad} %Gradient
\DeclareMathOperator{\rot}{rot} %Rotation
\DeclareMathOperator{\hess}{Hess} %Hesse-Matrix
\DeclareMathOperator{\supp}{supp}
\DeclareMathOperator{\aut}{Aut}
\DeclareMathOperator{\inn}{Inn}
\DeclareMathOperator{\sym}{Sym}
\DeclareMathOperator{\syl}{Syl}

%--Skalarprodukt (cooler Befehl, den ich im Internet gefunden habe; benutzt TeX-Befehle)
\makeatletter
\newcommand{\sprod}[2]{\ensuremath{%
  \setbox0=\hbox{\ensuremath{#2}}
  \dimen@\ht0
  \advance\dimen@ by \dp0
  \left\langle \left.#1 \,\rule[-\dp0]{0pt}{\dimen@}\right|#2\right\rangle}}
\makeatother

%--Norm (auch aus dem Internet, wird auch auf der Beispielseite verwandt)
\newcommand{\norm}[2][\relax]{
\ifx#1\relax \ensuremath{\left\Vert#2\right\Vert}
\else \ensuremath{\left\Vert#2\right\Vert_{#1}}
\fi}


%--selbstgeschriebenen Befehle
%--Betrag
\newcommand{\abs}[1]{\ensuremath{\left\vert#1\right\vert}}

%--Umklammern mit passender Größe der Klammern
\newcommand{\enbrace}[1]{\ensuremath{\left( #1\right)}}

%--Mengen
\newcommand{\penbrace}[1]{\ensuremath{\left\{#1\right\}}}

%--Differential
\newcommand{\diff}[2]{\ensuremath{\frac{\partial #1}{\partial #2} }}

\newcommand{\zz}{$\mathrm{Z\kern-.3em\raise-0.5ex\hbox{Z}}$} % zu zeigen ZZ aus dem inet
\setlength{\parindent}{0pt}%absatz nicht einrücken
\newcommand{\lh}[1]{\langle #1 \rangle} %lineare Hülle
\newcommand{\nt}{\trianglelefteqslant} %normalteiler
\newcommand{\pfs}{\mathds{P}-\text{f.s.}} %P-f.s. konvergenz
\newcommand{\dint}{\mathrm{d}} % d des integrals

\newcommand{\xfrac}[2]{%
	\mbox{\raisebox{-0.4ex}{\ensuremath{\displaystyle #1}\hspace{0.2ex}}%
		{\raisebox{-0.1ex}{\big \backslash}}%
		\raisebox{0.6ex}{\ensuremath{\displaystyle #2}}%
	}%
}
\newcommand{\Pw}{\mathds{P}}
\newcommand{\E}{\mathds{E}}
\newcommand{\R}{\mathds{R}}
\newcommand{\N}{\mathds{N}}
\newcommand{\Z}{\mathds{Z}}


\newcommand{\sect}[1]{\section*{#1}\addcontentsline{toc}{section}{#1}}
\newcommand{\ssect}[1]{\subsection*{#1}\addcontentsline{toc}{subsection}{#1}}

\newcommand{\vorlesung}{Finanzmathematik}
\newcommand{\Prof}{PD Dr. Paulsen}
\newcommand{\subt}{Aufarbeitung der Übungen}

\input{extra_files/headings.tex}

\begin{document}
\maketitle
\thispagestyle{empty}
\newpage

\thispagestyle{empty}
\vspace*{\fill}
\begin{center}
	Hierbei handelt es sich um eine \subt von \textbf{\Prof}, WWU Münster, aus der Vorlesung \textbf{\vorlesung} im Wintersemester 2014/15. Dies ist kein Skript der Vorlesung und keine eigene Arbeit des Autors.\\
	\vspace{2cm}
	Für Fehler in der Aufarbeitung wird keine Haftung übernommen. Hinweise auf Fehler sind gerne gesehen, hierfür kann man mich in der Uni ansprechen oder alternativ eine e-Mail an: \textit{tobias.wedemeier@gmx.de}\\
	Auch ist eine Mitarbeit über Github möglich.\\
	\vspace{2cm}
	Die Beweise stammen größtenteils vom Autor oder anderen Kommilitonen, und sind teilweise nur verkürzt oder vereinfacht dargestellt. Sie dienen nur dem Verständnis.
		
\end{center}
\vspace*{\fill}
\newpage
	
\pagenumbering{Roman}
	
\tableofcontents
\cleardoubleoddemptypage %sorgt dafür, dass alles folgende erst auf der nächsten freien "rechten" Seite steht
	
\pagenumbering{arabic}
\setcounter{page}{1}



\sect{Zettel 1}
\label{sub:zettel_1fima}

\ssect{Aufgabe 1}
\bet{Bear Spread}:\\ \index{Strategien!Bear Spread}
long put, strike $K_2$, short put, strike $K_1<K_2$
\begin{center}
	\begin{tikzpicture}[line cap=round,line join=round,>=triangle 45,x=.7cm,y=.7cm]
	\draw[->,color=black] (-1.,0.) -- (12.5,0.);
	\foreach \x in {3.,9.}
	\draw[shift={(\x,0)},color=black] (0pt,2pt) -- (0pt,-2pt);
	\draw[->,color=black] (0.,-2.) -- (0.,5.);
	\draw (12.5, -.1) node[anchor=north west] {$S_T$};
	\draw (.1,5.) node[anchor=north west] {Profit};
	\draw[red] (0.,3.) -- (3.,3.) node [black,midway,below] {Bear Spread};
	\draw[red] (3.,3.) -- (9.,-.5);
	\draw[red] (9.,-.5) -- (12.,-.5);
	\draw[black, style=dashed] (0.,4.3) -- (9.,-1.);
	\draw[black,style=dashed] (9.,-1.) -- (12.,-1.) node [right] {long put};
	\draw[black, style=dashed] (0.,-1.) -- (3.,2.5);
	\draw[black,style=dashed] (3.,2.5) -- (12.,2.5) node [right] {short put};
	\draw[thick,black,decorate,decoration={brace,amplitude=2pt}] (-0.01,0.) -- (-0.01,2.5) node[midway,left]{$p_1$};
	\draw[thick,black,decorate,decoration={brace,amplitude=2pt}] (-0.01,-1.) -- (-0.01,0.) node[midway,left]{$p_2$};
	\draw (3.,0.) node [below] {$K_1$};
	\draw (9.,0.) node [below] {$K_2$};
	\end{tikzpicture}
	\captionof{figure}{Bear Spread}
\end{center}
Profit: $(K_2-S_T)^+ - p_2+p_1-(K_1-S_T)^+=(K_2-K_1)+(p_1-p_2)$

\ssect{Aufgabe 3}
\Index{Exchange-Option}:\\
Analog zur Put-Call-Parität.
% subsection zettel 1 (end)

\sect{Zettel 2}
\label{sub:zettel_2fima}

\ssect{Aufgabe 1}
\bet{Eigenschaften des Put-Preises:}\\
\begin{enumerate}[(i)]
	\item innerer Wert: $P(S_0,T,K)\ge \max\{0,K\cdot B(0,T)-S_0\}$
	\item obere Grenze: $P(S_0,T,K)<K$
	\item Monoton im strike: $K_1\le K_2\Rightarrow P(S_0,T,K_1)\le P(S_0,T,K_2)$
	\item $B(0,T)(K_2-K_1)\ge P(K_2)-P(K_1)\quad \forall K_1\le K_2$
	\item Konvexität in $K$: 
	\[ P(K_2)\le \lambda P(K_1)+(1-\lambda)P(K_3)\quad \forall K_1<K_2<K_3\text{ mit } \lambda=\frac{K_3-K_2}{K_3-K_1} \]
\end{enumerate}
Beweise analog zum \hyperref[sub:call-preis]{Call-Preis}.

\ssect{Aufgabe 2}
Gelte \uline{No-Arbitrage} und \uline{keine Deflation},d.h. $0<B(0,T)<1$.\\
Dann ist der Call-Preis monoton in der Zeit, also \[ T_1<T_2:\quad C(T_1)\le C(T_2) \]
\bet{Beweis:}\\
Sei $C(T_1)$ der Preis eines Calls auf ein Underlying mit Laufzeit $T_1$, strike $K$ und Anfangspreis $S_0$ und $C(T_2)$ der Preis eines Calls auf ein Underlying mit demselben strike und Anfangspreis, aber mit Laufzeit $T_2>T_1$.\\
Angenommen, $C(T_1)>C(T_2)$.\\
Dann gehe am Anfang short im Call mit Laufzeit $T_1$ und long im Call mit Laufzeit $T_2$\\
$\Rightarrow$ risikoloser Gewinn von $C(T_1)-C(T_2)>0$.\\
\uline{in $T_1$:} Ist $S_{T_1}\le K$,wird der Call-Inhaber die Option nicht nutzen. Dann nutzen wir unsere Call-Option in $T_2$ ebenfalls nicht und haben insgesamt einen risikolosen Gewinn.\\
Ist $S_{T_1}>K$, leihen wie uns das Underlying von der Bank um die Verpflichtung des Calls zu erfüllen. Wir erhalten dafür $K$ und gehe damit $\frac{K}{B(T_1,T_2)}\times$long in einen Zero-Bond mit Laufzeit bis $T_2$.\\
\uline{in $T_2$:} Wir erhalten für die Anleihe ($\frac{K}{B(T_1,T_2)}$.\\
Ist $K<S_{T_2}$, nutze die Option und gebe das Underlying an die Bank zurück. Ist $K\ge S_{T_2}$, kaufe das Underlying für $S_{T_2}$ und lasse die Option verfallen, dann gebe das Underlying ebenfalls zurück.\\
In $T_2$ erhalten wir also \[ \frac{K}{B(T_1,T_2)}-\min\{S_{T_2,K}\}\ge \frac{K}{B(T_1,T_2)}-K\ge 0\]
da $B(T_1,T_2)\in (0,1)~\Rightarrow$ risikoloser Gewinn $\lightning_{\text{No-Arbitrage}}$\\
Also $C(T_1)\le C(T_2)$
\hfill $\square$

\ssect{Aufgabe 4}
\bet{Terminzinssatz}\\
Der Kunde zahlt jedes Jahr $K \texteuro$ an die Versicherung, die dafür eine bestimmte, im voraus festgelegte Rendite $R$ zusichert. Erstelle einen geeigneten Sparplan.\\
Annahme: Kunde zahlt immer am Jahresanfang. Die Versicherung muss heute, in $t_0$, $n\cdot K \texteuro$ anlegen um die garantierte Rendite zu gewährleisten.\\
in $t_0$ short in Zero-Bonds: \[ K\cdot B(t_0,1), K\cdot B(t_0,2), \cdot, K\cdot B(t_0,n-1) \]
Also zu jedem $j=1,\dots,n-1$ muss die Versicherung $K\texteuro$ an die Bank zurück zahlen, dies wird gerade durch die jährlichen Prämien der Kunden getilgt.\\
Also hat die Versicherung am Anfang ein Kapital von $K+\sum_{j=1}^{n-1}K\cdot B(t_0,j)$ zur Verfügung. Lege dies in Zero-Bonds an mit Laufzeit $n$ Jahren an: \[ \enbrace{K\cdot \sum_{j=1}^{n-1}K\cdot B(t_0,j)}\times \text{long in n-Zero-Bonds} \]
Daher Auszahlung bei $T=n$: \[ R=\enbrace{K\cdot \enbrace{1+\sum_{j=1}^{n-1}B(t_0,j)}}\cdot \frac{1}{B(t_0,n)} \]
$R$ ist dann die mögliche garantierte Auszahlung.

\sect{Zettel 3}
\ssect{Aufgabe 1}

\cleardoubleoddemptypage
\pagenumbering{Alph}
\setcounter{page}{1}


\printindex
\listoffigures
\end{document}