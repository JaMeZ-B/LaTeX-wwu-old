\section*{Aussagen aus den Übungen}
\addcontentsline{toc}{section}{Aussagen aus den Übungen}
\label{sec:übungenfima}

\subsection*{Zettel 1}
\label{sub:zettel_1fima}
\minisec{Aufgabe 1}
\bet{Bear Spread}:\\ \index{Strategien!Bear Spread}
long put, strike $K_2$, short put, strike $K_1<K_2$
\begin{center}
	\begin{tikzpicture}[line cap=round,line join=round,>=triangle 45,x=.7cm,y=.7cm]
	\draw[->,color=black] (-1.,0.) -- (12.5,0.);
	\foreach \x in {3.,9.}
	\draw[shift={(\x,0)},color=black] (0pt,2pt) -- (0pt,-2pt);
	\draw[->,color=black] (0.,-2.) -- (0.,5.);
	\foreach \y in {}
	\draw[shift={(\y,0)},color=black] (0pt,2pt) -- (0pt,-2pt);
	\draw (12.5, -.1) node[anchor=north west] {$S_T$};
	\draw (.1,5.) node[anchor=north west] {Profit};
	\draw[red] (0.,3.) -- (3.,3.) node [black,midway,below] {Bear Spread};
	\draw[red] (3.,3.) -- (9.,-.5);
	\draw[red] (9.,-.5) -- (12.,-.5);
	\draw[black, style=dashed] (0.,4.3) -- (9.,-1.);
	\draw[black,style=dashed] (9.,-1.) -- (12.,-1.) node [right] {long put};
	\draw[black, style=dashed] (0.,-1.) -- (3.,2.5);
	\draw[black,style=dashed] (3.,2.5) -- (12.,2.5) node [right] {short put};
	\draw[thick,black,decorate,decoration={brace,amplitude=2pt}] (-0.01,0.) -- (-0.01,2.5) node[midway,left]{$p_1$};
	\draw[thick,black,decorate,decoration={brace,amplitude=2pt}] (-0.01,-1.) -- (-0.01,0.) node[midway,left]{$p_2$};
	\draw (3.,0.) node [below] {$K_1$};
	\draw (9.,0.) node [below] {$K_2$};
	\end{tikzpicture}
\end{center}
Profit: $(K_2-S_T)^+ - p_2+p_1-(K_1-S_T)^+=(K_2-K_1)+(p_1-p_2)$

\minisec{Aufgabe 3}
\Index{Exchange-Option}:\\
Analog zur Put-Call-Parität.
% subsection zettel 1 (end)

\subsection*{Zettel 2}
\label{sub:zettel_2fima}

\minisec{Aufgabe 1}
\bet{Eigenschaften des Put-Preises:}\\
\begin{enumerate}[(i)]
	\item innerer Wert: $P(S_0,T,K)\ge max\{0,K\cdot B(0,T)-S_0\}$
	\item obere Grenze: $P(S_0,T,K)<K$
	\item Monoton im strike: $K_1\le K_2\Rightarrow P(S_0,T,K_1)\le P(S_0,T,K_2)$
	\item $B(0,T)(K_2-K_1)\ge P(K_2)-P(K_1)\quad \forall K_1\le K_2$
	\item Konvexität in $K$: 
	\[ P(K_2)\le \lambda P(K_1)+(1-\lambda)P(K_3)\quad \forall K_1<K_2<K_3\text{ mit } \lambda=\frac{K_3-K_2}{K_3-K_1} \]
\end{enumerate}
Beweise analog zum \hyperref[sub:call-preis]{Call-Preis}.

\minisec{Aufgabe 2}
Gelte \uline{No-Arbitrage} und \uline{keine Deflation},d.h. $0<B(0,T)<1$.\\
Dann ist der Call-Preis monoton in der Zeit, also \[ T_1<T_2:~~C(T_1)\le C(T_2) \]
\bet{Beweis:}\\
folgt!

\minisec{Aufgabe 4}
\bet{Terminzinssatz}\\
Der Kunde zahlt jedes Jahr $K \texteuro$ an die Versicherung, die dafür eine bestimmte, im voraus festgelegte Rendite $R$ zusichert. Erstelle einen geeigneten Sparplan.\\
Annahme: Kunde zahlt immer am Jahresanfang. Die Versicherung muss heute, in $t_0$, $n\cdot K \texteuro$ anlegen um die garantierte Rendite zu gewährleisten.\\
in $t_0$ short in Zero-Bonds: \[ K\cdot B(t_0,1), K\cdot B(t_0,2), \cdot, K\cdot B(t_0,n-1) \]
Also zu jedem $j=1,\dots,n-1$ muss die Versicherung $K\texteuro$ an die Bank zurück zahlen, dies wird gerade durch die jährlichen Prämien der Kunden getilgt.\\
Also hat die Versicherung am Anfang ein Kapital von $K+\sum_{j=1}^{n-1}K\cdot B(t_0,j)$ zur Verfügung. Lege dies in Zero-Bonds an mit Laufzeit $n$ Jahren an: \[ \enbrace{K\cdot \sum_{j=1}^{n-1}K\cdot B(t_0,j)}\times \text{long in n-Zero-Bonds} \]
Daher Auszahlung bei $T=n$: \[ R=\enbrace{K\cdot \enbrace{1+\sum_{j=1}^{n-1}B(t_0,j)}}\cdot \frac{1}{B(t_0,n)} \]
$R$ ist dann die mögliche garantierte Auszahlung.
