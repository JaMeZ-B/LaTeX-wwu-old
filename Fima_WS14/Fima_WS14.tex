\documentclass[a4paper, pagesize=pdftex, pdftex, twoside, headsepline, index=totoc,toc=listof, fontsize=10pt, cleardoublepage=empty, headinclude, DIV=13, BCOR=13mm]{scrartcl}

\usepackage[ngerman]{babel}
\usepackage{scrtime} % Bestandteil von KOMA-Skript, ermoeglicht Zugriff auf Uhrzeit des Kompilierens 
\usepackage{scrpage2} % ermöglicht Bearbeiten von Kopf- und Fusszeilen (wie fancyhdr, nur optimiert auf KOMA-Skript, leich andere Syntax)
\usepackage[utf8]{inputenc} % Gibt an in welcher Textcodierung der Code verstandne werden soll
\usepackage{etex} % sehr technisch, ermöglicht LaTeX mehr Speicher zu belegen
\usepackage[T1]{fontenc} % auch sehr technisch; ist wichtig, um die Schriftarten richtig zu behandeln
\usepackage{textcomp} %verhindert ein paar Fehler bei den Fonts
\usepackage{amsmath} % Packet der American Mathematical Society, das viele Mathematik-Umgebungen und -Befehle definiert
\usepackage{amssymb} %zusätzliche Symbole
\usepackage{latexsym} % nochmal zusätzliche Symbole
\usepackage{stmaryrd} % nochmal mehr zusätzliche Symbole, u.a. Blitz für Widerspruchsbeweise ;)
\usepackage{nicefrac} % schräge Brüche, benutzte ich für Quotienvektorräume
\usepackage{paralist} % redefiniert alle Listenbefehle, sodass diese einen optionalen Parameter haben, der die Nummerierung angibt
\usepackage{dsfont} % Schriftart für N,Z,Q,R die ich momentan benutze (mittels \mathds{R} z.B)
\usepackage[pdftex]{graphicx} % Packet, dass das Einbinden von Grafiken aus Dateien ermöglicht
\usepackage{makeidx}% ermöglicht das automatische Anlegen eines Index 
\usepackage{extarrows}
\usepackage{bbold}
\usepackage{mathtools}
%\usepackage{MnSymbol}

\flushbottom
\usepackage[normalem]{ulem}
\setlength{\ULdepth}{1.8pt}

%--Indexverarbeitung
\newcommand{\bet}[1]{\uline{\textbf{#1}}} %Betonung von Text
\newcommand{\Index}[1]{\uline{\textbf{#1}}\index{#1}} % Befehl, der gleichzeitg das Argument hervorhebt und in den Index mitaufnimmt
\makeindex % startet das automatische Sammeln der Index-Einträge
% Ein kleiner Text am Anfang des Index
\setindexpreamble{{\noindent \itshape Die \emph{Seitenzahlen} sind mit Hyperlinks zu den entsprechenden Seiten versehen, also anklickbar!} \par \bigskip}
\renewcommand{\indexpagestyle}{scrheadings} % Seitenstil für den Index festlegen

%--Farbdefinitionen
\usepackage[usenames, table, x11names]{xcolor} %usenames und x11names, aktivieren viele Farben; siehe Dokumentation von xcolor
% Es lassen sich natürlich auch eigene Farben definieren (hier nur Graustufen)
\definecolor{dark_gray}{gray}{0.45}
\definecolor{light_gray}{gray}{0.7}

%--Zum Zeichnen (ich habe es jetzt mal mit aufgenommen, aber es ist eigentlich nochmal ein ganz anderes Thema, sodass ich da jetzt nicht viel zu sagen werde)
\usepackage{tikz} % TikZ steht übrigens für "TikZ ist kein Zeichenprogramm", ein rekursives Akronym ...
\tikzset{>=latex}
\usetikzlibrary{shapes,arrows}
\usetikzlibrary{calc}
\usetikzlibrary{decorations.pathreplacing}
% Hiermit kann man ganz leicht kommutative Diagramme zeichnen (deswegen auch "cd")
\usepackage{tikz-cd}

%--Marginnote, ermöglicht es kleine Notizen an neben den eigentlichen Textkörper zu setzten
\usepackage{marginnote}
\renewcommand*{\marginfont}{\color{Honeydew4} \footnotesize }

%--Schriftarten
\usepackage{lmodern} % neuere Version der Standard-LaTeX-Schriftarten
\renewcommand{\familydefault}{\sfdefault} %Standardschriftart auf die serifenlose Schriftart setzen

%--Hyperref; aktiviert Hyperlinks in der erzeugten PDF-Datei und definiert deren Aussehen
\usepackage[colorlinks, pdfpagelabels, pdfstartview=FitH, bookmarksopen=true, bookmarksnumbered=true,linkcolor=black,urlcolor=SkyBlue2, plainpages=false, hypertexnames=false, citecolor=black, hypertexnames=true]{hyperref}

%--Römische Zahlen
\newcommand{\RM}[1]{\MakeUppercase{\romannumeral #1{}}}



%-- Definitionen von weiteren Mathe-Befehlen, die dann das "richtige" Aussehen haben. Hier sind der Phantasie keine Grenzen gesetzt
\DeclareMathOperator{\id}{id} %identische Abbildung
\DeclareMathOperator{\End}{End} %Endomorphismen
\DeclareMathOperator{\rg}{rg} %Rang
\DeclareMathOperator{\diam}{diam} %Durchmesser
\DeclareMathOperator{\dist}{dist} %Distanz
\DeclareMathOperator{\grad}{grad} %Gradient
\DeclareMathOperator{\rot}{rot} %Rotation
\DeclareMathOperator{\hess}{Hess} %Hesse-Matrix
\DeclareMathOperator{\supp}{supp}
\DeclareMathOperator{\aut}{Aut}
\DeclareMathOperator{\inn}{Inn}
\DeclareMathOperator{\sym}{Sym}
\DeclareMathOperator{\syl}{Syl}

%--Skalarprodukt (cooler Befehl, den ich im Internet gefunden habe; benutzt TeX-Befehle)
\makeatletter
\newcommand{\sprod}[2]{\ensuremath{%
  \setbox0=\hbox{\ensuremath{#2}}
  \dimen@\ht0
  \advance\dimen@ by \dp0
  \left\langle \left.#1 \,\rule[-\dp0]{0pt}{\dimen@}\right|#2\right\rangle}}
\makeatother

%--Norm (auch aus dem Internet, wird auch auf der Beispielseite verwandt)
\newcommand{\norm}[2][\relax]{
\ifx#1\relax \ensuremath{\left\Vert#2\right\Vert}
\else \ensuremath{\left\Vert#2\right\Vert_{#1}}
\fi}


%--selbstgeschriebenen Befehle
%--Betrag
\newcommand{\abs}[1]{\ensuremath{\left\vert#1\right\vert}}

%--Umklammern mit passender Größe der Klammern
\newcommand{\enbrace}[1]{\ensuremath{\left( #1\right)}}

%--Mengen
\newcommand{\penbrace}[1]{\ensuremath{\left\{#1\right\}}}

%--Differential
\newcommand{\diff}[2]{\ensuremath{\frac{\partial #1}{\partial #2} }}

\newcommand{\zz}{$\mathrm{Z\kern-.3em\raise-0.5ex\hbox{Z}}$} % zu zeigen ZZ aus dem inet
\setlength{\parindent}{0pt}%absatz nicht einrücken
\newcommand{\lh}[1]{\langle #1 \rangle} %lineare Hülle
\newcommand{\nt}{\trianglelefteqslant} %normalteiler
\newcommand{\pfs}{\mathds{P}-\text{f.s.}} %P-f.s. konvergenz
\newcommand{\dint}{\mathrm{d}} % d des integrals

\newcommand{\xfrac}[2]{%
	\mbox{\raisebox{-0.4ex}{\ensuremath{\displaystyle #1}\hspace{0.2ex}}%
		{\raisebox{-0.1ex}{\big \backslash}}%
		\raisebox{0.6ex}{\ensuremath{\displaystyle #2}}%
	}%
}
\newcommand{\Pw}{\mathds{P}}
\newcommand{\E}{\mathds{E}}
\newcommand{\R}{\mathds{R}}
\newcommand{\N}{\mathds{N}}
\newcommand{\Z}{\mathds{Z}}


\newcommand{\sect}[1]{\section*{#1}\addcontentsline{toc}{section}{#1}}
\newcommand{\ssect}[1]{\subsection*{#1}\addcontentsline{toc}{subsection}{#1}}

\newcommand{\vorlesung}{Finanzmathematik}
\newcommand{\Prof}{PD Dr. Paulsen}
\newcommand{\subt}{Mitschrift der Tafelnotizen}

\input{extra_files/headings.tex}



\begin{document}
\maketitle
\thispagestyle{empty}
\cleardoubleoddemptypage

\thispagestyle{empty}
\vspace*{\fill}
\begin{center}
	Hierbei handelt es sich um eine \subt von \textbf{\Prof}, WWU Münster, aus der Vorlesung \textbf{\vorlesung} im Wintersemester 2014/15. 
	Dies ist kein Skript der Vorlesung und keine eigene Arbeit des Autors.\\
	\vspace{2cm}
	Für Fehler in der Mitschrift wird keine Haftung übernommen. 
	Hinweise auf Fehler sind gerne gesehen, hierfür kann man mich in der Uni ansprechen oder alternativ eine e-Mail an: \textit{tobias.wedemeier@gmx.de}\\
	Auch ist eine Mitarbeit über Github möglich.\\
	\vspace{2cm}
	Wenn Teile aus der Vorlesung selber fehlen, können diese gerne an meine e-Mail versandt werden. 
	Ich werde diese dann einarbeiten.\\
\end{center}
\vspace*{\fill}
\newpage

\pagenumbering{Roman}

\tableofcontents
\cleardoubleoddemptypage %sorgt dafür, dass alles folgende erst auf der nächsten freien "rechten" Seite steht

\thispagestyle{empty}

\section*{Prolog}  % (fold)
\addcontentsline{toc}{section}{Prolog}
\label{sec:prolog}

\subsection*{Ziel} % (fold)
\addcontentsline{toc}{subsection}{Ziel}
\label{sub:ziel}

\begin{itemize}
	\item Bewertung von Finanzderivaten, dies entspricht der Bewertung von Finanzmarktrisiken
	\item aktuarielle Bewertung von Risiken, biometrische Risiken (Rente,$\dots$) $\leftrightarrow$ Personenversicherungen,
	sonstige Risiken (Unfall, $\dots$ ) $\leftrightarrow$ Schadenversicherungen
\end{itemize}
% subsection ziel (end)

\subsection*{Schlagwörter} % (fold)
\addcontentsline{toc}{subsection}{Schlagwörter}
\label{sub: schlagwörter}

\begin{itemize}
	\item Black-Scholes Formel
	\item äqivalentes Martingalmaß
	\item Hedging, Replizieren durch Handel
	\item Arbitage
	\item Äquivalenzprinzip
	\item Risikoausgleich im Kollektier
\end{itemize}
% subsection schlagwörter (end)

\subsection*{Hilfsmittel} % (fold)
\addcontentsline{toc}{subsection}{Hilfsmittel}
\label{sub:hilfsmittel}

Theorie der stochastischen Prozesse
\begin{itemize}
	\item mathem. Modellierung von zeitlich abhängigen Zufallsphänomenen
	\item notwendig zur Beschreibung von Finanzmärkten
\end{itemize}
% subsection hilfsmittel (end)

\subsection*{Themen} % (fold)
\addcontentsline{toc}{subsection}{Themen}
\label{sub: themen}

\begin{itemize}
	\item diskrete und kontinuierliche Martingaltheorie
	\item diskrete und kontinuierliche Markov-Prozesse
	\item Wiener-Prozess, Brownsche Bewegung
	\item geometrische Brownsche Bewegung als Modell für Aktienkurse
\end{itemize}

% subsection themen (end)
%section prolog (end)

\newpage

\pagenumbering{arabic}
\setcounter{page}{1}


\section{Informelle Einführung} % (fold)
\label{sec: informelle_einführung}

\begin{enumerate}[(i)]
	\item Zweiteilung von Finanzgütern in:
	\begin{enumerate}[(1)]
		\item Basisfinanzgüter
		\item derivative Finanzgüter
	\end{enumerate}
	\item zu (1) gehören:
	\begin{itemize}
		\item Aktien
		\item festverzinsliche Wertpapiere, Bonds
		\item Rohstoffe, Agrarprodukte
	\end{itemize}
diese werden gehandelt auf:
	\begin{itemize}
		\item Aktienmärkte
		\item Rentenmärkte
		\item Warenmärkte
	\end{itemize}
Diese werden als Kassamärkte bezeichnet.

	\item zu (2) gehören:
	\begin{itemize}
		\item Optionen auf Aktien
		\item Swaps (Zinsderivate)
		\item futures und forwards
	\end{itemize}

\end{enumerate}

\subsection{Option} % (fold)
\label{sub: option}
Unterscheidung in Kauf- und Verkaufoptionen
\begin{itemize}
	\item Eine Kaufoption (\Index{Call}) gibt das Recht ein Basisfinanzgut (\Index{Underlying}), zu einem im Voraus bestimmten fixen Preis,
	dem Ausübungspreis (\Index{strike}, Basis), während (\Index{amerikanische Option}) oder nur am Ende der Laufzeit der Option (\Index{europäische Option}) zu kaufen.
	\item  Eine Verkaufoption (\Index{Put}) gibt das Recht ein Basisfinanzgut (Underlying), zu einem im Voraus bestimmten fixen Preis, dem Ausübungspreis (strike, Basis), während (amerikanische Option) oder nur am Ende der Laufzeit der Option (europäische Option) zu verkaufen.
\end{itemize}
Dies sind \bet{unbestimmte Termingeschäfte} \index{Termingeschäft!unbestimmtes}, da keinerlei Verpflichtung zum Kauf bzw. Verkauf besteht.

% subsection option (end)

\subsection{long, short} % (fold)
\label{sub: long,_short}

In der Regel nimmt der Käufer eines Finanzgutes eine \bet{long-Position} ein, der Verkäufer eine \bet{short-Position}.\index{Position!long}\index{Position!short}
Der Verkäufer wird auch als writer (Zeichner) bezeichnet, da er die Option 'zeichnet'. 
Man kann zu jeder Zeit eine long oder short Position eingehen, insbesondere auch wenn man die Aktie gar nicht besitzt. 
Dies wird auch als \Index{Leerverkauf} (short selling) bezeichnet, hierbei leiht man sich die Aktie von der Bank um sie zu verkaufen.

\vfill

% subsection long, short (end)

\subsection{Payoff und Profit Diagramme} % (fold)
\label{sub:payoff_und_profit_diagramme}

\begin{itemize}
	\item Positionen in Finanzgütern bergen Chancen und Risiken.
	\item \Index{Payoff}: Wert der Position wird gegen den Preis des Underlyings aufgetragen
	\item \Index{Profit}: analog zum Payoff, unter Berücksichtigung von Kosten (Anfangswert der Postion)
	\item Beispiele: Option mit Laufzeit $T \in \mathds{N}$, Underlying mit Preis $S_T$ in $T$

\begin{enumerate}[(a)]
	\item long call: strike $K$ \\
		Payoff: $(S_T - K)^+$ \\
		$S_T \le K$ keine Ausübung, $S_T > K$ Ausübung der Option (Ablauf: leihe Geld, kaufe Aktie, verkaufe Aktie, zahle Geld zurück) \\
		
		\begin{center}
			\begin{tikzpicture}[line cap=round,line join=round,>=triangle 45,x=1.0cm,y=1.0cm]
			\draw[->,color=black] (-0.1,0.) -- (7.2,0.);
			\draw[->,color=black] (0.,-1.) -- (0.,3.);
			\draw [color=red] (3.,0.)-- (5.56,2.22);
			\draw [color=red] (0.,0.)-- (3.,0.);
			\draw (6.5,-0.04) node[anchor=north west] {$S_T$};
			\draw (0.1,3.) node[anchor=north west] {Profit};
			\draw [fill=red] (3.,0.) circle (1.5pt);
			\draw[color=red] (3.,-0.4) node {$K$};
			\end{tikzpicture}
			\captionof{figure}{Payoff long call}
		\end{center}
		Kosten: Anfangspreis des Calls $c>0$.
		Profit: $(S_T - K)^+ -c$ \\
		
		\begin{center}
		\begin{tikzpicture}[line cap=round,line join=round,>=triangle 45,x=1.0cm,y=1.0cm]
		\draw[->,color=black] (-0.1,0.) -- (7.2,0.);
		\draw[shift={(3,0)},color=black] (0pt,2pt) -- (0pt,-2pt);
		\draw[->,color=black] (0.,-1.) -- (0.,3.);
		\draw [color=red] (3.,-0.5)-- (5.56,2.22);
		\draw [color=red] (0.,-0.5)-- (3.,-0.5);
		\draw (6.5,-0.04) node[anchor=north west] {$S_T$};
		\draw (0.1,3.) node[anchor=north west] {Profit};
		\draw[thick,black,decorate,decoration={brace,amplitude=2pt}] (-0.01,-0.5) -- (-0.01,0.) node[midway,left]{$c$};
		\draw [fill=red] (3.,-0.5) circle (1.5pt);
		\draw[color=red] (3.,-1.) node {$K$};
		\end{tikzpicture}
		\captionof{figure}{Profit long call}
	\end{center}
	\newpage
	\item long put: strike $K$ \\
		Payoff: $(K - S_T)^+$ \\
		$S_T > K$ keine Ausübung, $S_T \le K$ Ausübung der Option (Ablauf: leihe Aktie, verkaufe Aktie, kaufe Aktie, gebe Aktie zurück) \\
		\begin{center}
			\begin{tikzpicture}[line cap=round,line join=round,>=triangle 45,x=1.0cm,y=1.0cm]
			\draw[->,color=black] (-0.1,0.) -- (7.2,0.);
			\draw[->,color=black] (0.,-0.5) -- (0.,3.);
			\draw [color=red] (0.,2.)-- (3.,0.);
			\draw [color=red] (3.,0.)-- (7.,0.);
			\draw (6.5,-0.04) node[anchor=north west] {$S_T$};
			\draw (0.1,3.) node[anchor=north west] {Profit};
			\draw [fill=red] (3.,0.) circle (1.5pt);
			\draw[color=red] (3.,-0.4) node {$K$};
			\end{tikzpicture}
			\captionof{figure}{Payoff long put}
		\end{center}
		Kosten: Anfangspreis de Option $p>0$.
		Profit: $(K - S_T)^+ - p$ \\
		\begin{center}
			\begin{tikzpicture}[line cap=round,line join=round,>=triangle 45,x=1.0cm,y=1.0cm]
			\draw[->,color=black] (-0.1,0.) -- (7.2,0.);
			\draw[shift={(3,0)},color=black] (0pt,1pt) -- (0pt,-1pt);
			\draw[->,color=black] (0.,-1.) -- (0.,3.);
			\draw [color=red] (0.,2.)-- (3.,-0.5);
			\draw [color=red] (3.,-0.5)-- (7.,-0.5);
			\draw (6.5,-0.04) node[anchor=north west] {$S_T$};
			\draw (0.1,3.) node[anchor=north west] {Profit};
			\draw [fill=red] (3.,-0.5) circle (1.5pt);
			\draw[color=red] (3.,-1) node {$K$};
			\draw[thick,black,decorate,decoration={brace,amplitude=2pt}] (-0.01,-0.5) -- (-0.01,0.) node[midway,left]{$p$};
			\end{tikzpicture}
			\captionof{figure}{Profit long put}
		\end{center}
		
	\item short call: \\
		Payoff: $- (S_T -K)^+$, Profit: $c - (S_T - K)^+$ \\
		
		\begin{minipage}[b]{5cm}
		\begin{tikzpicture}[line cap=round,line join=round,>=triangle 45,x=0.8cm,y=0.8cm]
			\draw[->,color=black] (-0.1,0.) -- (7.2,0.);
			\draw[shift={(3,0)},color=black] (0pt,1pt) -- (0pt,-1pt);
			\draw[->,color=black] (0.,-3.) -- (0.,1.);
			\draw [color=red] (0.,0.)-- (3.,0.);
			\draw [color=red] (3.,0.)-- (7.,-2.5);
			\draw (6.5,-0.04) node[anchor=north west] {$S_T$};
			\draw (0.1,1.3) node[anchor=north west] {Profit};
			\draw [fill=red] (3.,0.) circle (1.5pt);
			\draw[color=red] (3.,-0.5) node {$K$};
		\end{tikzpicture}
		\captionof{figure}{Payoff short call}
		\end{minipage}
		\hfill
		\begin{minipage}[b]{5cm}
		\begin{tikzpicture}[line cap=round,line join=round,>=triangle 45,x=0.8cm,y=0.8cm]
			\draw[->,color=black] (-0.1,0.) -- (7.2,0.);
			\draw[shift={(3,0)},color=black] (0pt,1pt) -- (0pt,-1pt);
			\draw[->,color=black] (0.,-3.) -- (0.,1.);
			\draw [color=red] (0.,0.5)-- (3.,0.5);
			\draw [color=red] (3.,0.5)-- (7.,-2.5);
			\draw (6.5,-0.04) node[anchor=north west] {$S_T$};
			\draw (0.1,1.3) node[anchor=north west] {Profit};
			\draw [fill=red] (3.,0.5) circle (1.5pt);
			\draw[color=red] (3.,1.) node {$K$};
			\draw[thick,black,decorate,decoration={brace,amplitude=2pt}] (-0.01,0.) -- (-0.01,0.5) node[midway,left]{$c$};
		\end{tikzpicture}
		\captionof{figure}{Profit short call}
		\end{minipage}
		\vfill
	\item short put: \\
		Payoff: $-(K - S_T)^+$, Profit: $p-(K - S_T)^+$ \\
		
		\begin{minipage}[b]{5cm}
			\begin{tikzpicture}[line cap=round,line join=round,>=triangle 45,x=0.8cm,y=0.8cm]
			\draw[->,color=black] (-0.1,0.) -- (7.2,0.);
			\draw[shift={(3,0)},color=black] (0pt,1pt) -- (0pt,-1pt);
			\draw[->,color=black] (0.,-3.) -- (0.,1.);
			\draw [color=red] (3.,0.)-- (7.,0.);
			\draw [color=red] (0.,-2.5)-- (3.,0.);
			\draw (6.5,-0.04) node[anchor=north west] {$S_T$};
			\draw (0.1,1.3) node[anchor=north west] {Profit};
			\draw [fill=red] (3.,0.) circle (1.5pt);
			\draw[color=red] (3.,0.5) node {$K$};
			\end{tikzpicture}
			\captionof{figure}{Payoff short put}
		\end{minipage}
		\hfill
		\begin{minipage}[b]{5cm}
			\begin{tikzpicture}[line cap=round,line join=round,>=triangle 45,x=0.8cm,y=0.8cm]
			\draw[->,color=black] (-0.1,0.) -- (7.2,0.);
			\draw[shift={(3,0)},color=black] (0pt,1pt) -- (0pt,-1pt);
			\draw[->,color=black] (0.,-3.) -- (0.,1.);
			\draw [color=red] (3.,0.5)-- (7.,0.5);
			\draw [color=red] (0.,-2.5)-- (3.,0.5);
			\draw (6.5,-0.04) node[anchor=north west] {$S_T$};
			\draw (0.1,1.3) node[anchor=north west] {Profit};
			\draw [fill=red] (3.,0.5) circle (1.5pt);
			\draw[color=red] (3.,1.) node {$K$};
			\draw[thick,black,decorate,decoration={brace,amplitude=2pt}] (-0.01,0.) -- (-0.01,0.5) node[midway,left]{$p$};
			\end{tikzpicture}
			\captionof{figure}{Profit short put}
		\end{minipage}
\end{enumerate}
\end{itemize}
% subsection payoff_und_profit_diagramme (end)

\subsection{Strategien}
\label{sub:strategien}

Durch Kombination von einfachen Positionen bildet man \Index{Strategien}.

\minisec{Beispiel}
\begin{itemize}
	\item Absicherung einer Aktie:
	\begin{itemize}
		\item Aktie zum heutigen Kurs kaufen mit strike $K$
		\item zur Absicherung gegen Kursverlust in $T$ wird eine Putoption zum strike $K$ gekauft
	\end{itemize}
	\item Gesamtposition: \\
	\begin{tabular}{r | c c c}
		& long Aktie & long put & Gesamt \\
		\hline
		Kosten & $K$ & $p$ & $K+p$ \\
		Payoff & $S_T$ & $(K-S_T)^+$ & $S_T+(K-S_T)^+ = \max\{K, S_T\}$ \\
	\end{tabular}
	\item Profit: \\
	\begin{equation*}
	\begin{aligned}
		S_T + (K-S_T)^+ -(K+p) = (S_T-K) + (K - S_T)^+ - p = -p\mathbb{1}_{\{S_T \le K\}} + (S_T- (K+p))\mathbb{1}_{\{S_T>K\}}
	\end{aligned}
	\end{equation*}
	

	\begin{center}
		\begin{tikzpicture}[line cap=round,line join=round,>=triangle 45,x=1.0cm,y=1.0cm]
		\draw[->,color=black] (-1.,0.) -- (7.2,0.);
		\draw[shift={(3,0)},color=black] (0pt,2pt) -- (0pt,-2pt);
		\draw[->,color=black] (0.,-1.) -- (0.,3.);
		\draw [color=red] (3.,-0.5)-- (5.56,2.22);
		\draw [color=red] (0.,-0.5)-- (3.,-0.5);
		\draw (6.5,-0.04) node[anchor=north west] {$S_T$};
		\draw (0.1,3.) node[anchor=north west] {Profit};
		\draw[thick,black,decorate,decoration={brace,amplitude=2pt}] (-0.01,-0.5) -- (-0.01,0.) node[midway,left]{$p$};
		\draw [fill=red] (3.,-0.5) circle (1.5pt);
		\draw[color=red] (3.,-1.) node {$K$};
		
		\end{tikzpicture}
		\captionof{figure}{Bsp. Profit Diagramm}
	\end{center}
\vfill
\end{itemize}
\minisec{long straddle} \index{Strategien!long straddle}
\begin{itemize}
	\item Idee: Spekulation auf eine starke Kursänderung \\
	
	\begin{tabular}{r | c c c}
		& long call & long put & Gesamt \\
		\hline
		Kosten & $c$ & $p$ & $c+p$ \\
		Payoff & $(S_T-K)^+$ & $(K-S_T)^+$ & \abs{S_T-K} \\
	\end{tabular}\\
	
	Profit: $\abs{S_T-K} - (c+p)$ \\

	\begin{center}
		\begin{tikzpicture}[line cap=round,line join=round,>=triangle 45,x=1.0cm,y=1.0cm]
		\draw[->,color=black] (-1.,0.) -- (7.2,0.);
		\draw[shift={(3,0)},color=black] (0pt,2pt) -- (0pt,-2pt);
		\draw[->,color=black] (0.,-1.) -- (0.,3.);
		\draw [color=red] (3.,-0.5)-- (5.56,2.22);
		\draw [color=red] (0.,2.22)-- (3.,-0.5);
		\draw (6.5,-0.04) node[anchor=north west] {$S_T$};
		\draw (0.1,3.) node[anchor=north west] {Profit};
		\draw[thick,black,decorate,decoration={brace,amplitude=2pt}] (-0.01,-0.5) -- (-0.01,0.) node[midway,left]{$c+p$};
		\draw [fill=red] (3.,-0.5) circle (1.5pt);
		\draw[color=red] (3.,-1.) node {$K$};
		
		\end{tikzpicture}
		\captionof{figure}{long straddle}
	\end{center}
\end{itemize}

\minisec{Bullish Vertical Spread}\index{Strategien!Bullish Vertical Spread}
Idee: Risikoarme Spekulation auf ein Anziehen des Kurses \\

\begin{tabular}{r | c c c}
	& long call & short call & Gesamt \\
	& mit strike $K_1$ & mit strike $K_2 > K_1$ & \\
	\hline
	Kosten & $c_1$ & $-c_2$ & $c_1-c_2 >0$ \\
	Payoff & $(S_T-K_1)^+$ & $-(S_T -K_2)^+$ & $(S_T-K_1)\mathbb{1}_{\{K_1<S_T<K_2\}}+(K_2-K_1)\mathbb{1}_{\{S_T>K_2\}}$ \\
\end{tabular}
\marginnote{Je kleiner der strike, desto teuerer ist der call.}
\begin{center}
	\begin{tikzpicture}[line cap=round,line join=round,>=triangle 45,x=1.0cm,y=1.0cm]
	\draw[->,color=black] (-0.1,0.) -- (7.2,0.);
	\foreach \x in {1.5,3.5}
	\draw[shift={(\x,0)},color=black] (0pt,2pt) -- (0pt,-2pt);
	\draw[->,color=black] (0.,-1.) -- (0.,3.);
	\draw [color=red] (1.5,-0.5)-- (3.5,2.);
	\draw [color=red] (0.,-0.5)-- (1.5,-0.5);
	\draw [color=red] (3.5,2.)-- (5.,2.);
	\draw (6.5,-0.04) node[anchor=north west] {$S_T$};
	\draw (0.1,3.) node[anchor=north west] {Profit};
	\draw[thick,black,decorate,decoration={brace,amplitude=5pt}] (5.,2.) -- (5.,0.);
	\draw[thick,black,decorate,decoration={brace,amplitude=2pt}] (-0.01,-0.5) -- (-0.01,0.) node[midway,left]{$c_1-c_2$};
	\draw[color=black] (6.1,1.2) node {$K_1-K_2$};
	\draw[color=black] (6.1,0.8) node {$-(c_1-c_2)$};
	\draw [fill=red] (1.5,-0.5) circle (1.5pt);
	\draw[color=red] (1.5,-1.) node {$K_1$};
	\draw[fill=red] (3.5,2.) circle (1.5pt);
	\draw[color=red] (3.5,2.5) node {$K_2$};
	\end{tikzpicture}
	\captionof{figure}{Bullish Vertical Spread}
	
\end{center}

\minisec{Butterfly Spread} \index{Strategien!Butterfly Spread}
Idee: Risikoarme Spekulation auf eine Seitwärtsbewegung des Kurses \\
strike: $K_1<K_2<K_3$ \\

\begin{tabular}{r | c c c c}
	& long call & long call & 2$\times$ short call \\
	& strike $K_1$ & strike $K_3$ & strike $K_2$ \\
	\hline
	Kosten & $c_1$ & $c_3$ & $-2c_2$ & $c_1+c_3-2c_2$ \\
\end{tabular}\\
Payoff: $(S_T-K_1)\mathbb{1}_{\{K_1<S_T<K_2\}}+(2K_2-K_1-S_T)\mathbb{1}_{\{K_2<S_T<K_3\}}+2K_2-(K_1+K_3)\mathbb{1}_{\{S_T>K_3\}}$ \\
Falls $K_2 = \frac{1}{2}(K_1+K_3) \Rightarrow c_1+c_3-2c_2 > 0$ \\
\begin{center}
	\begin{tikzpicture}[line cap=round,line join=round,>=triangle 45,x=1.0cm,y=1.0cm]
	\draw[->,color=black] (-0.1,0.) -- (7.2,0.);
	\foreach \x in {1.,2.25,3.5}
	\draw[shift={(\x,0)},color=black] (0pt,2pt) -- (0pt,-2pt);
	\draw[->,color=black] (0.,-1.) -- (0.,3.);
	\draw [color=red] (1.,-0.5)-- (2.25,2.25);
	\draw [color=red] (0.,-0.5)-- (1.,-0.5);
	\draw [color=red] (2.25,2.25)-- (3.5,-0.5);
	\draw [color=red] (3.5,-0.5)-- (7.,-0.5);
	\draw (6.5,-0.04) node[anchor=north west] {$S_T$};
	\draw (0.1,3.) node[anchor=north west] {Profit};
	\draw [fill=red] (1.,-0.5) circle (1.5pt);
	\draw[color=red] (1.,-1.) node {$K_1$};
	\draw[fill=red] (2.25,2.25) circle (1.5pt);
	\draw[color=red] (2.25,2.55) node {$K_2$};
	\draw[fill=red] (3.5,-0.5) circle (1.5pt);
	\draw[color=red] (3.5,-1) node {$K_3$};
	\end{tikzpicture}
	\captionof{figure}{long Butterfly Spread}
\end{center}
Für weitere Strategien klicken Sie \href{http://de.wikipedia.org/wiki/Optionsstrategie}{hier}.
% subsection strategien (end)

\subsection{Arbitrage} % (fold)
\label{sub:arbitrage}
\begin{itemize}
	\item Ein \Index{Arbitrage} ist eine Möglichkeit durch Handel mit Finanzgütern einen risikolosen Profit zu erzielen.
	\item \textbf{Beispiel} \\
	\begin{tabular}{r | c c}
		& New York & Frankfurt \\
		\hline
		Aktie & 130 \$ & 100 \texteuro \\
		Wechselkurs & \multicolumn{2}{c}{1,27 \$ $\mathrel{\hat=}$ 1 \texteuro } \\
	\end{tabular}
	\item Arbitragemöglichkeit: \\
	leihe 100 \texteuro $~\rightsquigarrow$ kaufe Aktie in Frankfurt $\rightsquigarrow$ verkaufe Aktie in New York\\ $\rightsquigarrow$ tausche 127 \$ in 100 \texteuro  $~\rightsquigarrow$ 100 \texteuro~  zurück zahlen $\rightsquigarrow$ risikolosen Profit von 3 \$
	\item Grundannahme: \\
	Im Handel mit Finanzgütern gibt es keine Arbitragen. Dies ist das sogenannte \bet{No-Arbitrage Prinzip}. \index{Arbitrage!No-}
	\item Aus dem No-Arbitrage Prinzip kann das \Index{Replikationsprinzip} gefolgert werden.	
\end{itemize}
% subsection arbitrage (end)

\subsection{Replikationsprinzip}
\label{sub: replikationsprinzip}
Haben zwei verschiedene Kombinationen $K,L$ von ausschüttungsfreien Finanzgütern zu einem zukünftigen Zeitpunkt $T \in \mathds{R}$ immer den gleichen Wert, so haben sie auch zum gegenwärtigen Zeitpunkt den gleichen Wert. \\
Die Kombination $K$ repliziert den Payoff der Kombination $L$, und umgekehrt.\\
\textbf{Argumentation:}\\
$K$,$L$ habe den Anfangswert $V_0,W_0 \in \mathds{R}$ und den zufälligen Wert $V_T,W_T \in \mathds{R}$ in $T$. \\
Es gelte: $V_T = W_T$: \\
\underline{Beh.:} $V_0 = W_0$ \\

\underline{$\mathds{A}$} \\
1.Fall: $V_0>W_0$. \\
Dann kann durch short selling von $K$ ein Arbitrage erzielt werden:
\begin{itemize}
	\item short selling in $K$
	\item gehe long in $L$
	\item[$\Rightarrow$ am Anfang Gewinn $V_0-W_0>0$]
	\item handeln entsprechend $L$ bis $T$
	\item[in $T$:]
	\item verkaufe $L$, erhalte $W_T = V_T$ 
	\item kaufe $K$ für $V_T$ und gebe die Position $K$ zurück
\end{itemize}
Am Ende: Glattstellen der Positionen $W_T - V_T = 0$ \textbf{$\lightning$} \\

2.Fall: $W_0>V_0$. Analog.
\hfill $\square$

% subsection replikationsprinzip (end)

\subsection{Nullkouponanleihe}
\label{sub: nullkouponanleihe}
\index{Anleihe!Nullkoupon-}
\begin{itemize}
	\item[festverzinsliches Wertpapier:]
	\item Fälligkeit $T$ (Maturity)
	\item Zahlung von 1 Euro
	\item keine Kouponzahlung während der Laufzeit
\end{itemize}
$B(t,T)$ bezeichne den Preis dieser Anleihe zum Zeitpunkt $t<T$. $0<B(t,T)<1$ ist der Regelfall.

% subsection nullkouponanleihe (end)

\subsection{Put-Call Parität}
\label{sub: put-call_parität}
Seien $c,p$ die Anfangspreise einer Call- bzw. Putoption mit Laufzeit $T$ und strike $K$.\\
Sei $S_0$ und $S_T$ die Preise des Underlyings heute und in $T$. \\
Dann gilt: 
\[
S_0 + p = c + K\cdot B(0,T)
\]

\textbf{Argumentation:} \\
Betrachte folgende Kombinationen:\\
I: long Aktie, long put\\
II: long call, $K \cdot $ long in eine Nullkouponanleihe mit Fälligkeit in $T$\\

Wert zum Zeitpunkt $T$: \\
I: $S_T + (K-S_T)^+ = \max\{S_T,K\}$ \\
II: $(S_T - K)^+ + K = \max\{S_T,K\}$ \\

Replikationsprinzip liefert:
\[
S_0 + p = c + K \cdot B(0,T)
\]
\hfill $\square$

% subsection put-call parität (end)

\subsection{forward} %fold
\label{sub:forward}

Ein \Index{forward} ist ein unbedingtes Termingeschäft mit Ausübungszeitpunkt $T$ (Maturity), für ein Underlying mit Preisen $S_0$ heute und $S_T$ in $T$. 
Zwei Parteien A und B, mit festem Terminpreis $F_T$ zum Vertragsabschluss. 
In $T$: A zahlt an B den Terminpreis $F_T$, B liefert das Underlying
\marginnote{zum Beispiel bei Agrargütern}

A hat die long-Position im forward, B die short-Position. Zusammenhang zwischen Termin- und Spotpreis des Underlyings.\\
\hspace{2cm} $S_0$ - gegenwärtiger Preis, \Index{Spotpreis}\\
\hspace{2cm} $F_T$ - Terminpreis zum Termin $T$\\
Dann gilt:
\[ 
F_T \cdot B(0,T) = S_0
\]
\bet{Argumentation:}\\
Betrachte folgende Kombinationen:\\
I: long im forward zum Zeitpunkt $T$, $F_T \times$ long in einer Nullkouponanleihe mit Fälligkeit $T$\\
II: long im Underlying\\
Wert zum Zeitpunkt $T$:
I: $\underbracket{S_T - F_T}_{\text{forward}} + \underbracket{F_T}_{\text{Nullkouponanleihe}} = S_T$\\
II: $S_T$\\
Replikationsprinzip liefert:
\[
F_T \cdot B(0,T) = S_0 
\]

%subsection forward (end)

\subsection{Digitale Position}
\label{sub: digitale_position}
\index{Digitale Position}
Recht auf Auszahlung eines festen Geldbetrags (etwa 1 \texteuro) bei Eintreten eines auslösenden Ereignisses (\bet{bedingtes Termingeschäft})\index{Termingeschäft!bedingtes}.\\
\uline{z.B.}\\
\begin{tabular}{c c}
	digitaler call & digtialer put\\
	$\mathbb{1}_{\{S_T\ge K\}}$ & $\mathbb{1}_{\{S_T\le K \}}$ \\
\end{tabular}

% subsection digitale position (end)

\subsection{Eigenschaften des Call-Preises}
\label{sub:call-preis}
Sei $C(S_0,T,K)$ der Preis eines Calls auf ein Underlying mit Laufzeit $T$, strike $K$ und Anfangspreis $S_0$.\\
Dann gilt:\\
\begin{enumerate}[(i)]
	\item $C(S_0,T,K) \ge \max\{0, S_0 - K\cdot B(0,T)\}$ \Index{innerer Wert} des Calls
	\item $C(S_0,T,K) \le S_0$ \Index{obere Grenze} des Calls
	\item $K_1 \le K_2 \Rightarrow C(S_0,T,K_1) \ge C(S_0,T,K_2)$
	\item $B(0,T)(K_2-K_1) \ge C(S_0,T,K_1)- C(S_0,T,K_2)~\forall K_1<K_2$
	\item $C(S_0,T,K_2) \le \frac{K_3-K_2}{K_3-K_1}\cdot C(S_0,T,K_1)+ \frac{K_2-K_1}{K_3-K_1} \cdot C(S_0,T,K_3)~\forall K_1<K_2<K_3$
		\Index{Konvexität in $K$}
\end{enumerate}
\begin{center}
	\begin{tikzpicture}[line cap=round,line join=round,>=triangle 45,x=.8cm,y=.8cm]
	\draw[->,color=black] (-0.1,0.) -- (7.2,0.);
	\draw[->,color=black] (0.,-1.) -- (0.,3.);
	\draw [color=red] (0.,2.)-- (3.5,0.);
	\draw[color=black, domain=0:5] plot (\x , {2*exp(-0.3*\x)}) node[right] {Preis des Underlyings};
	\draw (6.5,-0.04) node[anchor=north west] {$K$};
	\draw (0.1,3.) node[anchor=north west] {$C(K)$};
	\draw [fill=red] (3.5,0.) circle (1.5pt);
	\draw[color=black] (3.5,-0.4) node {$\frac{S_0}{B(0,T)}$};
	\end{tikzpicture}
	\captionof{figure}{Konvexität in K}
\end{center}
\bet{Argumentation:}\\
\begin{enumerate}[(i)]
	\item Falls $C(S_0,T,K) < 0$ gehe long im Call und halte bis $T$. Risikolosen Gewinn von $\abs{C(S_0,T,K)}> 0$ am Anfang. $\lightning_{\text{No-Arbitrage}}$\\
	Genauso sieht man ein, dass $P(S_0,T,K)\ge 0$\\
	Put-Call Parität liefert:
	\begin{equation*}
		\begin{aligned}
			C(S_0,T,K)&=S_0 + P(S_0,T,K)-K\cdot B(0,T)\\
			&\ge S_0 - K\cdot B(0,T)
		\end{aligned}
	\end{equation*}
	\item  Falls $C(S_0,T,K) >S_0$.\\
	long Aktie, short im call\\
	am Anfang: Gewinn von $C(S_0,T,K) -S_0>0$, benutze die Aktie um die Verpflichtung des Calls zu erfüllen: $S_T - (S_T-K)^+ \ge 0~~\lightning_{\text{No-Arbitrage}}$
	\item Sei $K_1 \le K_2$:\\
	Falls $C(K_1)<C(K_2)$ gehe  short in $K_2$ und long in $K_1$.\\
	am Anfang: $C(K_2) - C(K_1) > 0$\\
	am Ende: 
	\[
	(S_T-K_1)^+ -(S_T-K_2)^+ = (S_T-K_1)\mathbb{1}_{\{K_1\le S_T\le K_2\}}-(K_2-K_1)\mathbb{1}_{\{S_T>K_2\}}>0\quad \lightning_{\text{No-Arbitrage}}
	\]
	\item Sei $K_1<K_2$.\\
	Falls $C(K_1)-C(K_2)>(K_2-K_1)\cdot B(0,T)$ gehe short in $K_1$, long in $K_2$, long $(K_2-K_1)\times$ Nullkouponanleihe\\ 
	am Anfang: $C(K_1)-C(K_2)-(K_2-K_1)\cdot B(0,T)>0$\\
	am Ende: $(S_T-K_2)^+ -(S_T-K_1)^+ +K_2-K_1 \ge 0~~ \lightning_{\text{No-Arbitrage}}$
	\item $K_1<K_2<K_3$; $K_2=\lambda K_1+(1-\lambda)K_3$ mit $\lambda=\frac{K_3-K_2}{K_3-K_1}$\\
	Falls $C(K_2)> \lambda C(K_1)+ (1-\lambda)C(K_3)$, gehe short in $K_2$, $\lambda \times$ long in $K_1$, $(1-\lambda) \times$ long in $K_3$.\\
	am Anfang: $C(K_2)-\lambda C(K_1)-(1-\lambda)C(K_3)>0$\\
	am Ende: 
	\begin{equation*}
	\begin{aligned}
		\lambda (S_T -K_1)^+ +(1-\lambda)(S_T-K_3)^+-(S_T-K_2)^+
		&= \lambda(S_T-K_1)\mathbb{1}_{\{K_1<S_T<K_2\}} \\
		&+ \big[\lambda(S_T-K_1)-(S_T-K_2)\big]\mathbb{1}_{\{K_2<S_T<K_3\}} \\
		&+ \underbracket{\big[K_2 -(\lambda K_1+(1-\lambda)K_3)\big]\mathbb{1}_{\{S_T>K_3\}}}_{=0}
	\end{aligned}
	\end{equation*}
	da 
	\begin{equation*}
	\begin{aligned}
		\lambda (S_T-K_1)+(K_2-S_T) &= K_2-\lambda K_1-(1-\lambda)S_T \\
		&= \lambda K_1 + (1-\lambda)K_3-\lambda K_1-(1-\lambda)S_T \\
		&= (1-\lambda)(K_3-S_T) \ge 0 \\
		&\lightning_{\text{No-Arbitrage}}
	\end{aligned}
	\end{equation*}
	
\end{enumerate}
% subsection call-preis (end)

\subsection{Zinsmethoden}
\label{sub:zinsmethoden}
\uline{Frage:} Wie kann man Kapitalrenditen durch annualisierte \bet{Zinssätze} \index{Zinssatz} beschreiben?\\\marginnote{Zählkonvention für uns nicht wichtig}
\uline{Antwort:} Man vereinbart eine \Index{Zinsmethoden} und eine \Index{Zählkonvention} (Anzahl der Tage eines Jahres).\\
\uline{Genauer:} Kapital $N$ wird zum Zeitpunkt $t$ in eine Nullkouponanleihe mit Fälligkeit in $T$ angelegt.\\
\\
\\
\begin{minipage}[b]{7cm}
	\begin{tikzpicture}[line cap=round,line join=round,>=triangle 45,x=1.0cm,y=1.0cm]
	\draw[->,color=black] (0.,0.) -- (5.,0.);
	\foreach \x in {1., 4,}
	\draw[shift={(\x,0)},color=black] (0pt,2pt) -- (0pt,-2pt);
	\draw [black] (1.,-0.3) node {$t$};
	\draw [black] (4.,-0.3) node {$T$};
	\end{tikzpicture}
\end{minipage}
\begin{minipage}[b]{8cm}
	in $t$: erhalte für $N$:\\
	$\frac{N}{B(t,T)}$ $T$-Bonds \textcolor{light_gray}{ \{\footnotesize Nullkouponanleihe mit Fälligkeit in $T$}\normalsize \\

	in $T$: die Position hat einen Wert von $\frac{N}{B(t,T)}$
\end{minipage}
\uline{Gewinn:} $\frac{N}{B(t,T)}-N = N \enbrace{\frac{1}{B(t,T)}-1}$ \\
$R(t,T) = \frac{1}{B(t,T)}-1$ kann als Kapitalrendite interpretiert werden, die ein Investment zwischen $t$ und $T$ hervorbringt:\\
\uline{Ziel:} Beschreibung durch einen jährlichen Zinssatz:

\begin{enumerate}[(a)]
	\item \bet{lineare Zinsmethode:} \index{Zinsmethoden!lineare} \\
	lineare Verteilung der jährlichen Zinsen auf die Laufzeit $R(t,T)=\underbracket{(T-t)}_{Laufzeit} \cdot r_{lin}$, $r_{lin}$ ist der jährliche Zinssatz bei linearer Zinsmethode.\\
	\bet{Beispiel:} \\
	Anlage Zeitraum ein Monat
	\begin{itemize}
		\item Rendite von $0.5\% = 50~\text{bp}$ (ein Basispunkt $\mathrel{\hat=}$ 0,01\%)\index{Basispunkt}
		\item $r_{lin}=0.5\% \cdot 12 = 6\%$
	\end{itemize}
	\item \bet{periodische Zinsmethode:} \index{Zinsmethoden!periodische} \\
	\\
	\\
	\begin{minipage}[c]{10cm}
		\begin{tikzpicture}[line cap=round,line join=round,>=triangle 45,x=1.0cm,y=1.0cm]
		\draw[->,color=black] (-1.,0.) -- (7.5,0.);
		\foreach \x in {0.,1.,2.,3.,4.,5.,6.,7.}
		\draw[shift={(\x,0)},color=black] (0pt,2pt) -- (0pt,-2pt);
		\draw [black] (1.,-0.3) node {$t_1$};
		\draw [black] (2.,-0.3) node {$t_2$};
		\draw [black] (3.,-0.3) node {$t_3$};
		\draw [black] (4.,-0.3) node {$t_4$};
		\draw [black] (5.,-0.3) node {$t_5$};
		\draw [black] (6.,-0.3) node {$t_6$};
		\draw [black] (7.,-0.3) node {$T$};
		\draw [black] (0.,-0.3) node {$t=t_0$};
		\end{tikzpicture}
	\end{minipage}
	\begin{minipage}[c]{6cm}
		setze $t_i= t+i\cdot \frac{T-t}{m}, i=0,\dots,m$
	\end{minipage}
	Ein jährlicher Zins $r$ wird linear verteilt auf die Periodenlänge. Das Kapital wird unter Berücksichtigung von Zinseszinsen verzinst.\\
	\uline{Verzinsung:}
	\[
	K_m(r,t,T):=\enbrace{1+r \cdot \frac{T-t}{m}}^m = 1 + R(t,T) 
	\]
	\item \bet{stetige Zinsmethode:} \index{Zinsmethoden!stetige}\\
	\begin{enumerate}[(i)]
		\item \uline{Konstante} Zinsrate $r$ \index{Zinsmethoden!stetige!konstant} \\
		erhält man als Grenzübergang für $m \to \infty$\\
		$ \lim\limits_{m \to \infty} K_m(r,t,T) = e^{r(T-t)} = 1+ R(t,T)$
		\item \uline{nicht konstante} Zinsrate \index{Zinsmethoden!stetige!nicht konstant} \\
		$r: [0,\infty ) \to \mathds{R}$ liefert eine Kapitalentwicklung der Form:
		\[
		K(r,t,T)= e^{\int_{t}^{T} r(s)\mathrm{d}s} 
		\]
		zwischen $t$ und $T$.
	\end{enumerate}
	\newpage
	
	\bet{Veranschaulichung:}\\
	$r:[0,\inf) \to \mathds{R},~~~~t_i= t+i\cdot \frac{T-t}{m}, i=0,\dots,m$\\
	1 \texteuro~in $t_0$ ergibt bei Zinsrechnung:\\
	\begin{equation*}
	\begin{aligned}
		K_m(r,t,T) &= (1+r(t_0)\Delta t)\cdot(1+r(t_1)\Delta t)\cdots(1+r(t_m)\Delta t) \\
		\log~K_m(r,t,T) &= \sum_{i=1}^{m}\log(1+r(t_{i-1})\Delta t) \\
		&= \underbracket{\sum_{i=1}^{m}r(t_{i-1})\Delta t}_{\int_{t}^{T}r(s)\mathrm{d}s} +\underbracket{\mathcal{O}(\Delta t)}_{\stackrel{\Delta t \to 0}{\longrightarrow} 0}
	\end{aligned}
	\end{equation*}
\end{enumerate}
% subsection zinsmethoden (end)

\subsection{Festzinsanleihe}
\label{sub:festzinsanleihe}
\index{Anleihe!Festzins-}

\begin{itemize}
	\item festverzinsliches Wertpapier
	\item Nominal $N$
	\item Fälligkeit $T$
	\item Zinstermine $t_1<t_2<\dots <t_m\le T$
	\item Koupons $K_1,K_2,\dots,K_m$
\end{itemize}
In der Regel werden Koupons als Zins auf das Nominal gezahlt, d.h. $K_i=N\cdot R$, $R$ Zinssatz.\\
\begin{center}
	\begin{tikzpicture}[line cap=round,line join=round,>=triangle 45,x=1.0cm,y=1.0cm]
	\draw[->,color=black] (-1.,0.) -- (7.5,0.);
	\foreach \x in {0.,1.,2.,3.,4.,5.,6.,7.}
	\draw[shift={(\x,0)},color=black] (0pt,2pt) -- (0pt,-2pt);
	\draw [black] (1.,-0.3) node {$t_1$};
	\draw[->,color=black] (1.,-0.6) -- (1.,-1.);
	\draw [black] (1.,-1.3) node {$K_1$};
	\draw [black] (2.,-0.3) node {$t_2$};
	\draw[->,color=black] (2.,-0.6) -- (2.,-1.);
	\draw [black] (2.,-1.3) node {$K_2$};
	\draw [black] (3.,-0.3) node {$\dots$};
	\draw[->,color=black] (3.,-0.6) -- (3.,-1.);
	\draw [black] (3.,-1.3) node {$\dots$};
	\draw [black] (5.,-0.3) node {$\dots$};
	\draw[->,color=black] (5.,-0.6) -- (5.,-1.);
	\draw [black] (5.,-1.3) node {$\dots$};
	\draw [black] (6.,-0.3) node {$t_m$};
	\draw[->,color=black] (6.,-0.6) -- (6.,-1.);
	\draw [black] (6.,-1.3) node {$K_m$};
	\draw [black] (7.,-0.3) node {$T$};
	\draw[->,color=black] (7.,-0.6) -- (7.,-1.);
	\draw [black] (7.,-1.3) node {$N$};
	\draw [black] (0.,-0.3) node {$t$};
	\draw[->,color=black] (0.,0.5) -- (0.,0.1);
	\draw [black] (0.,1.) node {Preis};
	\end{tikzpicture}
	\captionof{figure}{Ablauf Festzinsanleihe}
\end{center}
Bewertung zu Zeitpunkt $t<t_1$:\\
Mit Hilfe einer Modifikation des Replikationsprinzips:\\
I: long in die Festzinsanleihe\\
II: long in $K_i \times$ $T_i$-Bonds, $i=1,\dots,m$, long in $N$ $T$-Bonds\\

Beide Strategien erzeugen den gleichen Zahlungsstrom an Auschüttungen:
\[
K_1 \text{ in } t_1, K_2 \text{ in } t_2, \dots K_m \text{ in } t_m 
\]
und haben den gleichen Endwert $N$ in $T$.\\
Replikationsprinzip liefert, dass die Preise in $t<t_1$ übereinstimmen müssen, d.h.
\[
\text{Preis der Festzinsanleihe in }t<t_1 \text{ ist } 
\]
\[
\sum_{i=1}^{m}K_i \cdot B(t,t_i)+N\cdot B(t,T) 
\]

% subsection festzinsanleihe (end)

\subsection{Variabelverzinsliche Anleihe}
\label{sub:variableverzinsliche_anleihe}
\index{Anleihe!Variabelverzinsliche}
\begin{itemize}
	\item[\bet{Floater FRN} (Floating Rate Note)]\index{Anleihe!Variabelverzinsliche!Floater FRN}
	\item Nominal $N$
	\item Fälligkeit $T$
	\item Startpunkt $t_0$
	\item Zinszahlungstermine $t_0<t_1<\dots<t_m=T$
	\item \Index{nachschüssige Kouponzahlungen} $K_1,K_2,\dots,K_m$ entsprechend dem für die Periode geltendem Marktzins
	\[
	F(t_{i-1},t_{i-1},t_i)=\frac{1}{t-t_{i-1}}\cdot \enbrace{\frac{1}{B(t_{i-1},t_i)}-1} 
	\]
	also
	\begin{equation*}
	\begin{aligned}
		K_i &= N\cdot F(t_{i-1},t_{i-1},t_i)(t_i-t_{i-1})\\
		&=N\enbrace{\frac{1}{B(t_{i-1},t_i)}-1},~~i=1,\dots,m
	\end{aligned}
	\end{equation*}
	\item Bewertung in $t_0$ durch folgende replizierende Handelsstrategie:\\
	\Index{Rollierende Anlage} des Nominals bis zum jeweiligen nächsten Zinstermin.\\
\end{itemize}
\uline{Genauer:}\\
\begin{center}
	\begin{tikzpicture}[line cap=round,line join=round,>=triangle 45,x=1.0cm,y=1.0cm]
	\draw[->,color=black] (-1.,0.) -- (7.5,0.);
	\foreach \x in {0.,1.,2.,3.,6.,7.}
	\draw[shift={(\x,0)},color=black] (0pt,2pt) -- (0pt,-2pt);
	\draw [black] (1.,-0.3) node {$t_1$};
	\draw [black] (2.,-0.3) node {$t_2$};
	\draw [black] (3.,-0.3) node {$\dots$};
	\draw [black] (6.,-0.3) node {$\dots$};
	\draw [black] (7.,-0.3) node {$t_m$};
	\draw [black] (0.,-0.3) node {$t_0$};
	\end{tikzpicture}
\end{center}
\begin{itemize}
	\item in $t_0$: Kaufe $\frac{N}{B(t_0,t_1)}~t_1$-Bonds und halte bis $t_1$
	\item in $t_1$: 
	\begin{itemize}
		\item Reinvestition von $N$ in die 2.-Zinsperiode durch Kauf von $\frac{N}{B(t_1.t_2)}~t_2$-Bonds
		\item Ausschüttung der Zinszahlung von $\frac{N}{B(t_0,t_1)}-N=N\cdot F(t_0,t_0,t_1)(t_1-t_0)=K_1$
	\end{itemize}
	\item $\dots$
	\item$\dots$
	\item in $t_m$:
	\begin{itemize}
		\item Rückzahlung von $N$
		\item Ausschüttung der letzten Zinszahlung $\frac{N}{B(t_{m-1},t_m)}-N=K_m$
	\end{itemize}
\end{itemize}
Das Halten der variabel verzinslichen Anleihe und das Durchführen der rollierenden Handelsstrategie liefern den gleichen Zahlungsstrom an Zinszahlungen und haben zur Fälligkeit das Nominal als Endwert. 
Für die Handelsstrategie wird in $t_0$ ein Kapital von $N$ benötigt. 
Deshalb ist der Preis der variabel verzinslichen Anleihe in $t_0$ durch $N$ gegeben. \\
In $t<t_0$ ist der Preis $N\cdot B(t,t_0)$, denn durch Kauf von $N~t_0$-Bonds in $t$ kann die rollierende Handelsstrategie von $t$ beginnend durchgeführt werden.
% subsection variabelverzinsliche anleihe (end)

\subsection{Swaps}
\label{sub:swaps}
Ein \bet{Zinsswap}\index{Swaps!Zinsswap} liefert die Möglichkeit das \Index{Zinsänderungsrisiko} einer Festzinsanleihe zu vermeiden:\\
\begin{itemize}
	\item Tauschgeschäft
	\item beim Zinsswap werden feste gegen variable Zinsen getauscht
	\item \Index{Tenorstruktur} $t_0<t_1<\dots<t_m$
	\item jährlichen Festzinssatz
	\item Nominal $N$, das nur zur Berechnung der Zinsen dient
	\item Unterscheidung in \bet{Payer- und Reciever-Swaps} \index{Swaps!Payer-}\index{Swaps!Reciever-} ausgehend von der Festzinsseite
\end{itemize}
Am Ende einer jeden Periode werden die festen Zinsen $N\cdot R(t_i-t_{i-1})$ gegen die variablen\\ 
$N\cdot F(t_{i-1},t_{i-1},t_i)(t_i-t_{i-1})$ getauscht.\\
Dies führt zum Zahlungsstrom
\[
N(t_i-t_{i-1})\enbrace{F(t_{i-1},t_{i-1},t_i)-R},~1\le i\le m 
\]
beim Payer-Swap und
\[
N(t_i-t_{i-1})(R-F(t_{i-1},t_{i-1},t_i)) 
\]
beim Reciever-Swap.\\
Ein Payer-Swap kann repliziert werden durch eine long-Position in der FRN, short in die Festzinsanleihe zum Nominal $N$ und Zinszahlungsterminen passend zur Tenorstruktur.\\
Deshalb gilt für den Preis Payerswap($t$) in $t\le t_0$:\\
\begin{equation*}
\begin{aligned}
	\text{Payerswap}(t) &= \underbracket{N\cdot B(t,t_0)}_{\text{FRN in }t}-\underbracket{\enbrace{\sum_{i=1}^{m}N\cdot R(t_{i-1}-t_i) \cdot B(t,t_i)+N\cdot B(t,t_m)}}_{\text{Festzinspreis}}\\
	&=N\enbrace{B(t,t_0)-B(t,t_m)-\sum_{i=1}^{m}R\cdot B(t_{i-1},t_i)}
\end{aligned}
\end{equation*}
Der 'faire' Festzinspreis $R$ liegt dann in $t$ vor, wenn Payerswap($t$)=0, also wenn
\[
R= \frac{B(t,t_0)-B(t,t_m)}{\sum_{i=1}^{m}B(t,t_i)(t_i-t_{i-1})} 
\]
$R$ ist dann die sogenannte \Index{Swaprate} in $t$.

% subsection swaps (end)
% section informelle einführung (end)
\newpage
\section{Aktuarielle Bewertung von Zahlungsströmungen}
\label{sec:zahlungsströmungen}
\bet{Ziel:} Bewertung von Zahlungsverpflichtungen, die durch biometrische Risiken verursacht werden. 
Biometrische Risiken sind zum Beispiel Todesfall, Invalidität,...\\
\subsection{Zahlungsströme und deren Bewertung}
\label{sub:zahlungsströme}
\index{Zahlungsströme}
\begin{itemize}
	\item zeitdiskrete periodische Sichtweise, Zeit wird in Jahren gemessen
	\item[\uline{Definition}]
	\item Ein Zahlungsstrom $Z$ ist eine Folge $(Z(n))_{n \in \mathds{N}}$ von nicht negativen reellen Zahlen, $Z(n) \mathrel{\hat=}$ Auszahlung zum Zeitpunkt $n$
	\item Frage: Was ist der Kapitalwert, der durch den Zahlungsstrom der verursachten Zahlungsverpflichtungen entsteht?
	\item Antwort: Summe der \Index{abdiskontierten} Zahlungen
	\item Genauer: Für jedes $n\in N$ gibt $B(k,n)$, den Preis des $n$-Bonds zum Zeitpunkt $k$, den Wert einer in $n$ fälligen Zahlungsverpflichtung von 1\texteuro$~$an.
	\item Deshalb definieren wir:\\
	\[
	V_0(Z)= \sum_{n=0}^{\infty}Z(n)\cdot B(0,n) 
	\]
	Summe aller auf den Anfang abdiskontierten Zahlungsverpflichtungen, Kapitalwert von heute.\\Und
	\[
	V_m(Z)= \sum_{k=0}^{\infty}Z(m+k)\cdot B(m,m+k) 
	\]
	Summe aller nach $m$ fälligen auf den Zeitpunkt $m$ abdiskontierten Zahlungsverpflichtungen.
	\item $V_m(Z)$ ist das Kapital, das zum Zeitpunkt $m$ benötigt wird, um die zukünftigen Zahlungsverpflichtungen erfüllen zu können.
	\item Praxis: Periodische Rendite $r$, periodische \Index{Diskontfaktor}
	\[
	v=\frac{1}{1+r} \Rightarrow B(m,n)=v^{n-m}~~~\forall 0\le m\le n 
	\]
\end{itemize}
% subsection zahlungsströme (end)

\subsection{Personenversicherung und deren Bewertung}
\label{sub:personenversicherung}
\bet{Ziel:}\\
Mathematische Beschreibung und Analyse einer Personenversicherung\\
\bet{Definition:}\\
Eine Personenversicherung ist ein Quadertupel $\Gamma=(t,s,b,T)$ mit Zahlungsströmen $(t(n))_{n \in \mathds{N}_0},~~(s(n))_{n\in \mathds{N}_0},~~(b(n))_{n \in \mathds{N}_0}$ und $(0,\infty)$-wertiger Zufallsvariabel $T$.\\
\bet{Interpretation:}\\
\begin{itemize}
	\item $T$ ist eine zufällige Ausfallzeit (etwa Restlebensdauer)
	\item \Index{Todesfallspektrum} $(t(n))_{n\in \mathds{N}_0}$
	\[
	t(n)\ge 0 \mathrel{\hat=} \text{ Auszahlung in $n$ bei Ausfall in der $n$-ten Periode} 
	\]
	\item \Index{Erlebensspektrum} $(s(n))_{n\in \mathds{N}_0}$
	\[
	s(n)\ge 0 \mathrel{\hat{=}} \text{ Auszahlung in $n$, wenn $n$ erreicht wird} 
	\]
	\item \Index{Beitragsspektrum} $(b(n))_{n \in \mathds{N}_0}$
	\[
	b(n)\ge 0 \mathrel{\hat{=}} \text{ Prämienzahlung in $n$, wenn $n$ erreicht wird} 
	\]
\end{itemize}
Aus Sicht des Versicherungsunternehmens erzeugt eine Personenverischerung die folgenden zufälligen Zahlungsströme:\\

\begin{minipage}[t]{8.7cm}
	Ausgabenstrom:\\
	$A(n)=s(n) \mathbb{1}_{\{T>n \}} + t(n) \mathbb{1}_{\{n-1<T\le n \}} \\ \forall n \in \mathds{N}_0,~~A(0)=s(0)$
\end{minipage}
\begin{minipage}[t]{5cm}
	Einnahmestrom:\\
	$I(n)=b(n)\mathbb{1}_{\{T>n \}}~~\forall n \in \mathds{N}_0$
\end{minipage}

Bewertung aus heutiger Sicht durch
\begin{equation*}
\begin{aligned}
	V_0(A) &= s(0)+\sum_{n=1}^{\infty}s(n) \mathbb{1}_{\{T>n\}}B(0,n)+\sum_{n=1}^{\infty}t(n)\mathbb{1}_{\{n-1<T\le n \}} B(0,n) \\
	V_0(I) &= \sum_{n=0}^{\infty}b(n)\mathbb{1}_{\{T>n\}} B(0,n)
\end{aligned}
\end{equation*}
$V_0(A) \mathrel{\hat{=}}$ heutiger Kapitalwert des zufälligen Zahlungsstroms\\
$\mathds{E}V_0(A) \mathrel{\hat{=}}$ mittlerer Kapitalwert der zukünftigen Zahlungsverpflichtung\\
$\mathds{E}V_0(I) \mathrel{\hat{=}}$ mittlerer Kapitalwert der Einnahmen\\
\bet{Definition:}\\
$\mathds{E}V_0(A)$ heißt \Index{Barwert} der durch die Versicherung induzierten Zahlungsverpflichtungen.
$\mathds{E}V_0(I)$ heißt \bet{Barwert} der durch die Versicherung induzierten Einnahmen.\\
Eine Personenversicherung heißt \Index{ausgewogen} oder \Index{fair}, wenn $\mathds{E}V_0(A)=\mathds{E}V_0(I)$ gilt und beide endlich sind.\\
Ist $\mathds{E}V_0(A)<\infty$ oder $\mathds{E}V_0(I)<\infty$, so ist $\mathds{E}V_0(A)-\mathds{E}V_0(I)$ der Barwert der Versicherung.\\
Dies ist als Ausgangspreis zu interpretieren, den ein Versicherungsunternehmen verlangt.\\
\bet{Äquivalenzprinzip:}\\
Man wähle $(t,s,b)$ so, dass die Versicherung fair ist.

% subsection personenversicherung (end)

\subsection{Klassische Beispiele}
\label{sub:klassische_beispiele}
\begin{itemize}
	\item versichert wird eine Person
	\item biometrisches Risiko ist das Todesfallrisiko
	\item Ausfallzeit ist deshalb die Restlebensdauer der Person
\end{itemize}
\begin{enumerate}[(a)]
	\item \Index{Todesfallversicherung:}
	\begin{itemize}
		\item Todesfallsumme $M$
		\item Laufzeit $n$
		\item konstante periodische Prämienzahlung
		\begin{equation*}
		\begin{aligned}
			t(k) &= M ~ k=1,\dots,n~~~~t(k)=0 \text{ sonst}\\
			s(k) &= 0 ~ \forall k \in \mathds{N}_0\\
			b(k) &= p ~k=0,\dots,n-1~~~~b(k)=0 \text{ sonst}
		\end{aligned}
		\end{equation*}
		\item Induzierte Zahlungsströme:
		\begin{equation*}
		\begin{aligned}
			A(k) &= M\cdot \mathbb{1}_{\{k-1<T\le k\}},~k=1,\dots,n~~~A(k)=0 \text{ sonst}\\
			I(k) &= p\cdot \mathbb{1}_{\{T>k\}},~k=0,\dots,n-1~~~I(k)=0 \text{ sonst}\\
			V_0(A) &= \sum_{k=1}^{n}M\cdot B(0,k) \mathbb{1}_{\{k-1<T\le k \}}\\
			V_0(I) &= \sum_{k=0}^{n-1} p\cdot B(0,k) \mathbb{1}_{\{T>k \}}\\
			\text{Also } \mathds{E}V_0(A) &= \sum_{k=1}^{n}M\cdot B(0,k) \mathds{P}(k-1<T\le k)\\
			\mathds{E}V_0(I) &= p\cdot \sum_{k=0}^{n-1}B(0,k) \mathds{P}(T>k)
		\end{aligned}
		\end{equation*}
		\item Praxis:
		\begin{itemize}
			\item Restlebenszeit wird durch das Alter bestimmt $T_x$ Restlebenszeit eines x-Jährigen
			\item Stationaritätsannahme $\mathds{P}(T_x>t+s~|~T_x>s)=\mathds{P}(T_{x+s}>t)$
			\item $q_x= \mathds{P}(T_x\le 1)$ 1-jährige Sterbew'keit eines x-Jährigen
			\item $p_x=1-q_x=\mathds{P}(T_x>1)$ 1-jährige Überlebensw'keit eines x-Jährigen
			\item $_kp_x:=\mathds{P}(T_x>k)=\mathds{P}(T_x>1)\mathds{P}(T_x>k~|~T_x>1)\stackrel{\text{Stationarität}}{=} p_x \mathds{P}(t_{x+1}>k-1)\\ =\dots=p_xp_{x+1}\dots p_{x+k-1}$
			\item $_kq_x = 1- _kp_x = \mathds{P}(T_x\le k)$
			\item Bezeichnung für $M=1$, Eintrittsalter $x$
			\[
			_{|n}A_x = \sum\limits_{k=1}^{n}v^k \mathds{P}(k-1<T_x\le k)
			\]
			für $p=1$:
			\[
			ä_{x:n\rceil} = \sum\limits_{k=0}^{n-1}v^k \mathds{P}(T_x>k)
			\]
			Die Todesfallversicherung ist fair, wenn $M\cdot~_{|n}A_x = p \cdot ä_{x:n\rceil}$\\
			$n \to +\infty$ entspricht Todesfallversicherung ohne zeitliche Beschränkung
			\item Bezeichnung\\
			\begin{equation*}
			\begin{aligned}
				A_x &= \sum_{k=1}^{\infty}v^k \mathds{P}(k-1<T_x \le k)\\
				ä_x &= \sum_{k=0}^{\infty}v^k \mathds{P}(T_x>k)
			\end{aligned}
			\end{equation*} 
			\end{itemize}
	\end{itemize}
	\newpage
	\item \Index{aufgeschobene Rentenversicherung}\\
	
	\begin{minipage}[b]{8cm}
		\begin{itemize}
			\item Eintrittsalter $x$
			\item Aufschubszeit $m$ Jahre		
			\item Bezugszeit $n$ Jahre
			\item Rentenhöhe $R$
			\item Beitragshöhe $p$
		\end{itemize}	
	\end{minipage}		
	\begin{minipage}[b]{9cm}
		\begin{tikzpicture}[line cap=round,line join=round,>=triangle 45,x=1.0cm,y=1.0cm]
		\draw[->,color=black] (0.,0.) -- (7.5,0.);
		\foreach \x in {0.,1.,2.,3.5,5.,7.}
		\draw[shift={(\x,0)},color=black] (0pt,2pt) -- (0pt,-2pt);
		\draw [black] (1.,-.2) node {1};
		\draw[->, black] (1.,0.2) -- (1.,.7) node[above] {$p$};
		\draw [black] (2.,-.2) node {2};
		\draw[->, black] (2.,0.2) -- (2.,.7) node[above] {$p$};
		\draw [black] (3.5,.2) node {$m$};
		\draw[->, black] (3.5,-.2) -- (3.5,-.7) node[below] {$R$};
		\draw [black] (5.,.2) node {$m+k$};
		\draw[->, black] (5.,-.2) -- (5.,-.7) node[below] {$R$};
		\draw [black] (7.,.2) node {$n+m$};
		\draw[thick,black,decorate,decoration={brace,amplitude=2pt}] (3.5,-1.2) -- (0.,-1.2) node[midway,below]{Aufschubszeit};
		\draw[thick,black,decorate,decoration={brace,amplitude=2pt}] (7.,-1.2) -- (3.5,-1.2) node[midway,below]{Bezugszeit};
		\end{tikzpicture}
	\end{minipage}
	\begin{itemize}
		\item[Modellierung:]
		\item $T=T_x$ Restlebenszeit eines x-Jährigen
		\item $t(k)=0 ~\forall k \in \mathds{N}_0$
		\item $s(k)=0~ k=0,\dots,m-1,~~~s(m+k)=R~ k=0,\dots,n-1$
		\item $b(k)=p~ k=0,\dots,m-1,~~~b(k)=0$ sonst
		\item[Induzierte Zahlungsströme:]
		\item Ausgaben: $A(m+k)= R\cdot \mathbb{1}_{\{T>m+k\}}~ k=0,\dots,n-1,~~ A(k)=0$ sonst
		\item Einnahmen: $I(k)=p\cdot \mathbb{1}_{\{T>k\}}~ k=0,\dots,m-1 ~~ I(k)=0$ sonst
		\item Barwert der Ausgaben: $\mathds{E}V_0(A)=\sum_{k=0}^{n-1}R\cdot v^{m+k}\mathds{P}(T_>m+k)$
		\item Barwert der Einnahmen: $\mathds{E}V_0(I)=p\sum_{k=0}^{m-1}v^k \mathds{P}(T>k)=p\cdot ä_{x:m\rceil}$ \marginnote{\tiny $ä_{x:m\rceil}=\sum_{k=0}^{m-1}v^k \mathds{P}(T>k)$}
		\item Bezeichnung für $R=1$: $_{m|n}ä_x:=\sum_{k=0}^{n-1}v^{m+k}\mathds{P}(T>m+k)$
		\item Die Versicherung ist fair, wenn $R\cdot~ _{m|n}ä_x=p\cdot ä_{x:m\rceil}$
		\item Für $n=\infty$, lebenslange Rente: $_{m|}ä_x=\sum_{k=0}^{\infty}v^{k+m}\mathds{P}(T>m+k)$
	\end{itemize}
	\item \Index{Erlebensfallversicherung}\\
	\begin{itemize}
		\item Eintrittsalter $x$
		\item Laufzeit $n$ Jahre
		\item \Index{Erlebensfallsumme} $M$, Auszahlung bei Überleben von $n$ Jahren
		\item konst. Prämie $p$, während der Laufzeit
		\item[Modellierung:]
		\item $T=T_x$ Restlebenszeit
		\item $t(k)=0~\forall k\in \mathds{N}_0$
		\item $s(k)=\left\{\begin{array}{cl} M, & k=n\\ 0, & \text{ sonst.} \end{array}\right.$
		\item $b(k)=\left\{\begin{array}{cl} p, & k=0,\dots,n-1 \\ 0, & \text{ sonst.}	\end{array}\right.$
		\item[Induzierte Zahlungsströme:]
		\item Ausgaben: $A(m)=M\cdot \mathbb{1}_{\{T>m\}}$, $A(k)=0$ sonst
		\item Einnahmen: $I(k)=p\cdot \mathbb{1}_{\{T>k\}},~~k=1,\dots,n-1$
		\item[Bewertung:]
		\[
		\mathds{E}V_0(A)= M \underbracket{v^n \mathds{P}(T_x>n)}_{= _nE_x}=M\cdot~_nE_x 
		\]		
		\[
		\mathds{E}V_0(I)= p\cdot ä_{x:n\rceil} 
		\]
		\item Versicherung ist fair, wenn $ M~_nE_x=pä_{x:m\rceil}$
	\end{itemize}
	\item \Index{gemischte Versicherung} (Kapitalgebundene Lenbensvers.)\\
	
	\begin{minipage}[t]{9cm}
		\begin{itemize}
			\item Kombination aus Todesfall- und Erlebensversicherung
			\item Eintrittsalter $x$
			\item Laufzeit $n$
			\item Versicherungssumme $M$, fällig bei Tod während der Laufzeit oder bei Überleben der Laufzeit
			\item konst. Prämie $p$, während der Laufzeit
		\end{itemize}
	\end{minipage}
	\begin{minipage}[t]{9cm}
		\begin{itemize}
			\item[Modellierung:]
			\item $T=T_x$ Restlebenszeit
			\item $t(k)=M,~~ k=1,\dots,n$, $t(k)=0$ sonst
			\item $s(n)=M,~~ s(k)=0$ sonst
			\item $b(k)=p,~~ k=0,\dots,n-1$, $b(k)=0$ sonst
		\end{itemize}
	\end{minipage}
	Induzierte Zahlungsströme:
	\begin{equation*}
	\begin{aligned}
		A(k) &= M\cdot \mathbb{1}_{\{k-1<T\le k \}}, ~~~~~k=1,\dots,n-1\\
		A(n) &= M\cdot (\mathbb{1}_{\{n-1<T \le n \}}+\mathbb{1}_{\{T>n\}}), ~~~~~A(k)=0 \text{ sonst}\\
		I(k) &= p\cdot \mathbb{1}_{\{T>k\}},~~~~k=0,\dots,n-1
	\end{aligned}
	\end{equation*}
	Bewertung: 
	\[ 
	\mathds{E}V_0(A)=M(~_{|n}A_x + ~_nE_x) \qquad \qquad \mathds{E}V_0(I)=p ä_{x:n\rceil}
	\]
	Versicherung ist fair, wenn $M(~_{|n}A_x + ~_nE_x)=p ä_{x:n\rceil}$
\end{enumerate}
% subsection klassische beispiele (end)

\subsection{Deckungskapital}
\label{sub:deckungskapital}
\index{Deckungskapital}
Betrachtet wird nur der Fall einer \Index{deterministischen Zinsentwicklung}, z.B. $B(k,n) \in (0,1)$ det. $\forall n \in \mathds{N},~k<n$\\
\bet{Beobachtung:}\\
Anfangs sind die Prämieneinnahmen pro Jahr höher, als die zu erwartenden Ausgaben pro Jahr. 
Dies führt zum Aufbau einer \Index{Prämienreserve}. 
Gegen Ende sind die zu erwartenden Leistungen pro Jahr höher, als die Prämien pro Jahr und werden durch die aufgebaute Prämienreserve finanziert.\\
Der \bet{Deckungskapitalverlauf}\index{Deckungskapital!-verlauf} spiegelt den Auf- und Abbau der Prämienreserve wieder.\\
\marginnote{Anfang und Ende bezieht sich auf die Versicherung, bzw. viele Versicherungen zum selben Zeitpunkt}
\bet{Definition:}\\
Gegeben sei eine allgemeine Prämienversicherung $\Gamma=(t,s,b,T)$.
Sei $(A(n))_{n\in \mathds{N}}$ und $(I(n))_{n\in \mathds{N}}$ der Zahlungsstrom der Ausgaben und Einnahmen. 
Das nach $m$ Jahren gebildete Deckungskapital $\mathcal{D}(m)$ ist definiert, als die Differenz der Barwerte, der dann zukünftigen Ausgaben und Einnahmen, wobei die Diskontierung auf das Ende des n-ten Jahres vorgegeben wird.\\
In mathematischen Formeln: 
\[
\mathcal{D}(m)= \mathds{E}(V_m(A)~|~\underbracket{T>m}_{\text{die nach $m$ noch leben}})- \mathds{E}(V_m(I)~|~T>m) ~~ \forall m\in \mathds{N}_0 
\]
Dies ist die Definition des sogenannten \bet{prospektiven Deckungskapitals}\index{Deckungskapital!prospektives} (vorausschauende Methode).\\
Für $m=0$ ist $\mathcal{D}(0)$ der Barwert der Versicherung. 
$\mathcal{D}(0)=0$ liegt bei einer fairen Versicherung vor.\\
\bet{Bemerkung:}\\
\begin{equation*}
\begin{aligned}
	\mathds{E}(V_m(A)~|~T>m) &= \mathds{E}\enbrace{\sum_{k=0}^{\infty}A(m+k)\underbracket{B(m,m+k)}_{=v^k}~|~(T>m)}\\
	&= \sum_{k=1}^{\infty}t(k+m)B(m,m+k) \mathds{P}(m+k-1<T<m+k~|~T>m)\\
	&+ \sum_{k=0}^{\infty}s(k+m)B(m,m+k) \mathds{P}(T>m+k~|~T>m)	
\end{aligned}
\end{equation*}
\[
\text{Da } A(m+k) = t(k+m)\cdot \mathbb{1}_{\{m+k-1<T \le m+k \}} + s(k+m)\cdot \mathbb{1}_{\{T>m+k \}}
\]
Analog:
\[
\mathds{E}(V_m(I)~|~T>m)= \sum_{k=0}^{\infty}b(m+k)B(m,m+k) \mathds{P}(T>m+k~|~T>m) 
\]
% subsection deckungskapital (end)

\subsection{Beispiele Deckungskapital}
\label{sub:bsp_deckungskapital}
periodischen konst. Diskontfaktor $v$\\
\begin{enumerate}[(a)]
	\item Todesfallverischerung:\\
	
	\begin{minipage}[c]{6cm}
		\begin{itemize}
			\item Eintrittsalter $x$
			\item VS $M=1$
			\item Laufzeit $n$ Jahre
		\end{itemize}
	\end{minipage}
	\begin{minipage}[c]{8cm}
		\begin{equation*}
		\begin{aligned}
			A(k) &= \mathbb{1}_{\{k-1<T_x\le k \}}, ~~~k=1,\dots,n\\
			I(k) &= p\cdot \mathbb{1}_{\{T_x>k \}}, ~~~k=0,\dots,n-1 \text{ mit } p=\frac{_{|n}A_x}{ä_{x:n\rceil}}
		\end{aligned}
		\end{equation*}
	\end{minipage}
	\begin{equation*}
	\begin{aligned}
		\mathcal{D}_x(m) &= \sum_{k=1}^{n-m}v^k\mathds{P}(m+k-1<T_x \le m+k~|~T_x>m)-p\sum_{k=0}^{n-m-1}v^k\mathds{P}(T_x>m+k~|~T_x>m)\\
		&= \sum_{k=1}^{n-m}v^k\mathds{P}(k-1<T_{x+m} \le k) - p \sum_{k=0}^{n-m-1}v^k\mathds{P}(T_{x+m}>k)\\
		&=~_{|n-m}A_{x+m} - p\cdot ä_{x+m:n-m\rceil}
	\end{aligned}
	\end{equation*}
	\begin{center}
	\begin{tikzpicture}[line cap=round,line join=round,>=triangle 45,x=1.0cm,y=1.0cm]
	\draw[->,color=black] (0.,0.) -- (5.5,0.);
	\foreach \x in {0,5}
	\draw[shift={(\x,0)},color=black] (0pt,2pt) -- (0pt,-2pt);
	\draw (5.,0.) node[below] {$n$};
	\coordinate (A) at (0,0);
	\coordinate (B) at (5,0);
	\coordinate (a) at (2,2);
	\coordinate (c) at (3,2);
	\draw[cap=round]
		(A) .. controls (a) and (c) ..(B);
	\end{tikzpicture}
	\captionof{figure}{Deckungskapital: Todesfall,begrenzt}
	\end{center}
	\item Todesfallversicherung, unbegrenzte Laufzeit
	\begin{equation*}
	\begin{aligned}
		\mathcal{D}_x(m) &= \sum_{k=1}^{\infty}v^k\mathds{P}(m+k-1<T_x \le m+k~|~T_x>m)-p\sum_{k=0}^{\infty}v^k\mathds{P}(T_x>m+k~|~T_x>m)\\
		&= \sum_{k=1}^{\infty}v^k\mathds{P}(k-1<T_{x+m} \le k) - p \sum_{k=0}^{\infty}v^k\mathds{P}(T_{x+m}>k)\\
		& =A_{x+m} - p\cdot ä_{x+m} \qquad\qquad m=0,1,2,3,\dots \qquad p\text{ erfüllt } A_x=pä_x
	\end{aligned}
	\end{equation*}
	\begin{center}
	\begin{tikzpicture}[line cap=round,line join=round,>=triangle 45,x=1.0cm,y=1.0cm]
	\draw[->,color=black] (0.,0.) -- (6,0.);
	\foreach \x in {0}
	\draw[shift={(\x,0)},color=black] (0pt,2pt) -- (0pt,-2pt);
	\coordinate (A) at (0,0);
	\coordinate (B) at (6,0);
	\coordinate (a) at (2,3);
	\coordinate (c) at (3,0);
	\draw[cap=round]
	(A) .. controls (a) and (c) ..(B);
	\end{tikzpicture}
	\captionof{figure}{Deckungskapital: Todesfall, unbegrenzt}
	\end{center}
	\item Erlebensfallversicherung:\\
	\begin{minipage}[c]{8cm}
		\begin{itemize}
			\item Eintrittsalter $x$
			\item Laufzeit $n$
			\item Versicherungssumme 1
		\end{itemize}
	\end{minipage}
	\begin{minipage}[c]{8cm}
		\begin{equation*}
		\begin{aligned}
			A(k) &= \mathbb{1}_{\{T_x>k \}} \text{ für } k=n\\
			A(k) &= 0 \text{ sonst}\\
			I(k) &= p\mathbb{1}_{\{T_x>k \}}\quad k=0,\dots,n-1
		\end{aligned}
		\end{equation*}
	\end{minipage}
	Deckungskapitalverlauf:
	\begin{equation*}
	\begin{aligned}
		\mathcal{D}_x(m)  &= v^{n-m}\mathds{P}(T_x>n~|~T_x>m)-p\sum_{k=0}^{n-m-1}v^k\mathds{P}(T_x>m+k~|~T_x>m)\\
		\text{Stationarität }\Rightarrow &=v^{n-m}\mathds{P}(T_{x+m}>n-m)-p\sum_{k=0}^{n-m-1}v^k\mathds{P}(T_{x+m}>k)\\
		&= ~_{n-m}E_{x+m}-pä_{x+m:n-m\rceil}
	\end{aligned}
	\end{equation*}
	\begin{center}
	\begin{tikzpicture}[line cap=round,line join=round,>=triangle 45,x=.6cm,y=.6cm]
	\draw[->,color=black] (-.5,0.) -- (5.5,0.);
	\foreach \x in {0,5}
	\draw[shift={(\x,0)},color=black] (0pt,2pt) -- (0pt,-2pt);
	\draw[->,color=black] (0.,-.5) -- (0,5.5);
	\foreach \y in {0,5}
	\draw[shift={(\y,0)},color=black] (0pt,2pt) -- (0pt,-2pt);
	\draw (5.,0.) node[below] {$n$};
	\coordinate (A) at (0,0);
	\coordinate (B) at (5,5);
	\coordinate (a) at (4,.5);
	\coordinate (b) at (4.5,3.5);
	\draw[cap=round]
	(A) .. controls (a) and (b) ..(B);
	\draw[thick,black,decorate,decoration={brace,amplitude=2pt}] (5,5) -- (5,0) node[midway,right]{1};
	\end{tikzpicture}
	\captionof{figure}{Deckungskapital: Erlebensfall}
	\end{center}
	\item gemischte Versicherung:\\
	\begin{itemize}
		\item Todesfall + Erlebensfall
		\item Deckungskapitalverlauf als Summe der Deckungskapitalien der einzelnen Versicherungen
		\item in Formeln: \quad $\cdot$Laufzeit $n$ \quad $\cdot$Eintrittsalter $x$
		\[
		\mathcal{D}_x(m)=A_{x+m:n-m\rceil}-pä_{x+m:n-m\rceil} \text{ mit } A_{x+m:n-m\rceil}=pä_{x+m:n-m\rceil} 
		\]
		Wobei $A_{x:n\rceil}=~_{|n}A_x+~_nE_x$ Barwert der gemischten Versicherung.
	\end{itemize}
	\begin{center}
		\begin{tikzpicture}[line cap=round,line join=round,>=triangle 45,x=.6cm,y=.6cm]
		\draw[->,color=black] (-.5,0.) -- (5.5,0.);
		\foreach \x in {0,5}
		\draw[shift={(\x,0)},color=black] (0pt,2pt) -- (0pt,-2pt);
		\draw[->,color=black] (0.,-.5) -- (0,5.5);
		\foreach \y in {0,5}
		\draw[shift={(\y,0)},color=black] (0pt,2pt) -- (0pt,-2pt);
		\draw (5.,0.) node[below] {$n$};
		\coordinate (A) at (0,0);
		\coordinate (B) at (5,5);
		\coordinate (a) at (.5,4);
		\coordinate (b) at (3.5,4.5);
		\draw[cap=round]
		(A) .. controls (a) and (b) ..(B);
		\draw[thick,black,decorate,decoration={brace,amplitude=2pt}] (5,5) -- (5,0) node[midway,right]{1};
		\end{tikzpicture}
		\captionof{figure}{Deckungskapital: gemischte Versicherung}
	\end{center}
	\newpage
	\item Aufgeschobene Rentenversicherung:\\
	\begin{minipage}[c]{8cm}
		\begin{itemize}
			\item Eintrittsalter $x$
			\item Aufschubszeit $n$
			\item Rentenbezugszeit bis zum Tod
			\item Rentenhöhe 1
		\end{itemize}
	\end{minipage}
	\begin{minipage}[c]{8cm}
		\begin{itemize}
			\item Ausgaben: 
			\[
			A(n+k)=\mathbb{1}_{\{T_x>m+k \}},\quad k=0,1,2,\dots 
			\]
			\item Einnahmen: 
			\[
			I(k)= p\mathbb{1}_{\{T_x>k \}}, \quad k=0,\dots,n-1 
			\]
		\end{itemize}
	\end{minipage}
	Versicherung ist fair, wenn $pä_{x:n\rceil}=~_{|n}ä_x$\\
	Deckungskapitalverlauf:
	\begin{equation*}
	\begin{aligned}
		m=0,\dots,n-1: \mathcal{D}_x(m) &= \sum_{k=0}^{\infty}v^{n-m+k} \mathds{P}(T_{x+m}>n-m+k) -p\sum_{k=0}^{n-m-1}v^k\mathds{P}(T_{x+m}>k)\\
		&=~_{|n-m}ä_{x+m}-pä_{x+m:n-m\rceil}\\
		m=n:\quad \mathcal{D}_x(n) &= ä_{x+n}\\
		m>n:\quad \mathcal{D}_x(m) &= ä_{x+m}
	\end{aligned}
	\end{equation*}
	\begin{center}
		\begin{tikzpicture}[line cap=round,line join=round,>=triangle 45,x=.6cm,y=.6cm]
		\draw[->,color=black] (-.5,0.) -- (5.5,0.);
		\foreach \x in {0,5}
		\draw[shift={(\x,0)},color=black] (0pt,2pt) -- (0pt,-2pt);
		\draw[->,color=black] (0.,-.5) -- (0,5.5);
		\foreach \y in {0,5}
		\draw[shift={(\y,0)},color=black] (0pt,2pt) -- (0pt,-2pt);
		\coordinate (A) at (0,0);
		\coordinate (B) at (2.5,5);
		\coordinate (C) at (5,0);
		\coordinate (a) at (1.5,.5);
		\coordinate (b) at (2,4);
		\coordinate (c) at (3,4);
		\coordinate (d) at (3.5,.5);
		\draw[cap=round]
		(A) .. controls (a) and (b) .. (B);
		\draw[cap=round]
		(B) .. controls (c) and (d).. (C);
		\draw[style=dotted] (2.5,5) -- (2.5,0) node[midway,right,sloped]{$ä_{x+m}$};
		\draw (2.5,0) node[below] {$n$};
		\end{tikzpicture}
		\captionof{figure}{Deckungskapital: Rentenversicherung}
	\end{center}
\end{enumerate}


% subsection beispiele deckungskapital (end)

Weitere Beispiele für Personenversicherungen bei denen die Ausfallzeiten nicht durch die Restlebenszeit einer einzelnen Person gegeben ist:

\subsection{Personengemeinschaften, verbundene Leben}
\label{sub:verbundene_leben}
\index{Personengemeinschaften}
\begin{itemize}
	\item $n$ Personen mit Restlebensdauer $T_1,\dots,T_n$
	\item Aus diesen wird eine Ausfallzeit der Gemeinschaft definiert durch $\Gamma=f(T_1,\dots,T_n)$ für eine geeignete Funktion $f$.
	\item \uline{Bsp:} $n=2$ $T=\min\{T_1,T_2\}=T_1\land T_2$ oder $T=\max\{T_1,T_2\}=T_1\lor T_2$
	\item \uline{Bem:} Bei unabhängigen $T_1,\dots,T_2$ kann die Verteilung von $\max\{T_1,\dots,T_n\}$ bzw. $\min\{T_1,\dots,T_n\}$ ausgerechnet werden, denn
	\begin{equation*}
	\begin{aligned}
		&\mathds{P}(\max\{T_1,\dots,T_n\}\le t)=\mathds{P}(T_1\le t,\dots,T_n\le t)=\prod_{i=1}^{n}\mathds{P}(T_i\le t)\\
		&\mathds{P}(\min\{T_1,\dots,T_n\}>t)=\mathds{P}(T_1>t,\dots,T_n>t)=\prod_{i=1}^{n}\mathds{P}(T_i>t)
	\end{aligned}
	\end{equation*}
	\item \uline{Beispiel:} Todesfallversicherung eines Ehepaares
	\begin{itemize}
		\item Eintrittsalter erste Person $x$, zweite Person $y$
		\item Laufzeit $n$ Jahre
		\item Versicherungssumme M wird fällig, wenn eine der beiden Personen stirbt (1. Tod)
		\item konst. Prämie solange wie beiden leben
	\end{itemize}
	\item \uline{Modell:}\\
	Setze $T_{xy}=T_x\land T_y$.\\
	$t(m)=M,\quad m=1,\dots,n;\qquad b(m)=p, \quad m=0,\dots,n-1;\qquad s(m)=0,\quad \forall n\in \mathds{N}_0$\\
	Dann beschreibt $\Gamma=(t,s,b,T_{xy})$ diese Versicherung.\\
	Bestimmung von $p$:\\
	\[
	A(k)=m\mathbb{1}_{\{k-1<T_{xy}<k\}}, \quad k=1,\dots,n
	\]
	\[
	I(K)=p\mathbb{1}_{\{T_{xy}<k\}}, \quad k=0,\dots,n-1
	\]
	Fair, wenn \[
	p\sum_{k=0}^{n-1}v^k\mathds{P}(T_{xy}>k)=M\sum_{k=1}^{n}v^k\mathds{P}(k-1<T_{xy}\le k) 
	\]
	Es gilt: 
	\begin{equation*}
	\begin{aligned}
		\mathds{P}(T_{xy}>k) &= \mathds{P}(T_{xy}>k~|~T_{xy}>k-1)\mathds{P}(T_{xy}>k-1)\\
		&= \mathds{P}(T_{x+k-1,y+k-1}>1)\mathds{P}(T_{xy}>k-1)\\
		&=\mathds{P}_{x+k-1}^{(1)}\mathds{P}_{y+k-1}^{(2)}\mathds{P}(T_{xy}>k-1)=\mathds{P}_{x+k-1}^{(1)}\mathds{P}_{y+k-1}^{(2)}\cdot \dots \cdot \mathds{P}_x^{(1)}\mathds{P}_y^{(2)}
	\end{aligned}
	\end{equation*}
	und
	\begin{equation*}
	\begin{aligned}
		\mathds{P}(k-1<T_{xy}\le k) &= \mathds{P}(T_{xy}\le k~|~T_{xy}>k-1)\mathds{P}(T_{xy}>k-1)\\
		&= \mathds{P}(T_{x+k-1,y+k-1}\le 1)\mathds{P}(T_{xy}>k-1)\\
		&=(1-\mathds{P}_{x+k-1}^{(1)}\mathds{P}_{y+k-1}^{(2)}\mathds{P}(T_{xy}>k-1))
	\end{aligned}
	\end{equation*}
\end{itemize}

%subsection personengemeinschaften (end)

\subsection{Konkurrierende Ausscheideursachen}
\label{sub:ausscheideursachen}
\begin{itemize}
	\item Ausfallzeit $T$
	\item mehrere konkurrierende Ausscheideursachen. Welche Ursache zum Ausscheiden führt ist zufällig und wird durch eine $\{1,\dots,m\}$-wertige Zufallsvariable $J$ beschrieben.
	\item Leistung bei Ausfall hängt von der Ausscheideursache ab
	\item Modellierung erfolgt dadurch, dass die Todesfallleistung ersetzt bzw. modifiziert wird durch Ausfallleistungen.
	\item \uline{Definition:} Sei $T$ eine $(0,\infty)$-wertige ZV und $J$ eine $\{1,\dots,m\}$-wertige ZV. Seien $(t_j)_{j=1,\dots,m},~s,~b$ Zahlungsströme.
	\item Dann heißt $\Gamma=\enbrace{(t_j)_{j=1,\dots,m},s,b,T,J}$ Personenversicherung unter $m$ konkurrierenden Risiken
	\item \uline{Interpretation:} Anfangszustand\\
	\begin{center}
		$T~\mathrel{\hat{=}}$ Verweilzeit im Anfangszustand\\
		$J~\mathrel{\hat{=}}$ zufällige Wahl einer Ausscheideursache\\
		$t_j(n)~\mathrel{\hat{=}}$ Leistung bei Ausfall in der $n$-ten Periode, wegen Ursache $j$\\
		$s(n)~\mathrel{\hat{=}}$ Leistung bei einer Verweildauer größer als $n$\\
		$b(n)~\mathrel{\hat{=}}$ Beitrag bei Ausfall nach n
	\end{center}
	\item \uline{Zahlungsströme:}
	\begin{equation*}
	\begin{aligned}
		A(n) &= \sum_{j=1}^{m}t_j(n)\mathbb{1}_{\{n-1<T\le n,~J=j \}}+s(n)\mathbb{1}_{\{T>n\}}\\
		I(n) &= b(n)\mathbb{1}_{\{T>n\}}
	\end{aligned}
	\end{equation*}
	\item \uline{Bewertung:}
	\begin{equation*}
	\begin{aligned}
		\mathds{E}V(A) &= \sum_{n=1}^{\infty}\sum_{j=1}^{m}t_j(n)v^n\mathds{P}(n-1<T\le n,~J=j)\\
		&+ \sum_{n=0}^{\infty}s(n)v^n\mathds{P}(T>n)\\
		\mathds{E}V(A) &= \sum_{n=0}^{\infty}b(n)v^n\mathds{P}(T>n)
	\end{aligned}
	\end{equation*}
	Für eine praktische Berechnung muss die Stationaritätsannahme geeignet modifiziert werden.\\
\end{itemize}

\minisec{Definition}
$\enbrace{(T_x)_{x\in \mathds{N}_0},J}$ ist stationär, falls gilt 
\[ 
\mathds{P}(T_x\le n+k,J=j~|~T_x>n)=\mathds{P}(T_{x+n}\le k,J=j)\quad \forall n\in \mathds{N}_0,~k\in \mathds{N},~j\in \{1,\dots,m\} 
\]

\minisec{Lemma}
Ist $\enbrace{(T_x)_{x\in \mathds{N}_0},J}$ stationär, so ist auch $(T_x)_{x\in \mathds{N}_0}$ stationär. 
Es gilt also 
\[ 
\mathds{P}(T_x\le n+k~|~T_x>n)=\mathds{P}(T_{x+n}\le k) 
\]

\bet{Beweis:}\\
\[ 
\mathds{P}(T_x\le n+k~|~T_x>n)= \sum_{j=1}^{m}\mathds{P}(T_x\le n+k,J=j~|~T_x>n)= \sum_{j=1}^{m}\mathds{P}(T_{x+n}\le k,J=j)= \mathds{P}(T_{x+n}\le k) 
\]

Setze $q_{x,j}=\mathds{P}(T_x\le 1,J=j)$ als W'keit eines $x$-Jährigen im nächsten Jahr wegen Ursache $j$ auszuscheiden.\\
$q_x=\mathds{P}(T_x\le 1)=\sum_{j=1}^{m}q_{x,j}$.\\
$p_x=1-q_x$ einjährige Verweildauer eines x-Jährigen.\\
Wegen der Stationaritätsannahme gilt dann: 
\[
\mathds{P}(T_x>n)=\mathds{P}(T_x>n~|~T_x>n-1)\cdot\dots \cdot \mathds{P}(T_x>1)= p_{x+n-1}\cdot \dots\cdot p_x
\]
bzw.
\begin{equation*}
\begin{aligned}
	\mathds{P}(n-1<T_x\le n,J=j) &= \mathds{P}(n-1<T_x\le n,J=j~|~T_x>n-1)\mathds{P}(T_x>n-1)\\
	&= \mathds{P}(T_{x+n-1}\le 1,J=j)\mathds{P}(T_x>n-1)\\
	&= q_{x+n-1,j}\mathds{P}(T_x>n-1)
\end{aligned}
\end{equation*}
Für eine Berechnung der Barwerte genügt es also die $q_{x,j}$ zu spezifizieren.\\

\uline{Bsp:} Invaliditätsrente:\\
\begin{itemize}
	\item Eintrittsalter $x$
	\item Grundzustand aktiv ($a$)
	\item $n$ Restlaufzeit zur gesetzlichen Rente
	\item mögliche Ausscheideursachen
	\begin{itemize}
		\item Invalidität
		\item Tod
	\end{itemize}
\end{itemize}
Bei Invalidität wird eine lebenslange Rente der Höhe $R$ gezahlt.\\
\uline{Modell:}\\
\begin{center}
	$T_x \mathrel{\hat{=}}$ Verweilzeit im Zustand $a$\\
	$\mathds{P}(T_x>k) \mathrel{\hat{=}}$ als aktiver $k$ Jahre zu überleben\\
	$J=1 \mathrel{\hat{=}}$ Invalidität\\
	$J=2 \mathrel{\hat{=}}$ Tod\\
	$t_1(k)=R\cdot ä_{x+k} \mathrel{\hat{=}}$ Leistung bei Invalidisierung im $k$-ten Jahr, Barwert des Rentenanspruchs, $k=1,\dots,n$\\
	$t_2(k)=0~\forall k\in \mathds{N}_0,~s(k)=0,~\forall k\in \mathds{N}_0,~b(k)=p,~ k=0,\dots,n-1$
\end{center}
\uline{Bewertung:}\\
\begin{equation*}
\begin{aligned}
	\mathds{E}V(A) &= R\sum_{k=1}^{n}ä_{x+k}v^k\mathds{P}(k-1<T_x\le k,J=j)\\
	&= R\sum_{k=1}^{n}ä_{x+k}v^kq_{x+k-1}\mathds{P}(T_x>k-1)\\
	\mathds{E}V(I) &= p\sum_{k=0}^{n-1}v^k\mathds{P}(T_x>k)
\end{aligned}
\end{equation*}
$i(y):=q_{y,1}$ einjährige Invalidisierungsw'keit eines $y$-Jährigen\\
$q_y^a:=q_{y,2}$ einjährige Sterbew'keit eines aktiven\\
$q_y=q_y^a+i(y)$ W'keit eines aktiven $y$-Jährigen im nächsten Jahr auszuscheiden.\\
$p_y=1-q_y$\\
% subsection ausscheideursachen (end)
% section zahlungsströme (end)
\newpage
\section{Exkurs stochastische Prozesse}
\label{sec:stoch_prozesse}

\subsection{Definitionen}
\label{sub:def}
Sei $(\Omega,\mathcal{F},\mathds{P})$ ein W'Raum. 
Sei $T\subseteq \mathds{R}$ eine Zeitparametermenge. 
Sei $(E,\mathcal{E})$ ein messbarer Raum als Zustandsraum. Eine Familie $(X_t)_{t \in T}$ von $E$-wertigen ZV'en heißt \Index{stochastischer Prozess}.\\
\marginnote{man glaubt es nicht ein W'Raum ;) }
Eine Familie $(\mathcal{F}_t)_{t\in T}$ von Unter-$\sigma$-Algebren von $\mathcal{F}$ heißt \Index{Filtration}, wenn 
\[
\mathcal{F}_s\subseteq \mathcal{F}_t~\forall s\le t,~s,t\in T
\]
$(\mathcal{F}_t)_{t\in T}$ gibt ein Informationsverlauf wieder. $\mathcal{F}_t \mathrel{\hat{=}}$ Information, die bis zum Zeitpunkt $t$ verfügbar ist.\\
$(X_t)_{t\in T}$ heißt \Index{adaptiert} bzgl. der Filtration $(\mathcal{F}_t)_{t\in T}$, falls gilt $X_t$ ist messbar bzgl. $\mathcal{F}_t~\forall t\in T$.\\ 
In der Regel: $T\subseteq \mathds{N}_0$ oder $T\subseteq [0,\infty)$, $E=\mathds{R}^d,~\mathcal{E}=\mathcal{L}(\mathds{R}^{^0d})$\\

\uline{Beispiel:}\\
Die Preisentwicklung von $d$ Finanzgütern kann man durch einen stochastischen Prozess $(X_t)_{t\in T}$ mit Werten in $\mathds{R}^d$ beschreiben.

%subsection definitionen (end)

\subsection{Das N-Perioden CRR-Modell}
\label{sub:crr_modell}
\begin{minipage}[c]{12cm}
	(Anmerkung: CRR steht für Coxe-Ross-Rubinstein)\\
	$\Omega=\{0,1\}^N,~\mathcal{F}=\mathds{P}(\Omega),~0<d<u$\\
	$Y_n:\Omega\to \mathds{R},~\omega\mapsto u^{\omega_n}d^{1-\omega_n}=\left\{\begin{array}{cl} u, & \text{falls }\omega_n=1\\ d, & \text{falls }\omega_n=0    \end{array}\right.$\\
	$S_n=Y_1Y_2\cdot\dots\cdot Y_n$ Preis nach $n$-Perioden, $(S_n)_{n=0,\dots,N}$ Verlauf einer Aktie über $N$ Perioden.\\
	Zusätzlich zur Aktie betrachtet man ein \Index{Geldmarktkonto} mit konst. periodischer Verzinsung $r$.\\
	$\begin{pmatrix} (1+r)^n \\ S_n \end{pmatrix}_{n=0,\dots,N}$ beschreibt im \Index{CRR-Modell} den Verlauf der Preise dieser beiden Basisfinanzgüter.\\
		
	Anmerkung zur Abbildung 20: bei sehr kleinen Zeitsprüngen ist es möglich anzunehmen, dass das Finanzgut nur einen kleinen Sprung nach oben oder unten machen kann (zetliche Darstellung der Black-Scholes Formel).
\end{minipage}
\begin{minipage}[t]{5cm}
	\begin{center}
	\begin{tikzcd}[column sep=small]
		& & & u^3 \\
		& & u^2 \ar{ru} \ar{rd} &\\
		& u \ar{ru} \ar{rd} & & u^2d  \\
		1 \ar{ru} \ar{rd}  &  & ud \ar{ru} \ar{rd} &\\
		& d \ar{ru} \ar{rd} & & ud^2\\
		& & d^2\ar{ru} \ar{rd} &\\
		& & & d^3 
	\end{tikzcd}
	\captionof{figure}{CRR-Modell}
	\end{center}
\end{minipage}

% subsection crr-modell (end)

\subsection{Random-Walk}
\label{sub:random_walk}
Sei $(Y_n)_{n\in \mathds{N}}$ eine Folge von iid ZV'en. Sei $Y_0$ unabhängig von $(Y_n)_{n\in \mathds{N}}$.\\
Durch $S_n=Y_0+\sum_{k=1}^{n}Y_k,~n\in \mathds{N}$ wird ein sogenannter \Index{Random-Walk} definiert.\\
Durch $S_n=Y_0\cdot \prod_{k=0}^{n}Y_k,~n\in \mathds{N}$ wird ein \bet{geometrischer Random-Walk}\index{Random-Walk!geometrischer} definiert. Die Aktie im CRR-Modell ist ein geom. Random-Walk.

% subsection random-walk (end)
\newpage
\subsection{Bedingter Erwartungswert}
\label{sub:bed_ew}
Sei $(\Omega,\mathcal{F},\mathds{P})$ ein W'Raum. $G$ Unter-$\sigma$-Algebra von $\mathcal{F}$. $X:\Omega\to \mathds{R}$ sei messbar bzgl. $\mathcal{F}$ und es existiert  $\mathds{E}X$.\\
Dann heißt $Z:\Omega\to\mathds{R}$ eine Version des \bet{bedingten Erwartungswertes}\index{bedingter Erwartungswert} von $X$ bzgl. $G$, wenn gilt:\\
\begin{enumerate}[(i)]
	\item $Z$ ist messbar bzgl. $G$
	\item $\int_{A}Z\mathrm{d}\mathds{P}=\int_{A}X\mathrm{d}\mathds{P}\quad \forall A\in G$
\end{enumerate}
Schreibweise $Z=\mathds{E}(X~|~G)$; 
ist $G=\sigma(Y)$ für eine ZV $Y$, so schreibt man $\mathds{E}(X~|~G)=\mathds{E}(X~|~Y)$

% subsection bed ew (end)

\subsection{Existenz \& Eindeutigkeit}
\label{sub:ex&eind}
Gegeben seien die Bezeichnungen von \hyperref[sub:bed_ew]{3.4}. Dann existiert der bedingte Erwartungswert von $X$ bzgl. $G$ und ist $\mathds{P}$-f.s. eindeutig bestimmt, d.h. erfüllen $Z_1,Z_2$ die Bedingungen (i), (ii) aus 3.4, so gilt $Z_1=Z_2,~\pfs$.\\

\bet{Beweis:}\\
\uline{Existenz:} 1.Fall: $X\ge 0$\\
$\mu(A)=\int_AX\dint\mathds{P},~A\in G$ definiert ein $\sigma$-endliches Maß auf $(\Omega,G)$ mit $\mu \ll \mathds{P}$.\\
Satz von Radon-Nikodym liefert ein $G$-messbares $Z$ mit $\mu(A)=\int_A Z\dint\mathds{P}$. 
Also $Z=\mathds{E}(X~|~G)~\forall A\in G$\\
2.Fall: $X=X^+-X^-$\\
$\mathds{E}(X^+~|~G),~\mathds{E}(X^-~|~G)$ existiert nach Fall 1.
\begin{equation*}
\begin{aligned}
	\int\limits_AX\dint\mathds{P} &= \int\limits_AX^+\dint\mathds{P}-\int\limits_AX^-\dint\mathds{P}\\
	&= \int\limits_A\mathds{E}(X^+~|~G)\dint\mathds{P}-\int\limits_A\mathds{E}(X^-~|~G)\dint\mathds{P}\\
	&= \int\limits_A\mathds{E}(X^+~|~G)-\mathds{E}(X^-~|~G)\dint\mathds{P}\quad A\in G
\end{aligned}
\end{equation*}
Also ist $\mathds{E}(X^+~|~G)-\mathds{E}(X^-~|~G)$ der bedingte Erwartungswert von $X$ bzgl. $G$.\\
\uline{Eindeutigkeit:} $Z_1,Z_2$ bedingte EW.
Sei $A=\{Z_1-Z_2>\frac{1}{n}\}$, ist $G$ messbar.
\begin{equation*}
\begin{aligned}
	&\int\limits_{A}Z_1-Z_2\dint\mathds{P}\stackrel{Z_1,Z_2\text{ EW}}{=} \int\limits_AX\dint\mathds{P}-\int\limits_AX\dint\mathds{P}=0\\
	&\Rightarrow \mathds{P}(Z_1-Z_2>\frac{1}{n})=0~\forall n\in \mathds{N}\\
	&\Rightarrow \mathds{P}(Z_1-Z_2>0)=0\\
	&\text{Genauso folgt } \mathds{P}(Z_2-Z_1>0)=0
\end{aligned}
\end{equation*}
\hfill $\square$

\minisec{Beispiel}
$X_1,\dots,X_n $ iid ZV'en, sei $S_n:=\sum_{i=1}^{n}X_i$.\\
\uline{Frage:} Was ist $\mathds{E}(X_1~|~S_n)$?\\
Vermutung: $\mathds{E}(X_1~|~S_n)=\mathds{E}(X_2~|~S_n)=\dots=\mathds{E}(X_n~|~S_n)$\\
Dann gilt: 
\[
n\mathds{E}(X_1~|~S_n)=\sum_{i=1}^{n}\mathds{E}(X_i~|~S_n)=\mathds{E}\enbrace{\sum_{i=1}^{n}X_i~|~S_n}=\mathds{E}(S_n~|~S_n)=S_n
\]
$\Rightarrow \mathds{E}(X_1~|~S_n)=\frac{1}{n}S_n$\\
Wieso ist $\mathds{E}(X_1~|~S_n)=\mathds{E}(X_k~|~S_n)$?\\

\zz: $\int_{\{S_n\in B\}}X_1\dint\mathds{P}= \int_{\{S_n\in B\}}X_k\dint\mathds{P}\quad \forall B\in \mathcal{L}$\\

Da die $X_1,\dots,X_n$ stochastisch unabhängig sind, ist die Verteilung durch $\mathds{P}^{(X_{\pi(1)},\dots,X_{\pi(n)})}$ gegeben. 
Daher folgt 
\[ 
\int\limits_{\penbrace{x\in \mathds{R}^n~|~\sum_{i=1}^{n}x_i \in B}}X_1\mathds{P}^{(X_{\pi(1)},\dots,X_{\pi(n)})}\mathrm{d}(x_1,\dots,x_n)
\]
Betrachte Permutation $\pi$ mit $\pi(1)=k$, dann folgt 
\[
=\int\limits_{\{S_n\in B\}}X_k\dint\mathds{P} 
\]

% subsection existenz und eindeutigkeit (end)

\subsection{Faktorisierter bedingter Erwartungswert}
\label{sub:fakt_ew}
Sei $X:(\Omega,\mathcal{F},\mathds{P}) \to (\mathds{R},\mathcal{L})$ eine Zufallsvariable, $Y:(\Omega,\mathcal{F},\mathds{P}) \to (E,\mathcal{E})$ messbar.\\ 
Sei $G=\sigma(Y)$. 
Dann gilt:\\
Eine Zufallsvariable $Z:\Omega\to \mathds{R}$ ist $G$-messbar genau dann, wenn es eine $\mathcal{E}$-messbare Abbildung $h:E\to \mathds{R}$ gibt mit $Z=h\circ Y$

\begin{minipage}[c]{7cm}
	\begin{center}
	\begin{tikzcd}[column sep=small]
		\Omega \ar{r}{Y} \ar{rd}[below,left]{\E(X~|~Y)} & (E,\mathcal{E}) \ar{d}[right]{h}\\
		& (\R,\mathcal{L})
	\end{tikzcd}
	\captionof{figure}{fakt. bed. Erwartungswert}
	\end{center}
\end{minipage}
\begin{minipage}[c]{5cm}
	\[h(y)=\E(X~|~Y=y)\]
\end{minipage}

Falls $Z$ eine Version der bedingten Erwartung von $X$ gegeben $Y$ ist, so gibt es also ein $h:E\to \mathds{R}$ mit $Z=h\circ Y$.\\
Schreibweise $h(y)=\mathds{E}(X~|~Y=y)$, $h$ heißt Version der \bet{faktorisierten}\index{bedingter Erwartungswert!faktorisierte bed. EW} bedingten Erwartung von $X$ gegeben $Y$.\\
Sind $h_1,h_2$ Versionen der bedingten Erwartung von $X$ gegeben $Y$, so gilt 
\[
h_1(y)=h_2(y) \text{ für }\mathds{P}^y-\text{f.a. }y\in E 
\]
$y\mapsto \mathds{E}(X~|~Y=y)$ ist eindeutig festgelegt für $\mathds{P}^y$-f.a. $y$ durch 
\[
\mathds{E}X\mathbb{1}_{\{Y\in B\}}=\int\limits_{\mathclap{\{Y\in B\}}}\mathds{E}(X~|~Y)\dint \mathds{P} 
\]
\[
\int\limits_{\mathclap{\{Y\in B\}}}hY\dint\Pw = \int\limits_B h(y) \Pw^y(\dint y) \quad \forall B\in \mathcal{L} 
\]

Ausrechnen des bedingten Erwartungswertes erfolgt häufig durch Spezifikation der bedingten Verteilung.

% subsection fakt. EW (end)

\subsection{Stochastischer Kern}
\label{sub:stoch_kern}
Seien $(\Omega,\mathcal{F}),(E,\mathcal{E})$ messbare Räume. 
Ein \Index{stochastischer Kern} ist eine Abbildung $K:E\times \F \to [0,1]$ mit folgenden Eigenschaften:
\begin{enumerate}[(i)]
	\item $K(y,.)$ ist ein W-Maß für alle $y\in E$
	\item $K(.,A)$ ist messbar für alle $A\in \mathcal{F}$
\end{enumerate}

%sub end

\subsection{bedingte W'keit und bedingte Verteilungen}
\label{sub:bed_wk_vert}
$(\Omega,\mathcal{F},\mathds{P})$ ein W'Raum, $\G$ Unter-$\sigma$-Algebra von $\mathcal{F}$. 
Für jedes $\Gamma\in \mathcal{F}$ heißt $\E(\mathbb{1}_\Gamma~|~\G)$ \Index{bedingte Wahrscheinlichkeit} von $\Gamma$ gegeben $\G$\\
Schreibweise: 
\[
\Pw(\Gamma~|~\G):=\E(\mathbb{1}_\Gamma~|~\G) 
\]
Seien $X:(\Omega,\mathcal{F})\to (E_1,\mathcal{E}_1)$, $Y:(\Omega,\mathcal{F})\to (E_2,\mathcal{E}_2)$ messbare Abbildungen.\\
Die \Index{bedingte Verteilung} von $X$ gegeben $Y$ ist ein stochastischer Kern $K:E_2\times \mathcal{E}_1\to [0,1]$ derart, dass $y\mapsto K(y,A)$ eine Version der faktorisierten bedingten Erwartung von $\Pw(X\in A~|~Y)$ ist für alle $A\in\mathcal{E}_1$.\\
Schreibweise: 
\[
K(y,A)=\Pw(X\in A~|~Y=y) 
\]
Durch Erweiterungsschluss kann man zeigen 
\[
\E(f(X)~|~Y=y)= \int f(x) K(Y,\dint x) 
\]
für jedes messbare $f:E_1\to(\Omega,\mathcal{L})$.
%sub end

\subsection{Beispiel: diskrete Zufallsvariablen}
\label{sub:bsp_dis_zv}
Sei $(E_1,\mathcal{E}_1)$ messbar, $E_2$ abzählbar
\begin{equation*}
\begin{aligned}
	\mathcal{E}_2 &= \mathcal{P}(E_2)\\
	X &: (\Omega,\mathcal{F})\to (E_1,\mathcal{E}_1) \text{ messbar}\\
	Y &: (\Omega,\mathcal{F})\to (E_2,\mathcal{E}_2)\text{ messbar}
\end{aligned}
\end{equation*}
Die bedingte Verteilung von $X$ gegeben $Y=y$ ist definiert durch 
\[
\Pw(X\in A~|~Y=y)=\frac{\Pw(X\in A,Y=y)}{\Pw(Y=y)}\quad \forall y\in E_2 \text{ mit } \Pw(Y=y)>0 
\]
Definiere den stochastischen Kern $K: E_2\times \mathcal{E}_1\to [0,1]$\\
$K(y,A)=\left\{\begin{array}{cl} 
\frac{\Pw(X\in A,Y=y)}{\Pw(Y=y)} & \forall A\in \mathcal{E}_1,y\in E_2 \text{ mit }\Pw(Y=y)>0\\
\text{irgendwie} & \text{d.h. wähle W'Maß auf } (E_1,\mathcal{E}_1)\\
& \text{ und setze } K(y,A)=\mu(A)~\forall A\in \mathcal{E}_1  \end{array}\right.$\\

Dann ist $K$ die bedingte Verteilung von $X$ gegeben $Y$.
%sub end
\newpage

\subsection{Lebesgue-Dichten}
\label{sub:lebesgue_dichten}
Sei $(X,Y)$ ein zweidimensionaler Zufallsvektor mit \Index{Lebesgue-Dichte} $h:\R^2\to \R_{\ge 0}$\\
\begin{equation*}
\begin{aligned}
	\Pw(X\in A,Y\in B) &= \int\limits_{A\times B}h(x,y)\lambda^2(\dint x,\dint y)\\
	&= \int\limits_A\int\limits_B h(x,y) \lambda(\dint y)\lambda(\dint x)\quad \forall B\in \mathcal{L}
\end{aligned}
\end{equation*}
Setze $f(y)=\int_{\R}h(x,y)\dint x$.\\
Dann ist $f$ messbar wegen Fubini und die Lebesgue-Dichte von $Y$, denn:
\begin{equation*}
\begin{aligned}
	\Pw(Y\in B) &= \Pw(Y\in B,X\in \R)= \int\limits_{\R\times B}h(x,y)\dint(x,y)\\
	&=\int\limits_B\underbracket{\int\limits_\R h(x,y)\dint x}_{f(y)}\dint y \quad \forall B\in \mathcal{L}
\end{aligned}
\end{equation*}
Definiere den stochastischen Kern $K:\R\times \mathcal{L}\to[0,1]$ durch
$\left\{\begin{array}{cl} \int_A \frac{h(x,y)}{f(y)}\lambda(\dint x) & \text{falls } f(y)>0\\ \text{irgendwie}    \end{array}\right.$
\[ 
\frac{\Pw(X\in A,Y=y)}{\Pw(Y=y)}\stackrel{X\text{ diskret}}{=} \frac{\sum_{x\in A}\Pw(X=x,Y=y)}{\Pw(Y=y)} 
\]
Dann ist $K$ eine bedingte Verteilung von $Y$,denn
\begin{equation*}
\begin{aligned}
	\Pw(X\in A,Y\in B) &= \int\limits_{A\times B} h(x,y)\lambda^2(\dint x, \dint y)\\
	&= \int\limits_B\int\limits_Ah(x,y)\dint x \dint y\\
	&= \int\limits_{B\cap\{f>0\}}\int\limits_A \frac{h(x,y)}{f(y)}\dint x \cdot f(y)\dint y\\
	&= \int\limits_{\mathclap{B\cap\{f>0\}}}K(y,A)f(y)\dint y\\
	&= \int\limits_BK(y,A)\Pw^y(\dint y)
\end{aligned}
\end{equation*}
Also gilt: $\omega\mapsto K(Y(\omega),A)$ ist eine Version von $\Pw(X\in A,Y)$\\
$\Rightarrow y\mapsto K(y,A)$ ist eine Version von $\Pw(X\in A~|~Y=y)$.

%sub end
\newpage

\subsection{Eigenschaften}
\label{sub:eigenschaften}
Sei $(\Omega,\mathcal{F},\mathds{P})$ W'Raum, $\G\subseteq \mathcal{F}$ Unter-$\sigma$-Algebra. 
Seien $X,X_1,X_2$ integrierbare ZV'en. Dann gilt:
\begin{enumerate}[(i)]
	\item $\E(\alpha X_1+\beta X_2~|~\G)=\alpha\E(X_1~|~\G)+\beta \E(X_2~|~\G)$ für alle $\alpha,\beta$
	\item $X_1\le X_2 \Rightarrow \E(X_1~|~\G)\le \E(X_2~|~\G)$
	\item Sei $Z$ eine $\G$-messbare ZV, so dass $\E ZX$ existiert. 
	Dann gilt 
	\[ 
	\E(ZX~|~\G)=Z\E(X~|~\G)
	\]
	\item Sind $G_1,G_2$ Unter-$\sigma$-Algebren mit $G_1\subseteq G_2$, so folgt 
	\[
	\E(X~|~G_1)=\E(\E(X~|~G_2)~|~G_1)\quad \text{ '\Index{Tower Property}'} 
	\]
	\item Sind $X$ und $\G$ stoch. unabhängig, so gilt 
	\[
	\E(X~|~\G)=\E X 
	\]
	\item Sind $Z_1,Z_2$ stoch. unabh. ZV'en mit Werten in $(E_1,\mathcal{E}_1),(E_2,\mathcal{E}_2)$ und ist $h:E_1\times E_2 \to (\R,\mathcal{L})$ messbar mit ex. $\E h(Z_1,Z_2)$, so gilt 
	\[
	\E(h(Z_1,Z_2)~|~Z_2=z_2)= \E h(Z_1,z_2)
	\]
	für $\Pw^{Z_2}$-f.a. $z_2\in E_2$
	\item $\E \E(X~|~\G)=\E X$
\end{enumerate}

\bet{Beweis:}\\
(i), (ii) einfach (\checkmark).\\
(iii): 1.Fall: $Z\ge 0,~X\ge0$\\
Ist $Z=\mathbb{1}_G$ mit $G\in \G$, so gilt für alle $A\in \G$ 
\begin{equation*}
\begin{aligned}
	\int\limits_A ZX\dint \Pw &=\int\limits_A \mathbb{1}_G X\dint \Pw =\int\limits_{A\cap G}X\dint\Pw\\
	&= \int\limits_{A\cap G}\E(X~|~\G)\dint\Pw =\int\limits_A Z\E(X~|~\G)\dint\Pw\\
	&\Rightarrow \E(ZX~|~\G)=Z\E(X~|~\G)
\end{aligned}
\end{equation*}
Ist $Z=\sum_{i=1}^{n}\alpha_i \mathbb{1}_{G_i},~ G_i\in \G,~\alpha_i\ge 0$ so gilt wegen (i)
\begin{equation*}
\begin{aligned}
	\E(ZX~|~\G)&=\E\enbrace{\sum_{i=1}^{n}\alpha_i\mathbb{1}_{G_i}X~|~\G}= \sum_{i=1}^{n}\alpha_i\E(\mathbb{1}_{G_i}X~|~\G)\\
	&= \sum_{i=1}^{n}\alpha_i\mathbb{1}_{G_i}\E(X~|~\G)=Z\E(X~|~\G)
\end{aligned}
\end{equation*}
Ist $Z\ge0$, so ex. eine Folge von Treppenfunktionen $(Z_n)_{n\in \N}$ mit $0\le Z_1\le Z_2\le \dots \uparrow Z \Rightarrow Z_nX\uparrow ZX$.\\
Für jedes $A\in G$ folgt mit monotoner Konvergenz 
\begin{equation*}
\begin{aligned}
	\int\limits_A ZX\dint\Pw &= \E ZX\mathbb{1}_A = \lim\limits_{n\to\infty}\E Z_nX\mathbb{1}_A =\lim\limits_{n\to\infty} \E\E(Z_nX~|~\G)\mathbb{1}_A\\
	&=\lim\limits_{n\to \infty}\E Z_n\E(X~|~\G)\mathbb{1}_A=\E Z\E(X~|~\G)\mathbb{1}_A\\
	&\Rightarrow \E(ZX~|~\G)=Z\E(X~|~\G)
\end{aligned}
\end{equation*}
Es ex. $\E ZX$.\\ $ZX=U-V$ mit $U=Z^+X^++Z^-X^-\ge0,~V=Z^+X^-+Z^-X^+\ge0$
\begin{equation*}
\begin{aligned}
	(Z^+-Z^-)(X^+-X^-)= Z^+X^++Z^-X^- -(Z^-X^++X^-Z^+)
\end{aligned}
\end{equation*}
Also \begin{equation*}
\begin{aligned}
	\E(ZX~|~\G) &= \E(U~|~\G)-\E(V~|~\G)= \E(Z^+X^+~|~\G)+\E(Z^-X^-~|~\G)-\big(\E(Z^-X^+~|~\G)+\E(Z^+X^-~|~\G)\big)\\
	&= Z^+\E(X^+~|~\G)-Z^-\E(X^+~|~\G)-\big(Z^+\E(X^-~|~\G)-Z^-\E(X^-~|~\G)\big)\\
	&= Z(\E(X^+~|~\G)-\E(X^-~|~\G)) = Z\E(X~|~\G)
\end{aligned}
\end{equation*}

(iv): Sei $A\in G_1$.
\begin{equation*}
\begin{aligned}
	&\int\limits_A \E(X~|~G_2)\dint\Pw \stackrel{A\in G_2}{=} \int\limits_A X\dint\Pw \stackrel{A\in G_1}{=} \int\limits_A \E(X~|~G_1)\dint\Pw\\
	&\Rightarrow \E(\E(X~|~G_2)~|~G_1)=\E(X~|~G_1)
\end{aligned}
\end{equation*}

(vi): \zz~$\E(h(Z_1,Z_2)~|~ Z_2=z_2)=\E h(Z_1,z_2)$\\
Für $B\in \mathcal{E}_2$ gilt:
\begin{equation*}
\begin{aligned}
	&\int\limits_{\{Z_2\in B\}} h(Z_1,Z_2)\dint\Pw = \E h(Z_1,Z_2)\mathbb{1}_{\{Z_2\in B,\Z_1\in E_1 \}}\\
	&=\int\limits_B\int\limits_{E_1}h(z_1,z_2)\Pw^{Z_1}(\dint z_1)\Pw^{Z_2}(\dint z_2)
\end{aligned}
\end{equation*}
da $\int_{E_1}h(z_1,z_2)\Pw^{Z_1}(\dint z_1)= \E h(Z_1,z_2)$ für $\Pw^{Z_2}$-f.a. $z_2$, ist 
\[
\E(h(Z_1,Z_2)~|~ Z_2=z_2)=\E h(Z_1,z_2) \text{ für }\Pw^{Z_2}-\text{f.a. } z_2 
\]
%sub end

\subsection{Bestapproximation}
Sei $(\Omega,\mathcal{F},\mathds{P})$, $X$ Zv mit $\E X^2<\infty$ $\G\subseteq \mathcal{F}$ Unter-$\sigma$-Algebra,
\[
L_2(\Omega,\mathcal{F},\mathds{P})= \{Y:\Omega\to \R~|~Y \text{ ist }\mathcal{F}-\text{messbar und }\E Y^2<\infty \} 
\]
\[
L_2(\Omega,\mathcal{G},\mathds{P})= \{Y:\Omega\to \R~|~Y \text{ ist }\G-\text{messbar und }\E Y^2<\infty \} 
\]
\[
L_2(\Omega,\mathcal{G},\mathds{P})\subseteq L_2(\Omega,\mathcal{F},\Pw)
\]
Durch $\sprod{Y}{Z}:=\E YZ $ wird ein Skalarprodukt auf $L_2(\Omega,\mathcal{F},\Pw)$ definiert.
Für $X\in L_2(\Omega,\mathcal{F},\mathds{P})$ ist $\hat{X}:= \E(X~|~\mathcal{G})$ die \Index{Orthogonalprojektion} auf $L_2(\Omega,\mathcal{F},\mathds{P})$, d.h. $\hat{X}\in L_2(\Omega,\mathcal{G},\mathds{P})$
\[
\norm{X-\hat{X}}_2^2=\inf\norm{X-Z}_2^2,~Z\in L_2(\Omega,\G,\Pw) 
\]

$\norm{Y}_2^2=\sprod{Y}{Y}$ für alle $Y\in L_2(\Omega,\mathcal{F},\mathds{P})$. $L_2(\Omega,\mathcal{G},\mathds{P})$ ist ein abgeschlossener Teilraum.\\
\begin{center}
\begin{tikzpicture}[line cap=round,line join=round,>=triangle 45,x=.4cm,y=.4cm]
\draw[->,color=black] (-1,0.) -- (10,0.);
\draw[->,color=black] (0.,-1) -- (0.,6.3);
\draw [shift={(3.90046420031,2.61888310592)},color=green,fill=green,fill opacity=0.1] (0,0) -- (-146.121460195:0.6) arc (-146.121460195:-56.1214601949:0.6) -- cycle;
\draw (3.90046420031,2.61888310592)-- (4.61409199606,1.55603319736);
\fill[color=green,fill=green,fill opacity=0.1] (3.83238555423,2.27256999327) circle (0.02);
\draw (5,3.3) node[anchor=north west] {\tiny$L_2(\Omega,\G,\Pw)$};
\draw (5,-.1) node[anchor=north west] {\tiny$\R\mathrm{\hat{=}}L_2(\Omega,\F,\Pw)$};
\draw (-1.4,-0.94)-- (7.20,4.83);
\draw [fill=blue] (3.90,2.618) circle (1pt);
\draw[color=blue] (3.9,3.2) node {\tiny$\hat{X}$};
\draw [fill=blue] (4.614,1.556) circle (1pt);
\draw[color=blue] (4.4,1.6) node[below] {\tiny$X$};
\end{tikzpicture}
\captionof{figure}{Orthogonalprojektion auf dem $L_2(\Omega,\G,\Pw)$}
\end{center}
\bet{Beweis:}\\
Es gilt $X=\hat{X}+X-\hat{X}$. \zz $X-\hat{X}\perp Z$ für alle $Z\in L_2(\Omega,\mathcal{G},\mathds{P})$\\
Eigenschaften des bedingten Erwartungswertes implizieren:
\begin{equation*}
\begin{aligned}
	\sprod{\mathbb{1}_A}{X }&= \int \mathbb{1}_A X\dint \Pw=\int \mathbb{1}_A\hat{X}\dint \Pw= \sprod{\mathbb{1}_A}{\hat{X}}~\forall A\in \mathcal{G}\\
	&\Rightarrow \sprod{\sum_{i=1}^{n}\alpha_i\mathbb{1}_{A_i}}{X}=\sprod{\sum_{i=1}^{n}\alpha_i\mathbb{1}_{A_i}}{\hat{X}}~\forall A_i\in \mathcal{G},~\alpha_i\in \R\\
	&\Rightarrow \sprod{Z}{X}=\sprod{Z}{\hat{X}}~\forall Z\in L_2(\Omega,\G,\Pw)
\end{aligned}
\end{equation*}
da $\sprod{.}{X}$ bzw. $\sprod{.}{\hat{X}}$ stetig sind. $\Rightarrow X-\hat{X}\perp L_2(\Omega,\G,\Pw)$\\
Mit Pythagoras folgt 
\[
\norm{X-Z}_2^2=\norm{X-\hat{X}+\hat{X}-Z}_2^2=\norm{X-\hat{X}}_2^2+\norm{\hat{X}-Z}_2^2\ge\norm{X-\hat{X}}_2^2 
\]
\hfill $\square$
%sub end

\subsection{Martingale}
\label{sub:martingale}
Sei $T$ eine Zeitparametermenge. 
$(\F_t)_{t\in T}$ eine Filtration und $(M_t)_{t\in T}$ ein \Index{adaptiv-stochastischer Prozess}.\\
$M=(M_t)_{t \in T}$ heißt \Index{Martingal}, falls gilt:
\begin{enumerate}[(i)]
	\item $\E\abs{M_t}<\infty~\forall t\in T$
	\item $\E(M_t~|~\F_s)=M_s~\forall s,t\in T,~s\le t$
	\marginnote{$\E(M_t~|~\F_s)$ Schätzung der zukünftigen Zahlung eines Spielers}
\end{enumerate}
$M$ heißt \bet{Submartingal}\index{Martingal!Sub-}, falls gilt:
\begin{enumerate}[(i)]
	\item $\E\abs{M_t}<\infty~\forall t\in T$
	\item $\E(M_t~|~\F_s)\ge M_s~\forall s,t\in T,~s\le t$
	\marginnote{ist günstig weiter zu spielen}
\end{enumerate}
$M$ heißt \bet{Supermartingal}\index{Martingal!Super-}, falls gilt:
\begin{enumerate}[(i)]
	\item $\E\abs{M_t}<\infty~\forall t\in T$
	\item $\E(M_t~|~\F_s)\le M_s~\forall s,t\in T,~s\le t$
	\marginnote{jetzt sollte man gehen}
\end{enumerate}
%sub end

\subsection{Beispiele Martingale}
\label{sub:bsp_martingale}
\begin{enumerate}[(i)]
	\item \uline{Random-Walk:}\index{Random-Walk}\\
	$S_n=S_0+\sum\limits_{i=1}^{n}X_i$, $S_0$ unabhängig von $(X_i)_{i\in \N}$ und iid, $\E\abs{X_i}<\infty,~\E\abs{S_0}<\infty$ und $\F_n=\sigma(S_0,S_1,S_2,\dots,S_n)=\sigma(S_0,X_1,\dots,X_n)$.\\
	Dann gilt:
	\begin{equation*}
	\begin{aligned}
		&\E(S_{n+1}~|~\F_n) = \E(S_n+X_{n+1}~|~\F_n) = \E(S_n~|~\F_n)+\E(X_{n+1}~|~\F_n)\\
		&\stackrel{S_n~\F_n\text{-messbar}}{=} S_n+\E(X_{n+1}~|~\F_n)\stackrel{\tiny X_{n+1}\text{ unabh. von }\F_n}{=}S_n+\E X_{n+1}\\
		&\left\{\begin{array}{lc} > & \\ = & S_n \\ < &   \end{array}\right. \Leftrightarrow \begin{array}{ccc}  & > & \\ \E X_{n+1} & = & 0\\ & < & \end{array}
	\end{aligned}
	\end{equation*}
	\item \uline{geometrischer Random-Walk:}\\
	$S_n=S_0\cdot\prod_{i=1}^{n}X_i,~(X_i)_{i\in \N}$ iid, $\E\abs{X_i}<\infty$, $\F_n=\sigma(S_0,X_1,\dots,X_n)$\\
	Dann gilt:
	\begin{equation*}
	\begin{aligned}
		&\E(S_{n+1}~|~\F_n) = \E(S_n\cdot X_{n+1}~|~S_0,\dots,S_n)\\
		&\stackrel{S_n~\F_n\text{-messbar}}{=} S_n\cdot\E(X_{n+1}~|~S_0,\dots,S_n)=S_n\cdot\E X_{n+1}
	\end{aligned}
	\end{equation*}
	$(S_n)$ ist ein Martingal genau dann, wenn $\E X_i=1~\forall i\in \N$, für Submartingal bzw. Supermartingal verschiedene Fälle betrachten.
\end{enumerate}
%sub end

\subsection{Stopzeit}
\label{sub:stopzeit}
Sei $(\F_t)_{t\in T}$ eine Filtration, $\tau:\Omega\to T\cup \{+\infty\}$ heißt \Index{Stopzeit}, falls 
\[
\{\tau\le t\}\in \F_t
\]

Stopzeiten kann man als Verkaufsstrategie interpretieren. Die Entscheidung über $t$ hinaus fortzufahren. 
$\{t\in T\}$ darf nur von der bis $t$ verfügbaren Informationen abhängen.

Beispiel: $(S_n)_{n\in \N}$ reellwertiger stoch. Prozess $\F_n=\sigma(S_0,\dots,S_n)~\forall n\in \N$. $\tau=\inf\{n\in \N_0~|~S_n>a\}$ ist eine Stopzeit, denn $\{\tau>n\}=\{S_0\le a,\dots,S_n\le a\}\subseteq \F_n$.

\begin{center}
	\begin{tikzpicture}[line cap=round,line join=round,>=triangle 45,x=.7cm,y=.5cm]
	\draw[->,color=black] (-1.,0.) -- (7.5,0.);
	\foreach \x in {2,4,7}
	\draw[shift={(\x,0)},color=black] (0pt,2pt) -- (0pt,-2pt);
	\draw[->,color=black] (0.,-1.) -- (0.,5.);
	\foreach \y in {4.5}
	\draw[shift={(0,\y)},color=black] (0pt,2pt) -- (0pt,-2pt);
	\draw (0,4.5) node [left] {$a$};
	\draw (4,0) node [below] {$\tau$};
	\draw (7,0) node [below] {$N$};
	\draw (0,2)--(2,3.5);
	\draw (2,3.5)--(2.5,2.7);
	\draw (2.5,2.7)--(4,3.8);
	\draw (4,3.8)--(5,2);
	\draw (5,2)--(6,4.5);
	\draw (6,4.5)--(7,2.5);
	\end{tikzpicture}
	\captionof{figure}{Stopzeiten}
\end{center}

$\tau=\inf\{0\le k\le N~|~S_k=\max\{S_0,\dots,S_n\} \}$ ist keine Stopzeit, da zur Stopentscheidung in die Zukunft geschaut werden muss.
%sub end

\subsection{Martingal als faires Glücksspiel}
Sei $(\F_n)_{n\in \N_0}$ eine Filtration und $(M_n)_{n\in \N_0}$ adaptierter stoch. Prozess mit $\E\abs{M_n}<\infty$ für alle $n\in \N_0$.

$M_n\mathrm{\hat{=}}$ Auszahlung, die ein Spieler erhält, wenn er das Spiel zum Zeitpunkt $n$ beendet.\\
Stopzeiten $\mathrm{\hat{=}}$ Strategien, die ein Spieler verwirklichen kann.\\
$\tau$ ist eine beschränkte Stopzeit, falls es ein $N\in \N$ ex. mit $\tau\le N$.\\
Beschränkte Stopzeiten $\mathrm{\hat{=}}$ real umsetzbare Strategien.\\

\minisec{Satz:}
Es gilt: $(M_n)_{n\in \N_0}$ ist ein Martingal genau dann, wenn 
\[
\E M_{\tau}=\E M_0 \text{ für jede beschränkte Stopzeit }\tau 
\]
d.h. durch Spielen des Glücksspiels kann sich ein Spieler im Mittel weder verbessern noch verschlechtern (faires Glücksspiel).\\

\bet{Beweis:}\\
\uline{$\Rightarrow$:} 
Sei $\tau$ Stopzeit mit $\tau\le N$.
\begin{equation*}
\begin{aligned}
	\E M_\tau &=\E \sum_{n=0}^{N}M_\tau \mathbb{1}_{\{\tau=n\}}= \E \sum_{n=0}^{N}M_n \mathbb{1}_{\{\tau=n\}}\\ 
	&=\sum_{n=0}^{N}\E M_n \mathbb{1}_{\{\tau=n\}}= \sum_{n=0}^{N}\E\E(M_N~|~\F_n) \mathbb{1}_{\{\tau=n\}}\\
	&= \sum_{n=0}^{N}\E\E(M_N\mathbb{1}_{\{\tau=n\}}~|~\F_n)= \sum_{n=0}^{N}\E M_N\mathbb{1}_{\{\tau=n\}}\\
	&=\E\sum_{n=0}^{N}M_N \mathbb{1}_{\{\tau=n\}}= \E M_N= \E M_0
\end{aligned}
\end{equation*}
Also: $\E M_\tau=\E M_0$.\\

\uline{$\Leftarrow$:} 
Sei $n<m$. 
\zz~$\E(M_m~|~\F_n)=M_n$, d.h. $\int_A M_m\dint \Pw = \int_A M_n\dint\Pw$ für alle $A\in \F_n$.\\
Zu $A\in \F_n$ definiere Stopzeit:
\[
\tau_A(\omega)=\left\{\begin{array}{cl}  m, & \omega\in A\\ n, & \omega \in A^c   \end{array}\right. 
\]
$\tau_A=m\mathbb{1}_A+n\mathbb{1}_{A^c}$\\
$\tau_A$ ist eine beschränkte Stopzeit. 
Es gilt also:
\begin{equation*}
\begin{aligned}
	\E M_0 &=\E M_{\tau_A}= \E M_m\mathbb{1}_A+ \E M_n \mathbb{1}_{A^c}\\
	&= \E  M_m\mathbb{1}_A+ \E M_n- \E M_n \mathbb{1}_{A}
\end{aligned}
\end{equation*}
Weiter gilt mit $\tau_A = n$ 
\[
\E M_0=\E M_{\tau_A} =\E M_n 
\]
Einsetzen  liefert 
\[
\E M_m\mathbb{1}_A=\E M_n\mathbb{1}_A 
\]
\hfill $\square$
%sub end

\subsection{Optional Sampling}
\label{sub:opt_sampling}
Frage: Wann gilt $\E M_\tau=\E M_0$, wenn $M$ ein Martingal ist. 
Für beschränkte Stopzeiten klar. 
Für unbeschränkte Stopzeiten braucht man Voraussetzungen.\\
\bet{Beispiel:} Irrfahrt auf $\Z$\\
$S_n=\sum_{i=1}^{n}X_i,~ (X_i)_{i\in \N}$ iid, $\Pw(X_i=1)=\frac{1}{2}=\Pw(X_i=-1)~, \tau=\inf\{n\in \N_0~|~S_n=1\}$\\
%grafik
Es gilt: $\Pw(\tau<\infty)=1$ $\Pw$-f.s. $S_\tau=1\Rightarrow \E S_\tau=1\not=\E S_0=0$.

Antwort liefert das \Index{Optional-Sampling Theorem}.\\
\minisec{Satz:}
Sei $(M_n)_{n\in \N_0}$ ein Martingal bzgl. einer Filtration $(\F_n)_{n\in \N_0}$. 
Sei $\tau$ eine Stopzeit mit den folgenden Eigenschaften:
\begin{enumerate}[(i)]
	\item $\Pw(\tau<\infty)=1$
	\item $\E\abs{M_\tau}<\infty$
	\item $\E\abs{M_n}\mathbb{1}_{\{\tau>n\}} \stackrel{n\to\infty}{\to}0$
\end{enumerate}
Dann gilt: 
\[
\E M_\tau=\E M_0 
\]

\bet{Beweis:}\\
Approximiere $\tau$ durch beschränkte Stopzeiten $\tau\land m$. 
Es gilt $\E M_{\tau\land m} =\E M_0$. 
Also 
\begin{equation*}
\begin{aligned}
	\abs{\E M_\tau-\E M_0}&=\abs{\E M_\tau-\E M_{\tau\land m}} =\abs{\E M_\tau-\E M_\tau\mathbb{1}_{\{\tau\le n\}}-\E M_n\mathbb{1}_{\{\tau> n\}}}\\
	&= \abs{\E M_\tau\mathbb{1}_{\{\tau> n\}}- \E M_n\mathbb{1}_{\{\tau >n\}}}\\
	&\le \E\abs{M_\tau}\mathbb{1}_{\{\tau> n\}}+ \E\abs{M_n}\mathbb{1}_{\{\tau>n\}} \to 0
\end{aligned}
\end{equation*}
Wegen (ii),(iii) und einer Anwendung des Satzes von der majorisierten Konvergenz.


%sub end

\subsection{Anwendung}

Berechnung von Ruinwahrscheinlichkeiten. Sei $(X_n)_{n\in \N}$ iid mit $\Pw(X_n=1)=p=1-\Pw(X_n=-1)$ $S_n=\sum_{i=1}^{n} X_i$ für alle $n\in \N$\\
Anfangskapital von $k$ Euro. $S_n^{(k)}=k+S_n=k+\sum_{i=0}^{n} X_i$ Vermögen nach $n$-Spielen bei Anfangskapital $k$.

Wir spielen solange, bis wir ein Vermögen von $l>k$ Euro erreichen oder ruiniert sind 
\[
\tau=\inf\{n\in \N~|~S_n^{(k)}=0\text{ oder } S_n^{(k)}=l \} = \inf\{n\in \N~|~S_n=0-k\text{ oder } S_n=l-k \}
\]
ist die Strategie.

\begin{equation*}
\begin{aligned}
	\{S_\tau=-k\} &= \{S_\tau^{(k)}=0\} \text{ ist Ruin}\\
	\{S_\tau=l-k\} &= \{S_\tau^{(k)}=l\} \text{ ist Gewinn}
\end{aligned}
\end{equation*}
Man kann zeigen: 
\[
\Pw(\tau<\infty)=1,~\E \tau<\infty 
\]

\begin{enumerate}
	\item $p=\frac{1}{2}$ der faire Fall:\\
	Dann ist $(S_n)_{n\in \N_0}$ ein Martingal.
	\[
	\E\abs{S_\tau}\le \max\{k,l-k\}<\infty
	\]
	\[
	\E\abs{S_n}\mathbb{1}_{\{\tau>n\}}\le \max\{k,l-k\}\Pw(\tau>n)\to 0 
	\]
	Optional-Sampling liefert 
	\[
	0=\E S_\tau= -k\Pw(S_\tau=-k)+(l-k)\Pw(S_\tau=l-k) 
	\]
	Zusammen mit $\Pw(S_\tau=-k)+\Pw(S_\tau=l-k)=1$ folgt
	\[
	\Pw(S_\tau=-k)=\frac{l-k}{l}\text{ Ruinw'keit} 
	\]
	\[
	\Pw(S_\tau=l-k)=\frac{k}{l}\text{ Gewinnw'keit} 
	\]
	\item $p\not= \frac{1}{2}$ der unfaire Fall:\\
	$S_n=\sum_{k=1}^{n}X_k$, $\tau=\inf\{n\in \N_0~|~S_n=-k\text{ oder }S_n=l-k\}$\\
	Betrachte den geometrischen Random-Walk 
	\[
	M_n=a^{S_n}=\prod_{i=1}^{n}a^{X_i}\text{ mit }a>0 
	\]
	$(M_n)_{n\in \N_0}$ ist ein Martingal $\Leftrightarrow \E a^{X_1}=1 \Leftrightarrow ap+\frac{1}{a}(1-p)=1 \Leftrightarrow a=1$ oder $a=\frac{1-p}{p}$\\
	Für $p\not= \frac{1}{2}$ ist $a\not=1$\\
	Weiter gilt:
	\begin{equation*}
	\begin{aligned}
		&\E \abs{M_\tau}\le \max\{a^{-k},a^{l-k}\}<\infty\\
		&\E\abs{M_n}\mathbb{1}_{\{\tau>n\}}\le \max\{a^{-k},a^{l-k}\}\Pw(\tau>n)\stackrel{n\to\infty}{\to}0\\
		&\text{Optional Sampling liefert: }\\
		&~1=\E M_\tau=a^{-k}\Pw(S_\tau=-k)+a^{l-k}\Pw(S_\tau=l-k)
	\end{aligned}
	\end{equation*}
	Zusammen mit $\Pw(S_\tau=-k)+\Pw(S_\tau=l-k)=1$ folgt
	\[
	P(S_\tau=-k)=\frac{a^k-a^l}{1-a^l}\text{ und } \Pw(S_\tau=l)=\frac{1-a^k}{1-a^l}
	\]
	Es folgt der Nachweis von $\Pw(\tau<\infty)=1$.\\
	Betrachte dazu für $b\in \Z$, $\tau_b=\inf\{n\in \N_0~|~S_n=b\}$\\
	Dann gilt:
	\begin{equation*}
	\begin{aligned}
		&\Pw(\tau_b<\infty)=\Pw(\tau_1<\infty)\Pw(\tau_{b-1}<\infty)=\Pw(\tau_1<\infty)^b\\
		&\Pw(\tau_{-a}<\infty)=\Pw(\tau_{-1}<\infty)^a \text{ für alle }a,b\in \N
	\end{aligned}
	\end{equation*}
	Weiter ist 
	\begin{equation*}
	\begin{aligned}	
		\Pw(\tau_1<\infty) &= \Pw(\tau_1<\infty,X_1=1)+\Pw(\tau_1<\infty,X_1=-1)\\
		&= \Pw(X_1=1)+\Pw(X_1=-1)P(\tau_2<\infty)\\
		&=p+(1-p)\Pw(\tau_1<\infty)^2
	\end{aligned}
	\end{equation*}
	Also ist $\Pw(\tau_1<\infty)$ Lösung von \[
	(1-p)x^2-x+p=0 
	\]
	\[
	\Rightarrow \Pw(\tau_1<+\infty)=1\text{ oder }\Pw(\tau_1<+\infty)=\frac{p}{1-p}
	\]
	Der Fall $p\ge \frac{1}{2}$:\\
	$p\ge \frac{1}{2}\Rightarrow\frac{p}{1-p}\ge 1 \Rightarrow\Pw(\tau_1<+\infty)=1$\\
	
	Der Fall $p<\frac{1}{2}$:\\
	Dann ist $\frac{p}{1-p}<1$\\
	\begin{equation*}
	\begin{aligned}
	\text{SLLN } &\Rightarrow \frac{S_n}{n}\stackrel{n\to\infty}{\to}\E X_1=2p-1<0\\
	&\Rightarrow S_n\to -\infty~\Pw-f.s.\\
	&\Rightarrow \Pw(\sup\limits_{n\in \N} S_n=+\infty)=0
	\end{aligned}
	\end{equation*}
	Wäre $\Pw(\tau_1<\infty)=1$, so wäre $\Pw(\tau_b<\infty)=1~\forall b\in \N$ und damit
	\begin{equation*}
	\begin{aligned}
		\Pw(\sup\limits_{n\in \N}S_n=+\infty) &= \lim\limits_{b\to\infty}\Pw(\sup\limits_{n\in \N}S_n\ge +\infty)\\
		&= \lim\limits_{b\to\infty}\Pw(\tau_b<\infty)=1 \lightning
	\end{aligned}
	\end{equation*}
	Widerspruch, also gilt $\Pw(\tau_1<\infty)=\frac{p}{1-p}$ falls $p<\frac{1}{2}$\\
	$(M_n)_{n\in \N_0}$ ist Martingal $\Leftrightarrow~ \E a^{X_1}=1\Leftrightarrow ap+\frac{1}{a}(1-p)=1\Leftrightarrow a=1$ oder $a=\frac{1-p}{p}$\\
	Analog kann man schließen $\Pw(\tau_{-1}<\infty)=\left\{\begin{array}{cl}  1, &\text{falls } p\le \frac{1}{2} \\ \frac{1-p}{p}, & \text{falls }p>\frac{1}{2}  \end{array}\right.$\\
	Insgesamt folgt für $a<0<b,~a,b\in \R$, und $\tau_{ab}=\inf\{n\in \N_0~|~S_n=a\text{ oder }S_n=b\}$
	\[
	\Pw(\tau_{ab}<\infty)=\Pw(\tau_a<\infty\text{ oder }\tau_b<\infty)=1
	\]
	
	Berechnung von $\E \tau_{ab}$:\\
	\begin{enumerate}[(1)]
		\item unfairer Fall:\\
		$S_n-n\E X_1= S_n-(2p-1),~n\in N_0$ ist ein \bet{zentrierter Random-Walk}\index{Random-Walk!zentrierter} und deshalb ein Martingal. Optional Sampling liefert:
		\[
		\E(S_{\tau\land n}-(\tau\land n)(2p-1))=0 \Leftrightarrow \E(\tau\land n)(2p-1)=\E S_{\tau\land n} 
		\]
		$\E \tau\land n \uparrow \E\tau$ monotone Konvergenz\\
		$\E S_{\tau\land n}\stackrel{\tiny n\to\infty}{\to} \E S_\tau$ majorisierte Konvergenz, da $\abs{S_{\tau\land n}}\le \max\{\abs{a},b \}$\\
		Es gilt $\E S_{\tau\land n}=a\Pw(S_\tau=a)+b\Pw(S_\tau=b)=(2p-1)\E \tau$\\
		Also folgt:
		\[
		\E\tau_{ab}= \frac{1}{2p-1}\cdot \enbrace{a\frac{\enbrace{\frac{1-p}{p}}^{\abs{a}}-\enbrace{\frac{1-p}{p}}^{\abs{a}+b}}{1-\enbrace{\frac{1-p}{p}}^{\abs{a}+b}}+ b\frac{1-\enbrace{\frac{1-p}{p}}^{\abs{a}}}{1-\enbrace{\frac{1-p}{p}}^{\abs{a}+b}}} 
		\]
		\item fairer Fall: $p=\frac{1}{2}$\\
		$(S_n^2-n\E X_1^2)_{n\in \N_0}$ ist ein Martingal, denn
		\begin{equation*}
		\begin{aligned}
			\E(S_{n+1}^2~|~\F_n) &= \E((S_n+X_{n+1})^2~|~\F_n)\\
			&= \E(S_n^2~|~\F_n)+2\E(X_{n+1}S_n~|~\F_n)+\E(X_{n+1}^2~|~\F_n)\\
			&= S_n^2+ 2S_n\E(X_{n+1}~|~\F_n)+ \E X_{n+1}^2\\
			&= S_n^2+ 2S_n+\underbracket{\E X_{n+1}}_{=0}\cdot \E X_{n+1}^2\\
			&= S_n^2+ \E X_{n+1}^2\\
			\E(S_{n+1}^2-(n+1)\E X_1^2~|~\F_n)
			&= \E(S_{n+1}^2~|~\F_n)-(n+1)\E(X_1^2~|~\F_n)\\
			&=S_n^2+\E X_{n+1}^2-(n+1)\E X_1^2\\
			&=S_n^2+n\E X_1^2
		\end{aligned}
		\end{equation*}
		Hieraus folgt die Martingaleigenschaft. Optioal Smapling liefert: 
		\[
		\E(S_{\tau\land n}^2-(\tau\land n)\E X_1^2=0 \Leftrightarrow \E S_{\tau\land n}^2= \underbracket{X_1^2}_{=1}\cdot \E(\tau\land n)
		\]
		$\E \tau\land n\uparrow \E\tau$ mon. Konv. und $\E S_{\tau\land n}^2 \to \E S_\tau^2$ maj. Konv.\\
		\begin{equation*}
		\begin{aligned}
			\Rightarrow \E\tau &= \E S_\tau^2 = a^2\Pw(S_\tau=a)+b\Pw(S_\tau=b)\\
			&= a^2\cdot \frac{b}{\abs{a}+b}+b^2\cdot \frac{\abs{a}}{\abs{a}+b}=\frac{\abs{a}b(\abs{a}+b)}{\abs{a}+b}\\
			&= \abs{a}b
		\end{aligned}
		\end{equation*}
		\end{enumerate}
\end{enumerate}
%sub end

\subsection{Vorhersehbare Prozesse}
\label{sub:vorher_prozesse}
Sei $(\F_n)_{n\in \N}$ eine Filtration. Ein stochastischer Prozess $(X_n)_{n\in \N}$ heißt \Index{vorhersehbar}, wenn gilt \marginnote{man kennt den Wert für $X_n$ schon vor Beginn der Periode}
\[ 
X_n\text{ ist } \F_{n-1}-\text{messbar }\forall n \in \N
\]

%sub end
\newpage
\subsection{Doob-Meyer Zerlegung}
\label{sub:doob-meyer}
Sei $(X_n)_{n\in \N_0}$ ein, zu einer Filtration $(\F_n)_{n\in \N_0}$, adaptierter Prozess, mit 
\[
\E\abs{X_n}<\infty~\forall n\in\N_0
\]
Dann existiert genau eine Zerlegung der Form 
\[
X_n= Y + M_n + \Lambda_n \quad \Pw-\text{f.s. }\forall n\in \N_0 
\]
mit $Y$ ist $\F_o$-messbare Startvariable, $(M_n)_{n\in \N_0}$ ist ein Martingal, mit $M_0=0$ und $(\Lambda_n)_{n\in \N}$ ist vorhersehbar, $\Lambda_0=0$.\\
Eindeutigkeit bedeutet, wenn es eine weiter Zerlegung $X=Y'+M_n'+\Lambda_n'$ gibt, so folgt: 
\[
Y=Y',M_n=M_n',~\Lambda_n=\Lambda_n'~\Pw\text{-f.s.},~\forall n\in \N_0  
\]

\bet{Beweis:}\\
\uline{Existenz:}\\
Setze $M_0=0,~\Lambda_0=0$ und $M_1=X_1-\E(X_1~|~\F_0),~\Lambda_1=\E(X_1~|~\F_0)-X_0$.\\
Definiere rekursiv:\\
\[ 
M_{n+1}=M_n+X_{n+1}-\E(X_{n+1}~|~\F_n) 
\]
\[
\Lambda_{n+1}=\Lambda_n+\E(X_{n+1}~|~\F_n)-X_n 
\]
Dann gilt $(\Lambda_n)_{n\in \N_0}$ ist vorhersehbar, $(M_n)_{n\in \N_0}$ ist ein Martingal, denn
\begin{equation*}
\begin{aligned}
	\E(M_{n+1}~|~\F_n) &= \E(M_n+X_{n+1}-\E(X_{n+1}~|~\F_n)~|~\F_n)\\
	&= M_n+\E(X_{n+1}~|~\F_n)-\E(X_{n+1}~|~\F_n)=M_n
\end{aligned}
\end{equation*}
\uline{Behauptung:} 
\[
X_n=X_0+M_n+\Lambda_n,~\forall n\in \N_0 
\]
Induktion nach $n$:\\
$n=0$ klar(\checkmark)\\
\uline{$n\to n+1$:}
\begin{equation*}
\begin{aligned}
	X_{n+1} &= X_{n+1}-X_n+X_n\\
	&\stackrel{\text{IV}}{=} X_{n+1}-X_n+X_0+M_n+\Lambda_n\\
	&= X_0+M_n+X_{n+1}-\E(X_{n+1}~|~\F_n)+\Lambda_n+\E(X_{n+1}~|~\F_n)-X_n\\
	&= X_0+M_{n+1}+\Lambda_{n+1}
\end{aligned}
\end{equation*}

Eindeutigkeit folgt aus der Tatsache, dass ein vorhersehbares Martingal konstant sein muss, d.h. sei $(Z_n)_{n\in\N}$ ein Martingal und vorhersehbar, dann folgt 
\[
\exists Y\in\R:~Y~ \F_0\text{-messbar}:~Z_n=Y~\Pw\text{-f.s.}
\]
\uline{Beweis:} 
\[
Z_{n+1}=\E(Z_{n+1}~|~\F_n)=Z_n=\dots=Z_1=Y ~ \F_0\text{-messbar} 
\] 
\hfill $\square$

\begin{equation*}
\begin{aligned}
	&\text{Sei } Y+M_n+\Lambda_n=Y'+M_n'+\Lambda_n'\\
	&\Rightarrow Y=Y', \text{ da } M_0=M_0'=0,~ \Lambda_0=\Lambda_0'=0\\
	&\Rightarrow M_n+\Lambda_n=M_n'+\Lambda_n'\\
	&\Rightarrow M_n-M_n'=\Lambda_n'-\Lambda_n
\end{aligned}
\end{equation*}
Also ist $(M_n-M_n')_{n\in \N_0}$ ein vorhersehbares Martingal. Also folgt $M_n-M_n'=Z~\forall n\in \N_0$ für ein $\F_0$-messbares $Z$.
\[
Z=\E(M_n-M_n'~|~\F_0)=M_0-M_0'=0 \Rightarrow Z\equiv 0 
\]
\hfill $\square$

%sub end

\section{Diskrete Finanzmarktmodelle}
\label{sec:disk_finanzmarktmodelle}
\begin{itemize}
	\item[Ziel:]
	\item Modellierung von Finanzmärkten in diskreter Zeit
	\item Formulierung des Arbitragebegriffs
	\item Arbitrage freie Bewertung von Derivaten
	\item Zusammenhang zur W'theorie, das No-Arbitrage Theorem
\end{itemize}

\subsection{Beschreibung von Finanzmärkten}
\label{sub:beschreibung_fimarkt}
\begin{itemize}
	\item periodische Sichtweise
	\item $N$ Perioden
	\item $N$ Handelszeitpunkte $0,1,\dots,N-1$
	\item Informationsverlauf wird gegeben durch eine Filtration $(\F_n)_{n=0,\dots,N}$
	\item $d$ risikobehaftete Finanzgüter, mit zu $(\F_n)_{n=0,\dots,N}$ adaptierten Preisprozessen
	\[
	S_1(n),\dots,S_d(n),~n=0,\dots,N
	\]
	$S=(S_1,\dots,S_d)$ beschreibt die Entwicklung der Basisfinanzgüter (\Index{risky assets}).\\
	$S(n)$ ist der zufällige Vektor der Preise nach $n$ Perioden für die risky assets.
	\item ein \Index{Numeraire Asset} (Verrechnungsgut) mit Preisprozess $S_0(n),~n=0,\dots,N$, wobei vorausgesetzt wird, dass $S_0(n)>0,~n=0,\dots,N$\\
	$S_0$ ist adaptiert bzgl. $(\F_n)_{n=0,\dots,N}$.\\
	Das Numeraire Asset dient zur Verrechnung. Häufig wird ein Geldmarktkonto hierzu benutzt, d.h.
	\begin{equation*}
	\begin{aligned}
		S_0(n)=\beta(n)&=(1+\rho(1))(1+\rho(2))\cdots(1+\rho(n)),~ n=1,\dots,N,~\beta(0)=1\\
		&=\prod_{i=1}^{n}1+\rho(i)
	\end{aligned}
	\end{equation*}
	wobei $(\rho(n))_{n=1,\dots,N}$ ein vorhersehbarer Prozess ist mit 
	\[
	\rho(n)>-1,~\Pw-\text{f.s. }n=1,\dots,N
	\]
	\begin{tikzpicture}[line cap=round,line join=round,>=triangle 45,x=1.0cm,y=1.0cm]
	\draw[->,color=black] (-1.,0.) -- (7.5,0.);
	\foreach \x in {0,1,5,6,7}
	\draw[shift={(\x,0)},color=black] (0pt,2pt) -- (0pt,-2pt);
	\draw (0,0) node[below] {$0$};
	\draw (1,0) node[below] {$1$};
	\draw (5,0) node[below] {$n-1$};
	\draw (6,0) node[below] {$n$};
	\draw (7,0) node[below] {$N$};
	\draw (1,0) node[above] {$(1+\rho(1))$};
	\end{tikzpicture}
	
	$\rho(n)$ beschreibt die zufällige Zinsrate der $n$-ten Periode
	\item gehandelt werden kann durch Erwerb bzw. Verkauf von Anteilen an den $(d+1)$ Basisfinanzgütern in den Handelszeitpunkten. Die Entwicklung der Anzahl an Anteilen der Basisfinanzgütern entspricht dabei vorhersehbaren Prozessen $(\varphi,H)$, mit $\varphi(n)$ entspricht der Anzahl an Anteilen des Numeriare Assets in der $n$-ten Periode und $H_j(n)$ entspricht der Anzahl an Anteilen im $j$-ten risky Assets in der $n$-ten Periode.\\
	\[
	H=(H_1,\dots,H_d)
	\]
	Ein solches Paar $(\varphi,H)$ heißt \Index{Handelsstrategie}. Vorhersehbar, da am Anfang einer Periode das Portfolio zusammengesetzt wird.
\end{itemize}
Eine Handelsstrategie $(\varphi,H)$ induziert eine \Index{Vermögensentwicklung}
\[
V(n)=\varphi(n)S_0(n)+\sum_{j=1}^{d}H_j(n)S_j(n),~n=1,\dots,N
\]
Wert nach $n$ Perioden.
\[
	V(0)=\varphi(1)S_0(0)+\sum_{j=1}^{d}H_j(1)S_j(0)
\]
Anfangsvermögen. $V(0)$ kann als Anfangskapital interpretiert werden, dass ein Investor einsetzen muss, um die Handelsstrategie $(\varphi,H)$ durchführen zu können. $V(n)$ entspricht dem Vermögen am Ende der $n$-ten Periode, vor Umschichtung des Portfolios.

%sub end

\subsection{Selbstfinanzierung}
\label{sub:selbstfinanzierung}
Wird beim Handel in den Handelszeitpunkten $1,\dots,N-1$ kein Kapital hinzugefügt bzw. entnommen, so nennt man diese Handelsstrategie \Index{selbstfinanzierend}.\\
\uline{Formal:} $(\varphi,H)$ heißt selbstfinanzierend, wenn
\begin{equation*}
\begin{aligned}
	V_n(H)&=\varphi(n)S_0(n)+\sum_{j=1}^{d}H_j(n)S_j(n)\\
	&=\varphi(n+1)S_0(n)+\sum_{j=1}^{d}H_j(n+1)S_j(n),~n=1,\dots,N-1
\end{aligned}
\end{equation*}
Beispiele für selbstfinanzierende Strategien
\begin{enumerate}[(a)]
	\item \uline{Buy and hold Strategie:}\\
	Ein Anfangskapital $x>0$ wird in das erste risky asset investiert und bis zum Ende gehalten.
	$H_1(1)=\frac{x}{S_1(0)}$ Kaufen am Anfang\\
	$H_1(n)=\frac{x}{S_1(0)},~n=2,\dots,N$ Halten über die Perioden\\
	$H_j\equiv 0$ für alle anderen $j$\\
	Wertentwicklung:\\
	\[
	V(n)=H_1(n)S_1(n)=\frac{x}{S_1(0)}S_1(n)
	\]
	\item \uline{short selling and hold einer Aktie:}\\
	$H_1(1)=-1$ short selling der Aktie, Verkauf am Anfang\\
	$H_1(n)=-1$ Halten der Verkaufoption von $n=2,\dots,N$\\
	Anfangskapital:
	\[
	-S_1(0)<0
	\]
	Wertentwicklung:
	\[
	-S_1(n),~n=2,\dots,N
	\]
	\newpage
	\item kaufe Aktie 1, halte diese $k$-Perioden und tausche danach in Aktie 2, falls $S_2(k)<S_1(k)$ und halte diese Position bis zum Ende.\\
	Also:
	$H_1(n)=1$, $H_2(n)=0$ für $n=1,\dots,k$\\
	\[
	H_2(k+1)=\frac{V(k)}{S_2(k)}\mathbb{1}_{\{S_2(k)<S_1(k)\}}=\frac{S_1(k)}{S_2(k)}\mathbb{1}_{\{S_2(k)<S_1(k)\}}
	\] 
	zufällige Umschichtung in Aktie 2 am Anfang der $(k+1)$-Periode.\\
	\[
	H_1(k+1)=\mathbb{1}_{\{S_2(k)>S_1(k)\}}
	\]
	\[
	H_2(n)=\frac{S_1(k)}{S_2(k)}\mathbb{1}_{\{S_2(k)<S_1(k)\}}
	\]
	\[
	H_1(n)=\mathbb{1}_{\{S_2(k)>S_1(k)\}},~ n=k+2,\dots,N
	\]
	Halten bis zum Ende.\\
	$H$ ist vorhersehbar und selbstfinanzierend, da 
	\begin{equation*}
	\begin{aligned}
		V(k) &= S_1(k)=\frac{S_1(k)}{S_2(k)}\mathbb{1}_{\{S_2(k)<S_1(k)\}}S_2(k)+\mathbb{1}_{\{S_2(k)>S_1(k)\}}S_1(k)\\
		&= H_2(k+1)S_2(k)+H_1(k+1)S_1(k)
	\end{aligned}
	\end{equation*}
\end{enumerate}
% sub end

\subsection{Beispiele CRR-Modell}
\label{sub:bsp_crr}
\begin{enumerate}[(a)]
	\item Das $N$-Perioden CRR-Modell:\\
	$N$-Perioden, $n=1,\dots,N$, $S_0$ Anfangskurs, Filtration $(\F_n)_{n=1,\dots,N}$.\\
	Sei $(Z_n)_{n=1,\dots,N}$ die Anzahl der Aufwärtssprünge in den ersten $n$ Perioden.\\
	Annahme: $Z_n=\sum_{i=1}^{n}Y_i$ mit iid. ZV'en $Y_1,\dots,Y_N$, $\Pw(Y_i=1)=p=1-\Pw(Y_i=0)$.\\
	Die Sprunghöhen sind $0<d<u$, damit ergibt sich der Preisprozess der risky assets (Aktie), der Form 
	\[
	S_n=S_0u^{Z_n}d^{n-Z_n}~ n=1,\dots,N	
	\]
	Also ergibt sich
	\[
	S_{n+1}=S_nu^{Y_{n+1}}d^{1-Y_{n+1}}
	\]
	$(Y_i)_{i=1,\dots,N}$ adaptiert bzgl. der Filtration.\\
	\uline{Bemerkung:} $(S_n)_{n=1,\dots,N}$ ist ein geometrischer Random-Walk, startend aus $S_0>0$.\\
	\uline{Andere Darstellung:}
	\[
	S_n=S_0\cdot \prod_{i=1}^{n}X_i \text{ mit } X_i=u^{Y_i}d^{1-Y_i}
	\]
	Das Numeriare Asset ist ein Geldmarktkonto mit konstanter periodischer Zinsrate $\rho>-1$, d.h.
	\[
	\beta(n)=(1+\rho)^n ~ n=1,\dots,N
	\]
	\item Mehrdimensionales CRR-Modell:\\
	Gegeben seien $l$ Aktien und $l$ Aktienpreisprozesse, entsprechend dem einfachen CRR-Modell.
	\[
	S_j(n) = S_j(0)u_j^{Z_j(n)}d_j^{n-Z_j(n)},~ n=1,\dots, N
	\]
	Sei $(\F_n)_{n=1,\dots,N}$ eine Filtration. $(Z_n)_{n=1,\dots,N}$ ist ein $l$-dim. Random-Walk mit 
	\[
	Z(n)=\sum_{i=1}^{n}Y(i)
	\]
	$Y(i)=(Y_1(i),\dots,Y_l(i))$ mit $\Pw(Y_j(i)=1)=p_j=1-\Pw(Y_j(i)=0)$. 
	Die $Y_1,\dots,Y_N$ sind iid., aber in einer Periode können die $Y_1(i),\dots,Y_l(i)$ abhängig sein.\marginnote{mehrdimimensionales Bernoulli-Experiment, aber Aktien beeinflussen sich gegenseitig}\\
	Das Numeraire Asset ist wie beim CRR-Modell ein Geldmarktkonto:
	\[
	\beta(n)=\prod_{i=1}^{n}(1+\rho)=(1+\rho)^n
	\]
	\item Das verallgemeinerte CRR-Modell:\\
	Idee: Ersetze den Random-Walk $(Z_n)$, der die Aufwärtssprünge zählt, durch eine zeitlich inhomogene \Index{Markov-Kette}.\\
	Genauer: $(\Omega,\F,\Pw)$ sei W'Raum, $(\F_n)_{n=0,\dots,N}$ eine Filtration und $(Z_n)_{n=0,\dots,N}$ ein \Index{Markov-Prozess}, adaptiert bzgl. $(\F_n)$ mit
	\begin{enumerate}[(i)]
		\item $Z_0=0$
		\item Übergangswahrscheinlichkeiten:
		\[
		\Pw(Z_{n+1}=k+1~|~Z_n=k)=p_n(k)=1-\Pw(Z_{n+1}=k~|~Z_n=k),~ k=0,\dots,n
		\]
	\end{enumerate}
	Markov-Eigenschaft bedeutet:
	\[
	\Pw(Z_{n+1}=k~|~\F_n)=\Pw(Z_{n+1}=k~|~Z_n)
	\]
	insbesondere folgt damit:
	\[
	\Pw(Z_{k+1}=k~|~Z_n=k_n,Z_{n-1}=k_{n-1},\dots,Z_1=k_1,Z_0=0)=\Pw(Z_{k+1}=k~|~Z_n=k_n)
	\]
	$Z(n)$ zählt die Aufwärtssprünge der ersten $n$-Perioden. Setze als Preisprozess des risky asset
	\[
	S(n)=S_0u^{Z_n}d^{n-Z_n}\text{ mit }0<d<u
	\]
	Für die Entwicklung des Geldmarktkontos wird angenommen, dass die Zinsrate in einer Periode von der bis dahin erfolgten Anzahl an Aufwärtssprüngen abhängt, d.h.
	\[
	\rho(n)=r(n,Z_{n-1}),~n=1,\dots,N \marginnote{kennen den Zins in der $n$-ten Periode, da wir $Z_{n-1}$ kennen}
	\]
	mit $r:\N\times \{0,\dots,N\}\to (-1,\infty)$. 
	$\rho(n)$ ist dann die zufällige Zinsrate in der $n$-ten Periode. 
	$\rho(n)$ ist $\F_{n-1}$-messbar für $n=1,\dots,N$, $\rho$ ist also vorhersehbar.\\
	Der Preisprozess des Geldmarktkontos ergibt sich durch
	\[
	\beta(n)=\prod_{i=1}^{n}(1+\rho(i)),~n=0,\dots,N
	\]
\end{enumerate}
% sub end

\subsection{Das diskontierte Finanzmarktmodell}
\label{sub:disk_finanzmarktmodell}
Gegeben sei ein Modell entsprechend \hyperref[sub:beschreibung_fimarkt]{4.1} mit $S=(S_1,\dots,S_d)$ als Preisprozess für die risky assets und $S_0$ als Preisprozess für das Numeraire Asset.\\
Alle Preise sind hier in Geldeinheiten (Euro) notiert. 
Eine weitere Möglichkeit Preise zu notieren besteht darin, diese in Anzahl an Anteilen des Numeraire Assets anzugeben. 
$x$ Geldeinheiten zum Zeitpunkt $t$ entsprechen $\frac{x}{S_0(t)}$-Anteilen des Numeraire Assets.\\
Im Falle das $S_0$ das Geldmarktkonto ist, $(S_0(n)=\beta(n))$, ist dies der übliche Diskontierungsvorgang. 
Dies drückt aus wie viel Geld zum Zeitpunkt $t$ heute wert ist. 
$x$ Euro in $t$ entsprechen $\frac{x}{\beta(t)}$ heute. 
Führt man diese '\Index{Diskontierung}' für die Basisfinanzgüter durch, erhält man ein Finanzmarktmodell, dessen Preise in Anteilen des Numeraire Assets notiert sind.
Definiere:
\[
S_j^*(t)=\frac{S_j(t)}{S_0(t)},~t=0,\dots,N,~1\le j \le d
\]
$S_j^*$ ist dann der Preisprozess des $j$-ten risky assets ausgedrückt in Anteilen des Numeraire Assets.
$(S_1^*,\dots,S_d^*)$ ist dann das 'abdiskontierte' Finanzmarktmodell.

Für eine Handelsstrategie $(\varphi,H)$ ist der Wertprozess, in Anteilen des Numeraire Assets ausgedrückt, gegeben durch
\[
V^*(n) = \varphi(n)+\sum_{j=1}^{d}H_j(n)S_j^*(n)
\]
% sub end

\subsection{Charakterisierung der Selbstfinazierung}
\label{sub:charkt_selbstfinanzierung}
Eine Handelsstrategie $(\varphi,H)$ ist selbstfinazierend genau dann, wenn sich ihre Vermögensentwicklung aus dem Anfangskapital und den Periodengewinnen bzw. Verlusten ergibt.\\
Genauer: Für eine Handelsstrategie $(\varphi,H)$ sind äquivalent:
\begin{enumerate}[(i)]
	\item $(\varphi,H)$ ist selbstfinanzierend
	\item 
	\[
	V(n) = V(0)+\sum_{k=1}^{n}\varphi(k)\Delta S_0(k) + \sum_{k=1}^{n}\underbracket{\sum_{j=1}^{d}H_j(k)\Delta S_j(k)}_{=\sprod{H(k)}{\Delta S(k)}},~n=1,\dots,N
	\marginnote{alternativ auch mit Skalarprodukt $\sprod{H(k)}{\Delta S(k)}$}
	\]
	\item 
	\[
	V^*(n)= V^*(0)+\sum_{k=1}^{n}\sum_{j=1}^{d} H_j(k)\Delta S_j^*(k),~ n=1,\dots,N
	\]
\end{enumerate}
Für ein stochastischen Prozess $(X(n))_{n=0,1,2,\dots}$ bedeutet 
\[
\Delta X(n)= X(n)-X(n-1)
\]
Prozess der Periodenzuwächse.\\

\bet{Beweis:}\\
Für jede Handelsstrategie $(\varphi,H)$ gilt:
\begin{equation*}
\begin{aligned}
	V(1)-V(0) &= \varphi(1)S_0(1)-\varphi(1)S_0(0)+\sprod{H(1)}{S(1)}-\sprod{H(1)}{S(0)}\\
	&= \varphi(1)\Delta S_0(1)+\sprod{H(1)}{\Delta S(1)}
\end{aligned}
\end{equation*}
Deshalb gilt: 
\begin{equation*}
\begin{aligned}
	\text{(ii) ist erfüllt} \Leftrightarrow &\Delta V(k)=\varphi(k)\Delta S_0(k)+\sprod{H(k)}{\Delta S(k)}~k=2,\dots,N\\
	\Leftrightarrow &\varphi(k)S_0(k)+\sprod{H(k)}{S(k)}- \varphi(k-1)S_0(k-1)-\sprod{H(k-1)}{S(k-1)}\\
	&=\varphi(k)S_0(k)-\varphi(k)S_0(k-1)+\sprod{H(k)}{S(k)}-\sprod{H(k)}{S(k-1)}~k=2,\dots,N\\
	\stackrel{l=k-1}{\Leftrightarrow} &\sprod{H(l+1)}{S(l)}+\varphi(l+1)S_0(l)=\sprod{H(l)}{S(l)}+\varphi(l)S_0(l)~l=1,\dots,N-1\\
	\Leftrightarrow &(\varphi,H) \text{ ist selbstfinanzierend}
\end{aligned}
\end{equation*}
Weiter gilt:
\begin{equation*}
\begin{aligned}
	\text{(iii) ist erfüllt} \Leftrightarrow &\Delta V^*(k)=\sprod{H(k)}{\Delta S^*(k)}~ k=2,\dots,N\\
	\Leftrightarrow &\varphi(k)+\sprod{H(k)}{S(k)}\frac{1}{S_0(k)}-\varphi(k-1)-\sprod{H(k-1)}{S(k-1)}\frac{1}{S_0(k-1)}\\
	&=\frac{1}{S_0(k)}\sprod{H(k)}{S(k)}-\frac{1}{S_0(k-1)}\sprod{H(k)}{S(k-1)}~ k=2,\dots,N\\
	\Leftrightarrow &\varphi(k)S_0(k-1)+\sprod{H(k)}{S(k-1)}= \varphi(k-1)S_0(k-1)+\sprod{H(k-1)}{S(k-1)}~ k=2,\dots,N\\
	\Leftrightarrow &(\varphi,H) \text{ ist selbstfinanzierend}
\end{aligned}
\end{equation*}
\hfill $\square$
%sub end

Wichtig ist, dass der Handel in den risky assets durch Aufbau einer geeigneten Position im Numeraire Asset zu einer selbstfinanzierenden Handelsstrategie gemacht werden kann.

\subsection{Satz 1}
\label{sub:satz_1fima}
Sei $S_0(0)=1$. Zu jedem $\R^d$-wertigem vorhersehbarem Prozess $H$ und jedem Anfangskapital $V_0$ ex. genau ein vorhersehbarer Prozess $\varphi$, so dass $(\varphi,H)$ selbstfinanzierend ist und 
\[
V^*(n)= V_0+\sum_{k=1}^{n}\sprod{H(k)}{\Delta S^*(k)}~n=1,\dots,N \tag*{$(\ast)$}
\]

\bet{Beweis:}\\
Zu bestimmen ist ein $\varphi$, so dass der abdiskontierte Wertprozess von $(\varphi,H)$ durch $(\ast)$ gegeben ist.\\
Damit ist $(\varphi,H)$ selbstfinanzierend wegen \hyperref[sub:charkt_selbstfinanzierung]{4.5}.\\
Wegen
\[
V_0=\varphi(1)S_0(0)+\sprod{H(1)}{S(0)}
\]
folgt
\[
\varphi(1)=\frac{\sprod{H(1)}{S(0)}-V_0}{S_0(0)}
\]
Bestimmung von $\varphi(n)$ für $n\ge 2$:\\
Wegen
\[
V_0+\sum_{k=1}^{n}\sprod{H(k)}{\Delta S^*(k)}=V^*(n)=\varphi(n)+\sprod{H(n)}{S^*(n)}
\]
erhält man
\[
V_0+\sum_{k=1}^{n-1}\sprod{H(k)}{\Delta S^*(k)}+\sprod{H(n)}{S^*(n)}-\sprod{H(n)}{S^*(n-1)}=\varphi(n)+\sprod{H(n)}{S^*(n)}
\]
Also setzt man
\[
\varphi(n)=V_0+\sum_{k=1}^{n-1}\sprod{H(k)}{\Delta S^*(k)}+\sprod{H(n)}{S^*(n-1)}
\]
\hfill $\square$\\
Bezeichne mit $\mathcal{H}$ die Menge aller $\R^d$-wertigen vorhersehbaren stoch. Prozesse.
Definiere den stoch. Prozess $H\cdot S^*$ durch
\[
(H\cdot S^*)(n)=\sum_{k=1}^{n}\sprod{H(k)}{\Delta S^*(k)}~ n=1,\dots,N,~ (H\cdot S^*)(0)=0
\]
$(H\cdot S^*)(n)$ ist die Summe der Periodengewinne bzgl. $S^*$ über die ersten $n$-Perioden.
$H\cdot S^*$ wird als \uline{diskreter stoch. Integralprozess} bezeichnet.

%sub end

\subsection{Arbitrage}
\label{sub:arbitrage_math}
Eine selbstfinanziernde Handelsstrategie $(\varphi,H)$ heißt \Index{Arbitrage}, wenn gilt 
\[
V_0\le 0,~V_N\ge 0 \text{ und } \Pw(V_N-V_0>0)>0
\]
Ausgedrückt in Anteilen des Numeraire Assets ist dies äquivalent zu, mit $S_0(0)=1$:
\[
V_0\le 0,~ V^*(N)=\frac{V(N)}{S_0(N)}\ge 0 \text{ und } \Pw(V^*(N)-V_0 > 0) > 0
\]
Da $V^*(N)-V_0=H\cdot S^*(N)$.\\
Es gibt eine Arbitragemöglichkeit genau dann, wenn es ein Anfangskapital $V_0\le 0$ und ein $H\in \mc{H}$ gibt mit
\[
V_0+(H\cdot S^*)(N)\ge 0 \text{ und } \Pw(H\cdot S^*(N)>0)>0
\]

\bet{Bemerkung:} 
Existiert ein Arbitrage, so existiert auch ein Arbitrage zum Anfangskapital 0.\\

\bet{Beweis:}\\
Sei $(\varphi,H)$ ein Arbitrage mit
\[
V(0)=\varphi(1)S_0(0)+\sprod{H(1)}{S(0)}<0
\]
Dann ist
\[
V^*(N)=V(0)+(H\cdot S^*)(N)\ge 0 \text{ und } \Pw(H\cdot S^*(N)>0)>0
\]
Zum Anfangskapital 0 existiert eine selbstfinanzierende Handelsstrategie $(\Psi,H)$ mit 
\[
V_{(\Psi,H)}^*(N)= 0+ (H\cdot S^*)(N)\ge -V(0)>0
\]
\hfill $\square$
%sub end

\subsection{Beispiele Arbitrage}
\label{sub:bsp_arbitrage}
\minisec{Satz}
Das CRR-Modell ist genau dann arbitragefrei, wenn $d<1+\rho<u$ gilt.\\

\bet{Beweis:}\\
\uline{"$\Rightarrow$":} per Kontraposition: Ist $1+\rho \le d<u$, so ist immer die Aktie besser als das Bankkonto.
Die buy and hold Strategie für die Aktie liefert dann ein Arbitrage.
Setze $H\equiv 1$.\\
Dann existiert zum Anfangskapital 0 eine selbstfin. Handelsstrategie $(\varphi,H)$ mit Wertprozess
\[
V^*(n)=0+(H\cdot S^*)(n),~ n=0,\dots,N
\]
Also
\[
V^*(n)=\sum_{k=1}^{n}H(k)\Delta S^*(k)=S^*(N)-S^*(0)\ge S(0)\frac{d^N}{(1+\rho)^N}-S(0)\ge 0
\]
und 
\[
\Pw(V^*(N)>0)>0
\]
Ist  $d<u\le 1+\rho$, so ist das Bankkonto immer besser als die Aktie.
Also kann man durch ein short selling der Aktie ein Arbitrage konstruieren.
Setze also $H(n)\equiv -1,~ n=1,\dots,N$.
Dann existiert zum Anfangskapital 0 eine selbstfin. Handelsstrategie $(\varphi,H)$ mit Wertprozess
\[
V^*(N)=(H\cdots S^*)(N)
\] 
also
\[
V^*(N)=(H\cdot S^*)(N)=-(S^*(N)-S(0))=S(0)-S^*(N)\ge S(0)-S(0)\frac{u^N}{(1+\rho)^N}\ge S(0)-S(0)=0
\]
und 
\[
\Pw(V^*(N)-V(0)>0)>0
\]
\newpage
\uline{"$\Leftarrow$":} per Kontraposition: Sei das Modell nicht arbitragefrei.\\
Dann existiert ein vorhersehbares $H$ mit
\[
V^*_N=(H\cdot S^*)(N)\ge 0 \text{ und } \Pw((H\cdot S^*)(N)>0)>0
\]
Wegen $V^*(N)=\sum_{k=1}^{N}H(k)\Delta S^*(k)$ existiert eine Periode $n$ mit 
\[
H(n)\Delta S^*(n)\ge 0 \text{ und } \Pw(H(n)\Delta S^*(n)>0)>0
\]
Also auch
\begin{equation*}
\begin{aligned}
	0\le H(n)\frac{\Delta S^*(n)}{S^*(n-1)}&=H(n)\left(\frac{S^*}{S^*(n-1)}-1\right)\\
	&= H(n)\left(\frac{1}{1+\rho}u^{X_n}d^{1-X_n}-1\right)
\end{aligned}
\end{equation*}
Daher folgt:
\begin{equation*}
\begin{aligned}
	S(n)&=u^{Z_n}d^{n-Z_n}S(0)\\
	Z_n&=\sum_{i=1}^{n}X_i,~ \Pw(X_i=1)=p=1-\Pw(X_i=0)\\
	S^*(n)&=\frac{S(n)}{(1+\rho)^n},~ \beta(n)=S_0(n)=(1+\rho)^n	
\end{aligned}
\end{equation*}
Annahme: $d<1+\rho<u$.\\
Dann ist mit $R(n)=\frac{1}{1+\rho}u^{X_n}d^{1-X_n}-1$:
\begin{equation*}
\begin{aligned}
	1=\Pw(H(n)R(n)\ge 0)&=\Pw(H(n)>0,R(n)>0)+\Pw(H(n)<0,R(n)<0)+\Pw(H(n)=0)\\
	&=\Pw(H(n)>0)\Pw(R(n)>0)+\Pw(H(n)<0)\Pw(R(n)<0)+\Pw(H(n)=0)
\end{aligned}
\end{equation*}
Da $H(n)$ $\F_{n-1}$-messbar ist und $R(n)$ unabhängig von $\F_{n-1}$ ist.
\[
=qp+r(1-p)+1-(q+r)
\]
mit $q=\Pw(H(n)>0),~r=\Pw(H(n)<0)$.\\
Aus $\Pw(H(n)=0)<1$ folgt
\[
q>0 \text{ oder } r>0
\]
Wegen
\begin{equation*}
\begin{aligned}
	1&=qp+r(1-p)+1-(q+r)\\
	&<q+r+1-(q+r)=1\lightning
\end{aligned}
\end{equation*}
\hfill $\square$\\
%sub end
\newpage
\uline{Ziel:} No-Arbitrage Theorem\\
Charakterisierung von arbitragefreien Märkten im probabilistischem Sinne.

\subsection{Äquivalente Maße}
\label{sub:eq_mas}
Sei $(\Omega,\F,\Pw)$ ein W'Raum.
\[
\mc{N}=\{N\in \R~|~\Pw(N)=0\}
\]
ist das System der $\Pw$-\Index{Nullmengen}.\\
Ein W'Maß $Q$ ist \Index{absolut-stetig} bzgl. $\Pw$
\[
Q\ll \Pw \Leftrightarrow \mc{N}_\Pw \subseteq \mc{N}_Q
\]

$Q$ heißt \Index{äquivalent} zu $\Pw$ genau dann, wenn $\mc{N}_\Pw=\mc{N}_Q$.\\
Ist $L\ge 0$ ZV mit $\int L\dint \Pw=1$, so wird durch
\[
Q(A)=\int\limits_A L\dint \Pw~\forall A\in \Omega
\]
ein W'Maß $Q$ definiert mit $Q\ll \Pw$.
$L$ ist die $\Pw$-Dichte von $Q$.
Schreibweise: $L=\frac{\dint Q}{\dint \Pw}$.\\
Gilt $\Pw(L>0)=1$ und $\frac{\dint Q}{\dint \Pw}=L$, so ist
\[
\Pw \sim Q \text{ und } \frac{\dint \Pw}{\dint Q}=\frac{1}{L}
\]
Weiter: Sind $L,L'$ Dichten von $Q$ bzgl. $\Pw$, so gilt $P(L=L')=1$. Für jede ZV $X$ gilt:
\[
\E_Q(X)=\int X\dint Q=\int XL\dint\Pw=\E_\Pw(LX)
\]
sofern obiger Erwartungswert existiert.\\

Zusammenhang zur Modellierung von Finanzmärkten:
\begin{itemize}
	\item Ein Finanzmarktmodell wird im westenlichen bestimmt durch die zufällige Entwicklung der Basisfinanzgüter.
	\item Dabei ist nicht entscheidend, welche Verteilung ein Akteur im Finanzmarkt postuliert.
	\item Zwei Akteure sind im gleichem Finanzmarkt, wenn die beiden postulierten Verteilungen für die Basisfinanzgüter die gleichen Ereignisse mit positiver W'keit eintreten lassen können.
	Das bedeutet, dass die Verteilungen zueinander äquivalent sind.
	\item Ein Übergang zu einem äquivalenten W'Maß ändert den Finanzmarkt nicht, wohl aber die Verteilung der Basisfinanzgüter.
	\item Ein endliches Finanzmarktmodell, z.B. $\abs{\Omega}<\infty$, wird nicht verändert, wenn die Menge der Elementarereignisse mit positiver W'keit unverändert bleibt.
\end{itemize}
%sub end
\newpage
\subsection{Äquivalentes Martingalmaß}
\label{sub:eq_martingalmas}
Gegeben sei ein Finanzmarktmodell mit Preisprozess $S=(S_1,\dots,S_d)$ der risky assets und ein Informationsverlauf $(\F_n)_{n=0,\dots,N}$.
Sei $S_0$ das Numeriare Asset und
\[
S_j^*:= \frac{S_j}{S_0},~j=1,\dots,d
\]
Ein W'Maß $\Pw^*$ auf $(\Omega,\F)$ heißt \Index{äquivalentes Martingalmaß}, wenn gilt 
\begin{enumerate}[(i)]
	\item $\Pw^* \sim \Pw$
	\item $(S_j^*(n))_{n=0,\dots,N}$ ist ein $\Pw^*$-Martingal für alle $j=1,\dots,d$
\end{enumerate}
Kurz: Bzgl. $\Pw^*$ ist der Finanzmarkt fair.\\

\uline{Ziel:} Arbitragefreier Markt $\Leftrightarrow$ Existiert ein äquivalentes Martingalmaß.\\
\uline{"$\Leftarrow$":} leicht.\\
\uline{"$\Rightarrow$":} etwas schwierig, math. Argument ist der Trennungssatz von Minkowski. \marginnote{vgl. Funktional Analysis}
%sub end

\subsection{Separationssatz von Minkowski}
\label{sub:separationssatz}
Seien $C_1$ und $C_2$ nicht leere konvexe Mengen des $\R^n$ mit $C_1\cap C_2=\emptyset$.
Sei $C_1$ abgeschlossen und $C_2$ kompakt.\\
Dann gibt es eine lineare Abbildung $\varphi:\R^n\to\R$ und reelle Zahlen $\beta_1<\beta_2$ mit
\[
\varphi(x)\le \beta_1<\beta_2\le \varphi(y),~\forall x\in C_1,~y\in C_2
\]
d.h.
\[
\sup\limits_{x\in C_1}\varphi(x)<\sup\limits_{y\in C_2}\varphi(y)
\]
\begin{minipage}[c]{8cm}
	Graphische Veranschaulichung:\\
	\begin{center}
		\begin{tikzpicture}[line cap=round,line join=round,>=triangle 45,x=1.0cm,y=1.0cm,scale=2]
			\draw [shift={(0.0677534717447,0.00367786116921)},color=black,fill=blue,fill opacity=0.1] (0,0) -- (9.28984473873:0.118518518519) arc (9.28984473873:99.2898447387:0.118518518519) -- cycle;
			\draw [rotate around={8.39909464785:(-0.887407407407,-0.186666666667)}] (-0.887407407407,-0.186666666667) ellipse (0.561303019894cm and 0.340449810227cm);
			\draw [shift={(2.04888888889,0.37037037037)}] plot[domain=2.49051593215:4.26928642542,variable=\t]({1.*1.09514253602*cos(\t r)+0.*1.09514253602*sin(\t r)},{0.*1.09514253602*cos(\t r)+1.*1.09514253602*sin(\t r)});
			\draw (-0.342613255233,-0.0634475846968)-- (0.975699051777,0.152194414474);
			\draw (-0.031736354862,0.611901970305)-- (0.142519994078,-0.453402053335);
			\draw (0.509395133542,-0.349407329683)-- (0.344476525251,0.658811068407);
			\draw (-1,-.2) node {$C_2$};
			\draw (1.1,.4) node {$C_1$};
		\end{tikzpicture}
	\end{center}
\end{minipage}
\begin{minipage}[c]{8cm}
	Wieso Kompaktheit von $C_2$:\\
	\begin{center}
		\begin{tikzpicture}[line cap=round,line join=round,>=triangle 45,x=.70cm,y=.70cm]
			\coordinate (C) at (-4,0.01);
			\coordinate (D) at (2,2);
			\coordinate (d) at (-1,0);
			\draw[cap=round]
			(C) .. controls (d) ..(D);
			\draw (-2,1) node {$C_2$};
			\coordinate (A) at (-4,-0.01);
			\coordinate (B) at (2,-2);
			\coordinate (a) at (-1,0);
			\draw[cap=round]
			(B) .. controls (a) ..(A);
			\draw (-2,-1) node {$C_1$};
			\draw (-4,0) -- (3,0);
		\end{tikzpicture}
	\end{center}
	$C_1$ und $C_2$ lassen sich nicht strikt trennen.
\end{minipage}


%sub end
Zur Bestimmung der zu trennenden konvexen Mengen wird die Arbitragefreiheit umformuliert.

\subsection{Umformulierung der Arbitragefreiheit}
\label{sub:umf_arbitragefreiheit}
Finanzmarktmodell über $N$ Perioden mit $S=(S_1,\dots,S_d)$ als Preisprozess der risky assets.
Mit $L^0(\Omega,\F,\Pw)$ sei die Menge der meßbaren Abb. von $\Omega$ nach $\R$.
$\G^*:=\{(H\cdot S^*)(N)~|~H\in\mc{H} \}$ bezeichnet die Menge der möglichen Gewinne, notiert in Anteile des Numeriare Assets, die beim Handel entsprechend einer selbstfinanzierenden Handelsstrategie erzielt werden können.
Dabei ist $\mc{H}$ die Menge der vorhersehbaren $\R^d$-wertigen Prozesse.\\

Der Markt ist arbitragefrei, wenn
\[
\G^*\cap L_+^0(\Omega,\F,\Pw)=\{0\}
\]
wobei $L_+^0(\Omega,\F,\Pw)=\{X\in L^0~|~X\ge 0\}$.\\
$\G^*$ ist ein Vektorraum.\\
\begin{center}
	\begin{tikzpicture}[line cap=round,line join=round,>=triangle 45,x=.35cm,y=.35cm]
	\draw[->,color=black] (-3.,0.) -- (3,0.);
	\draw[->,color=black] (0.,-3.) -- (0.,3.);
	\draw (-3,3) -- (3,-3) node[below] {$\G^*$};
	\draw[blue] (0,0)--(-1,-1) node[below] {$\mc{K}^*$};
	\draw[blue] (-2,2)--(-3,1);
	\draw[blue] (-1,1)--(-2,0);
	\draw[blue] (2,-2)--(1,-3);
	\draw[blue] (1,-1)--(0,-2);
	\end{tikzpicture}
	\captionof{figure}{Kegel im $\R^2$}
\end{center}
$\mc{K}^*:=\{C^*\in L^0~|~\text{es ex. ein }G^*\in \G^*\text{ mit }G^*\ge C^* \}$ $\mc{K}^*$ ist der Kegel aus Elementen aus $L^0$, die unterhalb von $\G^*$ liegen.
Es gilt:
\[
\G^*\cap L_+^0=\{0\}\Leftrightarrow \mc{K}^*\cap L_+^0=\{0\}
\]
Weiter
\[
\mc{K}^*\cap (-\mc{K}^*)=\G^*
\]
Mittels $\G^*$ und $\mc{K}^*$ können äquivalente Martingalmaße charakterisiert werden.

%sub end

\subsection{Satz 2}
\label{sub:satz_2fima}
Sei $\abs{\Omega}<\infty$.
Für ein zu $\Pw$ äquivalentes W'Maß $\Pw^*$ sind äquivalent:
\begin{enumerate}[(i)]
	\item $\Pw^*$ ist ein Martingalmaß, d.h. $S_j^*$ ist ein $\Pw^*$-Martingal für alle $j=1,\dots,d$.
	\item $\E^*C^*=0$ für alle $C^*\in \G^*$.
	\item $\E^*K^*\le 0$ für alle $K^*\in \mc{K}^*$.
\end{enumerate}

\bet{Beweis:}\\
\uline{(i)$\Rightarrow$(ii):}
Für $H\in \mc{H}$ ist $V^*(n)=(H\cdot S^*)(n),~n=1,\dots,N,~V^*(0)=0$ ein $\Pw^*$-Martingal, denn
\begin{equation*}
\begin{aligned}
	\E^*(\Delta V^*(k)~|~\F_{k-1}) &= \E^*(V^*(k)-V^*(k-1)~|~\F_{k-1})\\
	&= \E^*(\sprod{H(k)}{\Delta S^*(k)}~|~\F_{k-1}) = \E^*\left(\sum_{j=1}^{d}H_j(k)\Delta S_j^*(k)~|~\F_{k-1}\right)\\
	&= \sum_{j=1}^{d}\E^*(H_j(k)\Delta S_j^*(k)~|~\F_{k-1}) = \sum_{j=1}^{d}H_j(k)\underbracket{\E^*(\Delta S_j^*(k)~|~\F_{k-1})}_{=0} = 0~ k=1,\dots,N
\end{aligned}\marginnote{äquivalente Bedingung zur Martingaleigenschaft}
\end{equation*}
Für $C^*=(H\cdot S^*)(N)$ folgt also
\[
\E^* C^*= \E^*((H\cdot S^*)(N))=\E^* V^*(N)=\E^* V^*(0)=0
\]

\uline{(ii)$\Rightarrow$(i):}
Zeige die Martingaleigenschaft von $S_j^*$ bzgl. $\Pw^*$ für alle $j=1,\dots,d$.\\
$\zz: \E^*(S_j^*(k)~|~\F_{k-1})=S_j^*(k-1)$\\
$\Leftrightarrow \E^*(\Delta S_j^*(k)~|~\F_{k-1})=0\Leftrightarrow \E^*\mathbb{1}_A \ \Delta S_j^*(k)=0$ für alle $A\in \F_{k-1}$.\\

$\mathbb{1}_A \Delta S_j^*(k)$ ist der Gewinn der Handelsstrategie, die in der $k$-ten Periode long in $S_j$ geht, wenn $A$ eintritt, d.h.
\[
H_j(k)=\mathbb{1}_A,~H_j(n)=0 \text{ sonst}
\]
\[
(H\cdot S^*)(N)=H_j(k)\Delta S_j^*(k)=\mathbb{1}_A\Delta S_j^*(k),~ H_i\equiv 0\text{ für } i\neq j
\]
wegen (ii) folgt $\E^*\mathbb{1}_A \Delta S_j(k)=0$ für alle $A\in \F_{k-1}$.\\
Also ist $S_j^*$ ein $\Pw^*$-Martingal.\\

\uline{(ii)$\Rightarrow$(iii):}
Klar, wegen Monotonie des Erwartungswertes.\\

\uline{(iii)$\Rightarrow$(ii):}
Ist $C^*\in \G^*$ dann folgt $C^*\in \mc{K}^*\Rightarrow \E^*C^*\le 0$.
$\G^*$ ist ein Vektorraum:
\[
\Rightarrow -C^*\in\G^*\Rightarrow-C^*\in\mc{K}^*\Rightarrow\E^*(-C^*)\le 0\Rightarrow\E^*C^*\ge 0
\]
\hfill $\square$
%sub end

Zusammen mit dem Separationssatz kann man das No-Arbitrage Theorem beweisen.

\subsection{Das No-Arbitrage Theorem}
\label{sub:no-arbitrage_theorem}
1. Fundamentalsatz der Preistheorie:\\
Gegeben sei ein Finanzmarkt $S$ über einem endlichem $\Omega$ mit Informationsverlauf $(\F_n)_{n=0,\dots,N}$ bzgl. einem W'Raum $(\Omega,\F,\Pw)$.\\
Dann sind äquivalent
\begin{enumerate}[(i)]
	\item Der Markt ist arbitragefrei
	\[
	\G^*\cap L_+^0=\{0\}.
	\]
	\item Es existiert ein äquivalentes Martingalmaß $\Pw^*$.
\end{enumerate}

\bet{Beweis:}\\
\uline{(ii)$\Rightarrow$(i):}
ist einfach: $C^*\in\G^*\cap L_+^0$.
Dann gilt:
\[
C^*\ge 0\text{ und } \E^*C^*=0\Rightarrow C^*=0~ \Pfs[*]\Rightarrow C^*=0~ \Pfs \Rightarrow C^*=\{0\}
\]

\uline{(i)$\Rightarrow$(ii):}
O.E.d.A. $\Pw(\{\omega\})>0~\forall \omega\in \Omega$.\\
$\G^*$ ist als Teilraum von $L^0(\Omega,\F,\Pw)$ konvex und abgeschlossen.
\[
\Pw=\{Q:\Omega\to \R~|~Q(\omega)\ge 0~\forall \omega\in \Omega,~\sum_{\omega\in \Omega}Q(\omega)=1 \}
\]
ist die konvexe und kompakte Menge der W'Maße auf $(\Omega,\F)$.\\
Wegen (i) ist $\G^*\cap\Pw=\emptyset$.\\
Nach dem Separationssatz existiert eine lineare Abbildung $\varphi:L^*(\Omega,\F,\Pw)\to \R$ mit
\[
\varphi(C^*)<\inf\limits_{Q\in \Pw}\varphi(Q)
\]
Da der Dualraum von $L^*(\Omega,\F,\Pw)$ durch $L^1(\Omega,\F,\Pw)$ gegeben ist, ist $\varphi\in L^1(\Omega,\F,\Pw)$, d.h.
\[
\varphi(X)=\sum_{\omega\in \Omega} X(\omega)\varphi(\omega)
\]
Da $\G^*$ ein Teilraum ist und $\varphi(C^*)\le \alpha,~\forall C^*\in \G^*$ folgt $\varphi(C^*)=0,~\forall C^*\in \G^*$.\\
Für $e_\omega=\mathbb{1}_{\{\omega\}}$ gilt $e_\omega\in \Pw$ und
\[
0<\varphi(e_\omega)=\varphi(\omega)
\]
Da dies für alle $\omega\in\Omega$ gilt kann man ein $\Pw^*$ durch
\[
\Pw^*(\omega)=\frac{\varphi(\omega)}{\sum_{\omega'\in\Omega}\varphi(\omega')}~\forall \omega\in\Omega
\]
definieren.
\[
\Pw^*(\omega)>0~\forall \omega\in\Omega \Rightarrow \Pw^*\sim \Pw
\]
Wegen
\begin{equation*}
\begin{aligned}
\E^*C^* &= \sum_{\omega\in\Omega}C^*(\omega)\Pw^*(\omega) = \sum_{\omega\in \Omega}C^*(\omega)\frac{\varphi(\omega)}{\sum_{\omega'\in\Omega}\varphi(\omega')}\\
&= \frac{1}{\sum_{\omega'\in\Omega}\varphi(\omega')}\underbracket{\varphi(C^*)}_{=0}=0
\end{aligned}
\end{equation*}
$\stackrel{\hyperref[sub:umf_arbitragefreiheit]{4.12}}{\Rightarrow} \Pw^*$ ist ein Martingalmaß.
\hfill $\square$

%sub end

\subsection{Bestimmung von äquivalenten Martingalmaßen}
\label{sub:bestimmung_mmase}
\begin{enumerate}[(a)]
	\item CRR-Modell:
	\begin{itemize}
		\item $N$ Perioden.
		\[
		Z_n=\sum_{i=1}^{N}X_i,~X_1,\dots,X_N,\text{ iid, }\Pw(X_i=1)=p=1-\Pw(X_i=0)
		\]
		\item $S(n)=S_0u^{Z_n}d^{1-Z_n}$
		\item $\F_n=\sigma(X_1,\dots,X_n)=\sigma(Z_1,\dots,Z_n)$
		\item $\beta(n)=\prod_{i=1}^{n}1+p=(1+p)^n,~p>-1$,
	\end{itemize}
	Gesucht ist ein $\Pw^*$ mit 
	\begin{enumerate}[(i)]
		\item $\Pw \sim \Pw^*$
		\item $S^*$ ist ein $\Pw^*$-Martingal
	\end{enumerate}
	\[
	S^*(n)=\frac{S(n)}{\beta(n)}=S_0u^{Z_n}d^{n-Z_n}\frac{1}{(1+p)^n}=S_0\prod_{i=1}^{n}\frac{u^{X_i}d^{1-X_i}}{1+p}
	\]
	ist ein geometrischer Random-Walk.
	\marginnote{ab hier könnte sich einiges wiederholen, da er in zwei Vorlesungen das selbe jeweils entwas anders gemacht hat}
	$S^*$ ist ein Martingal genau dann, wenn
	\begin{equation*}
	\begin{aligned}
		&\E^*\frac{u^{X_i}d^{1-X_i}}{1+p}=1 \Leftrightarrow u\cdot p^*+d(1-p^*)=1+p\\
		&\Leftrightarrow p^*=\frac{1+p-d}{u-d}\in (0,1) \Leftrightarrow d<1+p<u
	\end{aligned}
	\end{equation*}
	Durch $p\in(0,1)$ werden alle äquivalenten CRR-Modelle parametrisiert und genau für den Parameter
	\[
	\Pw^*=\frac{(1+p)-d}{u-d}
	\]
	ist das Modell risikoneutral.
	Dies bedeutet, dass $S^*$ ein Martingal ist, bzgl. dem Parameter $\Pw^*$.\\
	\uline{Genauer:}\\
	Wegen 
	\[
	\Pw^*(X=x)=\frac{\Pw^*(X=x)}{\Pw(X=x)}\cdot \Pw(X=x),~\forall x\in \{0,1\}
	\]
	ist die Dichte von $(\Pw^*)^X$ bzgl. $\Pw^X$ gegeben durch
	\[
	l(x)=\frac{(p^*)^{Z_N}(1-p^*)^{N-Z_N}}{p^{Z_N}(1-p)^{N-Z_N}},~Z_N=\sum_{i=1}^{N}X_i
	\]
	Hieraus erhält man durch $L:\Omega\to(0,\infty),~\omega\mapsto l(X_1(\omega),\dots,X_N(\omega))=\frac{(p^*)^{Z_N(\omega)}(1-p^*)^{N-Z_N(\omega)}}{p^{Z_N(\omega)}(1-p)^{N-Z_N(\omega)}}$ die Dichte von $\Pw^*$ bzgl. $\Pw$.\\
	Setze also $\Pw^*$ an mittels
	\[
	\Pw^*(A)=\int\limits_A L\dint \Pw,~\forall A\in \F_N
	\]
	Dann ist $\Pw^*$ äquivalent zu $\Pw$, da $L>0~\Pfs$ und es gilt
	\begin{equation*}
	\begin{aligned}
		\Pw^*(X=x) &= \int\limits_{\{X=x\}}L\dint \Pw=\enbrace{\frac{p^*}{p}}^{\sum_{i=1}^{N}X_i}\enbrace{\frac{(1-p^*)}{1-p}}^{N-\sum_{i=1}^{N}X_i} \Pw(X=x)\\
		&= (p^*)^{\sum_{i=1}^{N}X_i}(1-p^*){N-\sum_{i=1}^{N}X_i}
	\end{aligned}
	\end{equation*}
	Bzgl. des so definierten Maßes $\Pw^*$ ist $S(n)=S_0u^{Z_n}d^{n-Z_n},~n=0,\dots,N$ ein geometrischer Random-Walk mit $\E^*S(1)=S_0(1+\rho)$.
	Deshalb ist $(S^*(n))$ ein $\Pw^*$-Martingal und damit $\Pw^*$ ein äquivalentes Martingalmaß.
\end{enumerate}
%sub end
%sec end

\newpage
\section{Bewerten von Derivaten}
\label{sec:bewerten_derivate}
Gegeben sei ein Finanzmarktmodell über $N$ Perioden mit Preisprozess $(S_1,\dots,S_d)$ der risky assets und $S_0$ des Numeraire Assets.\\

\uline{Grundannahme:}\\
Der Finanzmarkt ist arbitragefrei $\Leftrightarrow$ Existenz eines äquivalenten Martingalmaßes.\\
Bezeichne mit $\mc{P}$ die Menge aller äquivalenten Martingalmaße.
Dann ist $\mc{P}$ eine konvexe Teilmenge, der Menge aller zu $\Pw$ äquivalenten Maße.

\subsection{Claim und Hedge}
\label{sub:claim_hedge}
Ein Derivat ist ein Wertpapier, das eine Auszahlung am Ende der Laufzeit ($N$) verbrieft.\\
Mathematisch gesehen entspricht dies einer $\F_N$-messbaren Abbildung $C$.
Diese wird auch als \Index{Claim} bezeichnet.
Zum Beispiel $C=(S(N)-K)^+$.\\
$C^*=\frac{C}{S_0(N)}$ ist dann die \bet{Claimauszahlung}\index{Claim!-auszahlung}, notiert in Einheiten des Numeraire Assets.\\
Denkt man an das Replikationsprinzip, so ist gesucht eine Strategie, durch Handel am Finanzmarkt den Claim zu replizieren.
Im Modell bedeutet dies:\\
Gesucht ist ein Anfangskapital $V_0$ und ein $H\in \mc{H}$ mit 
\[
V_0+\sum_{n=1}^{N}\sprod{H(n)}{\Delta S^*(n)}=V_0+(H\cdot S^*)(N)=C^*
\]
$V_0$ und $H$ definieren dann eindeutig eine selbstfinanzierende Handelsstrategie $(\varphi,H)$ mit
\[
V_0\big((\varphi,H)\big)=V_0,~ V_N\big((\varphi,H)\big)=C\marginnote{Vorausgesetzt wird $S_0(0)=1$.}
\]

Ist dies möglich, so heißt $C$ \bet{hedgebar}\index{Hedge!hedgebar} und $(\varphi,H)$ bzw. $V_0$ und $H$ definieren eine \bet{Hedgestrategie}\index{Hedge!Hedgestrategie}.\\
In Analogie zum Replikationsprinzip kann man fragen:\\
Ist $V_0$ der eindeutige arbitragefreie Anfangspreis für $C$?\\

\bet{Informelle Argumentation:}\\
Ein Anfangspreis $x>V_0$ liefert ein Arbitrage für den Verkäufer, denn 
\begin{itemize}
	\item gehe short ins Claim, erhalte $x$
	\item investiere $V_0$ in die selbstfinanzierende Handelsstrategie und handle entsprechend dieser Strategie
\end{itemize}
$x-V_0$ ist der Gewinn am Anfang.
Benutze am Ende das Vermögen aus der Handelsstrategie, um die short-Position im Claim aufzulösen.
\[
V_N(H)-C=0
\]
Ein Anfangspreis $x<V_0$ liefert ein Arbitrage für den Käufer, denn
\begin{itemize}
	\item gehe short im Hedge, erhalte $V_0$.
	\item investiere $x$ in den Claim.
\end{itemize}
$V_0-x$ ist Gewinn am Anfang, handle nun entsprechend der short-Position im Hedge.
Am Ende benutze den Claim, um die short-Position im Hedge aufzulösen.
\[
C-V_N(H)=0
\]
Diese Argumentation legt nahe, dass $x=V_0$ der arbitragefreie Anfangspreis eines hedgebaren Claims $C$ ist.\\
Im Folgendem wird dies mathematisch präzisiert.
%sub end

\subsection{Satz 3}
\label{sub:satz_3}
Sei $C$ ein hedgebarer Claim und $H,H'\in \mc{H}$ Hedgestrategien zu Anfangskapitalien $V_0$ bzw. $V_0'$.\\
Dann gilt:
\[
V_0=V_0'=\E^*C^*,~\forall \Pw^*\in\mc{P}
\]
und
\[
V_0+(H\cdot S^*)(n)=V_H^*(n)=V_{H'}^*(n)=V_0'+(H'\cdot S^*)(n)=\E^*(C^*~|~\F_n),~\forall n=1,\dots,N
\]

\bet{Beweis:}\\
Dies folgt aus der Martingaleigenschaft von $S^*$ bzw. $(H\cdot S^*)$ bzgl. $\Pw^*$, da $\E^*(H\cdot S^*)(N)=0$:
\begin{equation*}
\begin{aligned}
	V_0&+(H\cdot S^*)(N)=C^*=V_0'+(H'\cdot S^*)(N)\\
	\Rightarrow V_0=\E^*&(V_0+(H\cdot S^*)(N))=\underbracket{\E^*C^*}_{=V_0'}=\E^*(V_0'+(H'\cdot S^*)(N))
\end{aligned}
\end{equation*} 
Das gleiche Argument liefert:
\begin{equation*}
\begin{aligned}
	V_H^*(n) &= V_0+(H\cdot S^*)(n)=\E^*(V_0+(H\cdot S^*)(N)~|~\F_n)\\
	&= \E^*(C^*~|~\F_n)=\E^*(V_0'+(H'\cdot S^*)(N)~|~\F_n)\\
	&= V_0'+(H'\cdot S^*)(N)= V_{H'}^*(n)
\end{aligned}
\end{equation*}
\hfill $\square$
%sub end

\subsection{Superreplizierbare Claims}
\label{sub:superrepl_claims}
Ein Claim $C$ heißt \bet{upper hedgebar}\index{Hedge!upper hedgebar} zum Anfangskapital $V_0$, falls ein $H\in\mc{H}$ gibt mit
\[
V_0+(H\cdot S^*)(N)\ge C^*
\]
Dies ist der Fall, wenn
\[
C^*-V_0\in \mc{K}^*
\]
$\G^*=\{(H\cdot S^*)(N)~|~H\in\mc{H}\}$ Gewinne, $\mc{K}^*=\{C^*~|~\text{es ex. }G\in\G\text{ mit }G\ge C^* \}$.\\
$C$ heißt \bet{strikt upperhedgebar} zum Anfangskapital $V_0$, falls es eine $H\in \mc{H}$ gibt mit
\[
V_0+(H\cdot S^*)(N)\ge C^*\text{ und }\Pw(V_0+(H\cdot S^*)(N)>C^*)>0
\]

$C$ heißt \bet{lower hedgebar}\index{Hedge!lower hedgebar} zum Anfangskapital $V_0$, wenn es ein $H\in \mc{H}$ gibt mit
\[
V_0-(H\cdot S^*)(N)\le C^*
\]
(\bet{strikt} für $\le$ und $\Pw(V_0-(H\cdot S^*)(N)<C^*)>0$)
\begin{center}
	\begin{tikzpicture}[line cap=round,line join=round,>=triangle 45,x=.5cm,y=.5cm]
	\draw[->,color=black] (-3.,0.) -- (3,0.);
	\draw[->,color=black] (0.,-3.) -- (0.,3.);
	\draw (-3,3)--(3,-3) node[below]{\tiny$G^*$};
	\draw (-3,4)--(3,-2) node[anchor=north west]{\tiny$V_0\neq G^*$};
	\draw[->] (0,0)--(2,2);
	\draw (4,2) node{\tiny$\Omega=\penbrace{\omega_1,\omega_2}$};
\end{tikzpicture}
\end{center}
Der Kegel $\mc{K}^*$, der zum Anfangskapital 0 upper hedgebaren Claims, lässt sich durch die erwarteten Auszahlungen bzgl. der äquivalenten Martingalmaße charakterisieren.
%sub end

\subsection{Satz 4}
\label{sub:satz_4fima}
Für einen Claim $C$ sind äquivalent:
\begin{enumerate}[(i)]
	\item $C^*\in \mc{K}^*$
	\item $\E^*C^*\le 0~\forall \Pw^*\in \mc{P}$
\end{enumerate}

\bet{Beweis:}\\
\uline{(i)$\Rightarrow$(ii):}
Ist $C^*\in\mc{K}^*\Rightarrow\exists H\in \mc{H}$ mit $(H\cdot S^*)(N)\ge C^*\Rightarrow 0=\E^*[(H\cdot S^*)(N)]\ge \E^*C^*~\forall \Pw\in \mc{P}$.\\
\uline{(ii)$\Rightarrow$(i):}
Dies ergibt sich aus dem Bipolartheorem.
% sub end

\subsection{Das Bipolartheorem}
\label{sub:bipolartheorem}
Betrachte den $\R^n$ mit einem Skalarprodukt.
Eine Teilmenge $C\in \R^n$ heißt \Index{Kegel}, wenn
\[
\lambda x\in C ~\forall \lambda>0,~x\in C
\]
(heißt \bet{konvexer Kegel}\index{Kegel!konvexer}, wenn $C$ ein Kegel ist und konvex ist.
Dies ist der Fall, wenn gilt
\[
x\in C,~x>0\Rightarrow \lambda x\in C
\]
\[
x,y\in C\Rightarrow x+y\in C
\]
)\\
Zu einem Kegel $C$ ist die \Index{Polarmenge} $C^0$ definiert durch
\[
C^0=\penbrace{y\in \R^n~|~\sprod{x}{y}\le 0~\forall x\in C}
\]
%grafik
\begin{minipage}[c]{7cm}
	\begin{tikzpicture}[line cap=round,line join=round,>=triangle 45,x=.60cm,y=.60cm]
	\draw[->,color=black] (-4.3,0.) -- (4,0.);
	\draw[->,color=black] (0.,-4.3) -- (0.,4);
	\fill[color=red,fill=red,fill opacity=0.1] (1.42,2.46) -- (0.,0.) -- (2.58,1.12) -- cycle;
	\fill[color=blue,fill=blue,fill opacity=0.1] (-2.54,1.24) -- (0.,0.) -- (1.62,-2.94) -- (0.36,-3.8) -- (-3.32,0.16) -- cycle;
	\draw (0.,0.)-- (1.42,2.46);
	\draw (0.,0.)-- (2.58,1.12);
	\draw [dash pattern=on 5pt off 5pt] (-1.62,2.9)-- (1.62,-2.94);
	\draw [dash pattern=on 2pt off 2pt] (-2.54,1.24)-- (3.02,-1.48);
	\draw (-1,-1) node {$C^0$};
	\draw(1.4,1.3) node {$C$};
	\end{tikzpicture}
\end{minipage}
\begin{minipage}[c]{7cm}
$C^0$ ist ein abgeschlossener Kegel, denn
\[
y\in C^0,~\lambda>0\Rightarrow\sprod{x}{\lambda y}=\lambda \sprod{x}{y}\le 0~\forall x\in C
\]
Die Abgeschlossenheit folgt aus der Stetigkeit des Skalarprodukts.
\end{minipage}

Das Bipolartheorem besagt, dass das Bipolar von $C$ mit dem Abschluss von $C$ übereinstimmt
\[
(C^0)^0=\overline{C}.
\]
Anwendung von \hyperref[sub:satz_4fima]{5.4}:
Für eine Menge $E\subseteq \R^n$ sei $\cone(E)$ der von $E$ erzeugte Kegel.
Definiere
\[
\cone(E):= \bigcap\ablim{C\text{ Kegel},~E\subseteq C}C=\penbrace{\lambda x~|~ \lambda>0,~x\in E}.
\]
Es gilt: (wegen \hyperref[sub:satz_2fima]{4.13}) $\Pw^*$ ist ein Martingalmaß genau dann, wenn
\[
\E^*C^*\le 0~\forall C^*\in\mc{K}^*.
\]
Deshalb ist $(\mc{K}^*)^0=\overline{\cone(\mc{P})}=\cone(\overline{\mc{P}})$
\[
\Rightarrow (\mc{K}^*)^{0^0}=\overline{\cone(\mc{P})}^0
\]
\[
\stackrel{\text{Bip.t.}}{\Rightarrow} \mc{K}^*=(\mc{K^*})^{0^0}=\overline{\cone(\mc{P})}^0=\cone(\overline{\mc{P}})^0
\]
Also gilt:
\[
\E^*C^*\le 0 ~\forall \Pw^*\in\mc{P}\Rightarrow C^*\in \cone(\overline{\mc{P}})^0\Rightarrow C^*\in\mc{K}^*
\]
%sub end

\subsection{Upper and lower hedging Preise}
\label{sub:upper_lower_preise}
Sei $C$ ein Claim.
$P_+(C)=\inf\{x\in \R~|~C^*$ ist upper hedgbar zum Anfangskapital $x\}$ und\\
$P_-(C)=\sup\{x\in \R~|~C^*$ ist lower hedgebar zum Anfangskapital $x \}$.\\
Aus \hyperref[sub:satz_4fima]{5.4} folgt, dass das Infimum bzw. Supremum angenommen wird.

\minisec{Satz}
\begin{enumerate}[(i)]
	\item Die Menge der upper hedging Preise ist ein abgeschlossenes Intervall $[P_+(C),+\infty)$
	\item Die Menge der lower hedging Preise ist ein abgeschlossenes Intervall $(-\infty,P_-(C)]$
	\item $P_-(C)\le P_+(C)$
\end{enumerate}

\bet{Beweis:}\\
(i): Klar ist, dass die Menge der upper hedging Preise ein nach oben unbeschränktes Intervall bildet.
Für die Abgeschlossenheit betrachte upper hedging Preise $a_n$ mit $a_n\searrow a$.\\
\zz~a ist upper hedging Preis.
Dann gilt 
\[
C^*-a_n\in \mc{K}^*~\forall n\in\N
\]
Wegen \hyperref[sub:satz_2fima]{4.13} gilt für $\Pw^*\in\mc{P}$: $\E^*(C^*-a_n)\le 0~\forall n\in N$, außerdem folgt wegen $C^*-a_n\to C^*-a$, dass $0\ge \E^*(C^*-a_n)\to \E^*(C^*-a) ~\forall \Pw^*\in \mc{P}$.\\
\hyperref[sub:satz_4fima]{5.4} liefert $C^*-a\in \mc{K}^*$ also ist $a$ ein upper hedging Preis.\\

(ii): geht wie (i): $a$ ist lower hedging Preis genau dann, wenn es ein $K\in \G^*$ gibt mit
\[
a+K\le C^* \Leftrightarrow -(C^*-a)\le-K \Leftrightarrow -(C^*-a)\in \mc{K}^*
\]
$a_n \nearrow a,~a_n$ lower hedging Preise $\Rightarrow -(C^*-a_n)\in\mc{K}^*\Rightarrow\E^*(-(C^*-a_n))\le 0~\forall\Pw^*\in\mc{P},~\forall n\in\N$\\
$C^*-a_n\to C^*-a\Rightarrow\E^*(-(C^*-a))\le0~\forall \Pw^*\in\mc{P}\Rightarrow-(C^*-a)\in\mc{K}^*$\\
$\Rightarrow a$ ist lower hedging Preis (für $C^*$).\\

(iii): Ist $a$ lower hedging Preis und $b$ upper hedging Preis,so gilt
\begin{equation*}
\begin{aligned}
	(C^*-b)\in \mc{K}^* &\Leftrightarrow \E^*(C^*-b)\le 0 ~\forall \Pw^*\in\mc{P}\\
	&\Leftrightarrow \sup\limits_{\Pw^*\in\mc{P}} \E^*C^*\le b\\
	-(C^*-a)\in\mc{K}^* &\Leftrightarrow \E^*(-(C^*-a))\le 0~\forall\Pw^*\in\mc{P}\\
	&\Leftrightarrow \inf\limits_{\Pw^*\in\mc{P}}\E^*C^*\ge a\\
	a\le \inf\limits_{\Pw^*\in\mc{P}}\E^*C^* &\le \sup\limits_{\Pw^*\in\mc{P}} \E^*C^* \le b
\end{aligned}
\end{equation*}
\hfill $\square$\\
Prinzipiell ergeben sich zwei Fälle:
\begin{enumerate}[(a)]
	\item $P_-(C)=P_+(C)$, dies ergibt sich, wenn $C$ hedgebar ist.
	\item $P_-(C)<P_+(C)$, dies ergibt sich, wenn $C$ nicht hedgebar ist.
\end{enumerate}
%sub end


\subsection{Charakterisierung der arbitragefreien Preise}
\label{sub:arbitragefreier_preis}
Sei $C$ ein Claim.
Kann aus $x\in \R$ ein strikter upper hedge finanziert werden, so ergibt sich eine Arbitragemöglichkeit für den Verkäufer.
Kann aus $x\in\R$ ein strikter lower hedge finanziert werden, so ergibt sich eine Arbitragemöglichkeit für den Käufer.\\
Dies ist die Motivation für die folgende Definition:\\
$x\in \R$ heißt \Index{arbitragefreier Preis} für $C$, falls durch $x$ weder ein striker upper noch ein strikter lower hedge finanziert werden kann.\\
Mit $\pi(C)$ bezeichne die Menge aller arbitragefreien Preise für $C$.
$\pi(C)$ kann mittels der upper und lower hedging Preise charakterisiert werden.

\minisec{Theorem}
Für ein Claim $C$ gilt:
\begin{enumerate}[(i)]
	\item $C$ ist hedgebar zum Anfangskapital $x$ genau dann, wenn $P_-(C)=x=P_+(C)$.
	\item Ist $C$ hedgebar zum Anfangskapital $x$, so ist
	\[
	\E^*C^*=x~\forall \Pw^*\in\mc{P} \text{ und } \pi(C)=\{x\}
	\]
	\item Ist $C$ nicht hedgebar, so gilt:
	\[
	P_-(C)<P_+(C)\text{ und } \pi(C)=\big(P_-(C),P_+(C)\big)=\penbrace{\E^*C^*~|~\Pw^*\in\mc{P}}
	\]
\end{enumerate}

\bet{Beweis:}\\
(i): "$\Rightarrow$":
Sei $C$ hedgebar zum Anfangskapital $x$.
Dann folgt:
\[
\exists H\in\mc{H}: x+(H\cdot S^*)(N)=C^* \Rightarrow C^*-x\in \mc{K}^* \Rightarrow x\text{ ist ein upper hedging Preis}
\]
Auch gilt:
\[
x-((-H)\cdot S^*(N)\le C^* \Rightarrow -(C^*-x)\in\mc{K}^* \Rightarrow x\text{ ist ein lower hedging Preis}
\]
Also gilt:
\[
x\le P_-(C)\le P_+(C)\le x \Rightarrow P_-(C)=P_+(C)
\]

"$\Leftarrow$":
Sei $P_-(C)=P_+(C)$.
Wegen \hyperref[sub:upper_lower_preise]{5.6} ist $x$ ein upper hedging Preis und ein lower hedging Preis.
Also ist $C^*-x\in \mc{K}^*$ und $-(C^*-x)\in\mc{K}^*$.
Daher gilt:
\[
C^*-x\in \mc{K}^*\cap (-\mc{K^*})=\G^*
\]
Also existiert ein $H\in\mc{H}$ mit 
\[
C^*= x+(H\cdot S^*)(N)
\]
Also ist $C$ hedgebar zum Anfangskapital $x$.
\hfill $\square$\\

(ii):
Sei $C$ hedgebar zum Anfangskapital $x$, so ist $C^*-x\in\G^*$.
Somit gilt
\[
0=\E^*(C^*-x) \Leftrightarrow \E^*C^*=x~\forall \Pw^*\in\mc{P}
\]
\uline{Behauptung:}
$\pi(C)=\{x\}$\\
"$\supseteq$":
Es ist kein strikter upper hedge noch ein strikter lower hedge aus $x$ finanzierbar, denn 
\[
\E^*C^* = x =  \E^*(x+(H\cdot S^*)(N))~\forall H\in\mc{H}
\]
Also ist $x\in\pi(C)$.\\

"$\subseteq$":
$C^*$ ist hedgebar zum Anfangskapital $x$, da $P_-(C)=x=P_+(C)$, also existiert $H\in\mc{H}$ mit
\[
x+(H\cdot S^*)(N)=C^*
\]
Jedes $y>x$ kann man nutzen zur Finanzierung eines strikten upper hedges, denn 
\[
y+(H\cdot S^*)(N)>x+(H\cdot S^*)=C^*
\]
Also ist $(x,\infty)\cap \pi(C)=\emptyset$.\\
Jedes $y<x$ lässt ein strikten lower hedge finanzieren.
Denn:
\[
y+(H\cdot S^*)(N)<x+(H\cdot S^*)(N)=C^*
\]
Also ist $(-\infty,x)\cap\pi(C)=\emptyset$.
Da $x\in\pi(C)$ ist $\pi(C)\subseteq \{x\}$.
\hfill $\square$\\

(iii):
Sei $C$ nicht hedgebar.
Wegen (i) gilt $P_-(C)<P_+(C)$.\\
"$\supseteq$":
Gilt $P_-(C)<x<P_+(C)$, so kann weder ein strikter upper noch ein strikter lower hedge aus $x$ finanziert werden.
Also ist $x\in \pi(C)$.\\
"$\subseteq$":
Ist $x\in\pi(C)$, so kann weder ein strikter upper noch ein strikter lower hedge finanziert werden.
Hieraus folgt:
\[
x\in [P_-(C),P_+(C)]
\]
Im Falle $P_-(C)<P_+(C)$ kann $x$ kein Randwert sein, da für $x=P_+(C)$ ein strikter upper hedge und aus $x=P_-(C)$ ein strikter lower hedge finanziert werden kann.
Also gilt:
\[
\pi(C)\subseteq (P_-(C),P_+(C))
\]
Begründung für den strikten upper hedge:
Wegen der Abgeschlossenheit existiert ein $H\in\mc{H}$ mit $P_+(C)+(H\cdot S^*)(N)\ge C^*$.
Es gilt $\Pw^*(P_+(C)+(H\cdot S^*)(N)>C^*)>0$, denn sonst wäre $H$ eine Hedgestrategie, was $P_-(C)=P_+(C)$ implizieren würde.
Für den strikten lower hedge analog.\\

Zeige weiter:
\[
\sup\limits_{\Pw^*\in\mc{P}}\E^*C^*=P_+(C) \text{ und } \inf\limits_{\Pw^*\in\mc{P}}\E^*C^*=P_-(C)
\]
Ist $x<P_+(C)$, so existiert kein upper hedge für $C$ mit Anfangskapital $x$.
Daraus folgt:
\[
C^*-x\notin \mc{K}^*
\]
Wegen \hyperref[sub:satz_4fima]{5.4} existiert ein $\Pw^*\in\mc{P}$ mit
\[
\E^*C^*-x>0\Rightarrow \E^*C^*>x.
\]
Ist $x>P_-(C)$, so existiert kein lower hedge für $C$ zum Anfangskapital $x$.
Also ist
\[
-(C^*-x)\notin \mc{K}^*
\]
Also existiert ein $\Pw^*\in\mc{P}$ mit 
\[
\E^*(C^*-x)<0 \Rightarrow \E^*C^*<x.
\]
\uline{Behauptung:}
$(P_-(C),P_+(C))=\penbrace{\E^*C^*~|~\Pw^*\in\mc{P}}$\\

\bet{Beweis:}\\
"$\subseteq$":
Klar, da $\inf \E^*C^*=P_-(C)$ und $\sup\E^*C^*=P_+(C)$.\\
"$\supseteq$":
Für $x=P_+(C)$ existiert ein strikter upper hedge, also existiert ein $H\in\mc{H}$ mit 
\[
x+(H\cdot S^*)(N)\ge C^* \text{ und }\Pw^*(x+(H\cdot S^*)(N)>C^*)>0
\]
und damit gilt:
\[
x>\E^*C^*~\forall \Pw^*\in\mc{P}
\]
Für $x=P_-(C)$ existiert ein strikter lower hedge.
Also existiert ein $H\in\mc{H}$ mit
\[
x+(H\cdot S^*)(N)\le C^* \text{ und } \Pw^*(x+(H\cdot S^*)(N)<C^*)>0
\]
\[
\Rightarrow x=\E^*(x+(H\cdot S^*)(N))<\E^*C^*
\]
Also
\[
P_-(C)<\E^*C^*<P_+(C)~\forall \Pw^*\in\mc{P}
\]
\hfill $\square$
%sub end

\subsection{Erweitertes Finanzmarktmodell}
\label{sub:erw_finanzmarktmodell}
Arbitragefreier Finanzmarkt mit $S=(S_1\dt{,}S_d)$ als Preisprozess der Basisgüter.
$S_0$ bezeichne das Numeraire Asset, $(\F_n)_{n=0,\dots,N}$ Informationsverlauf.\\
Im Markt sei $C$ die Auszahlung eines Claims zum Zeitpunkt $N$.
Das arbitragefreie Anfangspreisintervall für $C$ sei gegeben durch
\[
\pi(C)=(P_-(C),P_+(C)\text{ bzw. } \pi(C)=P(C) \text{ falls }P_-(C)=P_+(C)
\]
Der Finanzmarkt soll um den Handel mit $C$ erweitert werden, sodass der erweiterte Finanzmarkt arbitragefrei bleibt.
Der Claim wird als $(d+1)$-tes risky asset angesehen.
Bezeichne dessen Preisprozess mit $(S_{d+1}(n))_{n=0\dt{,}N}$.\\
Ist $x\in \pi(C)$, so existiert ein $\Pw^*\in\mc{P}$ mit $x=\E^*C^*$.
Durch
\[
S_{d+1}(n)=S_0(n)\cdot \E^*(C^*~|~\F_n)~n=1\dt{,}N
\]
kann dann ein Preisprozess definiert werden, für den gilt:
\[
S_{d+1}(N)=S_0(N)C^*=C \text{ und }S_{d+1}(0)=\underbracket{S_0(0)}_{=1}\E^*(C^*~|~\F_0)=\E^*C^*=x
\]
$\F_0$ sei trivial, d.h. $\Pw(A)\in \{0,1\}~\forall A\in \F_0$.
Weiter ist $S_{d+1}^*(n)=\E^*(C^*~|~\F_n),~n=0\dt{,}N$ ein $\Pw^*$-Martingal und damit definiert $\Pw^*$ ein äquivalentes Martingalmaß für das erweiterte Modell $(S_0\dt{,}S_d,S_{d+1})$.\\
Umgekehrt ist $(S_{d+1}(n))_{n=0\dt{,}N}$ ein Preisprozess für $C$ mit $S_{d+1}(N)=C$ und ist das erweiterte Modell arbitragefrei, so existiert ein äquivalentes Martingalmaß $\Pw^*$ für das erweiterte Modell.
Insbesondere gilt:
\[
S_{d+1}(n)=S_0(n)\E^*(C^*~|~\F_n)~n=0\dt{,}N.
\]
Da $\Pw^*$ auch ein äquivalentes Martingalmaß für das Ausgangsmodell ist und $S_{d+1}(0)=\E^*C^*$, gilt
\[
S_{d+1}(0)\in \pi(C).
\]
Insgesamt erhält man:\\
\bet{Theorem:}\\
Der Finanzmarkt ist um den Handel mit $C$ arbitragefrei erweiterbar, genau dann, wenn es ein $\Pw^*\in \mc{P}$ gibt mit
\[
S_{d+1}^*(n)=\E^*(C^*~|~\F_n)~n=0\dt{,}N.
\]
%sub end

\subsection{Vollständigkeit}
\label{sub:vollstaendigkeit}
Für hedgebare Claims ist der arbitragefreie Anfangspreis eindeutig bestimmt.
Finanzmärkte, in denen jeder Claim hedgebar ist, nennt man \Index{vollständig}.\\
Definition: Ein Finanzmarkt heißt vollständig, falls $P_-(C)=P_+(C)$ gilt für alle Claims $C$.
%sub end

\subsection{2. Fundamentalsatz der Preistheorie}
\label{sub:2_fundamentalsatz_preistheorie}
Für ein arbitragefreies Finanzmarktmodell mit äquivalentem Martingalmaß $\Pw^*$ sind äquivalent:
\begin{enumerate}[(i)]
	\item Das Modell ist vollständig.
	\item Das äquivalente Martingalmaß ist eindeutig, i.e.
	\[
	\mc{P}=\penbrace{\Pw^*}
	\]
	\item Zu jedem $\Pw^*$-Martingal $M$ existiert eine Darstellung der Form
	\[
	M_n=\E M_0 + (H\cdot S^*)(n) ~\forall 1\le n\le N
	\]
	mit vorhersehbarem $H\in\mc{H}$.
\end{enumerate}

\bet{Beweis:}\\
\uline{(i)$\Rightarrow$(ii):}
Sei $\Pw^*$ ein äquivalentes Martingalmaß.
Für $A\in \F_N$ ist $C=\mathbb{1}_A S_0(N)$ ein Claim mit $C^*=\mathbb{1}_A$.
Also gilt:
\[
\Pw_1^*(A)=\E_1^*C^*\stackrel{C\text{ ist hedgebar}}{=} \E^*C^*=\Pw^*(A) \Rightarrow \Pw_1^*=\Pw^*
\]

\uline{(ii)$\Rightarrow$(i):}
$\mc{P}=\penbrace{\Pw^*}\Rightarrow P_+(C)=P_-(C)$ für alle $C$ und damit ist jedes $C$ hedgebar.\\

\uline{(i)$\Rightarrow$(iii):}
Sei $M$ ein $(\F_n)$ Martingal bzgl. $\Pw^*$.
Dann ist $C=M(N)S_0(N)$ ein Claim mit $C^*=M(N)$.
Wegen der Vollständigkeit ist $C$ hedgebar.
Also existiert ein $V_0\in \R$ und ein preversibles $H$ mit 
\[
V_0+(H\cdot S^*)(N)=C^*=M(N)
\]
Also gilt:
\[
V_0=\E^*(V_0+(H\cdot S^*)(N))=\E^*C^*=\E^* M(N)=\E^* M(0)
\]
und
\[
\E^* M_0+ (H\cdot S^*)(n)=V_0+(H\cdot S^*)(n)= \E^*(V_0+(H\cdot S^*)(N)~|~\F_n)=\E^*(M(N)~|~\F_n)=M_n
\]

\uline{(iii)$\Rightarrow$(i):}
Ist $C$ ein Claim, so ist
\[
M_n=\E^*(C^*~|~\F_n),~n=0\dt{,}N
\]
ein $\Pw^*$-Martingal.
Wegen (iii) gibt es zu $\E^*C^*=\E^*M_0$ ein preversibles $H$ mit
\[
\E^*C^*+(H\cdot S^*)(N)=C^*
\]
Also ist $C$ hedgebar, also ist der Finanzmarkt vollständig.
\hfill $\square$
%sub end

\subsection{Satz 5}
\label{sub:satz_5fima}
Das arbitragefreie CRR und das verallgemeinerte arbitragefreie CRR-Modell sind vollständig.\\

\bet{Beweis:}\\
Dies folgt aus der Eindeutigkeit des äquivalenten Martingalmaßes.
\hfill $\square$
%sub end

\subsection{Hedgen im CRR-Modell}
\label{sub:hedgen_im_crr-modell}
Gegeben sei ein $N$-Perioden CRR-Modell und ein Claim
\[
C=g(S(N))
\]
Problem: Wie kann man algorithmisch den Hedge und damit auch den Preis ausrechnen?\\
Es gilt:
\[
S(n)=S(0)u^{Z_n}d^{n-Z_n},~Z_n=\sum_{k=1}^{n}X_k
\]
Gesucht ist ein preversibles $H$ und Anfangskapital $V_0$ mit
\[
V_0+(H\cdot S^*)(N)=C^*=\frac{g(S(N))}{(1+\rho)^N}
\]
Für den Wert der Hedgestrategie nach $n$-Perioden gilt:
\begin{equation*}
\begin{aligned}
V_n^*=V_0+(H\cdot S^*)(n) &= \E^*(C^*~|~\F_n) = \E^*(\E^*(C^*~|~\F_{n+1})~|~\F_n) = \E^*(V_{n+1}^*~|~\F_n)\\
&\stackrel{\text{Markov-Eigenschaft}}{=} \E^*(V_{n+1}^*~|~S(n))\\
&= \sum_{k=0}^{n}\E^*(V_{n+1}^*~|~Z_n=k)\mathbb{1}_{\penbrace{S(n)=S(0)u^kd^{n-k}}}
\end{aligned}
\end{equation*}
setze
\[
v^*(n,k)=\E^*(V_{n+1}^*~|~Z_n=k)=\E^*(C^*~|~Z_n=k),~ n=0\dt{,}N,~k=0\dt{,}n
\]
$v^*$ kann dann rekursiv berechnet werden durch
\[
v^*(N,k)=\enbrace{\frac{1}{1+\rho}}^N\cdot g\big(S(0)u^kd^{N-k}\big),~k=0\dt{,}N
\]
'Initialisierung':\\
for $n=N-1$ downto 0 do
\[
v^*(n,k)= p^*v^*(n+1,k+1)+(1-p^*)v^*(n+1,k) \text{ für } k=n,n-1\dt{,}0
\]
Es gilt:
\[
V_n^*=\sum_{k=0}^{n}v^*(n,k)\mathbb{1}_{\penbrace{Z_n=k}}
\]
Damit ist der Wertprozess der Hedgestrategie algorithmisch berechnet.\\
Berechnen der Hedgestrategie im CRR-Modell $C=g(S(N))$.
Gesucht ist $(H_n)_{n=1\dt{,}N}$ mit
\[
V_0+\sum_{k=1}^{N}H(k)\Delta S^*(k)=C^*
\]
Ansatz: $Z(n-1)=k$\\
\[
H(n)=\sum_{k=0}^{n-1}H(n)\mathbb{1}_{\penbrace{Z(n-1)=k}}=\sum_{k=0}^{n-1}h(n,k)\mathbb{1}_{\penbrace{Z(n-1)=k}}
\]
$h(n,k)$ wird rekursiv berechnet.\\
$v^*(n-1,k)$ ist der Preis in Einheiten des Numeraire Assets und $v^*(n,k+X_n)$ zum Zeitpunkt $n+1$ zu hedgen.
Der Hedge berechnet sich aus
\[
v^*(n-1,k)+h(n,k)\Delta S^*(n)=v^*(n,Z_n)\text{ auf }\penbrace{Z_{n-1}=k}
\]
Dies führt zu den Gleichungen
\begin{equation*}
\begin{aligned}
	v^*(n-1,k)+h(n,k)S^*(n-1)\enbrace{\frac{u}{1-\rho}-1}=v^*(n,k+1)\\
	v^*(n-1,k)+h(n,k)S^*(n-1)\enbrace{\frac{d}{1-\rho}-1}=v^*(n,k)	
\end{aligned}
\end{equation*}
Hieraus ergibt:
\begin{equation*}
\begin{aligned}
	h(n,k)=\frac{v^*(n,k+1)-v^*(n-1,k)}{S^*(n-1)\enbrace{\frac{u}{1+\rho}-1}}=\frac{v^*(n,k)-v^*(n-1,k)}{S^*(n-1)\enbrace{\frac{d}{1+\rho}-1}},~k=0\dt{,}n-1
\end{aligned}
\end{equation*}
Beachte: Auf $\penbrace{Z(n-1)=k}$ ist
\[
S^*(n-1)=\enbrace{\frac{1}{1+\rho}}^{n-1}S(0)u^kd^{n-1-k}
\]
Man erhält also den Wertprozess und Hedge für den Claim $C=g(S(N))$ durch folgenden Algorithmus.\\
Initialisierung:
\[
v^*(N,k)=\enbrace{\frac{1}{1+\rho}}^Ng(S(0)u^kd^{N-k}),~k=0\dt{,}N
\]
Rekursionsschritt:\\
for $n=N-1$ down to 0 do
\[
v^*(n,k)=p^*v^*(n+1,k+1)+(1-p^*)v^*(n+1,k)
\]
\[
h(n+1,k)=\frac{v^*(n+1,k+1)-v^*(n,k)}{S^*(n)\enbrace{\frac{u}{1+\rho}-1}},~S^*(n)=S(0)u^kd^{n-k}\frac{1}{(1+\rho)^n}\text{ für }k=0\dt{,}n
\]
Den Wertprozess, in Einheiten des Numeraire Assets, für den Hedge erhält man durch
\[
V_n^*=\sum_{k=0}^{n}v^*(n,k)\mathbb{1}_{\penbrace{Z(n)=k}},~n=0\dt{,}N
\]
und die Hedgestrategie durch
\[
H(n)=\sum_{k=0}^{n-1}h(n,k)\mathbb{1}_{\penbrace{Z(n-1)=k}}
\]
%sub end

\subsection{Algorithmische Berechnung des upper und lower hedging Preises im Trinomialmodell}
\label{sub:alg_berechnung}
\uline{1. Schritt:} Einperiodenfall:\\
$N=1$: Anfangskurs $S_0$, Endkurse $(uS_0,mS_0,dS_0),~0<d<m<u$, Zinsrate $\rho,~d<1+\rho<u$.
\[
\Delta S^*(1)=S^*(1)-S^*(0)=(\frac{uS_0}{1+\rho}-S_0,\frac{mS_0}{1+\rho}-S_0,\frac{dS_0}{1+\rho}-S_0)=S_0(\frac{u}{1+\rho}-1,\frac{m}{1+\rho}-1,\frac{d}{1+\rho}-1)=S_0\cdot R
\]
Ein Claim $C$ entspricht einem Vektor $C=(c_1,c_2,c_3)^T$
\[
C^*=\left(\begin{array}{c}c_1^*\\c_2^*\\c_3^* \end{array}\right),~c_i^*=\frac{c_i}{1+\rho}
\]
Ein Anfangskapital $V_0$ und $H$ Anteile im risky asset liefern einen upper Hedge, wenn 
\[
V_0+H\Delta S^*(1)\ge C^*\Leftrightarrow \left(\begin{array}{c}V_0+HS_0R_1\\V_0+HS_0R_2\\V_0+HS_0R_3 \end{array}\right)\ge \left(\begin{array}{c}c_1^*\\c_2^*\\c_3^*     \end{array}\right)
\]
$\penbrace{(V_0,H)~|~V_0+HS_0R_i\ge c_i^*,~i=1,2,3}$ ist der Durchschnitt von 3 Halbräumen im $\R^2$
\begin{center}
\begin{tikzpicture}[line cap=round,line join=round,>=triangle 45,x=.5cm,y=.5cm]
\draw[->,color=black] (-1.,0.) -- (10.,0.);
\draw[->,color=black] (0.,-5.) -- (0.,7.);
\foreach \y in {-.85,-.45,.7}
\draw[shift={(0,\y)},color=black] (2pt,0pt) -- (-2pt,0pt);
\fill[color=red,fill=red,fill opacity=0.1] (4.8819602175,6.047215913) -- (3.5406387431,2.69391222701) -- (2.89713036698,-1.71830439674) -- (3.47015654654,-3.386000471) -- (7.96407185629,-4.62408516301) -- (7.83100465735,5.11643379907) -- cycle;
\draw (0.,0.)-- (9.93346640053,3.41317365269);
\draw (0.,0.)-- (10.1729873586,-1.48369926813);
\draw (0.,0.)-- (10.,-4.);
\draw (1.27893523472,2.99117968212)-- (4.00894324528,-4.95404830962);
\draw (1.36513180535,-2.74485511735)-- (5.01989125198,6.39204349921);
\draw (4.0485792799,6.17660765644)-- (2.63596555097,-3.50898154322);
\draw [fill=blue] (2.0631344229,0.708900173442) circle (1.5pt);
\draw [fill=blue] (3.08217779981,-0.449526258573) circle (1.5pt);
\draw [fill=blue] (2.12333952784,-0.849335811136) circle (1.5pt);
%\draw [fill=black] (2.37896512266,-0.210271824093) circle (1.5pt);
%\draw [fill=black] (3.5406387431,2.69391222701) circle (1.5pt);
\draw [fill=black] (2.89713036698,-1.71830439674) circle (1.5pt);
\draw[<-,black] (3,-1.72)--(11,-2)node[right]{\tiny{upper hedge Preis}};
\draw(5,4)node[right]{\tiny{upper Hedgestrategien}};
\draw (10,-0.04) node[anchor=north west] {$V_0$};
\draw (-0.1,7.) node[anchor=north west] {$H$};
\draw (0,-.85)node[left]{\tiny$S_0R_3$};
\draw (0,-.45)node[left]{\tiny$S_0R_2$};
\draw (0,.7)node[left]{\tiny$S_0R_1$};
\end{tikzpicture}
\end{center}
\[
V_0+HS_0R_1=c_1^*=\sprod{(V_0,H)}{(1,S_0R_1)}=c_1^*
\]
Numerische Umsetzung: Berechnung der Schnittpunkte:
\begin{equation*}
\begin{aligned}
	V_0^{(1)}+H^{(1)}S_0R_2=c_2^*,~V_0^{(1)}+H^{(1)}S_0R_3=c_3^* &\Leftrightarrow H^{(1)}=\frac{c_3^*-c_2^*}{S_0(R_3-R_2)},~V_0^{(1)}c_2^*-\frac{c_3^*-c_2^*}{R_3-R_2}R_2
\end{aligned}
\end{equation*}
entsprechend
\begin{equation*}
\begin{aligned}
	V_0^{(2)}+H^{(2)}S_0R_1=c_1^*,~V_0^{(2)}+H^{(2)}S_0R_3=c_3^*
\end{aligned}
\end{equation*}
und
\begin{equation*}
\begin{aligned}
V_0^{(3)}+H^{(3)}S_0R_2=c_2^*,~V_0^{(3)}+H^{(3)}S_0R_2=c_2^*
\end{aligned}
\end{equation*}
Ist $V_0^{(1)}=V_0^{(2)}=V_0^{(3)}$, so ist 
\[
P_-(C)=P_+(C)=V_0^{(1)}\text{ und }H^-=H^+=H=H^{(1)}=H^{(2)}=H^{(3)}
\]
der Hedge für $C$.\\
Andernfalls bestimme $l,m,r$, sodass
\[
V_0^{(l)}\le V_0^{(m)}\le V_0^{(r)}
\]
Entscheide, ob $(V_0^{(m)},H^{(m)})$ ein upper Hedge ist durch
\[
V_0^{(m)}+H^{(m)}S_0R_m>c_m^*
\]
Ist dies der Fall, so ist
\[
P_+(C)=V_0^{(m)}\text{ und }H^+=H^{(m)}
\]
der upper Hedge und $P_-(C)=V_0^{(l)},~H^-=H^{(l)}$ der lower Hedge.\\
ist dies nicht der Fall, so ist
\[
P_-(C)=V_0^{(m)}\text{ und }H^-=H^{(m)}
\]
der lower Hedge.
Weiter ist dann $P_+(C)=V_0^{(r)},~H^{(r)}=H^+$ der upper Hedge.\\

\uline{2. Schritt:} Mehr Perioden-Fall ($N$-Perioden):
\begin{equation*}
\begin{aligned}
	S(n)&=S_0\prod_{i=1}^{n}Y_i,~(Y_i)\text{ iid, }Y_i\text{ hat nur Werte in }\penbrace{m,u,d}\\
	&= S_0u^{Z_1(n)}d^{Z_2(n)}m^{n-(Z_1(n)+Z_2(n)}\text{ mit }Z_1(n)=\sum_{k=1}^{n}\mathbb{1}_{\penbrace{Y_k=u}},~ Z_2(n)=\sum_{k=1}^{n}\mathbb{1}_{\penbrace{Y_k=d}}
\end{aligned}
\end{equation*}
Claim $C$ der Form $C=g(S(N))$.\\
Rekursiv wird die upper und lower hedging Strategie berechnet:\\
Initialisierung:
\[
v^-(N,(k,l))=v^+(N,(k,l))=(1+\rho)^{-N}g(S_0u^kd^lm^{N-(k+l)})
\]
for $n=N-1$ down to 0 do\\
	for $k=0\dt{,}n$\\
	for $l=0,\dt{,}n-k$\\
	Berechne den upper hedging Preis, sowie upper Hedge im Trinomialmodell mit Anfangskurs $S_0u^kd^lm^{n-(k+l)}$ und Claim $C^+=(v^+(n+1,(k+1,l)),~v^+(n+1,(k,l)),v^+(n+1,(k,l-1)))$\\
	Setze $v^+(n,(k,l))=P_+(C)$ und $h^+(n+1,(k,l))=H^+$\\
	Berechne den lower hedging Preis, sowie lower Hedge im Trinomialmodell mit Anfangskurs $S_0u^kd^lm^{n-(k+l)}$ und Claim $C^-=(v^-(n+1,(k+1,l)),~v^-(n+1,(k,l)),v^-(n+1,(k,l-1)))$\\
	Setze $v^-(n,(k,l))=P_-(C)$ und $h^-(n+1,(k,l))=H^-$\\

Es gilt:
\[
v^+\big(0,(0,0)\big)=P_+(C)
\]
ist das Anfangskapital des minimalen upper hegdes und 
\[
H_n^+:=\sum_{k=0}^{n-1}\sum_{l=0}^{n-k-1}h^+(n,(k,l))\mathbb{1}_{\penbrace{Z_1(n-1)=k}}\mathbb{1}_{\penbrace{Z_2(n-1)=l}}
\]
die minimale upper hedge Strategie, insbesondere gilt damit
\[
P_+(C)=\sum_{k=1}^{N}H(k)\Delta S^*(k)\ge C^*
\]
Entsprechend
\[
v^-(0,(0,0))=P_-(C)
\]
das Anfangskapital für den maximalen lower hedge und
\[
H_n^-=\sum_{k=0}^{n-1}\sum_{l=0}^{n-k-1}h^-(n,(k,l))\mathbb{1}_{\penbrace{Z_1(n-1)=k}}\mathbb{1}_{\penbrace{Z_2(n-1)=l}}
\]
ist der maximale lower hedge
\[
P_-(C)=\sum_{k=1}^{N}H(k)\Delta S^*(k)\le C^*
\]
%sub end

\subsection{Allgemeine Call-Formel}
\label{sub:allg_call-formel}
Betrachte Finanzmarkt über $N$-Perioden mit $(S(n))_{n=0\dt{,}N}\subseteq \R_+$ als Preisprozess für das risky asset und $(\beta(n))_{n=0\dt{,}N}$ als Geldmarktkonto, $(\F_n){n=0\dt{,}N}$ Filtration und damit
\[
\beta(n)=\prod_{k=1}^{n}(1+\zeta(k))
\]
mit vorhersehbarem Prozess $\zeta>-1$.
Wir betrachten einen Call mit strike $K$, d.h. $C=(S(N)-K)^+$ ist die Claimauszahlung nach $N$ Perioden.\\
Annahme: $C$ sei hedgebar und das Modell arbitragefrei.\\
Dann gilt: $\E^*C^*$ ist der eindeutige arbitragefreie Anfangspreis für $C$, wobei $\Pw^*\in \mc{P}$ beliebig gewählt werden kann.
\begin{equation*}
\begin{aligned}
	P(C)&=\E^*C^*=\E^*\enbrace{\frac{(S(N)-K)^+}{\beta(N)}}\\
	&= \E^*\frac{S(N)}{\beta(N)}\mathbb{1}_{\penbrace{S(N)>K}}-\E^*\frac{K}{\beta(N)}\mathbb{1}_{\penbrace{S(N)>K}}\\
	&= \E^*S^*(N)\mathbb{1}_{\penbrace{S(N)>K}}-K\cdot\E^*\frac{1}{\beta(N)}\mathbb{1}_{\penbrace{S(N)>K}}\\
	&= S_0\cdot \E^*\frac{S^*(N)}{S_0}\mathbb{1}_{\penbrace{S(N)>K}}-K\cdot B(0,N)\cdot \E^*\frac{1}{\beta(N)}\cdot\frac{1}{B(0,N)}\mathbb{1}_{\penbrace{S(N)>K}} \text{ mit } B(0,N):=\E^*\frac{1}{\beta(N)}
\end{aligned}
\end{equation*}
Definiere äquivalente Maße $\Pw_1^*$ und $\Pw_2^*$ durch
\[
\left.\frac{d\Pw_1^*}{d\Pw^*}\right|_{\F_N}=\frac{S^*(N)}{S_0}\text{ und } \left.\frac{d\Pw_2^*}{d\Pw^*}\right|_{\F_N}=\frac{1}{\beta(N)}\frac{1}{B(0,N)}
\]
Dann gilt:
\[
P(C)=S(0)\Pw_1^*(S(N)>K)-KB(0,N)\Pw_2^*(S(N)>K)
\]
Im CRR-Modell ist
\[
\beta(N)=(1+\zeta)^N,\text{ also }\Pw_2^*=\Pw^*
\]
somit folgt:
\[
P(C)=S(0)\Pw_1^*(S(N)>K)-K(1+\zeta)^{-N}\Pw^*(S(N)>K)
\]
\[
\frac{S(N)}{S(0)}=u^{Z(N)}d^{N-Z(N)}
\]
Dann folgt:
\[
S(N)>K\Leftrightarrow Z(N)>\frac{\ln\frac{K}{S(0)}-N\ln d}{\ln u-\ln d}=b_N
\]
Bzgl. $\Pw^*$ ist $Z(N)$ eine $\Bin(N,p^*)$ verteilte Zufallsvariable mit
\[
p^*=\frac{(1+\zeta)-d}{u-d}
\]
Im CRR-Modell ist bzgl. $\Pw_1^*$ der Zählprozess $(Z(n))_{n=0\dt{,}N}$ ein Random-Walk $Z(n)=\sum_{k=1}^{n}X_k$, mit 
\[
\Pw_1^*(X_k=1)=p_1^*=\frac{p^*u}{1+\zeta}
\]
Bzgl. $\Pw_1^*$ ist $Z(N)$ eine $\Bin(N,p_1^*)$ verteilte Zufallsvariable. Also gilt:
\[
P(C)=S(0)\Bin(N,p_1^*)((b_N,\infty))-K\Bin(N,p^*)((b_N,\infty))
\]
Dies ist die diskrete \Index{Black-Scholes Formel}
%sub end
%sec end
\newpage

\section{Das Black-Scholes Modell}
\label{sec:black_scholes_modell}
Ziel: Modellierung von Finanzmärkten in stetiger Zeit.

\subsection{Beschreibung des Modells}
\label{sub:beschreibung_des_modells}
Finanzmarktmodell besteht aus
\begin{itemize}
	\item einem Geldmarktkonto
	\item ein risky asset
	\item Laufzeit $T$
	\item[Geldmarktkonto:]
	\item Annahme: deterministische stetige Verzinsung mit Rate $r$.
	Daher entwickelt sich das Geldmarktkonto gemäß
	\[
	\beta(t)=e^{rt},~0\le t\le T
	\]
	\item[risky asset:]
	\item Anfangskurs $S_0>0$. Annahme:
	\begin{enumerate}[(a)]
		\item die relativen Kursänderungen sind unabhängig und zeitlich stationär
		\item die Kursänderungen sind stetig
	\end{enumerate} 
	Hieraus folgt, dass der Kursverlauf $(S(t))_{0\le t\le T}$ des risky assets durch einen stochastischen Prozess der Form
	\[
	S(t)=S(0)\exp\enbrace{\sigma W(t)-\frac{1}{2}\sigma^2t}e^{\mu t},~t\le T\text{ mit } \mu\in \R,\sigma>0
	\]
	beschrieben werden kann. $(W(t))_{t\ge 0}$ bezeichnet dabei den \Index{Wiener-Prozess}. Dieser ist definiert durch die folgenden Bedingungen
	\begin{enumerate}[(i)]
		\item $W(0)=0~\pfs$
		\item Für beliebige $0\le t_0<t_1\dt{<}t_n,~n\in \N$ sind 
		\[
		W_{t_1}-W_{t_0},W_{t_2}-W_{t_1}\dt{,}W_{t_n}-W_{t_{n-1}}
		\]
		stochastisch unabhängig.
		\item Für alle $0\le s,~t>0$ gilt
		\[
		W_{s+t}-W_s\sim W_t-W_0=W_t\sim N(0,t)
		\]
		\item $(W_t)_{t\ge0}$ hat stetige Pfade
	\end{enumerate}
	Wieso erfüllt das Modell die Annahmen?\\
	Die relativen Kursänderungen in $t_1\dt{<}t_n$ sind gegeben durch
	\[
	\frac{S(t_1)-S(t_0)}{S(t_0)},\frac{S(t_2)-S(t_1)}{S(t_1)}\dt{,}\frac{S(t_n)-S(t_{n-1})}{S(t_{n-1})}.
	\]
	Da
	\[
	S\frac{S(t_i)-S(t_{i-1})}{S(t_{i-1})}= \exp\enbrace{\sigma(W(t_i)-W(t_{i-1}))-\frac{1}{2}\sigma^2(t_i-t_{i-1})}e^{\mu(t_i-t_{i-1})}-1
	\]
	folgt die Unabhängigkeit und zeitliche Stationarität der relativen Kursänderungen aus (ii) und (iii).
	Die Annahme (b) ist erfüllt wegen (iv).
	Das das Modell aus den Annahmen folgt, ist nicht ganz so einfach zu beweisen.
	Das folgt aus der Tatsache, dass ein stochastischer Prozess $X$ mit unabhängigen stationären Zuwächsen der stetige Pfade hat, notwendigerweise ein Wiener-Prozess mit Drift sein muss, d.h.
	\[
	X(t)=\sigma W(t)+\nu t\text{ mit }\sigma>0\text{ und }\nu\in\R
	\]
	Nur aus der Annahme (a) ergeben sich sogenannte \Index{Levy-Prozess Modelle}.
\end{itemize}
%sub end

\subsection{Approximation eines Black-Scholes Modells durch ein CRR-Modell}
\label{sub:appox_bs-modell}
Gegeben: BS-Modell mit Parametern $\sigma>0$ für die \Index{Volatilität}, $T>0$ für die Laufzeit, $\mu>0$ für den Trend und $R>0$ für die Zinsrate, daraus ergibt sich
\[
S(t)=S(0)e^{\mu t}\exp\enbrace{\sigma W_t-\frac{1}{2}\sigma^2t}
\]
Es soll in geeigneter Weise ein CRR-Modell angepasst werden.
Teile hierzu den Zeitbereich in äquidistante Intervalle $[t_{i-1},t_i]$ mit $0=t_0,~T=t_n$, erhalten $\frac{T}{n}=\Delta_n$ als Intervalllänge.
Approximiere $S(t_j)=S(j\Delta)$ für $j=1\dt{,}n$ durch
\[
S_n(t_j)=S(0)u_n^{Z_n(j)}d_n^{j-Z_n(j)}
\]
mit $Y_1\dt{,}Y_n$ iid. $\Pw(Y_i=u_n)=p_n=1+\Pw(Y_i=d_n)$ und $Z_n(j)=\sum_{k=1}^{j}\mathbb{1}_{\penbrace{Y_k=u_n}}$.
$(S_n(t_j))_{j=0\dt{,}n}$ definiert einen Aktienpreisprozess in einem CRR-Modell.\\
Frage: Wie kann man $u_n,d_n,p_n$ sinnvoll wählen.\\
Ansatz: Wähle $u_n,d_n,p_n$ so, dass der Erwartungswert und die Varianz der log Rendite bis $T$ übereinstimmen.
Es gilt:
\begin{equation*}
\begin{aligned}
	\E\log\enbrace{\frac{S(T)}{S(0)}}&=\E\enbrace{\mu T+\sigma W_t-\frac{1}{2}\sigma^2T}\\
	&=\enbrace{\mu-\frac{1}{2}\sigma^2}T\\
	\V \log\frac{S(T)}{S(0)}&=\sigma^2\V W_t=\sigma^2T
\end{aligned}
\end{equation*}
Im CRR-Modell:
\begin{equation*}
\begin{aligned}
	\E \log\frac{S_n(T)}{S_n(0)}&=\E\log\prod_{k=1}^{n}Y_k = \E\sum_{k=1}^{n}\log Y_k\\
	&=n\E\log Y_1 = n((\log u_n)p_n+(\log d_n)(1-p_n))\\
	\V \log\frac{S_n(T)}{S_n(0)}&= \sum_{k=1}^{n}\V(\log Y_1)\\
	&=n\bigg(p_n(\log u_n)^2+(1-p_n)(\log d_n)^2-\big((\log u_n)p_n+(\log d_n)(1-p_n)\big)^2 \bigg)
\end{aligned}
\end{equation*}
Das führt auf die Gleichungen
\begin{equation*}
\begin{aligned}
	p_n\log u_n + (1-p_n)\log d_n &= \enbrace{\mu-\frac{1}{2}\sigma^2}\frac{T}{n}\\
	p_n\log^2 u_n + (1-p_n)\log^2 d_n &= \frac{\sigma^2T}{n}+\enbrace{\frac{\enbrace{\mu-\frac{1}{2}\sigma^2}T}{n}}^2
\end{aligned}
\end{equation*}
welche durch
\begin{equation*}
\begin{aligned}
	\log u_n &= \frac{\enbrace{\mu-\frac{1}{2}\sigma^2}T}{n}+\enbrace{\frac{1-p_n}{p_n}\frac{\sigma^2T}{n}}^{\frac{1}{2}}\\
	\log d_n &= \frac{\enbrace{\mu-\frac{1}{2}\sigma^2}T}{n}-\enbrace{\frac{1-p_n}{p_n}\frac{\sigma^2T}{n}}^{\frac{1}{2}}
\end{aligned}
\end{equation*}
Strebt $p_n\to p\in (0,1)$ für $n\to \infty$, so ist
\[
d_n<\underbracket{e^{r\frac{T}{n}}}_{=1+\zeta_n}< u_n
\]
denn 
\begin{equation*}
\begin{aligned}
	\log u_n &\sim \enbrace{\frac{1-p_n}{p_n}\frac{\sigma^2T}{n}}^{\frac{1}{2}}>r\frac{T}{n}\\
	\log d_n &\sim -\enbrace{\frac{1-p_n}{p_n}\frac{\sigma^2T}{n}}^{\frac{1}{2}}<r\frac{T}{n}
\end{aligned}
\end{equation*}
Im folgenden setze die Sprungwahrscheinlichkeit
\[
p_n=p\in (0,1)~\forall n\in \N
\]
setze $\eta=\mu-\frac{1}{2}\sigma^2$.\\
Definiere mit dem diskreten CRR-Aktienprozess $(S_n(t_j))_{j=0\dt{,}n}$ einen stochastischen Prozess, mit $(S_n(t))_{0\le t\le T}$ durch
\[
S_n(t)=S_n(t_{i-1})\text{ für } t_{i-1}\le t<t_i~\forall 1\le i\le n
\]
Für festes $t\in[0,T]$ und für $i_n=\lfloor n\frac{t}{T}\rfloor$ gilt:
\[
i_n\frac{T}{n}\le t<(i_n+1)\frac{T}{n},~\frac{i_n}{n}\to \frac{t}{T}
\]
Mit Hilfe des zentralen Grenzwertsatzes für Dreiecksschemata gilt:
\begin{equation*}
\begin{aligned}
	\log\frac{S_n(t)}{S_n(0)} &= \log\enbrace{\frac{S_n\enbrace{i_n\frac{T}{n}}}S_n(0)}\\
	&=\underbracket{\log\enbrace{\frac{S_n\enbrace{i_n\frac{T}{n}}}S_n(0)}-i_n\frac{1}{n}\enbrace{\mu-\frac{1}{2}\sigma^2}T}_{\to N(0,\sigma^2t)\text{ in Vert. nach CLT}}+\underbracket{\frac{i_n}{n}\enbrace{\mu-\frac{1}{2}\sigma^2}T}_{\to \enbrace{\mu-\frac{1}{2}\sigma^2}t}
\end{aligned}
\end{equation*}
da
\begin{equation*}
\begin{aligned}
	\E\log\enbrace{\frac{S_n\enbrace{i_n\frac{T}{n}}}{S_n(0)}}&= i_n\frac{1}{n}\enbrace{\mu-\frac{1}{2}\sigma^2}T\\
	\V\log\enbrace{\frac{S_n\enbrace{i_n\frac{T}{n}}}{S_n(0)}}&= i_n\frac{1}{n}\sigma^2T
\end{aligned}
\end{equation*}
Also gilt:
\[
\log\enbrace{\frac{S_n(t)}{S_n(0)}}\stackrel{n\to\infty}{\to} N\enbrace{\enbrace{\mu-\frac{1}{2}\sigma^2}t,\sigma^2t}\text{ in Verteilung}
\]
Da 
\[
\log\enbrace{\frac{S(t)}{S(0)}}\sim N\enbrace{\enbrace{\mu-\frac{1}{2}\sigma^2}t,\sigma^2t}
\]
folgt hieraus
\[
S_n(t)\stackrel{n\to\infty}{\to}S(t)\text{ Konvergenz in Verteilung (mit d bezeichnet).}
\]
Für $0<s_1<s_2\dt{<}s_k\le T$ folgt wegen der Unabhängigkeit und Stationarität von $\enbrace{\log\frac{S_n(t_j)}{S_n(0)}}_{j=0\dt{,}n}$ analog mit dem CLT, dass
\[
\enbrace{\log\enbrace{\frac{S_n(s_1)}{S_n(0)}}\dt{,}\log\enbrace{\frac{S_n(s_k)}{S_n(0)}}}\stackrel{\text{d}}{\to}\enbrace{\log\enbrace{\frac{S(s_1)}{S(0)}}\dt{,}\log\enbrace{\frac{S(s_k)}{S(0)}}}
\]
Hieraus erhält man, dasss die Familie der endlich dimensionalen Verteilungen von $S_n$ gegen die Familie der endlich dimensionalen Verteilungen von $S$ konvergieren.
Genauer: Für alle $0<t_1\dt{<}t_k\le T,~k\in \N$ gilt
\[
\enbrace{S_n(t_1)\dt{,}S_n(t_k)}\stackrel{\text{d}}{\to}\enbrace{S(t_1)\dt{,}S(t_k)}
\]
Zusammen mit einer \Index{Straffheitsbedingung} folgt hieraus die schwache Konvergenz von $(S_n)_{n\in \N}$ gegen $S$ in $\mathds{D}[0,T]$, wobei
\[
\mathds{D}[0,T]=\penbrace{x:[0,T]\to\R~|~x \text{ ist rechtsseitig stetig und hat linksseitige Limites}}
\]
%sub end

\subsection{Eigenschaften des Wiener-Prozesses}
\label{sub:eig_weiner-prozess}
Sei $(\Omega,\F,\Pw)$ ein W'Raum und $(\F_t)_{t\ge0}$ eine Filtration.\\
Ein stochastischer Prozess $(W_t)_{t\ge0}$ heißt \Index{Wiener-Prozess} bzgl. $(\F_t)_{t\ge0}$, wenn gilt:
\begin{enumerate}[(i)]
	\item $W$ ist adaptiert bzgl. $(\F_t)_{t\ge0}$
	\item $W_0=0$ \Pfs
	\item $W_t-W_s$ ist stochastisch unabhängig von $\F_s$ für alle $0\le s<t$
	\item $W_t-W_s\sim W_{t-s}\sim N(0,t-s)~\forall 0\le s<t$
	\item $W$ hat \Pfs stetige Pfade
\end{enumerate}
Im folgenden sollen Martingale bestimmt werden.

\minisec{Satz}
Sei $W$ ein Wiener-Prozess bzgl. $(\F_t)_{t\ge0}$.
Dann gilt:
\begin{enumerate}[(i)]
	\item $W$ ist ein Martingal
	\item $(W_t^2-t)$ ist ein Martingal
	\item $\enbrace{\exp\enbrace{\nu W_t-\frac{1}{2}\nu^2 t}}_{t\ge 0}$ ist ein Martingal
\end{enumerate}

\bet{Beweis:}\\
\begin{enumerate}[(i)]
	\item 
	\begin{equation*}
	\begin{aligned}
		\E(W_t~|~\F_s)&= \E(W_s+W_t-W_s~|~\F_s)=\E(W_s~|~\F_s)+\E(W_t-W_s~|~\F_s)\\
		&= W_s+\underbracket{\E(W_t-W_s)}_{=0}~\forall s\le t
	\end{aligned}
	\end{equation*}
	\item\begin{equation*}
	\begin{aligned}
		\E(W_t^2~|~\F_s)&= \E((W_s+W_t-W_s)^2~|~\F_s)=\E(W_s^2+2W_s(W_t-W_s)+(W_t-W_s)^2~|~\F_s)\\
		&= W_s^2+ \E(2W_s(W_t-W_s)~|~\F_s)+ \E((W_t-W_s)^2~|~\F_s)\\
		&= W_s^2+2W_s\underbracket{\E(W_t-W_s~|~\F_s)}_{=\E(W_t-W_s)=0}+ \E(W_t-W_s)^2\\
		&= W_s^2+ \E(W_{t-s})^2 = W_s^2+t-s
	\end{aligned}
	\end{equation*}
	\item $s\le t$
	\begin{equation*}
	\begin{aligned}
		\E(\exp(\nu W_t)~|~\F_s) &= \E(\exp(\nu(W_s+W_t-W_s))~~\F_s)= \E(\exp(\nu W_s)\exp(\nu(W_t-W_s)~|~\F_s))\\
		&= \exp(\nu W_s)\E(\exp(\nu(W_t-W_s)~|~\F_s)) = \exp(\nu W_s)\E(\exp(\nu(W_t-W_s)))\\
		&= \exp(\nu W_s)\E(\exp(\nu W_{t-s})) = \exp(\nu W_s)\exp\enbrace{\frac{1}{2}\nu^2(t-s)}
	\end{aligned}\marginnote{letzter Schritt stoch. Analysis (vgl.brownsche Bewegung)}
	\end{equation*}
\end{enumerate}
%sub end

\uline{Ziel:} 
Konstruktion des äquivalenten Martingalmaßes im Black-Scholes Modell.

\subsection{Maßwechsel}
\label{sub:masswechsel}
Sei $(\Omega,\F,\Pw)$ ein W'Raum, $(\F_t)_{t\ge 0}$ eine Filtration, sei $(L_t)_{t\ge0}$ ein positives Martingal bzgl. $\Pw$ und $\bar{\Pw}$ ein weiteres W'Maß auf $(\Omega,\F)$ mit:
\[
\left.\frac{\dint \bar{\Pw}}{\dint \Pw}\right|_{\F_t}=L_t~\forall t\ge 0
\]
Dann gilt:
\begin{enumerate}[(i)]
	\item Ist $Y$ $\F_t$-messbar und existiert $\bar{\E}Y$, so gilt
	\[
	\bar{\E}(Y~|~\F_s)=\frac{\E(YL_t~|~\F_s)}{L_s}~\forall s\le t
	\]
	dabei ist $\bar{\E}Y=\int Y\dint \bar{\Pw}$ und $\bar{\E}(Y~|~\F_t)$ der bedingte Erwartungswert von $Y$ bzgl. $\bar{\Pw}$.
	\item $(M_t)_{t\ge0}$ ist ein $\bar{\Pw}$-Martingal genau dann, wenn $(M_tL_t)_{t\ge0}$ ein $\Pw$-Martingal ist.
	\item Ist $(R_t)_{t\ge 0}$ ein positives $\Pw$-Martingal mit $\E R_t=1~\forall t\ge 0$, so kann auf jedem $\F_T$ ein W'Maß $Q_T$ definiert werden, mit
	\[
	\left.\frac{\dint Q_T}{\dint \Pw}\right|_{\F_t}=R_t~\forall t\le T
	\]
\end{enumerate}

\bet{Beweis:}\\
\uline{zu (i):}
Sei $Y$ $\F_t$-messbar und $A\in \F_s$
\begin{equation*}
\begin{aligned}
	\int\ablim{A}Y\dint \Pw &= \int\ablim{A}YL_t\dint\Pw = \int\ablim{A}\E(YL_t~|~\F_s)\dint\Pw\\
	&= \int\ablim{A}\E(YL_t~|~\F_s)\frac{1}{L_s}\dint\bar{\Pw}, \text{ da } \left.\frac{\dint \bar{\Pw}}{\dint \Pw}\right|_{\F_s}=L_s
\end{aligned}
\end{equation*}

\uline{zu (ii:)}
$(M_t)_{t\ge0}$ ist ein $\bar{\Pw}$-Martingal
\begin{equation*}
\begin{aligned}
	&\Leftrightarrow \bar{\E}(M_t~|~\F_s)=M_s~\forall s\le t \Leftrightarrow \E(M_tL_t~|~\F_s)\frac{1}{L}=M_s~\forall s\le t\\
	&\Leftrightarrow \E(M_tL_t~|~\F_s)=M_sL_s~\forall s\le t \Leftrightarrow ML \text{ ist ein }\Pw\text{-Martingal}
\end{aligned}
\end{equation*}

\uline{zu (iii):}
Wegen $\E R_T=1$ definiert
\[
Q_T(A)=\int\ablim{A}R_T\dint\Pw~\forall A\in \F_T
\]
ein zu $\Pw$ äquivalentes W'Maß auf $(\Omega,\F_T)$.
Für $A\in \F_t$ mit $t\le T$ gilt:
\begin{equation*}
\begin{aligned}
	Q_T(A) &= \int\ablim{A}R_T\dint\Pw \stackrel{A\in F_t}{=} \int\ablim{A}\E(R_T~|~\F_t)\dint\Pw\\
	&= \int\ablim{A}R_t\dint\Pw
\end{aligned}
\end{equation*}
%sub end

\subsection{Girsanov Transformation}
\label{sub:girsanov_transformation}
Sei $(W_t)_{t\ge 0}$ ein Wiener-Prozess bzgl. einer Filtration $(\F_t)_{t\ge0}$.
Sei für $\nu\in \R$ ein weiteres Maß $\Pw_\nu$ auf $(\Omega,\F_\infty)$ gegeben mit
\[
\left.\frac{\dint\Pw_\nu}{\dint \Pw}\right|_{\F_t}=\exp\enbrace{\nu W_t-\frac{1}{2}\nu^2t}~\forall t\ge0
\]
und $\F_\infty:=\sigma\enbrace{\bigcup\ablim{t\ge0}\F_t}$.\\
Dann gilt:
\[
\bar{W}_t=W_t-\nu t,~t\ge 0
\]
ist ein Wiener-Prozess bzgl. $\Pw_\nu$.\\

\bet{Beweis:}\\
Zeige die definierenden Eigenschaften des Wiener-Prozesses:
\begin{enumerate}[(i)]
	\item $(\bar{W}_t)_{t\ge 0}$ hat stetige Pfade mit $\bar{W}_0=0$.
	\item $\bar{W}_t-\bar{W}_s$ ist unabhängig von $\F_s$ und verteilt wie eine $N(0,t-s)$ Verteilung
\end{enumerate}
\uline{zu (i):}
ist klar.\\
\uline{zu (ii):}
Sei $g:\R\to\R$ beschränkt messbar.
\begin{equation*}
\begin{aligned}
	\E_\nu (g(\bar{W}_t-\bar{W}_s)~|~\F_s) &= \E (g(\bar{W}_t-\bar{W}_s)L_t~|~\F_s)\frac{1}{L_s} \text{ mit } L_t=\exp\enbrace{\nu W_t-\frac{1}{2}\nu^2t}\\
	&= \E\enbrace{g(W_t-W_s-\nu(t-s))\frac{L_t}{L_s}~|~\F_s}\\
	&= \E\enbrace{g(W_t-W_s-\nu(t-s))\exp\enbrace{\nu(W_t-W_s)-\frac{1}{2}\nu^2(t-s)}~|~\F_s}\\
	&= \E g(W_t-W_s-\nu(t-s))\exp\enbrace{\nu(W_t-W_s)-\frac{1}{2}\nu^2(t-s)}\\
	&= \E g(W_{t-s}-\nu(t-s))\exp\enbrace{\nu W_{t-s}-\frac{1}{2}\nu^2(t-s)}\\
	&= \E_\nu g(\bar{W}_{t-s})
\end{aligned}
\end{equation*}
Hieraus folgt: $\bar{W}_t-\bar{W}_s$ ist stochastisch unabhängig von $\F_s$ und genauso verteilt wie $\bar{W}_{t-s}$.
Dies ist eine $N(0,t-s)$ Verteilung, denn
\begin{equation*}
\begin{aligned}
	\E_\nu g(\bar{W}_t) &= \E g(W_t-\nu t)\exp\enbrace{\nu W_t-\frac{1}{2}\nu^2 t}\\
	&= \E g(W_t-\nu t)\exp\enbrace{\nu (W_t-\nu t)+\frac{1}{2}\nu^2t} = e^{\frac{1}{2}\nu^2t}\int g(x)e^{\nu x}N(-\nu t,t)(\dint x)\\
	&= e^{\frac{1}{2}\nu^2t}\int g(x)e^{\nu x}\frac{1}{\sqrt{2\pi t}}\exp\enbrace{-\frac{1}{2t}(x+\nu t)^2}\dint x\\
	&= \frac{1}{\sqrt{2\pi t}}\int g(x)e^{-\frac{1}{2t}x^2}\dint x = \int g(x)N(0,t)(\dint x)
\end{aligned}
\end{equation*}
Also ist $ \bar{W}_t$ $N(0,t)$-verteilt bzgl. $\Pw_\nu$.
%sub end

\subsection{Äquivalentes Martingalmaß im Black-Scholes Modell}
\label{sub:aquiv_martingmass_im_b_s_modell}
Sei $(\Omega,\F,\Pw)$ ein W'Raum, $(\F_t)_{t\ge 0}$ eine Filtration und $W$ ein Wiener-Prozess bzgl. $(\F_t)_{t\ge0}$.
Sei $[0,T]$ der Handelszeitraum eines Finanzmarktes,
\[
S(t)=S_0 e^{\mu t}\exp\enbrace{\sigma W_t-\frac{1}{2}\sigma^2t},~0\le t<T
\]
sei der Preisprozess eines risky assets, $S_0>0$ der Anfangspreis, $\mu\in \R$ Trendparameter, $\sigma>0$ Volatilität und $\beta(t)=e^{rt},~t\ge 0$ der Preisprozess eines Geldmarktkontos mit Zinsrate $r$.\\
Definition: Ein W'Maß $\Pw^*$ auf $(\Omega,\F_T)$ heißt äquivalentes Martingalmaß genau dann, wenn
\begin{enumerate}[(i)]
	\item $\Pw^*$ ist äquivalent zu $\Pw$ auf $\F_T$
	\item $S^*(t):=\frac{S(t)}{\beta(t)}=e^{-rtS(t)},~0\le t\le T$ ist ein $\Pw^*$-Martingal.
\end{enumerate}

\bet{Beweis:}\\
Ansatz: 
\[
\left.\difff{\Pw^*}{\Pw}\right|_{\F_t}=\exp\enbrace{\nu W(t)-\frac{1}{2}\nu^2t},
\]
zu bestimmen ist $\nu$.\\
Girsanov liefert $W^*(t)=W(t)-\nu t,~t\ge 0$ ist ein Wiener-Prozess bzgl. $\Pw^*$.
Bzgl. $\Pw^*$ gilt:
\begin{equation*}
\begin{aligned}
	S(t)&=S_0 e^{\mu t}\exp\enbrace{\sigma W_t-\frac{1}{2}\sigma^2t}\\
	&=S_0 e^{\mu t}\exp\enbrace{\sigma (W_t^*+\nu t)-\frac{1}{2}\sigma^2t}\\
	&=S_0 \exp\enbrace{\sigma W_t^*-\frac{1}{2}\sigma^2t} e^{(\mu+\sigma \nu)t}
\end{aligned}
\end{equation*}
Also
\[
S_t^*=e^{-rt}S_t=S_0 \exp\enbrace{\sigma W_t^*-\frac{1}{2}\sigma^2t} e^{(\mu+\sigma \nu)t}
\]
und damit $(S_t^*)$ ist ein $\Pw^*$-Martingal genau dann, wenn
\[
\mu -r+\sigma \nu \Leftrightarrow \nu = -\frac{\mu-r}{\sigma}
\]
\uline{Ergebnis:}\\
Für $\nu = -\frac{\mu-r}{\sigma}$ ist $\Pw^*$ ein äquivalentes Martingalmaß.\\

\bet{Bemerkung:}\\
Bzgl. $\Pw^*$ gilt:
\[
S(t)=S_0 e^{r t}\exp\enbrace{\sigma W_t^*-\frac{1}{2}\sigma^2t},~t\ge 0
\]
Also ist $S$ ein geometrischer Wiener-Prozess mit Trend $r$ und Volatiliät $\sigma$.\\
$\enbrace{\frac{S_t^*}{S_0}}_{t\ge0}$ ist ein $\Pw^*$-Martingal und damit ein positives Martingal mit
\[
\E^*\frac{S_t^*}{S_0}=\frac{S_0^*}{S_0}=1
\]
Deshalb kann ein Maßwechsel durchgeführt werden:
\[
\left.\difff{\Pw_\sigma^*}{\Pw}\right|_{\F_t}=\frac{S_t^*}{S_0}=\exp\enbrace{\sigma W_t^*-\frac{1}{2}\sigma^2t}
\]
Da $W^*$ ein Wiener-Prozess bzgl. $\Pw^*$ ist, gilt nach Girsanov
\[
W_t^**=W_t^*-\sigma t,~t\ge0
\]
ist ein Wiener-Prozess bzgl. $\Pw_\sigma^*$.
Weiter ist:
\begin{equation*}
\begin{aligned}
	S(t)&= S_0 e^{r t}\exp\enbrace{\sigma W_t^*-\frac{1}{2}\sigma^2t}\\
	&= S_0 e^{rt}\exp\enbrace{\sigma(W_t^**+\sigma t)-\frac{1}{2}\sigma^2t}\\
	&= S_0 e^{(r+\sigma^2)t}\exp\enbrace{\sigma W_t^**-\frac{1}{2}\sigma^2t}
\end{aligned}
\end{equation*}
\uline{Ergebnis:}\\
Der Aktienpreisprozess $(S(t))_{t\ge0}$ ist ein geometrischer Wiener-Prozess mit Trend $\mu$ und Volatilität $\sigma$ bzgl. $\Pw$, dann $r$ und $\sigma$ bzgl. $\Pw^*$ und abschließend $r+\sigma^2$ und $\sigma$ bzgl. $\Pw_\sigma^*$. 
%sub end

\subsection{Bewertung von Claims}
\label{sub:bewertung_claims}
Ein Derivat ist ein Wertpapier, das eine zufällige Auszahlung $C$ zum Zeitpunkt $T$ garantiert.
Im mathematischem Modell entspricht dies einer $\F_T$-messbaren Zufallsvariablen $C$.\\
Annahme:
\[
\E^*\abs{C^*}<\infty, \text{ wobei } C^*:=e^{-rt}C
\]
Klar
\[
\E^*\abs{C^*}<\infty \Leftrightarrow \E^*\abs{C}<\infty
\]
Es gilt:
$C$ ist durch eine selbstfinanzierende Handelsstrategie replizierbar.
Zum Nachweis hierfür benötigt man die stochastische Analysis (siehe Höhere Finanzmathematik).
Dies folgt aus dem \bet{Martingaldarstellungssatz}.\\
Deshalb gibt es einen eindeutigen arbitragefreien Preisprozess $(P_t(C))_{0\le t\le T}$.
Analog zum diskreten ist dieser gegeben durch
\[
e^{-rt} P_t(C)= \E^*(C^*~|~\F_t),~0\le t\le T
\]
Insbesondere ist damit $P_0(C)=\E^*(C^*)=\E^*e^{-rT}C$.
%sub end

\subsection{Black-Scholes Formel}
\label{sub:black_scholes_formel}
\index{Black-Scholes Formel}
Betrachtet wird eine Call-Option
\[
C=(S_T-K)^+
\]
Zu bestimmen ist:
\[
\E^*(e^{-rT}(S_T-K)^+~|~\F_t)=P_t(C)e^{-rt}
\]
Zunächst für $t=0$:
\begin{equation*}
\begin{aligned}
	\E^*e^{-rt}(S_T-K)^+ &= \E^*e^{-rt}S_T\mathbb{1}_{\penbrace{S_T>K}}-e^{-rt}K\Pw^*(S_T>K)\\
	S_0\cdot \E\frac{S_T^*}{S_0}\mathbb{1}_{\penbrace{S_T>K}}-e^{-rt}K\Pw^*(S_T>K)\\
	&= S_0\underbracket{\Pw_\sigma^*(S_T>K)}_{(1)}-e^{-rt}K\underbracket{\Pw^*(S_T>K)}_{(2)}\\
\end{aligned}
\end{equation*}
Mit $S_T=S_0\exp\enbrace{\sigma W_T^**-\frac{1}{2}\sigma^2 T}e^{(r+\sigma^2)T}$ folgt
\begin{equation*}
\begin{aligned}
	\Pw_\sigma^*(S_T>K) &= \Pw_\sigma^*\enbrace{\log\frac{S_T}{S_0}>\log\frac{K}{S_0}}\\
	&= \Pw_\sigma^*\enbrace{\sigma W_T^**-\frac{1}{2}\sigma^2 T+(r+\sigma^2)T>\log\frac{K}{S_0}}\\
	&= \Pw_\sigma^*\enbrace{\frac{W_T^**}{\sqrt{T}}>\frac{\log\frac{K}{S_0}+\frac{1}{2}\sigma^2T-(r+\sigma^2)T}{\sigma\sqrt{T}}}\\
	&= \phi\enbrace{\frac{\log\frac{S_0}{K}+\enbrace{r+\frac{1}{2}\sigma^2}T}{\sigma\sqrt{T}}}\\
	&\text{mit } \phi(y)=\frac{1}{\sqrt{2\pi}}\int\ablim{-\infty}^ye^{-\frac{1}{2}z^2}\dint z
\end{aligned}
\end{equation*}
Weiter:
\begin{equation*}
\begin{aligned}
	\Pw^*(S_T>K) &= \Pw^*\enbrace{\log\frac{S_t}{S_0}>\log\frac{K}{S_0}}\\
	&= \Pw^*\enbrace{\sigma W_T^*-\frac{1}{2}\sigma^2T+rT>\log\frac{K}{S_0}}\\
	&= \Pw^*\enbrace{\frac{W_T}{\sqrt{T}}>\frac{\log\frac{K}{S_0}-rT+\frac{1}{2}\sigma^2T}{\sigma\sqrt{T}}}\\
	&= \phi\enbrace{\frac{\log\frac{S_0}{K}+\enbrace{r-\frac{1}{2}\sigma^2}T}{\sigma\sqrt{T}}}
\end{aligned}
\end{equation*}
\uline{Ergebnis:}\\
Bezeichnet $c(S_0,T,K)$ den ANfangspreis einer Call-Option mit Laufzeit $T$, strike $K$ und Anfangsaktienkurs $S_0$, so gilt:
\[
c(S_0,T,K)= S_0\phi(h_1(S_0,T))-Ke^{-rt}\phi(h_2(S_0,T))
\]
mit 
\[
h_1(S_0,T)=\frac{\log\frac{S_0}{K}+\enbrace{r+\frac{1}{2}\sigma^2}T}{\sigma\sqrt{T}},~h_2(S_0,T)=\frac{\log\frac{S_0}{K}+\enbrace{r-\frac{1}{2}\sigma^2}T}{\sigma\sqrt{T}}
\]
Da der Aktienpreisprozess - gegeben $\F_t$ - sich verhält wie in einem Black-Scholes Modell, mit Laufzeit $T-t$ und Anfangskurs $S_t$, ergibt sich für den Call-Preis zum Zeitpunkt $t$:
\[
P_t(C)=c(S_t,T_t,K)
\]
Genauer kann man zeigen, mit Hilfe der Markov-Eigenschaft:
\begin{equation*}
\begin{aligned}
	P_t(C) &= \E^*\enbrace{e^{-rt}(S_T-K)^+~|~\F_t}e^{rt} = e^{-r(T-t)}\E^*((S_T-K)^+~|~\F_t)\\
	&= e^{-r(T-t)}\E^*((S_T-K)^+~|~S_t) = \E^*(e^{-r(T-t)}(S_T-K)^+~|~S_t)\\
	&= c(S_t,T-t,K)
\end{aligned}
\end{equation*}
denn
\[
\E^*((S_T-K)^+~|~S_t=x) = \E^*((S_{T-t}-K)^+~|~S_0=x)
\]
%sub end

\subsection{Greeks}
\label{sub:greeks}
\uline{Eigenschaften des Call-Preises:}\\
Sei $c(x,t,\sigma,K)$ der Preis einer Call-Option mit Laufzeit $t$, Volatilität $\sigma$, strike $K$ und Anfangskurs $x$.
\[
c(x,t,\sigma,K)= S_0\phi(h_1(x,t))-Ke^{-rt}\phi(h_2(x,t))
\]
\begin{enumerate}[(i)]
	\item $\lim\ablim{t\downarrow 0}c(x,t,\sigma,K)=(x-K)^+$
	\item $\partial_tc+\frac{1}{2}\sigma^2x^2\partial_x^2c+x\partial_xc=rc$ auf $(0,\infty)\times(0,\infty)$.
	Dies ist die \Index{Black-Scholes Differenzialgleichung}.
	Dann gilt:
	\[
	x\varphi(h_1(x,t))-Ke^{-rt}\varphi(h_2(x,t))=0
	\]
	\item $c$ ist strikt wachsend als Funktion des Aktienanfangskurses mit
	\[
	\Delta =\partial_xc=\phi(h_1)>0
	\]
	$\partial_xc$ ist das sogenannte \Index{Delta der Option}.
	Das Delta bestimmt die Replikationsstrategie.\\
	Setze $H(t)=\partial_xc(S(t),T-t)$ und $\psi(t)=-Ke^{-rt}\phi(h_2(S(t),T-t))$.
	Dann wird durch $(\psi(t),H(t))_{0\le t\le T}$ eine selbstfinanzierende Handelsstrategie definiert, die die Call-Option repliziert.
\end{enumerate}





\cleardoubleoddemptypage
\pagenumbering{Alph}
\setcounter{page}{1}


\printindex
\listoffigures
\end{document}